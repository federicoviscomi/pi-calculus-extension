
\section{Syntax}

As we did whit $\pi$ calculus, we suppose that we have a countable set of names $\mathbb{N}$, ranged over by lower case letters $a,b, \cdots, z$. This names are used for communication channels and values. Furthermore we have a set of identifiers, ranged over by $A$. We represent the agents or processes by upper case letters $P,Q, \cdots $. A multi $\pi$ process, in addiction to the same actions of a $\pi$ process, can perform also a strong prefix input:
\begin{center}
  $\pi$ ::= $\overline{x}y$ | $x(z)$ | $\underline{x(y)}$ | $\tau$ 
\end{center}
The process are defined, just as original $\pi$ calculus, by the following grammar:
\begin{center}
  \begin{tabular}{l}
    $P,Q$ ::= $0$ | $\pi.P$ | $P|Q$ | $P+Q$ | $(\nu x) P$ | $A(y_{1}, \cdots, y_{n})$
  \end{tabular}
\end{center}
and they have the same intuitive meaning as for the $\pi$ calculus. The strong prefix input allows a process to make an atomic sequence of actions, so that more than one process can synchronize on this sequence. For the moment we allow the strong prefix to be on input names only. Also one can use the strong prefix only as an action prefixing for processes that can make at least a further action. Since the strong prefix can be on input names only, the only synchronization possible is between a process that executes a sequence of $n$ actions(only the last action can be an output) with $n\geq 1$ and $n$ other processes each executing one single action(at least $n-1$ process execute an output and at most one executes an input).

Multi $\pi$ calculus is a conservative extension of the $\pi$ calculus in the sense that: any $\pi$ calculus process $p$ is also a multi $\pi$ calculus process and the semantic of $p$ according to the SOS rules of $\pi$ calculus is the same as the semantic of $p$ according to the SOS rules of multi $\pi$ calculus. 
We have to extend the following definition to deal with the strong prefix:
\begin{center}
  \begin{tabular}{ll}
	$B(\underline{x(y)}.Q, I)\; =\; \{y,\overline{y}\}\cup B(Q, I)$
      &
	$F(\underline{x(y)}.Q, I)\; =\; \{x,\overline{x}\}\cup (F(Q, I)-\{y,\overline{y}\})$
    \\
  \end{tabular}
\end{center}


In this setting two process cannot synchronize on a sequence of actions with length greater than one so we cannot have transactional synchronization but we can have multi-party synchronization.


\section{Operational semantic}

\subsection{Early operational semantic with structural congruence}

The semantic of a multi $\pi$ process is labeled transition system such that
\begin{itemize}
  \item 
    the nodes are multi $\pi$ calculus process. The set of node is $\mathbb{P}_{m}$
  \item
    the actions are multi $\pi$ calculus actions. The set of actions is $\mathbb{A}_{m}$, we use $\alpha, \alpha_{1}, \alpha_{2},\cdots $ to range over the set of actions, we use $\sigma, \sigma_{1}, \sigma_{2}, \cdots $ to range over the set $\mathbb{A}_{m}^{+} \cup \{\tau\}$.
  \item
    the transition relations is $\rightarrow\subseteq \mathbb{P}_{m}\times (\mathbb{A}_{m}^{+} \cup \{\tau\})\times \mathbb{P}_{m}$
\end{itemize}

In this case, a label can be a sequence of prefixes, whether in the original $\pi$ calculus a label can be only a prefix. We use the symbol $\cdot$ to denote the concatenation operator.

\begin{definition}\index{transition relation! multipi! input only! early! with structural congruence}
  The \emph{early transition relation with structural congruence} is the smallest relation induced by the rules in table \ref{multipisoloinputearlywith} where $inpSeq$ is a non empty sequence of input actions and $\sigma$ is a sequence of any action.
  \begin{table}
    \begin{tabular}{ll}
	  \hline\\
	  $\inferrule* [left=\bf{Out}]{
	  }{
	    \overline{x}y.P \;\xrightarrow{\overline{x}y} P
	  }$
	&
	  $\inferrule* [left=\bf{EInp}]{
	  }{
	    x(y).P \;\xrightarrow{xz} P\{z/y\}
	  }$
      \\\\
	  $\inferrule* [left=\bf{Tau}]{
	  }{
	    \tau.P \;\xrightarrow{\tau} P
	  }$
	&
	  $\inferrule* [left=\bf{SInp}]{
	      P\{y/z\} \xrightarrow{\sigma} P^{'}
	    \\
	      \sigma\neq \tau
% 	    \\
% 	      y\notin fn((\nu z) P)
	  }{
	    \underline{x(z)}.P \xrightarrow{xy \cdot \sigma} P^{'}
	  }$
      \\\\
	  $\inferrule* [left=\bf{Sum}]{
	    P \;\xrightarrow{\sigma} P^{'}
	  }{
	    P+Q \;\xrightarrow{\sigma} P^{'}
	  }$
	&
	  $\inferrule* [left=\bf{Str}]{
	      P\equiv P^{'}
	    \\
	      P^{'} \xrightarrow{\alpha} Q
	  }{
	      P \xrightarrow{\alpha} Q
	  }$
      \\\\
	  $\inferrule* [left=\bf{Com}]{
	      P \xrightarrow{\overline{x}y} P^{'}
	    \\
	      Q \xrightarrow{xy} Q^{'}
	  }{
	    P|Q \xrightarrow{\tau} P^{'}|Q^{'}
	  }$
	&
	  $\inferrule* [left=\bf{ComSeq}]{
	      P \xrightarrow{xy\cdot \sigma} P^{'}
	    \\
	      Q \xrightarrow{\overline{x}y} Q^{'}
	  }{
	    P|Q \xrightarrow{\sigma} P^{'}|Q^{'}
	  }$
      \\\\
	  $\inferrule* [left=\bf{Res}]{
	      P\; \xrightarrow{\sigma}\; P^{'}
	    \\
	      z\notin n(\sigma)
	  }{
	    (\nu z) P \;\xrightarrow{\sigma} (\nu z) P^{'}
	  }$
	&
	  $\inferrule* [left=\bf{SInpTau}]{
	      P\{y/z\}\; \xrightarrow{\tau}\; P^{'}
	  }{
	    \underline{x(z)}.P \;\xrightarrow{xy} P^{'}
	  }$
      \\\\
	  $\inferrule* [left=\bf{Par}]{
	      P \;\xrightarrow{\sigma}\; P^{'}
	    \\ 
	      bn(\sigma)\cap fn(Q)=\emptyset
	  }{
	      P|Q \;\xrightarrow{\sigma} P^{'}|Q
	  }$
	&
	  $\inferrule* [left=\bf{Opn}]{
	      P \xrightarrow{\overline{x}z}\; P^{'}
	    \\ 
	      z\neq x
	  }{
	      (\nu z)P \xrightarrow{\overline{x}(z)}\; P^{'}
	  }$
      \\\\
	&
	  $\inferrule* [left=\bf{OpnSeq}]{
	      P \xrightarrow{inpSeq \cdot \overline{x}z}\; P^{'}
	    \\ 
	      z\neq x
	  }{%lo strong prefixing e' solo sull'input e quindi ci puo' essere un solo output e deve essere alla fine
	      (\nu z)P \xrightarrow{inpSeq \cdot \overline{x}(z)}\; P^{'}
	  }$
      \\\hline
    \end{tabular}
    \caption{Multi $\pi$ early semantic with structural congruence}
    \label{multipisoloinputearlywith}
  \end{table}
\end{definition}



\begin{example}[Multi-party synchronization]
  We show an example of a derivation of three processes that synchronize.
  \begin{center}
  $
      \inferrule* [left=\bf{EComSng}]{
	\underline{x(a)}.x(b).P|\overline{x}y.Q)
	  \xrightarrow{xz}
	    P\{y/a\}\{z/b\}|Q
	\\
	  \inferrule* [left=\bf{Out}]{
	  }{
	    \overline{x}z.R	
	      \xrightarrow{\overline{x}z} 
		R
	  }
      }{
	(\underline{x(a)}.x(b).P|\overline{x}y.Q)|\overline{x}z.R
	  \xrightarrow{\tau}
	    (P\{y/a\}\{z/b\}|Q)|R
      }
  $
  \end{center}
  
  \begin{center}
  $\inferrule* [left=\bf{EComSeq}]{
      \inferrule* [left=\bf{SInp}]{
	\inferrule* [left=\bf{EInp}]{
	}{
	  (x(b).P)\{y/a\} \xrightarrow{xz} P\{y/a\}\{z/b\}
	}
      }{
	\underline{x(a)}.(x(b).P) 
	  \xrightarrow{xy \cdot xz} 
	    P\{y/a\}\{z/b\}
      }
    \\
      \inferrule* [left=\bf{Out}]{
      }{
	\overline{x}y.Q\; \;\xrightarrow{\overline{x}y}\; Q
      }
  }{
	\underline{x(a)}.x(b).P|\overline{x}y.Q)
	  \xrightarrow{xz}
	    P\{y/a\}\{z/b\}|Q
  }$
  \end{center}

\end{example}

\begin{lemma}\label{lemmastrongsequence}
  If $P\xrightarrow{\sigma} Q$ then only one of the following cases hold: 
  \begin{itemize}
    \item 
      $|\sigma|=1$
    \item
      $|\sigma|>1$, the first $|\sigma|-1$ actions are input and the last actions can be an input or an output.
  \end{itemize}
\end{lemma}



\subsection{Late operational semantic with structural congruence}

\begin{definition}\index{transition relation! multipi! input only! late! with structural congruence}
  The \emph{late transition relation with structural congruence} is the smallest relation induced by the rules in table \ref{multipisoloinputlateywith}.
  \begin{table}
    \begin{tabular}{ll}
	\hline\\
     	  $\inferrule* [left=\bf{Pref}]{
	    \alpha\; not\; a\; strong\; prefix
	  }{
	    \alpha.P \;\xrightarrow{\alpha} P
	  }$
	&
	  $\inferrule* [left=\bf{LComSeq}]{
	      P \;\xrightarrow{x(y)\cdot \sigma} P^{'}
	    \\
	      Q\;\xrightarrow{\overline{x}z} Q^{'}
	    \\
	      z\notin fn(\sigma)\cup fn(P)
	  }{
	    P|Q \;\xrightarrow{\sigma\{z/y\}} P^{'}\{z/y\}|Q^{'}
	  }$
      \\\\
	  $\inferrule* [left=\bf{SInp}]{
	      P\; \xrightarrow{\sigma}\; P^{'}
	    \\
	      \sigma\neq \tau
	  }{
	    \underline{x(y)}.P \;\xrightarrow{x(y) \cdot \sigma} P^{'}
	  }$
	&
	  $\inferrule* [left=\bf{LComSng}]{
	      P \;\xrightarrow{x(y)} P^{'}
	    \\
	      Q\;\xrightarrow{\overline{x}z} Q^{'}
	    \\
	      z\notin fn(P)
	  }{
	    P|Q \;\xrightarrow{\tau} P^{'}\{z/y\}|Q^{'}
	  }$

      \\\\
	  $\inferrule* [left=\bf{Sum}]{
	    P\; \xrightarrow{\sigma}\; P^{'}
	  }{
	    P+Q \;\xrightarrow{\sigma} P^{'}
	  }$
	&
	  $\inferrule* [left=\bf{Str}]{
	      P\equiv P^{'}
	    \\
	      P^{'}\; \;\xrightarrow{\alpha}\; Q^{'}
	    \\
	      Q\equiv Q^{'}
	  }{
	      P\; \;\xrightarrow{\alpha}\; Q
	  }$
      \\\\
	  $\inferrule* [left=\bf{Res}]{
	      P\; \xrightarrow{\sigma}\; P^{'}
	    \\
	      z\notin n(\alpha)
	  }{
	    (\nu z) P \;\xrightarrow{\sigma} (\nu z) P^{'}
	  }$
	&
	  $\inferrule* [left=\bf{Par}]{
	      P\; \xrightarrow{\sigma}\; P^{'}
	    \\
	      bn(\sigma)\cup fn(Q)=\emptyset
	  }{
	    P|Q \;\xrightarrow{\sigma} P^{'}|Q
	  }$
      \\\\
	  $\inferrule* [left=\bf{Opn}]{
	      P \xrightarrow{\overline{x}z}\; P^{'}
	    \\ 
	      z\neq x
	  }{
	      (\nu z)P \xrightarrow{\overline{x}(z)}\; P^{'}
	  }$
	&
	  $\inferrule* [left=\bf{OpnSeq}]{
	      P \xrightarrow{inpSeq \cdot \overline{x}z}\; P^{'}
	    \\ 
	      z\neq x
	  }{
	      (\nu z)P \xrightarrow{inpSeq \cdot \overline{x}(z)}\; P^{'}
	  }$
      \\\hline
    \end{tabular}
    \caption{Multi $\pi$ late semantic with structural congruence}
    \label{multipisoloinputlateywith}
  \end{table}
\end{definition}

\begin{example}[Multi-party synchronization]
  We show an example of a derivation of three processes that synchronize with the late semantic. The three processes are $\underline{x(a)}.x(b).P$, $\overline{x}y.Q$ and $\overline{x}z.R$. We assume that:
  \begin{center}
      $a\notin fn(x(b))\cup fn (\underline{x(a)}.x(b).P)$
  \end{center}
  and
  \begin{center}
      $b\notin fn(\underline{x(a)}.x(b).P|\overline{x}y.Q)$
  \end{center}

  \begin{center}
  $
      \inferrule* [left=\bf{LComSng}]{
	\underline{x(a)}.x(b).P|\overline{x}y.Q
	  \xrightarrow{x(b)}
	    P\{y/a\}|Q
	\\
	  \inferrule* [left=\bf{Pref}]{
	  }{
	    \overline{x}z.R	
	      \xrightarrow{\overline{x}z} 
		R
	  }
      }{
	(\underline{x(a)}.x(b).P|\overline{x}y.Q)|\overline{x}z.R
	  \xrightarrow{\tau}
	    (P\{y/a\}|Q)\{z/b\}|R
      }
  $
  \end{center}
  
  \begin{center}
  $\inferrule* [left=\bf{LComSeq}]{
      \inferrule* [left=\bf{SInp}]{
	\inferrule* [left=\bf{Pref}]{
	}{
	  x(b).P \xrightarrow{x(b)} P
	}
      }{
	\underline{x(a)}.x(b).P
	  \xrightarrow{x(a) \cdot x(b)} 
	    P
      }
    \\
      \inferrule* [left=\bf{Out}]{
      }{
	\overline{x}y.Q\; \;\xrightarrow{\overline{x}y}\; Q
      }
  }{
	\underline{x(a)}.x(b).P|\overline{x}y.Q)
	  \xrightarrow{x(b)}
	    P\{y/a\}|Q
  }$
  \end{center}

\end{example}



\subsection{Low level semantic}
This section contains the definition of an alternative semantic for multi $\pi$. First we define a low level version of the multi $\pi$ calculus(here with strong prefixing on input only), we call this language low multi $\pi$. The low multi $\pi$ is the multi $\pi$ enriched with a marked or intermediate process $*P$:
\begin{center}
   \begin{tabular}{l}
     $P,Q$ ::= $0$ | $\pi.P$ | $P|Q$ | $P+Q$ | $(\nu x) P$ | $A(x_{1}, \cdots, x_{n})$ | $*P$
   \\\\
     $\pi$ ::= $\overline{x}y$ | $x(z)$ | $\underline{x(y)}$ | $\tau$ 
   \end{tabular}
\end{center}
\begin{definition}
  The low level transition relation is the smallest relation induced by the rules in table \ref{lowleveltransitionrelation} in which $P$ stands for a process without mark, $L$ stands for a process with mark and $S$ can stand for both. 
  \begin{table}
    \begin{tabular}{ll}
      \hline\\
	  $\inferrule* [left=\bf{Out}]{
	  }{
	    \overline{x}y.P \stackrel{\overline{x}y}{\longmapsto} P
	  }$
	  &
	  $\inferrule* [left=\bf{EInp}]{
	  }{
	    x(y).P \stackrel{xz}{\longmapsto} P\{z/y\}
	  }$
      \\\\
	  $\inferrule* [left=\bf{Star}]{
	      S \stackrel{\gamma}{\longmapsto} S^{'}
	  }{
	      *S \stackrel{\gamma}{\longmapsto} S^{'}
	  }$
	  &
	  $\inferrule* [left=\bf{Tau}]{
	  }{
	    \tau.P \stackrel{\tau}{\longmapsto} P
	  }$
      \\\\
	  $\inferrule* [left=\bf{Sum}]{
	    P \stackrel{\gamma}{\longmapsto} S
	  }{
	    P+Q \stackrel{\gamma}{\longmapsto} S
	  }$
	&
	  $\inferrule* [left=\bf{SInpLow}]{
%	      y\notin fn(P)-\{z\}
	  }{
	    \underline{x(z)}.P \stackrel{xy}{\longmapsto} * P\{y/z\}
	  }$
      \\\\\\
	  $\inferrule* [left=\bf{Com1}]{
	      P \stackrel{\overline{x}y}{\longmapsto} P^{'}
	    \\
	      Q \stackrel{xy}{\longmapsto} Q^{'}
	  }{
	    P|Q \stackrel{\tau}{\longmapsto} P^{'}|Q^{'}
	  }$
	&
      \\\\
	  $\inferrule* [left=\bf{Com2}]{
	      L_{1} \stackrel{xy}{\longmapsto} L_{1}^{'}
	    \\
	      L_{2} \stackrel{\overline{x}y}{\longmapsto} P
	  }{
	    L_{1}|L_{2} \stackrel{\epsilon}{\longmapsto} L_{1}^{'}|P
	  }$
	&
	  $\inferrule* [left=\bf{Com2R}]{
	      L_{1} \stackrel{xy}{\longmapsto} L_{1}^{'}
	    \\
	      L_{2} \stackrel{\overline{x}y}{\longmapsto} P
	  }{
	    L_{2}|L_{1} \stackrel{\epsilon}{\longmapsto} P|L_{1}^{'}
	  }$
      \\\\
	  $\inferrule* [left=\bf{Com3L}]{
	      P \stackrel{xy}{\longmapsto} L
	    \\
	      Q \stackrel{\overline{x}y}{\longmapsto} Q^{'}
	  }{
	    P|Q \stackrel{\epsilon}{\longmapsto} L|Q^{'}
	  }$
	&
	  $\inferrule* [left=\bf{Com3R}]{
	      P \stackrel{xy}{\longmapsto} L
	    \\%serve questa regola???
	      Q \stackrel{\overline{x}y}{\longmapsto} Q^{'}
	  }{
	    Q|P \stackrel{\epsilon}{\longmapsto} Q^{'}|L
	  }$
      \\\\
	  $\inferrule* [left=\bf{Com4L}]{
	      L_{1} \stackrel{\overline{x}y}{\longmapsto} P
	    \\
	      L_{2} \stackrel{xy}{\longmapsto} Q
	  }{
	    L_{1}|L_{2} \stackrel{\tau}{\longmapsto} P|Q
	  }$
	&
	  $\inferrule* [left=\bf{Com4R}]{
	      L_{1} \stackrel{xy}{\longmapsto} P
	    \\
	      L_{2} \stackrel{\overline{x}y}{\longmapsto} Q
	  }{
	    L_{1}|L_{2} \stackrel{\tau}{\longmapsto} P|Q
	  }$
      \\\\\\
	  $\inferrule* [left=\bf{Res}]{
	      S \stackrel{\gamma}{\longmapsto} S^{'}
	    \\
	      y\notin n(\gamma)
	  }{
	    (\nu y) S \stackrel{\gamma}{\longmapsto} (\nu y) S^{'}
	  }$
	&
	  $\inferrule* [left=\bf{Opn}]{
	      S \stackrel{\overline{x}y}{\longmapsto} P
	    \\ 
	      y\neq x
	  }{
	      (\nu y)S \stackrel{\overline{x}(y)}{\longmapsto} P
	  }$
      \\\\\\
	  $\inferrule* [left=\bf{Par1L}]{
	      S \stackrel{\gamma}{\longmapsto} S^{'}
% 	    \\ 
% 	      bn(\gamma)\cap fn(Q)=\emptyset
	  }{
	      S|Q \stackrel{\gamma}{\longmapsto} S^{'}|Q
	  }$
	&
	  $\inferrule* [left=\bf{Par1R}]{
	      S \stackrel{\gamma}{\longmapsto} S^{'}
% 	    \\ 
% 	      bn(\gamma)\cap fn(Q)=\emptyset
	  }{
	      Q|S \stackrel{\gamma}{\longmapsto} Q|S^{'}
	  }$
      \\\\\\
	  $\inferrule* [left=\bf{Par2L}]{
	     P \stackrel{\gamma}{\longmapsto} L
% 	    \\ 
% 	      bn(\gamma)\cap fn(Q)=\emptyset
	  }{
	      P|Q \stackrel{\gamma}{\longmapsto} L|*Q
	  }$
	&
	  $\inferrule* [left=\bf{Par2R}]{
	     P \stackrel{\gamma}{\longmapsto} L
% 	    \\ 
% 	      bn(\gamma)\cap fn(Q)=\emptyset
	  }{
	      Q|P \stackrel{\gamma}{\longmapsto} *Q|L
	  }$

      \\\\
	  $\inferrule* [left=\bf{Par3L}]{
	      L_{1} \stackrel{\gamma}{\longmapsto} L_{1}^{'}
% 	    \\ 
% 	      bn(\gamma)\cap fn(L_{2})=\emptyset
	  }{
	      L_{1}|L_{2} \stackrel{\gamma}{\longmapsto} L_{1}^{'}|L_{2}
	  }$
	&
	  $\inferrule* [left=\bf{Par3R}]{
	      L_{2} \stackrel{\gamma}{\longmapsto} L_{2}^{'}
% 	    \\ 
% 	      bn(\gamma)\cap fn(L_{2})=\emptyset
	  }{
	      L_{1}|L_{2} \stackrel{\gamma}{\longmapsto} L_{1}|L_{2}^{'}
	  }$
      \\\\\\
	&
	  $\inferrule* [left=\bf{Cong}]{
	      P\equiv P^{'}
	    \\
	      P^{'} \stackrel{\gamma}{\longmapsto} S
	  }{
	      P \stackrel{\gamma}{\longmapsto} S
	  }$
      \\\\\hline
    \end{tabular}
    \caption{Low multi $\pi$ early semantic with structural congruence}
    \label{lowleveltransitionrelation}
  \end{table}
\end{definition}



\begin{lemma}\label{lemmacom3}
  For all unmarked processes $P,Q$ and marked processes $L_{1}, L_{2}$.
  \begin{itemize}
    \item
      if $P\stackrel{\alpha}{\longmapsto}L_{1}$ or $L_{1}\stackrel{\alpha}{\longmapsto}L_{2}$ then $\alpha$ can only be a free input or an $\epsilon$
    \item
      if $L_{1}\stackrel{\alpha}{\longmapsto}P$ of $P\stackrel{\alpha}{\longmapsto}Q$ then $\alpha$ is not an $\epsilon$
  \end{itemize}
  \begin{proof}
    DA FARE
  \end{proof}
\end{lemma}


\begin{lemma}\label{lemmalabellowhigh}
  If $P\stackrel{\gamma_{1}}{\longmapsto} L_{1} \stackrel{\gamma_{2}}{\longmapsto} \cdots  \stackrel{\gamma_{k}}{\longmapsto} L_{k} \stackrel{\gamma_{k+1}}{\longmapsto} Q$ and $\sigma = \gamma_{1}\cdot \ldots \cdot \gamma_{k+1}$ with $k\geq 1$ then $\sigma$ is a sequence of input and the last action in $\sigma$ is an input or an output.
  \begin{proof}
    DA FARE
  \end{proof}
\end{lemma}

  
\begin{definition}\label{low}
  Let $P, Q$ be unmarked processes and $L_{1}, \cdots, L_{k-1}$ marked processes. We define the derivation relation $\rightarrow_{s}$ in the following way:
  \begin{center}
    $\inferrule* [left=\bf{Low}]{
	P \stackrel{\gamma_{1}}{\longmapsto} L_{1} \stackrel{\gamma_{2}}{\longmapsto} L_{2} \cdots L_{k-1} \stackrel{\gamma_{k}}{\longmapsto} Q
      \\
	k\geq 1
    }{
      P \xrightarrow{\gamma_{1} \cdots \gamma_{k}}_{s}  Q
    }$
  \end{center}
  We need to be precise about the concatenation operator $\cdot$ since we have introduced the new label $\epsilon$. Let $a$ be an action such that $a\neq \tau$ and $a\neq \epsilon$ then the following rules hold:
  \begin{center}
      \begin{tabular}{lll}
	  $\epsilon \cdot a = a \cdot \epsilon = a$
	&
	  $\epsilon \cdot \epsilon = \epsilon$
	&
	  $\tau \cdot \epsilon = \epsilon \cdot \tau = \tau$
	\\
	  $\tau \cdot a = a \cdot \tau = a$
	&
	  $\tau \cdot \tau = \tau$
	&
      \end{tabular}
  \end{center}

  HA PIU' SENSO DIRE $a \cdot \tau \neq a$ e $\tau \cdot \tau \neq \tau$?

\end{definition}

\begin{example}[Multi-party synchronization]
  We show an example of a derivation of three processes that synchronize.
 
  \begin{center}
  $\inferrule* [left=\bf{Par2}]{\inferrule* [left=\bf{Com3}]{
      \inferrule* [left=\bf{SInpLow}]{
      }{
	\underline{x(a)}.x(b).P
	  \stackrel{xy}{\longmapsto}
	    *(x(b).P\{y/a\})
      }
    \\
      \inferrule* [left=\bf{Out}]{
      }{
	\overline{x}y.Q \stackrel{\overline{x}y}{\longmapsto} Q
      }
  }{
	\underline{x(a)}.x(b).P|\overline{x}y.Q
	  \stackrel{\epsilon}{\longmapsto}
	    *(x(b).P\{y/a\})|Q
  }}{
	(\underline{x(a)}.x(b).P|\overline{x}y.Q) | \overline{x}z.R
	  \stackrel{\epsilon}{\longmapsto}
	    (*(x(b).P\{y/a\})|Q)|*(\overline{x}z.R)
  }$
  \end{center}

  \begin{center}
    $\inferrule* [left=\bf{Par1}]{
      \inferrule*[left=\bf{Star}]{
	\inferrule* [left=\bf{EInp}]{
	}{
	  x(b).P\{y/a\} \stackrel{xz}{\longmapsto} P\{y/a\}\{z/b\}
	}
      }{
	*(x(b).P\{y/a\}) \stackrel{xz}{\longmapsto} P\{y/a\}\{z/b\}      
      }
    }{
      *(x(b).P\{y/a\}) | Q \stackrel{xz}{\longmapsto} P\{y/a\}\{z/b\} | Q
    }$
  \end{center}

  \begin{center}
  $
      \inferrule* [left=\bf{Com4}]{
	  *(x(b).P\{y/a\}) | Q \stackrel{xz}{\longmapsto} P\{y/a\}\{z/b\} | Q
	\\
	  \inferrule* [left=\bf{Star}]{
	    \inferrule* [left=\bf{Out}]{
	    }{
	      \overline{x}z.R	
		\stackrel{\overline{x}z}{\longmapsto}
		  R
	    }
	  }{
	    *\overline{x}z.R	
	      \stackrel{\overline{x}z}{\longmapsto}
		R
	  }
      }{
	(\underline{x(a)}.x(b).P|\overline{x}y.Q)|*(\overline{x}z.R)
	  \stackrel{\tau}{\longmapsto}
	    (P\{y/a\}\{z/b\}|Q)|R
      }
  $
  \end{center}

  \begin{center}
  $
      \inferrule* [left=\bf{Low}]{
	(\underline{x(a)}.x(b).P|\overline{x}y.Q) | \overline{x}z.R
	  \stackrel{\epsilon}{\longmapsto}
	    (*(x(b).P\{y/a\})|Q)|*(\overline{x}z.R)
	      \xrightarrow{\tau}
		(P\{y/a\}\{z/b\}|Q)|R
      }{
	(\underline{x(a)}.x(b).P|\overline{x}y.Q)|\overline{x}z.R
	  \xrightarrow{\tau}_{s}
	    (P\{y/a\}\{z/b\}|Q)|R
      }
  $
  \end{center}

\end{example}




\begin{proposition}
  Let $\rightarrow$ be the relation defined in table \ref{multipisoloinputearlywith} and let $P$ be a multi $\pi$ process. Then 
  \begin{center}
    $P\xrightarrow{\sigma} Q \Rightarrow P\xrightarrow{\sigma}_{s} Q$
  \end{center}
  \begin{proof}
    DA MODIFICARE PERCHE' LE REGOLE COM2 E COM3 SONO CAMBIATE 
    The proof is by induction on the depth of the derivation tree of $P\xrightarrow{\sigma} Q$:
    \begin{description}
      \item[base case]
    \end{description}
	If the depth is one then the rule used have to be one of: $EInp$, $Out$, $Tau$. These rules are also in table \ref{lowleveltransitionrelation} so we can derive $P \stackrel{\sigma}{\longmapsto} Q$ and then apply $Low$ to get the result $P\xrightarrow{\sigma}_{s} Q$.
    \begin{description}
      \item[inductive case]
    \end{description}
	If the depth is greater than one then the last rule used in the derivation can be:
	\begin{description}
	  \item[$SInp$]: 
	    the last part of the derivation tree looks like this:
	    \begin{center}
	      $\inferrule* [left=\bf{SInp}]{
		  P_{1}\{y/z\} \xrightarrow{\sigma} Q
		\\
		  \sigma\neq \tau
%		\\
%	  	  y\notin fn((\nu z) P)
	      }{
		\underline{x(z)}.P_{1} \xrightarrow{xy \cdot \sigma} Q
	      }$	      
	    \end{center}
	    for inductive hypothesis $P_{1}\{y/z\} \xrightarrow{\sigma}_{s} Q$ which implies that there exist $L_{1}, \cdots, L_{k}$ and $\gamma_{1}, \cdots, \gamma_{k+1}$ with $k\geq 0$ such that 
	    \begin{center}
	      $P_{1}\{y/z\} \stackrel{\gamma_{1}}{\longmapsto} L_{1}  \stackrel{\gamma_{2}}{\longmapsto} L_{2} \cdots L_{k-1} \stackrel{\gamma_{k}}{\longmapsto} L_{k} \stackrel{\gamma_{k+1}}{\longmapsto} Q$ 
	    \end{center}
	    and 
	    \begin{center}
	      $\gamma_{1} \cdot \ldots \cdot \gamma_{k+1} =  \sigma$
	    \end{center}
	    since
	    \begin{center}
	      \begin{tabular}{ll}
		$\inferrule* [left=\bf{SInpLow}]{
 		}{
 		  \underline{x(z)}.P_{1} \stackrel{xy}{\longmapsto} *P_{1}\{y/z\}
 		}$
	      &
		$\inferrule* [left=\bf{Star}]{
 		  P_{1}\{y/z\} \stackrel{\gamma_{1}}{\longmapsto} L_{1}
 		}{
 		  *P_{1}\{y/z\} \stackrel{\gamma_{1}}{\longmapsto} L_{1}
 		}$
	      \end{tabular}
	    \end{center}
	    then a proof of $P\xrightarrow{\sigma}_{s} Q$ is:
	    \begin{center}
	      $\inferrule* [left=\bf{Low}]{
		    \underline{x(z)}.P_{1} \stackrel{xy}{\longmapsto} *P_{1}\{y/z\} \stackrel{\gamma_{1}}{\longmapsto} L_{1}  \stackrel{\gamma_{2}}{\longmapsto} L_{2} \cdots L_{k-1} \stackrel{\gamma_{k}}{\longmapsto} L_{k} \stackrel{\gamma_{k+1}}{\longmapsto} Q
	      }{
		\underline{x(z)}.P_{1} \xrightarrow{xy \cdot \gamma_{1} \cdot \ldots \cdot \gamma_{k+1}}_{s} Q
	      }$
	    \end{center}
	    and $xy \cdot \gamma_{1} \cdot \ldots \cdot \gamma_{k+1}=xy \cdot \sigma$
	  \item[$SInpTau$]: this case is similar to the previous.
	  \item[$Sum$]: 
	the last part of the derivation tree looks like this:
	\begin{center}
	  $\inferrule* [left=\bf{Sum}]{
	    P_{1} \xrightarrow{\sigma} Q
	  }{
	    P_{1}+P_{2} \xrightarrow{\sigma} Q
	  }$
	\end{center}
	for the inductive hypothesis $P_{1} \xrightarrow{\sigma}_{s} Q$ which implies that there exist $L_{1}$, $\cdots$, $L_{k}$ and $\gamma_{1}$, $\cdots$, $\gamma_{k+1}$ with $k\geq 0$ such that 
	\begin{center}
	  $P_{1} \stackrel{\gamma_{1}}{\longmapsto} L_{1}  \stackrel{\gamma_{2}}{\longmapsto} L_{2} \cdots L_{k-1} \stackrel{\gamma_{k}}{\longmapsto} L_{k} \stackrel{\gamma_{k+1}}{\longmapsto} Q$ 
	\end{center}
	and 
	\begin{center}
	  $\gamma_{1} \cdot \ldots \cdot \gamma_{k+1} =  \sigma$
	\end{center}
	A proof of the conclusion is:
	\begin{center}
	  $\inferrule* [left=\bf{Low}]{
	    \inferrule* [left=\bf{Sum}]{
	      P_{1} \stackrel{\gamma_{1}}{\longmapsto} L_{1}
	    }{
	      P_{1}+P_{2} \stackrel{\gamma_{1}}{\longmapsto} L_{1}
	    }
	    \\
	      L_{1}  \stackrel{\gamma_{2}}{\longmapsto} L_{2} \cdots L_{k-1} \stackrel{\gamma_{k}}{\longmapsto} L_{k} \stackrel{\gamma_{k+1}}{\longmapsto} Q
	  }{
	    P+R \xrightarrow{\sigma}_{s} Q
	  }$
	\end{center}
      \item[$Str$]: this case is similar to the previous.
      \item[$Com$]: 
	the last part of the derivation tree looks like this:
	\begin{center}
	  $\inferrule* [left=\bf{Com}]{
	      P_{1} \xrightarrow{\overline{x}y} P_{1}^{'}
	    \\
	      Q_{1} \xrightarrow{xy} Q_{1}^{'}
	  }{
	    P_{1}|Q_{1} \xrightarrow{\tau} P_{1}^{'}|Q_{1}^{'}
	  }$
	\end{center}
	for inductive hypothesis $P_{1} \xrightarrow{\overline{x}y}_{s} P_{1}^{'}$ and $Q_{1} \xrightarrow{xy}_{s} Q_{1}^{'}$ which imply that there exist $L_{1}, \cdots, L_{k}$ and $\gamma_{1}, \cdots, \gamma_{k+1}$ with $k\geq 0$ such that 
	\begin{center}
	  $P_{1} \stackrel{\gamma_{1}}{\longmapsto} L_{1}  \stackrel{\gamma_{2}}{\longmapsto} L_{2} \cdots L_{k-1} \stackrel{\gamma_{k}}{\longmapsto} L_{k} \stackrel{\gamma_{k+1}}{\longmapsto} P_{1}^{'}$ 
	\end{center}
	and 
	\begin{center}
	  $\gamma_{1} \cdot \ldots \cdot \gamma_{k+1} = \overline{x}y$
	\end{center}
	and there exist $R_{1}, \cdots, R_{h}$ and $\delta_{1}, \cdots, \delta_{h+1}$ with $h\geq 0$ such that 
	\begin{center}
	  $Q_{1} \stackrel{\delta_{1}}{\longmapsto} R_{1}  \stackrel{\delta_{2}}{\longmapsto} R_{2} \cdots R_{h-1} \stackrel{\delta_{h}}{\longmapsto} R_{h} \stackrel{\delta_{h+1}}{\longmapsto} Q_{1}^{'}$ 
	\end{center}
	and 
	\begin{center}
	  $\delta_{1} \cdot \ldots \cdot \delta_{k+1} = xy$
	\end{center}
	we can have nine different cases now: 
	\begin{center}
	$\{\gamma_{1}=\overline{x}y,\; \gamma_{i}=\overline{x}y\; 1<i\leq k, \gamma_{k+1}=\overline{x}y\} \times \{\delta_{1}=xy,\; \delta_{j}=xy\; 1<j\leq h, \delta_{h+1}=xy\}$
	\end{center}
	We show just the first three since the others are similar:
	\begin{description}
	  \item[$\gamma_{1}=\overline{x}y$ $and$ $\delta_{1}=xy$]:
	    in this case: $\gamma_{k+1}=\tau$ and the other $\gamma$s are $\epsilon$; $\delta_{h+1}=\tau$ and the other $\delta$s are $\epsilon$. A proof of the conclusion is:
	    \begin{center}
	      $\inferrule* [left=\bf{Low}]{
		  P_{1}|Q_{1} \stackrel{\tau}{\longmapsto} L_{1}|R_{1} 
			      \stackrel{\epsilon}{\longmapsto} L_{2}|R_{1}
		  \cdots
		\\
		  L_{k-1}|R_{1} \stackrel{\epsilon}{\longmapsto} L_{k}|R_{1}
				\stackrel{\epsilon}{\longmapsto} L_{k}|R_{2}
		  \cdots 
				\stackrel{\epsilon}{\longmapsto} L_{k}|R_{h}
				\stackrel{\tau}{\longmapsto} P_{1}^{'}|R_{h}
				\stackrel{\tau}{\longmapsto} P_{1}^{'}|Q_{1}^{'}
	      }{
		P_{1}|Q_{1} \xrightarrow{\tau}_{s} P_{1}^{'}|Q_{1}^{'}
	      }$	  
	    \end{center}
	    we derive the first transaction of the premises with rule $Com3$, whether for the other transactions we use the rule $Par1$ and when necessary $Cong1$.
	  \item[$\gamma_{i}=\overline{x}y$ $and$ $\delta_{1}=xy$]:
	    in this case: $\gamma_{1}=\gamma_{k+1}=\tau$ and the other $\gamma$s are $\epsilon$; $\delta_{h+1}=\tau$ and the other $\delta$s are $\epsilon$. A proof of the conclusion is:
	    \begin{center}
	      $\inferrule* [left=\bf{Low}]{
		  P_{1}|Q_{1} \stackrel{\tau}{\longmapsto} L_{1}|Q_{1} 
			      \stackrel{\epsilon}{\longmapsto} L_{2}|Q_{1}
		  \cdots
			      \stackrel{\epsilon}{\longmapsto} L_{i-1}|Q_{1} 
			      \stackrel{\tau}{\longmapsto} L_{i}|R_{1}
			      \stackrel{\epsilon}{\longmapsto} L_{i+1}|R_{1}
		  \\\cdots 
			      \stackrel{\epsilon}{\longmapsto} L_{k}|R_{1}
			      \stackrel{\epsilon}{\longmapsto} L_{k}|R_{2}
		  \cdots 
			      \stackrel{\epsilon}{\longmapsto} L_{k}|R_{h}
			      \stackrel{\tau}{\longmapsto} P_{1}^{'}|R_{h}
			      \stackrel{\tau}{\longmapsto} P_{1}^{'}|Q_{1}^{'}
	      }{
		P_{1}|Q_{1} \xrightarrow{\tau}_{s} P_{1}^{'}|Q_{1}^{'}
	      }$	  
	    \end{center}
	    we derive the transaction $ L_{i-1}|Q_{1} \stackrel{\tau}{\longmapsto} L_{i}|R_{1}$ with rule $Com3$, whether for the other transactions of the premises we use the rule $Par1$ and when necessary $Cong1$.
	  \item[$\gamma_{k+1}=\overline{x}y$ $and$ $\delta_{1}=xy$]:
	    in this case: $\gamma_{1}=\tau$ and the other $\gamma$s are $\epsilon$; $\delta_{h+1}=\tau$ and the other $\delta$s are $\epsilon$. A proof of the conclusion is:
	    \begin{center}
	      $\inferrule* [left=\bf{Low}]{
		  P_{1}|Q_{1} \stackrel{\tau}{\longmapsto} L_{1}|Q_{1} 
			      \stackrel{\epsilon}{\longmapsto} L_{2}|Q_{1}
		  \cdots
			      \stackrel{\epsilon}{\longmapsto} L_{k}|Q_{1} 
		\\	      \stackrel{\tau}{\longmapsto} P_{1}^{'}|R_{1}
			      \stackrel{\epsilon}{\longmapsto} P_{1}^{'}|R_{2}
		  \cdots 
			      \stackrel{\epsilon}{\longmapsto} P_{1}^{'}|R_{h}
			      \stackrel{\tau}{\longmapsto} P_{1}^{'}|Q_{1}^{'}
	      }{
		P_{1}|Q_{1} \xrightarrow{\tau}_{s} P_{1}^{'}|Q_{1}^{'}
	      }$	  
	    \end{center}
	    we derive the transaction $L_{k}|Q_{1} \stackrel{\tau}{\longmapsto} P_{1}^{'}|R_{1}$ with rule $Com3$, whether for the other transactions of the premises we use the rule $Par1$ and when necessary $Cong1$.
	\end{description}
      \item[$Res$]: 
	the last part of the derivation tree looks like this:
	\begin{center}
	  $\inferrule* [left=\bf{Res}]{
	      P_{1}\; \xrightarrow{\sigma}\; Q_{1}
	    \\
	      z\notin n(\sigma)
	  }{
	    (\nu z) P_{1} \;\xrightarrow{\sigma} (\nu z) Q_{1}
	  }$
	\end{center}
	for the inductive hypothesis $P_{1} \xrightarrow{\sigma}_{s} Q_{1}$ which implies that there exist $L_{1}, \cdots, L_{k}$ and $\gamma_{1}, \cdots, \gamma_{k+1}$ with $k\geq 0$ such that 
	\begin{center}
	  $P_{1} \stackrel{\gamma_{1}}{\longmapsto} L_{1}  \stackrel{\gamma_{2}}{\longmapsto} L_{2} \cdots L_{k-1} \stackrel{\gamma_{k}}{\longmapsto} L_{k} \stackrel{\gamma_{k+1}}{\longmapsto} Q_{1}$ 
	\end{center}
	and 
	\begin{center}
	  $\gamma_{1} \cdot \ldots \cdot \gamma_{k+1} =  \sigma$
	\end{center}
	We can apply the rule $Res$ to each of the previous transitions because 
	\begin{center}
	  $z\notin n(\sigma)$ implies $z\notin n(\gamma_{i})$ for each $i$
	\end{center}
	and then apply the rule $Low$ to get a proof of the conclusion:
	\begin{center}
	  $\inferrule* [left=\bf{Low}]{
	    (\nu z)P_{1} \stackrel{\gamma_{1}}{\longmapsto} (\nu z)L_{1}  \stackrel{\gamma_{2}}{\longmapsto} (\nu z)L_{2} \cdots (\nu z)L_{k-1} \stackrel{\gamma_{k}}{\longmapsto} (\nu z)L_{k} \stackrel{\gamma_{k+1}}{\longmapsto} (\nu z)Q_{1}
	  }{
	    (\nu z)P_{1} \xrightarrow{\sigma}_{s} (\nu z)Q_{1}
	  }$
	\end{center}
      \item[$Par$]: this case is similar to the previous.
      \item[$ComSeq$]: 
	the last part of the derivation tree looks like this:
	\begin{center}
	  $\inferrule* [left=\bf{Com}]{
	      P_{1} \xrightarrow{xy \cdot \sigma} P_{1}^{'}
	    \\
	      Q_{1} \xrightarrow{\overline{x}y} Q_{1}^{'}
	  }{
	    P_{1}|Q_{1} \xrightarrow{\sigma} P_{1}^{'}|Q_{1}^{'}
	  }$
	\end{center}
	for inductive hypothesis $P_{1} \xrightarrow{xy \cdot \sigma}_{s} P_{1}^{'}$ which implies that there exist $L_{1}$, $\cdots$, $L_{k}$ and $\gamma_{1}$, $\cdots$, $\gamma_{k+1}$ with $k\geq 0$ such that 
	\begin{center}
	  $P_{1} \stackrel{\gamma_{1}}{\longmapsto} L_{1}  \stackrel{\gamma_{2}}{\longmapsto} L_{2} \cdots L_{k-1} \stackrel{\gamma_{k}}{\longmapsto} L_{k} \stackrel{\gamma_{k+1}}{\longmapsto} P_{1}^{'}$ 
	\end{center}
	and 
	\begin{center}
	  $\gamma_{1} \cdot \ldots \cdot \gamma_{k+1} = xy \cdot \sigma$
	\end{center}
	Again for inductive hypothesis $Q_{1} \xrightarrow{xy}_{s} Q_{1}^{'}$ which implies that there exist $R_{1}, \cdots, R_{h}$ and $\delta_{1}, \cdots, \delta_{h+1}$ with $h\geq 0$ such that 
	\begin{center}
	  $Q_{1} \stackrel{\delta_{1}}{\longmapsto} R_{1}  \stackrel{\delta_{2}}{\longmapsto} R_{2} \cdots R_{h-1} \stackrel{\delta_{h}}{\longmapsto} R_{h} \stackrel{\delta_{h+1}}{\longmapsto} Q_{1}^{'}$ 
	\end{center}
	and 
	\begin{center}
	  $\delta_{1} \cdot \ldots \cdot \delta_{k+1} = \overline{x}y$
	\end{center}
	We assume that $\sigma$ is not $\tau$ otherwise we can replace this instance of the rule $ComSeq$ with an instance of the rule $Com$ and proceed as in the previous case. We can have six different cases now: 
	\begin{center}
	$\{\gamma_{1}=xy,\; \gamma_{i}=\overline{x}y\; 1<i\leq k\} \times \{\delta_{1}=\overline{x}y,\; \delta_{j}=\overline{x}y\; 1<j\leq h, \delta_{h+1}=\overline{x}y\}$
	\end{center}
	We show just the first two since the others are similar:
	\begin{description}
	  \item[$\gamma_{1}=xy$ $and$ $\delta_{1}=\overline{x}y$]:
	    in this case: $\delta_{h+1}=\tau$ and the other $\delta$s are $\epsilon$. A proof of the conclusion is:
	    \begin{center}
	      $\inferrule* [left=\bf{Low}]{
		  P_{1}|Q_{1} \stackrel{\tau}{\longmapsto} L_{1}|R_{1} 
			      \stackrel{\gamma_{2}}{\longmapsto} L_{2}|R_{1}
		  \cdots
			      \stackrel{\gamma_{k}}{\longmapsto} L_{k}|R_{1}
			\\      \stackrel{\gamma_{k+1}}{\longmapsto} P_{1}^{'}|R_{1}
			      \stackrel{\epsilon}{\longmapsto} P_{1}^{'}|R_{2}
		  \cdots 
			      \stackrel{\epsilon}{\longmapsto} P_{1}^{'}|R_{h}
			      \stackrel{\tau}{\longmapsto} P_{1}^{'}|Q_{1}^{'}
	      }{
		P_{1}|Q_{1} \xrightarrow{\tau \cdot \gamma_{2}\cdot \ldots \cdot \gamma_{k+1}\cdot \epsilon \cdot \ldots \epsilon \cdot \tau}_{s} P_{1}^{'}|Q_{1}^{'}
	      }$	  
	    \end{center}
	    we derive the first transaction of the premises with rule $Com3$, whether for the other transactions we use the rule $Par1$ and when necessary $Cong1$. Since $\gamma_{1} \cdot \ldots \cdot \gamma_{k+1} = xy \cdot \sigma$ and $\gamma_{1}=xy$ then $\tau \cdot \gamma_{2}\cdot \ldots \cdot \gamma_{k+1}\cdot \epsilon \cdot \ldots \epsilon \cdot \tau=\sigma$
	  \item[$\gamma_{i}=xy$ $and$ $\delta_{1}=\overline{x}y$]:
	    in this case: $\gamma_{1}=\gamma_{k+1}=\tau$, $\gamma_{j}=\epsilon$ if $1<j<i$ and $\gamma_{i+1}\cdot \ldots \cdot \gamma_{k+1}=\sigma$; $\delta_{h+1}=\tau$ and the other $\delta$s are $\epsilon$. A proof of the conclusion is:
	    \begin{center}
	      $\inferrule* [left=\bf{Low}]{
		  P_{1}|Q_{1} \stackrel{\tau}{\longmapsto} L_{1}|Q_{1} 
			      \stackrel{\epsilon}{\longmapsto} L_{2}|Q_{1}
		  \cdots
			      \stackrel{\epsilon}{\longmapsto} L_{i-1}|Q_{1} 
			      \stackrel{\tau}{\longmapsto} L_{i}|R_{1}
			      \stackrel{\gamma_{i+1}}{\longmapsto} L_{i+1}|R_{1}
		  \\\cdots 
			      \stackrel{\gamma_{k}}{\longmapsto} L_{k}|R_{1}
			      \stackrel{\gamma_{k+1}}{\longmapsto} P_{1}^{'}|R_{1}
			      \stackrel{\epsilon}{\longmapsto} P_{1}^{'}|R_{2}
		  \cdots 
			      \stackrel{\epsilon}{\longmapsto} P_{1}^{'}|R_{h}
			      \stackrel{\tau}{\longmapsto} P_{1}^{'}|Q_{1}^{'}
	      }{
		P_{1}|Q_{1} \xrightarrow{\tau \cdot \epsilon \cdot \ldots \cdot \epsilon \cdot \tau \cdot \gamma_{i+1}\cdot \ldots \cdot \gamma_{k+1} \cdot \epsilon \cdot \ldots \cdot \epsilon \cdot \tau}_{s} P_{1}^{'}|Q_{1}^{'}
	      }$	  
	    \end{center}
	    we derive the transaction $ L_{i-1}|Q_{1} \stackrel{\tau}{\longmapsto} L_{i}|R_{1}$ with rule $Com3$ and $Cong1$, whether for the other transactions of the premises we use the rule $Par1$ and when necessary $Cong1$.
	\end{description}
      \item[$Opn$]: 
	the last part of the derivation tree looks like this:
	\begin{center}
	  $\inferrule* [left=\bf{Opn}]{
	      P_{1} \xrightarrow{\overline{x}z}\; Q
	    \\ 
	      z\neq x
	  }{
	      (\nu z)P \xrightarrow{\overline{x}(z)}\; Q
	  }$	 
	\end{center}
	for the inductive hypothesis $P_{1} \xrightarrow{\sigma}_{s} Q$ which implies that there exist $L_{1}$, $\cdots$, $L_{k}$ and $\gamma_{1}$, $\cdots$, $\gamma_{k+1}$ with $k\geq 0$ such that 
	\begin{center}
	  $P_{1} \stackrel{\gamma_{1}}{\longmapsto} L_{1}  \stackrel{\gamma_{2}}{\longmapsto} L_{2} \cdots L_{k-1} \stackrel{\gamma_{k}}{\longmapsto} L_{k} \stackrel{\gamma_{k+1}}{\longmapsto} Q$ 
	\end{center}
	and 
	\begin{center}
	  $\gamma_{1} \cdot \ldots \cdot \gamma_{k+1} =  \overline{x}z$
	\end{center}
	We can have three different cases now: 
	\begin{description}
	  \item[$\gamma_{1}=\overline{x}z$]:
	    in this case $\gamma_{h+1}=\tau$ and the other $\gamma$s are $\epsilon$. A proof of the conclusion is:
	    \begin{center}
	      $\inferrule* [left=\bf{Low}]{
		  \inferrule* [left=\bf{Open}]{
		      P_{1} \stackrel{\overline{x}(z)}{\longmapsto} L_{1}
		  }{
		    (\nu z)P_{1} \stackrel{\overline{x}(z)}{\longmapsto} L_{1}
		  }
		  \\
			      L_{1} \stackrel{\epsilon}{\longmapsto} L_{2}
		  \cdots
			      \stackrel{\epsilon}{\longmapsto} L_{k}
			      \stackrel{\epsilon}{\longmapsto} Q
	      }{
		(\nu z)P_{1} \xrightarrow{\overline{x}(z) \cdot \epsilon \cdot \ldots \cdot \epsilon}_{s} Q
	      }$	  
	    \end{center}
	  \item[$\gamma_{i}=\overline{x}z$]:
	    in this case $\gamma_{1}=\gamma_{k+1}=\tau$ and the other $\gamma$s are $\epsilon$. A proof of the conclusion is:
	    \begin{center}
	      $\inferrule* [left=\bf{Low}]{
		  (\nu z)P_{1}\stackrel{\tau}{\longmapsto} (\nu z)L_{1}
			      \stackrel{\epsilon}{\longmapsto} (\nu z)L_{2}
		  \cdots
			      \stackrel{\epsilon}{\longmapsto} (\nu z) L_{i-1}
		\\
		(\nu z) L_{i-1} \stackrel{\overline{x}(z)}{\longmapsto} L_{i}
		\\
			      L_{i} \stackrel{\epsilon}{\longmapsto} L_{i+1}
		  \cdots 
			      \stackrel{\epsilon}{\longmapsto} L_{k}
			      \stackrel{\tau}{\longmapsto} Q
	      }{
		(\nu z)P_{1} \xrightarrow{\tau \cdot \epsilon \cdot \ldots \cdot \epsilon \cdot \tau \cdot \gamma_{i+1}\cdot \ldots \cdot \gamma_{k+1} \cdot \epsilon \cdot \ldots \cdot \epsilon \cdot \tau}_{s} Q
	      }$	  
	    \end{center}
	    we derive the transition $ (\nu z) L_{i-1} \stackrel{\overline{x}(z)}{\longmapsto} L_{i}$ with rule $Open$ and the transitions $(\nu z)P_{1}\stackrel{\tau}{\longmapsto}(\nu z)L_{1} \stackrel{\epsilon}{\longmapsto} (\nu z)L_{2}\cdots \stackrel{\epsilon}{\longmapsto} (\nu z) L_{i-1}$ with rule $Res$.
	\end{description}
      \item[$OpnSeq$]: 
	the last part of the derivation tree looks like this:
	\begin{center}
	  $\inferrule* [left=\bf{OpnSeq}]{
	      P_{1} \xrightarrow{inpSeq \cdot \overline{x}z} Q
	    \\ 
	      z\neq x
	  }{
	      (\nu z)P_{1} \xrightarrow{inpSeq \cdot \overline{x}(z)} Q
	  }$	 
	\end{center}
	for the inductive hypothesis $P_{1} \xrightarrow{\sigma}_{s} Q$ which implies that there exist $L_{1}$, $\cdots$, $L_{k}$ and $\gamma_{1}$, $\cdots$, $\gamma_{k+1}$ with $k\geq 0$ such that 
	\begin{center}
	  $P_{1} \stackrel{\gamma_{1}}{\longmapsto} L_{1}  \stackrel{\gamma_{2}}{\longmapsto} L_{2} \cdots L_{k-1} \stackrel{\gamma_{k}}{\longmapsto} L_{k} \stackrel{\gamma_{k+1}}{\longmapsto} Q$ 
	\end{center}
	and 
	\begin{center}
	  $\gamma_{1} \cdot \ldots \cdot \gamma_{k+1} =  inpSeq \cdot \overline{x}z$
	\end{center}
	We can have three different cases now: 
	\begin{description}
	  \item[$\gamma_{1}=\overline{x}z$]: 
	    A proof of the conclusion is:
	    \begin{center}
	      $\inferrule* [left=\bf{Low}]{
		  \inferrule* [left=\bf{Opn}]{
		    P_{1}\stackrel{\overline{x}z}{\longmapsto} L_{1}
		  }{
		    (\nu z)P_{1}\stackrel{\overline{x}(z)}{\longmapsto} L_{1}
		  }
		  \\
			      L_{1} \stackrel{\gamma_{2}}{\longmapsto} L_{2}
		  \cdots
			      \stackrel{\gamma_{k}}{\longmapsto} L_{k}
			      \stackrel{\gamma_{k+1}}{\longmapsto} Q
	      }{
		(\nu z)P_{1} \xrightarrow{\overline{x}(z) \cdot \gamma_{2}\cdot \ldots \cdot \gamma_{k+1}}_{s} Q
	      }$	  
	    \end{center}
	  \item[$\gamma_{i}=\overline{x}z$]:
	    with $1<i\leq k$ in this case $\gamma_{i+1}= \cdot = \gamma_{k}=\epsilon$, $\gamma_{k+1}=\tau$ and $\gamma_{1}\cdot \ldots \cdot \gamma_{i-1}=inpSeq$. A proof of the conclusion is:
	    \begin{center}
	      $\inferrule* [left=\bf{Low}]{
		  (\nu z)P_{1}\stackrel{\gamma_{1}}{\longmapsto} (\nu z)L_{1}
			      \stackrel{\gamma_{2}}{\longmapsto} (\nu z)L_{2}
		  \cdots
			      \stackrel{\gamma_{i-1}}{\longmapsto} (\nu z) L_{i-1}
			      \stackrel{\overline{x}(z)}{\longmapsto} L_{i}
			      \stackrel{\epsilon}{\longmapsto} L_{i+1}
		  \cdots 
			      \stackrel{\epsilon}{\longmapsto} L_{k}
			      \stackrel{\tau}{\longmapsto} Q
	      }{
		(\nu z)P_{1} \xrightarrow{\gamma_{1}\cdot \ldots \cdot \gamma_{i-1} \cdot \overline{x}(z) \cdot \epsilon \cdot \ldots \cdot \epsilon \cdot \cdot \tau}_{s} Q
	      }$	  
	    \end{center}
	    we derive the transition $ (\nu z) L_{i-1} \stackrel{\overline{x}(z)}{\longmapsto} L_{i}$ with rule $Open$ and the transitions $(\nu z)P_{1}\stackrel{\gamma_{1}}{\longmapsto}(\nu z)L_{1} \stackrel{\gamma_{2}}{\longmapsto} (\nu z)L_{2}\cdots \stackrel{\gamma_{i-1}}{\longmapsto} (\nu z) L_{i-1}$ with rule $Res$.
	  \item[$\gamma_{k+1}=\overline{x}z$]:
	    in this case $\gamma_{1}\cdot \ldots \cdot \gamma_{k}=inpSeq$. A proof of the conclusion is:
	    \begin{center}
	      $\inferrule* [left=\bf{Low}]{
		  (\nu z)P_{1}\stackrel{\gamma_{1}}{\longmapsto} (\nu z)L_{1}
			      \stackrel{\gamma_{2}}{\longmapsto} (\nu z)L_{2}
		  \cdots
			      \stackrel{\gamma_{k}}{\longmapsto} (\nu z)L_{k}
			      \stackrel{\overline{x}(z)}{\longmapsto} Q
	      }{
		(\nu z)P_{1} \xrightarrow{\gamma_{1}\cdot \ldots \cdot \gamma_{k} \cdot \overline{x}(z)}_{s} Q
	      }$	  
	    \end{center}
	    we derive the transition $ (\nu z) L_{k} \stackrel{\overline{x}(z)}{\longmapsto} L_{i}$ with rule $Open$ and the others with rule $Res$.
	\end{description}

    \end{description}
    
    
  \end{proof}
\end{proposition}


\begin{proposition}
  Let $\rightarrow$ be the relation defined in table \ref{multipisoloinputearlywith} and let $P$ be a multi $\pi$ process. Then 
  \begin{center}
    $P\xrightarrow{\sigma}_{s} Q  \Rightarrow P\xrightarrow{\sigma} Q$
  \end{center}
  \begin{proof}
    DA MODIFICARE PERCHE' LE REGOLE COM2 E COM3 SONO CAMBIATE     
    The proof is by induction on the sum of the length of the sequence of transitions used to derive $P\xrightarrow{\sigma}_{s} Q$ and on the depth of the proof tree of the first transition in such a sequence:
    \begin{description}
      \item[base case]
    \end{description}
	in this case $P\stackrel{\sigma}{\longmapsto} Q$ and the derivation of this transition has depth one. The last(and only) rule used to derive $P\stackrel{\sigma}{\longmapsto} Q$ can be: $Out$, $EInp$ or $Tau$; these rules are also in table \ref{multipisoloinputearlywith} so we can derive $P\xrightarrow{\sigma} Q$.
    \begin{description}
      \item[inductive case]
    \end{description}
	we have two cases: the first is $P\stackrel{\sigma}{\longmapsto} Q$ and the depth of the derivation is greater than one. In this case the last rule in the derivation can be: $Sum$, $Com1$, $Res$, $Opn$, $Par1$, $Cong1$:
	\begin{description}
	  \item[$Sum$]:
		\begin{center}
		  $\inferrule* [left=\bf{Sum}]{
		      P_{1} \stackrel{\gamma_{1}}{\longmapsto} Q
		    }{
		      P_{1}+P_{2} \stackrel{\gamma_{1}}{\longmapsto} Q
		  }$ 
		\end{center}
	    for rule $Low$ $P_{1} \xrightarrow{\gamma_{1}}_{s} Q$, for inductive hypothesis $P_{1} \xrightarrow{\gamma_{1}} Q$ and finally for rule $Sum$ $P_{1}+P_{2} \xrightarrow{\gamma_{1}} Q$.
	  \item[$others$] 
	    the other cases are similar to the previous since these rules are in common. 
	\end{description}
	The second case is that there exist $L_{1}, \cdots, L_{k}$ and $\gamma_{1}, \cdots, \gamma_{k+1}$ with $k\geq 1$ such that 
	\begin{center}
	  $P \stackrel{\gamma_{1}}{\longmapsto} L_{1}  \stackrel{\gamma_{2}}{\longmapsto} L_{2} \cdots L_{k-1} \stackrel{\gamma_{k}}{\longmapsto} L_{k} \stackrel{\gamma_{k+1}}{\longmapsto} Q$ 
	\end{center}
	and 
	\begin{center}
	  $\gamma_{1} \cdot \ldots \cdot \gamma_{k+1} = \sigma$ and $k\geq 1$
	\end{center} 
	we proceed by induction on the depth of the derivation of $P \stackrel{\gamma_{1}}{\longmapsto} L_{1}$. Let us call $R$ the last instance of a rule used to derive $P\stackrel{\gamma_{1}}{\longmapsto} L_{1}$. $R$ cannot be $Out$, $EInp$, $Tau$, $Com1$, $Cong1$ because $L_{1}$ is a marked process but these rules lead to a non marked process. Also $R$ cannot be $Star$, $Com2$, $Com4$, $Cong3$ because $P$ is not marked whether the head of the conclusion of these rules is marked.
	\begin{description}
	  \item[base case] 
	\end{description}
	    $R$ can be only $SInpLow$, so 
	    \begin{center}
	      $\underline{x(z)}.P_{1} \stackrel{xy}{\longmapsto} *P_{1}\{y/z\} \stackrel{\gamma_{2}}{\longmapsto} L_{2} \cdots L_{k-1} \stackrel{\gamma_{k}}{\longmapsto} L_{k} \stackrel{\gamma_{k+1}}{\longmapsto} Q$ 
	    \end{center}
	    Since $*P_{1}\{y/z\}$ has a mark on top, the last rule used in a derivation of $ *P_{1}\{y/z\} \stackrel{\gamma_{2}}{\longmapsto} L_{2}$ has to be $Star$. This means that $P_{1}\{y/z\} \stackrel{\gamma_{2}}{\longmapsto} L_{2}$ and so
	    \begin{center}
	      $P_{1}\{y/z\} \stackrel{\gamma_{2}}{\longmapsto} L_{2} \cdots L_{k-1} \stackrel{\gamma_{k}}{\longmapsto} L_{k} \stackrel{\gamma_{k+1}}{\longmapsto} Q$ 
	    \end{center} 
	    which for rule $Low$ and inductive hypothesis implies $P_{1}\{y/z\} \xrightarrow{\gamma_{2}\cdot \ldots \cdot \gamma_{k+1}} Q$. A proof of the conclusion is:
	    \begin{center}
	      $\inferrule* [left=\bf{SInp}]{
		  P_{1}\{y/z\} \xrightarrow{\gamma_{2}\cdot \ldots \cdot \gamma_{k+1}} Q
		\\
		  \gamma_{2}\cdot \ldots \cdot \gamma_{k+1} \neq \tau
	      }{
		\underline{x(z)}.P_{1} \xrightarrow{xy \cdot \gamma_{2}\cdot \ldots \cdot \gamma_{k+1}} Q
	      }$
	    \end{center}
	    or
	    \begin{center}
	      $\inferrule* [left=\bf{SInpTau}]{
		  P_{1}\{y/z\} \xrightarrow{\gamma_{2}\cdot \ldots \cdot \gamma_{k+1}} Q
		\\
		  \gamma_{2}\cdot \ldots \cdot \gamma_{k+1} = \tau
	      }{
		\underline{x(z)}.P_{1} \xrightarrow{xy} Q
	      }$
	    \end{center}
	\begin{description}
	  \item[inductive case] 
	\end{description}
	    $R$ can be:
	    \begin{description}
	      \item[$Sum$] the first transition is:
		\begin{center}
		  $\inferrule* [left=\bf{Sum}]{
		      P_{1} \stackrel{\gamma_{1}}{\longmapsto} L_{1}
		    }{
		      P_{1}+P_{2} \stackrel{\gamma_{1}}{\longmapsto} L_{1}
		  }$ 
		\end{center}
		so we can build the following chain of transition:
		\begin{center}
		  $P_{1} \stackrel{\gamma_{1}}{\longmapsto} L_{1} \stackrel{\gamma_{2}}{\longmapsto} L_{2} \cdots L_{k-1} \stackrel{\gamma_{k}}{\longmapsto} L_{k} \stackrel{\gamma_{k+1}}{\longmapsto} Q$ 
		\end{center}
		apply the rule $Low$ and the inductive hypothesis to get $P_{1} \xrightarrow{\gamma_{1}\cdot \ldots \cdot \gamma_{k+1}} Q$. Now a proof of the conclusion can be
		\begin{center}
		  $\inferrule* [left=\bf{Sum}]{
		      P_{1} \xrightarrow{\gamma_{1}\cdot \ldots \cdot \gamma_{k+1}} Q
		    }{
		      P_{1}+P_{2} \xrightarrow{\gamma_{1}\cdot \ldots \cdot \gamma_{k+1}} Q
		  }$
		\end{center}
	      \item[$Res$] the first transition is:
		\begin{center}
		  $\inferrule* [left=\bf{Res}]{
			P_{1} \stackrel{\gamma_{1}}{\longmapsto} L_{1}^{'}
		      \\
			z\notin n(\gamma_{1})
		    }{
		      (\nu z) P_{1} \stackrel{\gamma_{1}}{\longmapsto} (\nu z)L_{1}^{'}
		  }$ 
		\end{center}
		given that $L_{1}$ has a restriction at the top level, all the other intermediate processes $L_{2}, \cdots, L_{k}$ have the same restriction at the top level. So the last rule used to prove $L_{i}\stackrel{\gamma_{i+1}}{\longmapsto} L_{i+1}$ for each $1\leq i \leq k-1$ is $Res$. The last rule used to derive $L_{k} \stackrel{\gamma_{k+1}}{\longmapsto} Q$ can be one of: 
		\begin{description}
		  \item[$Res$]:
		    \begin{center}
		      $\inferrule* [left=\bf{Res}]{
			  L_{k}^{'} \stackrel{\gamma_{k+1}}{\longmapsto} Q^{'}
			\\
			  z\notin n(Q^{'})
		      }{
			(\nu z)L_{k}^{'} \stackrel{\gamma_{k+1}}{\longmapsto} (\nu z)Q^{'}
		      }$
		    \end{center}
		    we can build the following chain of transitions:
		    \begin{center}
		      $P_{1} \stackrel{\gamma_{1}}{\longmapsto} L_{1}^{'} \stackrel{\gamma_{2}}{\longmapsto} L_{2}^{'} \cdots L_{k-1}^{'} \stackrel{\gamma_{k}}{\longmapsto} L_{k}^{'}\stackrel{\gamma_{k+1}}{\longmapsto} Q^{'}$ 
		    \end{center}
		    then apply the rule $Low$ and the inductive hypothesis to get $P_{1} \xrightarrow{\gamma_{1}\cdot \ldots \cdot \gamma_{k+1}} Q^{'}$. A proof of the conclusion can be
		    \begin{center}
		      $\inferrule* [left=\bf{Res}]{
			  P_{1} \xrightarrow{\gamma_{1}\cdot \ldots \cdot \gamma_{k+1}} Q^{'}
			\\
			  z\notin n(\gamma_{1}\cdot \ldots \cdot \gamma_{k+1})
		      }{
			(\nu z)P_{1} \xrightarrow{\gamma_{1}\cdot \ldots \cdot \gamma_{k+1}} (\nu z)Q^{'}
		    }$
		    \end{center}
		  \item[$Cong3$]: 
		    \begin{center}
		      $\inferrule* [left=\bf{Cong3}]{
			  (\nu z)L_{k}^{'} \stackrel{\gamma_{k+1}}{\longmapsto} Q^{'}
			\\
			  Q^{'} \equiv Q
		      }{
			(\nu z)L_{k}^{'} \stackrel{\gamma_{k+1}}{\longmapsto} Q
		      }$
		    \end{center}
		    for transitivity of $\equiv$ we can assume that there is a derivation of $(\nu z)L_{k}^{'} \stackrel{\gamma_{k+1}}{\longmapsto} Q^{'}$ that does not end with an instance of $Cong3$ and so it does end with $Res$ or $Opn$:
		    \begin{description}
		      \item[$Res$]:
			\begin{center}
			  $\inferrule* [left=\bf{Cong3}]{
			      \inferrule* [left=\bf{Res}]{
				L_{k}^{'} \stackrel{\gamma_{k+1}}{\longmapsto} Q^{''}
			      }{
				(\nu z)L_{k}^{'} \stackrel{\gamma_{k+1}}{\longmapsto} (\nu z)Q^{''}
			      }
			    \\
			      (\nu z)Q^{''} \equiv Q
			  }{
			    (\nu z)L_{k}^{'} \stackrel{\gamma_{k+1}}{\longmapsto} Q
			  }$
			\end{center}
			we can derive the following chain of transition:
			\begin{center}
			  $P_{1} \stackrel{\gamma_{1}}{\longmapsto} L_{1}^{'} \stackrel{\gamma_{2}}{\longmapsto} L_{2}^{'} \cdots L_{k-1}^{'} \stackrel{\gamma_{k}}{\longmapsto} L_{k}^{'}\stackrel{\gamma_{k+1}}{\longmapsto} Q^{''}$ 
			\end{center}
			and then apply the rule $Low$ and the inductive hypothesis to get 
			\begin{center}
			  $P_{1} \xrightarrow{\gamma_{1}\cdot \cdots \cdot \gamma_{k+1}} Q^{''}$ 
			\end{center}
			A proof of the conclusion is
			\begin{center}
			  $\inferrule* [left=\bf{Cong}]{
			      \inferrule* [left=\bf{Res}]{
				P_{1} \xrightarrow{\gamma_{1}\cdot \cdots \cdot \gamma_{k+1}} Q^{''}
			      }{
				(\nu z)P_{1} \xrightarrow{\gamma_{1}\cdot \ldots \cdot \gamma_{k+1}} (\nu z)Q^{''}
			      }
			    \\
			      (\nu z)Q^{''} \equiv Q
			  }{
			    (\nu z)P_{1} \xrightarrow{\gamma_{1}\cdot \ldots \cdot \gamma_{k+1}} Q
			  }$
			\end{center}
		      \item[$Opn$]:
			\begin{center}
			  $\inferrule* [left=\bf{Cong3}]{
			      \inferrule* [left=\bf{Opn}]{
				L_{k}^{'} \stackrel{\overline{x}z}{\longmapsto} Q^{'}
			      }{
				(\nu z)L_{k}^{'} \stackrel{\overline{x}(z)}{\longmapsto} Q^{'}
			      }
			    \\
			      Q^{'} \equiv Q
			  }{
			    (\nu z)L_{k}^{'} \stackrel{\overline{x}(z)}{\longmapsto} Q
			  }$
			\end{center}
			we can derive the following chain of transition:
			\begin{center}
			  $P_{1} \stackrel{\gamma_{1}}{\longmapsto} L_{1}^{'} \stackrel{\gamma_{2}}{\longmapsto} L_{2}^{'} \cdots L_{k-1}^{'} \stackrel{\gamma_{k}}{\longmapsto} L_{k}^{'}\stackrel{\overline{x}z}{\longmapsto} Q^{'}$ 
			\end{center}
			and then apply the rule $Low$ and the inductive hypothesis to get 
			\begin{center}
			  $P_{1} \xrightarrow{\gamma_{1}\cdot \cdots \cdot \gamma_{k}\cdot \overline{x}z} Q^{'}$ 
			\end{center}
			A proof of the conclusion is
			\begin{center}
			  $\inferrule* [left=\bf{Cong}]{
			      \inferrule* [left=\bf{OpnSeq}]{
				P_{1} \xrightarrow{\gamma_{1}\cdot \cdots \cdot \gamma_{k}\cdot \overline{x}z} Q^{'}
			      }{
				(\nu z)P_{1} \xrightarrow{\gamma_{1}\cdot \ldots \cdot \gamma_{k}\cdot \overline{x}(z)} Q^{'}
			      }
			    \\
			      Q^{'} \equiv Q
			  }{
			    (\nu z)P_{1} \xrightarrow{\gamma_{1}\cdot \ldots \cdot \gamma_{k}\cdot \overline{x}(z)} Q
			  }$
			\end{center}			
		    \end{description}
		  \item[$Opn$]:
		    \begin{center}
		      $\inferrule* [left=\bf{Opn}]{
			  L_{k}^{'} \stackrel{\overline{x}z}{\longmapsto} Q
			\\
			  z\notin n(Q)
		      }{
			(\nu z)L_{k}^{'} \stackrel{\overline{x}(z)}{\longmapsto} Q
		      }$
		    \end{center}
		    we can build the following chain of transitions:
		    \begin{center}
		      $P_{1} \stackrel{\gamma_{1}}{\longmapsto} L_{1}^{'} \stackrel{\gamma_{2}}{\longmapsto} L_{2}^{'} \cdots L_{k-1}^{'} \stackrel{\gamma_{k}}{\longmapsto} L_{k}^{'}\stackrel{\overline{x}z}{\longmapsto} Q$ 
		    \end{center}
		    then apply the rule $Low$ and the inductive hypothesis to get $P_{1} \xrightarrow{\gamma_{1}\cdot \ldots \cdot \gamma_{k} \cdot \overline{x}z} Q$. A proof of the conclusion can be
		    \begin{center}
		      $\inferrule* [left=\bf{OpnSeq}]{
			  P_{1} \xrightarrow{\gamma_{1}\cdot \ldots \cdot \gamma_{k} \cdot \overline{x}z} Q
			\\
			  z\notin n(\gamma_{1}\cdot \ldots \cdot \gamma_{k})
		      }{
			(\nu z)P_{1} \xrightarrow{\gamma_{1}\cdot \ldots \cdot \gamma_{k} \cdot \overline{x}(z)} Q
		      }$
		    \end{center}
		\end{description}
	      \item[$Opn$] the first transition is:
		\begin{center}
		  $\inferrule* [left=\bf{Opn}]{
			P_{1} \stackrel{\overline{x}z}{\longmapsto} L_{1}
		      \\
			z\notin n(\overline{x}(z))
		    }{
		      (\nu z) P_{1} \stackrel{\overline{x}(z)}{\longmapsto} L_{1}
		  }$ 
		\end{center}		
		since $\gamma_{1}$ is an output for lemma \ref{lemmastrongsequence} $\gamma_{2}\cdot \ldots \cdot \gamma_{k+1}=\tau$ and $\sigma=\overline{x}(z)$. We derive the following chain of transition:
		\begin{center}
		  $P_{1} \stackrel{\overline{x}z}{\longmapsto} L_{1} \stackrel{\gamma_{2}}{\longmapsto} L_{2} \cdots L_{k-1} \stackrel{\gamma_{k}}{\longmapsto} L_{k}\stackrel{\gamma_{k+1}}{\longmapsto} Q$ 
		\end{center}
		for rule $Low$ and inductive hypothesis
		\begin{center}
		  $P_{1} \xrightarrow{\overline{x}z} Q$ 
		\end{center}
		A proof of the conclusion is
		\begin{center}
		  $\inferrule* [left=\bf{Opn}]{
		    P_{1} \xrightarrow{\overline{x}z} Q
		  }{
		    (\nu z)P_{1} \xrightarrow{\overline{x}(z)} Q
		  }$ 
		\end{center}
	      \item[$Par1$]: 
		the first transition is:
		\begin{center}
		  $\inferrule* [left=\bf{Par1}]{
			P_{1} \stackrel{\gamma_{1}}{\longmapsto} L_{1}^{'}
		      \\
			bn(\gamma_{1})\cap fn(P_{2})=\emptyset
		    }{
		      P_{1}|P_{2} \stackrel{\gamma_{1}}{\longmapsto} L_{1}^{'}|P_{2}
		  }$
		\end{center}
		the transitions 
		\begin{center}
		  $P \stackrel{\gamma_{1}}{\longmapsto} L_{1} \stackrel{\gamma_{2}}{\longmapsto} \cdots \stackrel{\gamma_{k}}{\longmapsto} L_{k}$ 
		\end{center}
		are such that the last rule used in their derivation is $Par1$. Whether the derivation of the last transition can end with $Par1$ or $Cong3$:
		\begin{description}
		  \item[$Par1$]
		    We derive the following chain of transition:
		    \begin{center}
		      $P_{1} \stackrel{\gamma_{1}}{\longmapsto} L_{1}^{'} \stackrel{\gamma_{2}}{\longmapsto} L_{2}^{'} \cdots L_{k-1}^{'} \stackrel{\gamma_{k}}{\longmapsto} L_{k}^{'}\stackrel{\gamma_{k+1}}{\longmapsto} Q_{1}$ 
		    \end{center}
		    for rule $Low$ and inductive hypothesis
		    \begin{center}
		      $P_{1} \xrightarrow{\sigma} Q_{1}$ 
		    \end{center}
		    A proof of the conclusion is
		    \begin{center}
		      $\inferrule* [left=\bf{Par}]{
			  P_{1} \xrightarrow{\sigma} Q_{1}
			\\
			  bn(\sigma)\cap fn(P_{2})=\emptyset
		      }{
			P_{1}|P_{2} \xrightarrow{\sigma} Q_{1}|P_{2}
		      }$ 
		    \end{center}
		  \item[$Cong3$]
		    the last part of the derivation of the last transition is
		    \begin{center}
		      $\inferrule* [left=\bf{Cong3}]{
			  L_{k}^{'}|P_{2} \stackrel{\gamma_{k+1}}{\longmapsto} Q^{'}
			\\
			  Q^{'} \equiv Q
		      }{
			L_{k}^{'}|P_{2} \stackrel{\gamma_{k+1}}{\longmapsto} Q
		      }$ 		      
		    \end{center}
		    because of the transitivity of $\equiv$ we can assume that there is a derivation of $L_{k}^{'}|P_{2} \stackrel{\gamma_{k+1}}{\longmapsto} Q^{'}$ which does not use $Cong3$ as its last instance. So this last instance can only be $Par1$ so $Q^{'}=Q^{''}|P_{2}$. We derive the following chain of transition:
		    \begin{center}
		      $P_{1} \stackrel{\gamma_{1}}{\longmapsto} L_{1}^{'} \stackrel{\gamma_{2}}{\longmapsto} L_{2}^{'} \cdots L_{k-1}^{'} \stackrel{\gamma_{k}}{\longmapsto} L_{k}^{'}\stackrel{\gamma_{k+1}}{\longmapsto} Q^{''}$ 
		    \end{center}
		    for rule $Low$ and inductive hypothesis
		    \begin{center}
		      $P_{1} \xrightarrow{\sigma} Q^{''}$ 
		    \end{center}
		    A proof of the conclusion is
		    \begin{center}
		      $\inferrule*[left=\bf{Cong3}]{
			  \inferrule* [left=\bf{Par}]{
			      P_{1} \xrightarrow{\sigma} Q^{''}
			    \\
			      bn(\sigma)\cap fn(P_{2})=\emptyset
			  }{
			    P_{1}|P_{2} \xrightarrow{\sigma} Q^{''}|P_{2}
			  }
			\\
			  Q^{''}|P_{2} \equiv Q
		      }{
			P_{1}|P_{2} \xrightarrow{\sigma} Q
		      }$ 
		    \end{center}
		\end{description}
	      \item[$Cong2$]: 
		the last rule of the derivation of the first transition is:
		\begin{center}
		  $\inferrule* [left=\bf{Cong2}]{
		      P^{'} \stackrel{\gamma_{1}}{\longmapsto} L_{1}
		    \\
		      P^{'} \equiv P
		  }{
		    P \stackrel{\gamma_{1}}{\longmapsto} L_{1}
		  }$ 		      
		\end{center}
		We derive the following chain of transition:
		\begin{center}
		  $P^{'} \stackrel{\gamma_{1}}{\longmapsto} L_{1} \stackrel{\gamma_{2}}{\longmapsto} L_{2} \cdots L_{k-1} \stackrel{\gamma_{k}}{\longmapsto} L_{k}\stackrel{\gamma_{k+1}}{\longmapsto} Q$
		\end{center}
		for rule $Low$ and inductive hypothesis
		\begin{center}
		  $P^{'} \xrightarrow{\sigma} Q$ 
		\end{center}
		A proof of the conclusion is
		\begin{center}
		  $\inferrule* [left=\bf{Cong2}]{
			P^{'} \xrightarrow{\sigma} Q
		    \\
		      P^{'} \equiv P
		  }{
		    P \xrightarrow{\sigma} Q
		  }$ 
		\end{center}
	      \item[$Com3$]: 
		the last part of the derivation of the first transition, having in mind lemma \ref{lemmacom3}, is:
		\begin{center}
		  $\inferrule* [left=\bf{Com3}]{
		      P_{1} \stackrel{xy}{\longmapsto} L_{1}^{'}
		    \\
		      P_{2} \stackrel{\overline{x}y}{\longmapsto} P_{2}^{'}
		  }{
		    P_{1}|P_{2} \stackrel{\epsilon}{\longmapsto} L_{1}^{'}|P_{2}^{'}
		  }$ 		      
		\end{center}
		the derivations of the transitions $L_{1} \stackrel{\gamma_{2}}{\longmapsto} L_{2} \cdots L_{k-1} \stackrel{\gamma_{k}}{\longmapsto} L_{k}$ can end only with an instance of $Par1$ so 
		\begin{center}
		  $\inferrule* [left=\bf{Par1}]{
		      L_{i}^{'} \stackrel{\gamma_{i+1}}{\longmapsto} L_{i+1}^{'}
		    \\
		      bn(\gamma_{i+1})\cap fn(P_{2}^{'})=\emptyset
		  }{
		    L_{i}^{'}|P_{2}^{'} \stackrel{\gamma_{i+1}}{\longmapsto} L_{i+1}^{'}|P_{2}^{'}
		  }$ 		      		  
		\end{center}
		The derivation of the last transition $L_{k}\stackrel{\gamma_{k+1}}{\longmapsto} Q$ can end with $Par1$ or with $Cong3$:
		\begin{description}
		  \item[$Par1$]:
		    We derive the following chain of transition:
		    \begin{center}
		      $P_{1} \stackrel{xy}{\longmapsto} L_{1}^{'} \stackrel{\gamma_{2}}{\longmapsto} L_{2}^{'} \cdots L_{k-1}^{'} \stackrel{\gamma_{k}}{\longmapsto} L_{k}^{'}\stackrel{\gamma_{k+1}}{\longmapsto} Q_{1}$
		    \end{center}
		    for rule $Low$ and inductive hypothesis
		    \begin{center}
		      $P_{1} \xrightarrow{\sigma} Q_{1}$ 
		    \end{center}
		    A proof of the conclusion is
		    \begin{center}
		      $\inferrule* [left=\bf{EComSeq}]{
			  P_{1} \xrightarrow{xy \cdot \gamma_{2}\cdot \ldots \cdot \gamma_{k+1}} Q_{1}
			\\
			  P_{2}\xrightarrow{\overline{x}y} P_{2}^{'}
		      }{
			P_{1}|P_{2} \xrightarrow{\gamma_{2}\cdot \ldots \cdot \gamma_{k+1}} Q_{1}|P_{2}^{'}
		      }$ 
		    \end{center}
		    and $\gamma_{2}\cdot \ldots \cdot \gamma_{k+1}=\sigma$ because the first action $\gamma_{1}$ is $\epsilon$.
		  \item[$Cong3$]: e' facile da scrivere a questo punto
		    
		\end{description}
	      \item[$Par2$]: 
		the last part of the derivation of the first transition is:
		\begin{center}
		  $\inferrule* [left=\bf{Par2}]{
		      P_{1} \stackrel{\gamma_{1}}{\longmapsto} L_{1}^{'}
		    \\
		      bn(\gamma_{1})\cap fn(P_{2})=\emptyset
		  }{
		    P_{1}|P_{2} \stackrel{\gamma_{1}}{\longmapsto} L_{1}^{'}|*P_{2}
		  }$ 		      
		\end{center}
% 		caso1: sequenza(eventualmente vuota) di par3 terminata da una com4
% 		caso2: sequenza(eventualmente vuota) di par3 seguita da com2 seguita da ?
	    \end{description}	    
  \end{proof}
\end{proposition}


da sistemare la parte sulle regole simmetriche di com? par? e sum?



