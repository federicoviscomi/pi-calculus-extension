
%guardare l'introduzione del libro per TLP
%leggere per bene il libro del martini e immergere qualcosa nella tesi

\section{syntax}

Multi ccs is introduced in \cite{gorrieriMCCS}. Let $\mathbb{L}$ be a numerable set of channel names, let $\overline{\mathbb{L}}$ be the set of the complementary of channel names $\overline{\mathbb{L}}=\{\overline{l}: l\in \mathbb{L}\}$, let $Act=\mathbb{L}\cup\overline{\mathbb{L}}\cup\{\tau\}$ be the set of all action, where $\tau \notin \mathbb{L}\cup \overline{\mathbb{L}}$. The process terms are generated by the following grammar:
\begin{center}
  $p$ ::= $0$ | $\mu.p$ | $\underline{\mu}.p$ | $p+p$ | $p|p$ | $(\nu a)p$ | $C$
\end{center}
and we have also expression in the form
\begin{center}
  $C\; =\; p$ 
\end{center}
in which the constant $C$ is defined as the process $p$.
The meaning of each production is the following:
\begin{description}
  \item[$0$] is terminated process
  \item[$\mu.p$] is action prefixing, $\mu$ is an action
  \item[$\underline{\mu}.p$] is strong action prefixing
  \item[$p+p$] is non deterministic sum of two processes
  \item[$p|p$] is parallel composition of two processes
  \item[$(\nu a)p$] is restriction: the name $a$ is private in p. $\nu a$ is a binder for the name $a$ in the process $p$
  \item[$C$] is a constant
\end{description}

\begin{definition}
\label{guarded process}
A \emph{guarded process} is 
\end{definition}


We denote with $\mathbb{P}$ the set of all process that are guarded and closed with respect to constant.



\section{semantic}

\begin{definition}
\label{labeled transition system} 
A \emph{labeled transition system} is a labeled graph, that is a triple 
\begin{center}
  $(N, L, LE)$ 
\end{center}
such that:
\begin{description}
  \item[$N$] is a set of nodes
  \item[$L$] is a set of labels
  \item[$LE$] is a set of labeled edges(also called transition relation): $LE\subseteq N\times L\times N$
\end{description}
\end{definition}
\begin{definition}
The \emph{semantic of a multi-CCS process} is a labeled transition system 
\begin{center}
  $(\mathbb{P}, \mathbb{A}, \rightarrow)$
\end{center}
such that
\begin{description}
  \item[$\mathbb{P}$] is the set of all multi ccs processes
  \item[$\mathbb{A}$] is the set $(\mathbb{L}\cup \overline{\mathbb{L}})^{+}\cup \{\tau\}$
  \item[$\rightarrow$] is the minimal relation contained in $\mathbb{P}\times \mathbb{A} \times \mathbb{P}$ that is generated by the rules of the operational semantic of multi-CCS
\end{description}
\end{definition}

\begin{definition}
\label{operational semantic of multi-CCS}
The \emph{operational semantic of multi-CCS} is given by the following rules:
\begin{description}
  \item[$Pref$]:
    \begin{center}
      $\mu.p {\rightarrow}^{\mu} p $
    \end{center}
  \item[$S-Pref_{1}$]:
    \[
      \frac{p\rightarrow^{\sigma}p^{'}}{\underline{\tau}.p\rightarrow^{\sigma}p^{'}}
    \]
\end{description}

\end{definition}


\begin{example}
  We show an example of three processes that synchronize. In particular we prove that
  \[
	    \inferrule* [left=Com]{
		\inferrule* [left=SPref3]{
		    \inferrule* [left=Inp]{
		    }{
		      a.0\; \xrightarrow{a}\; 0
		    }
		  \\
		    a\neq \tau
		}{
		  \underline{a}.a.0\; \xrightarrow{aa}\; 0
		}
	      \\
		\inferrule* [left=Out]{
		}{
		  \overline{a}.0\; \xrightarrow{\overline{a}}\; 0
		}
	      \\
		Sync(aa,\overline{a}, a)
	    }{
	      \underline{a}.a.0|\overline{a}.0\; \xrightarrow{a}\; 0|0
	    }   
  \]

  \[
    \inferrule* [left=Res]{
	\inferrule* [left=Com]{
	    \underline{a}.a.0|\overline{a}.0\; \xrightarrow{a}\; 0|0
	  \\
	    \inferrule* [left=Out]{
	    }{
	      \overline{a}.0\; \xrightarrow{\overline{a}}\; 0
	    }
	  \\
	    Sync(a,\overline{a},\tau)
	}{
	  ((\underline{a}.a.0|\overline{a}.0)|\overline{a}.0)\; \xrightarrow{\tau}\; ((0|0)|0)\;
	}
      \\
	a\notin n(\tau)
    }{
      (\nu a)((\underline{a}.a.0|\overline{a}.0)|\overline{a}.0)\; \xrightarrow{\tau}\; (\nu a)((0|0)|0)\;
    }
  \]

\end{example}

