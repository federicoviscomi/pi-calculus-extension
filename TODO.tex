
0- aggiungere un esempio sullo scope extension nella parte sulla congruenza strutturale
1- leggere gli articoli di honda e yashimi
 - aggiungere la parte sul multi ccs
2- definire una semantica step per multi pi
3- definire delle relazioni di bisimulazione congruenza per multi pi
4- assicurarsi che ci siano almeno 2 esempi per ogni semantica
5- aggiungere proprieta' delle semantiche e delle bisimulazioni, quali?
6- cercare di trasportare multi pi sullo spi calcolo?
7- modificare la sintassi dei nomi delle regole in modo da renderla uniforme


fare esempio di almeno due derivazioni per ogni semantica, scegliere esempi significativi che usino almeno due prefissi e una sincronizzazione oppure che mostrino come avviene lo scope extension.
O meglio quali sono gli esempi notevoli, che suggeriscono che la semantica e' giusta?

ci sono delle cose che dovrei studiare per capire meglio l'algebra dei processi?
si puo' calcolare la semantica usando un motore di inferenza ad esempio Jena?
mettere tutte le regole in tabelle con riferimento invece che tabelle ancorate al testo?
usare inferrule invece delle tabelle per le regole? SI!
controllare dove e se viene usata ogni definizione ed eventuaelmente usarla o rimuoverla
usare matita o coq o isabella per dimostrare tutto incluse le transizioni?
ha senso cercare di capire che rapporto c'e' tra multi pi calcolo e reti di petri?
nella parte di pi calcolo, aggiungere esempi che chiariscano le differenze e le similitudini tra le varie bisimulazioni



