




\begin{itemize}
  \item
    dimostrare(o negare) l'equivalenza del pi calcolo con e senza congruenza strutturale
  \item
    nel multi pi calcolo con strong prefixing solo su input o solo su output: definire una semantica di basso livello sulla falsariga di quell'articolo
  \item
    fare un quadro generale sulle equivalenze nel pi calcolo
  \item
    scegilere una equivalenza(forse la open va bene) per multi pi calcolo(quale?) che sia una congruenza per input(ma non lo sara' per il parallelo)
  \item
    trovare equivalenza che sia una congruenza(es: open step) per tutti gli operatori
  \item
    trovare la congruenza coarsest contenuta nella bisimulazione scelta in precedenza
\end{itemize}

perche' nella regola originale sull'input nel pi calcolo early c'e' quella premessa?


\begin{itemize}
  \item 	
    stampare e rileggere
  \item
    riscrivere tutte le regole usando inferrule
  \item
    aggiungere la definizione precisa induttiva di alfa conversione anche per il multipi
  \item
    aggiungere a tutti i pi e multi pi la derivazione di
    \begin{itemize}
      \item 
	$(x(z).P|Q)|\overline{x}y.R \xrightarrow{\tau} (P\{w/z\}|Q)\{y/w\}|R$ 
      \item
	$y=z$ or $y\notin fn(Q)$: $((\nu y) \overline{x}y.P)| x(z).Q \xrightarrow{\tau} (\nu y)(P|Q\{y/z\})$
      \item
	$y\neq z$ and $y\in fn(Q)$  $y^{'}$ fresh: $((\nu y) \overline{x}y.P)| x(z).Q \xrightarrow{\tau} (\nu y^{'})(P\{y^{'}/y\}|Q\{y^{'}/z\})$
    \end{itemize}
  \item	
    le side condition delle par servono solo nella late? mi sembra di si
  \item
    sistemare index: o lo tolgo completamente o lo metto a tutte le definizioni.
  \item	
    cercare di equilibrare l'importanza del contenuto con gli spazi usati
\end{itemize}
