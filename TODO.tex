
fare esempio di almeno due derivazioni per ogni semantica, scegliere esempi significativi che usino almeno due prefissi e una sincronizzazione oppure che mostrino come avviene lo scope extension.
O meglio quali sono gli esempi notevoli, che suggeriscono che la semantica e' giusta?

capire il significato delle regole open e close, fare un discorso generale sullo scope estrusion. in particolare in che semantica serve

ci sono delle cose che dovrei studiare per capire meglio l'algebra dei processi?

si puo' calcolare la semantica usando un motore di inferenza ad esempio Jena?

scrivere il codice usando un pacchetto apposta?

mettere tutte le regole in tabelle con riferimento invece che tabelle ancorate al testo?

usare inferrule invece delle tabelle per le regole?

controllare dove e se viene usata ogni definizione ed eventuaelmente usarla o rimuoverla

usare matita o coq o isabella per dimostrare tutto incluse le transizioni?



CONTROLLARE TUTTE LE REGOLE:
> L’idea è che nella early
> quello che ricevi (cioe’ z) è un valore e come tale non ha vincoli
> rispetto ai free name esistenti, mentre nella late z sarebbe un place
> holder e quindi devi tenerlo distinto da tutti gli altri nomi free (onde
> evitare che quando andrai a fare la sostituzione in comunicazione, tu non
> vada a legare altri nomi che non hanno partecipato alla comunicazione). 


La regola OPEN serve quando e' solo l'oggetto dell'azione ad essere ristretto e non il soggetto. Ma perche'?
