
leggere fullCCSn.pdf

leggere 

Cédric Fournet and Georges Gonthier. The reflexive
CHAM and the join-calculus. In POPL, pages 372–
385, 1996.

Cédric Fournet, Georges Gonthier: The Join Calculus:
A Language for Distributed Mobile Programming.
APPSEM 2000: 268-332



leggere la parte sulle bisimulazioni del sangiorgi



















far funzionare le dimostrazioni di equivalenza
modificare la semantica late del pi calcolo in modo da uniformarla a quella early
la definizione con le regole di inferenza per l'alfa transizione va bene? posso dimostrare che l'alfa equivalenza cosi' definita e' davvero una equivalenza per tutti i termini intuitivamente alfa equivalenti?
fare la parte di equivalenza per la parte late
aggiungere un esempio di uso della regola OpnAlp
trasportare gli esempi early al caso late
studiare tutti gli articoli che il prof mi ha detto di studiare, dovrebbero essere 5



0- aggiungere un esempio sullo scope extension nella parte sulla congruenza strutturale
1- leggere gli articoli di honda e yashimi
 - aggiungere la parte sul multi ccs
2- definire una semantica step per multi pi
3- definire delle relazioni di bisimulazione congruenza per multi pi
4- assicurarsi che ci siano almeno 2 esempi per ogni semantica
5- aggiungere proprieta' delle semantiche e delle bisimulazioni, quali?
6- cercare di trasportare multi pi sullo spi calcolo?
7- modificare la sintassi dei nomi delle regole in modo da renderla uniforme
8- ha senso scrivere tutto al plurale?


fare esempio di almeno due derivazioni per ogni semantica, scegliere esempi significativi che usino almeno due prefissi e una sincronizzazione oppure che mostrino come avviene lo scope extension.
O meglio quali sono gli esempi notevoli, che suggeriscono che la semantica e' giusta?
i nomi di tutte le regole devono essere in UpperCamelCase, inoltre eventuali L per left ed R per right vanno alla fine!
ci sono delle cose che dovrei studiare per capire meglio l'algebra dei processi?
si puo' calcolare la semantica usando un motore di inferenza ad esempio Jena?
mettere tutte le regole in tabelle con riferimento invece che tabelle ancorate al testo? TABELLE CON RIFERIMENTO!
usare inferrule invece delle tabelle per le regole? SI!
controllare dove e se viene usata ogni definizione ed eventuaelmente usarla o rimuoverla
usare matita o coq o isabella per dimostrare tutto incluse le transizioni?
ha senso cercare di capire che rapporto c'e' tra multi pi calcolo e reti di petri?
nella parte di pi calcolo, aggiungere esempi che chiariscano le differenze e le similitudini tra le varie bisimulazioni



