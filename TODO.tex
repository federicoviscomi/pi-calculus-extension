




\begin{enumerate}
  \item
    dimostrare(o negare) l'equivalenza del pi calcolo con e senza congruenza strutturale e con e senza alfa conversione. FATTO MA NON COME SPERATO.
  \item
    nel multi pi calcolo con strong prefixing solo su input o solo su output: definire una semantica di basso livello sulla falsariga di quell'articolo. FATTO MA NON COME SPERATO.
    raggiungere un qualche risultato simile anche per multipiInpOut
  \item
    terminare la parte sulle bisimulazioni nel multipiOut senza congruenza strutturale.
    fare una cosa simile anche per multipiInp senza congruenza strutturale?
  \item
    terminare la parte sulle bisimulazioni nel multipiInp con congruenza strutturale.
    fare una cosa simile anche per multipiOut con congruenza strutturale?
  \item
    dare una semantica open step e provare a definire una bisimulazione open sulla semantica step.
    per multipiOut con e senza congruenza strutturale e per multipiInp con e senza congruenza strutturale
  \item
    trovare la congruenza coarsest contenuta nella bisimulazione scelta in precedenza
  \item
    ripetere i ragionamenti fatti in precedenza anche per multipiInpOut
  \item
    ha senso una semantica che conserva la proprieta' di essere una forma normale definita su coppie (insieme di nomi ristretti, processo senza restrizioni unguarded)?
\end{enumerate}


% perche' nella regola originale sull'input nel pi calcolo early c'e' quella premessa?
% \begin{itemize}
%   \item 
%     controllare tutti i ref
%   \item 
%     nel multipi solo output aggiungere stareps e modificare star in starout e modificare le dimostrazioni relative 
%   \item 	
%     stampare e rileggere
%   \item
%     riscrivere tutte le regole usando inferrule
%   \item
%     aggiungere la definizione precisa induttiva di alfa conversione anche per il multipi
%   \item
%     aggiungere a tutti i pi e multi pi la derivazione nella semantica late di
%     \begin{itemize}
%       \item 
% 	$(x(z).P|Q)|\overline{x}y.R \xrightarrow{\tau} (P\{w/z\}|Q)\{y/w\}|R$ 
%       \itemcd 
% 	$y=z$ or $y\notin fn(Q)$: $((\nu y) \overline{x}y.P)| x(z).Q \xrightarrow{\tau} (\nu y)(P|Q\{y/z\})$
%       \item
% 	$y\neq z$ and $y\in fn(Q)$  $y^{'}$ fresh: $((\nu y) \overline{x}y.P)| x(z).Q \xrightarrow{\tau} (\nu y^{'})(P\{y^{'}/y\}|Q\{y^{'}/z\})$
%     \end{itemize}
%   \item	
%     le side condition delle par servono solo nella late? mi sembra di si. aggiungere una parte che spiega questo argomento
%   \item
%     sistemare index: o lo tolgo completamente o lo metto a tutte le definizioni.
%   \item	
%     qual'e' l'ortografia corretta di multi-parti?
%   \item	
%     ci sono delle parti che sono uguali in multipioutput e multipiinput. fattorizzare.
%   \item	
%     cosa succede nella regola opnseq di multiout se sigma contiene $\overline{z}u$? per adesso ho deciso di eliminarle perche' $(\nu z)(\underline{\overline{x}z}.\underline{\overline{x}y}.\overline{x}z.0) \xrightarrow{\overline{x}(z).\overline{x}y.\overline{x}(z)} 0$ non sembra essere derivabile in multioutlow. aggiungere un paragrafo alla fine che spiega quasta situazione ed eventualmente propone una soluzione con relative dimostrazioni. Fare una cosa simile per multi inp e out.
%     la regola opn che avevo usato e' giusta? o ha piu' senso dire che se l'azione ristretta e' il sogetto di una azione in sigma allora non si puo' applicare la open?
%   \item
%     aggiugere le regole per la comunicazione L|P in muti IO
% \end{itemize}
