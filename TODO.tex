
leggere fullCCSn.pdf

leggere 

Cédric Fournet and Georges Gonthier. The reflexive
CHAM and the join-calculus. In POPL, pages 372–
385, 1996.

Cédric Fournet, Georges Gonthier: The Join Calculus:
A Language for Distributed Mobile Programming.
APPSEM 2000: 268-332



leggere la parte sulle bisimulazioni del sangiorgi



















far funzionare le dimostrazioni di equivalenza
modificare la semantica late del pi calcolo in modo da uniformarla a quella early
la definizione con le regole di inferenza per l'alfa transizione va bene? posso dimostrare che l'alfa equivalenza cosi' definita e' davvero una equivalenza per tutti i termini intuitivamente alfa equivalenti?
fare la parte di equivalenza per la parte late
aggiungere un esempio di uso della regola OpnAlp
trasportare gli esempi early al caso late




-1 - ricontrollare tutte le transizioni: chiarire quando l'input deve essere bound e quando no
0- aggiungere un esempio sullo scope extension nella parte sulla congruenza strutturale
1- leggere gli articoli di honda e yashimi
 - aggiungere la parte sul multi ccs

3- definire delle relazioni di bisimulazione congruenza per multi pi
4- assicurarsi che ci siano almeno 2 esempi per ogni semantica
5- aggiungere proprieta' delle semantiche e delle bisimulazioni, quali?

7- modificare la sintassi dei nomi delle regole in modo da renderla uniforme
8- ha senso scrivere tutto al plurale?
9- manca la regola I4 nel multi pi in out


i nomi di tutte le regole devono essere in UpperCamelCase, inoltre eventuali L per left ed R per right vanno alla fine!

controllare dove e se viene usata ogni definizione ed eventuaelmente usarla o rimuoverla




