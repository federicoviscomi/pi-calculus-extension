\documentclass{llncs}


\usepackage{epsfig}
\usepackage{latexsym}
\usepackage{graphicx}
\usepackage{color}
\usepackage{url}
\usepackage{stmaryrd}
\usepackage{pgf,pgfarrows,pgfnodes,pgfautomata,pgfheaps,pgfshade}
\usepackage{amsmath}
\usepackage{amsfonts}
%\usepackage{harvard}
%\bibliographystyle{alpha}
%\textheight = 20.8cm
%\textwidth = 12.5cm
\usepackage{tikz}

\newcommand{\post}[1]{\mbox{$#1^{\bullet}$}}
\newcommand{\pre}[1]{\mbox{$^{\bullet}#1$}}
\newcommand{\postr}[1]{#1^{\circ}}
\newcommand{\prer}[1]{\mbox{$^{\circ}#1$}}

%\newcommand{\deriv}[1]{{\mbox{${\:\stackrel{#1}{\rightarrow}\:}$}}}
\newcommand{\hide}{/}
%\newcommand{\restr}{\backslash}
%\newcommand{\fine}{\hfill{$\rule{2mm}{2mm}$}}
%\newcommand{\proof}{\noindent{\em Proof.}\ \  }

\newcommand{\lv}[1]{\Lambda_{#1}}
\newcommand{\hv}[1]{\mathcal{ H}_{#1}}
\newcommand{\lvtrace}{\stackrel{\Lambda}{\approx}_{tr}}
\newcommand{\lvbis}{\stackrel{\Lambda}{\approx}_{bis}}

\newcommand{\redStr}[1]{\color{red}#1\color{black}\ }
\newcommand{\angbr}[1]{\langle #1 \rangle}
\renewcommand{\labelitemi}{$\bullet$}
\renewcommand{\labelitemii}{$\bullet$}
\renewcommand{\labelitemiii}{$\bullet$}

\newtheorem{adefinizione}{Definition}[section]
\newtheorem{afact}[adefinizione]{Fact}
\newtheorem{alemma}[adefinizione]{Lemma}
\newtheorem{teorema}[adefinizione]{Theorem}
\newtheorem{corollario}[adefinizione]{Corollary}
\newtheorem{proposizione}[adefinizione]{Proposition}
\newtheorem{esempio}[adefinizione]{Example}

\newcommand{\mmuo}[1] {\underline{ \overline{ #1 } } }
\newcommand{\ouno}[2]{ \mmuo{ #1 } \mmang{ #2}  }
\newcommand{\mmo}[1] {\overline{ #1 } } 
\newcommand{\ozero}[2]{ \mmo{ #1 } \mmang{ #2}  }
\newcommand{\mmang}[1]{ \langle { #1 } \rangle }

\newcommand{\mmu}[1] {\underline{ #1 } }
\newcommand{\iuno}[2]{ \mmu{ #1 } ( { #2 } )  }
\newcommand{\izero}[2]{ { #1 } ( { #2 } )  }
\newcommand{\parallelo}{ \; \, \big| \, \; }
\newcommand{\bang}{!}
\newcommand{\new}[1]{ \nu {#1}  }
\newcommand{\parallelobig}{\quad \Big| \quad }
\newcommand{\dueduepuntiuguale}{ \quad ::= \quad }
\newcommand{\buco}[2]{\begin{figure}[!hbp]\makebox[\textwidth]{\framebox[#1]{\rule{0pt}{#2}}}\end{figure}}
\newcommand{\barb}{\downarrow}
\newcommand{\nbarb}{\downarrow \hspace{-9pt} \smallsetminus \;}
\newcommand{\ramencoding}[1]{ { \langle \spazio \big[ \;  { #1 }  \; \big] \spazio \rangle}}
\newcommand{\mdef}{ \triangleq }
\newcommand{\mydef}{\mdef}
\newcommand{\myto}{\;\rightarrow\;}
\newcommand{\myslowto}{\;\mapsto\;}
\newcommand{\myprioto}{\;\twoheadrightarrow\;}
\newcommand{\nmyprioto}{\myprioto\mspace{-24mu}\raisebox{0pt}[0pt][0pt]{\text{\raisebox{1.4pt}{\tiny /}}}\;\;\;}


%\newcommand{\proof}{{\it Proof: }}
\newcommand{\linib}{\mbox{$\sqsubset$}}
\newcommand{\ginib}{\mbox{$\sqsupset$}}
\newcommand{\leqinib}{\mbox{$\sqsubseteq$}}
\newcommand{\geqinib}{\mbox{$\sqsupseteq$}}
\newcommand{\emlim}{EL}
\newcommand{\fine}{{\mbox{ }\nolinebreak\hfill{$\Box$}}}
%\newcommand{\post}[1]{#1^{\bullet}}
%\newcommand{\pre}[1]{{^{\bullet}#1}}
\newcommand{\inib}[1]{{^{\circ}#1}}
\newcommand{\test}[1]{\widehat{#1}}
\newcommand{\init}{*}

\newcommand{\postn}[2]{#1^{\bullet_{#2}}}
\newcommand{\pren}[2]{{^{\bullet_{#2}}#1}}
%\newcommand{\testn}[2]{\whidehat_{#2}{#1}}

% macro causal semantics
\newcommand{\Max}[1]{Max(#1)}
\newcommand{\Min}[1]{Min(#1)}
\newcommand{\ctop}{\mbox{${\cal F}$}}
\newcommand{\ptoc}{\mbox{${\cal G}$}}

% macro sem picalcolo
\newcommand{\Conc}{\mbox{$Conc$}}
\newcommand{\dec}{\mbox{$dec$}}
\newcommand{\sost}[2]{\mbox{$\{#1/#2\}$}}
\newcommand{\sostd}[4]{\mbox{$\{#1/#2,#3/#4\}$}}
\newcommand{\deriv}[1]{{\mbox{${\:\stackrel{#1}{\longrightarrow}\:}$}}}
\newcommand{\sderiv}[1]{{\mbox{${\:\stackrel{#1}{\rightarrow}\:}$}}}
\newcommand{\Deriv}[1]{{\mbox{${\:\stackrel{#1}{\Longrightarrow}\:}$}}}
\newcommand{\derivas}[1]{{\mbox{${\:\stackrel{#1}{\longrightarrow_s}\:}$}}}
\newcommand{\derivls}[1]{{\mbox{${\:\stackrel{#1}{\longrightarrow_{ls}}\:}$}}}
%\newcommand{\derivstar}[1]{{\mbox{${\:\stackrel{#1}{\longrightarrow^*}\:}$}}}
\newcommand{\derivstar}[1]{{\mbox{${\:\stackrel{#1}{\longrightarrow}\!\phantom{}^*\:}$}}}
\newcommand{\nonderiv}[1]{{\mbox{${\:\not\stackrel{#1}{\longrightarrow}\:}$}}}
\newcommand{\nonderivas}[1]{{\mbox{${\:\not\stackrel{#1}{\longrightarrow_{s}}\:}$}}}
\newcommand{\nonderivls}[1]{{\mbox{${\:\not\stackrel{#1}{\longrightarrow_{ls}}\:}$}}}
\newcommand{\Nderiv}[1]{{\mbox{${\:\stackrel{#1}{\nrightarrow}\:}$}}}
\newcommand{\nderiv}[1]{\nrightarrow}
\newcommand{\cderiv}[1]{{\mbox{${\:\stackrel{#1}{-\!-\!\!\!\longrightarrow}\:}$}}}
\newcommand{\eqdef}{{\mbox{${\stackrel{def}{=}}$}}}
\newcommand{\spazio}{\hspace{2.4em}}
\newcommand{\bigfrac}[2]{
\renewcommand{\arraystretch}{1.5}
\begin{array}{c}#1\\
\hline
#2
\end{array}}
\newcommand{\restr}[1]{\mbox{$({\bf\nu} #1)$}}
\newcommand{\para}{\mbox{$\,|\,$}}
\newcommand{\recx}[1]{\mbox{rec$\, X.#1$}}
\newcommand{\parti}{\mbox{\raisebox{.4ex}{$\wp$}}}
%\newcommand{\minus}{\stackrel
%{\raisebox{-.1cm}[.1cm][.1cm]{.}}{\raisebox{-.01cm}[.1cm][.1cm]{-}}}
\newcommand{\minus}{\stackrel{\cdot}{-}}
\newcommand{\myequiv}{\stackrel{\rm def}{=}}
\newcommand{\nil}{\mbox{\bf 0}}
\newcommand{\1}{\mbox{\bf 1}}
\newcommand{\alphaconv}{\equiv_{\alpha}}

\renewcommand{\mid}{\;\;\big|\;\;}

\newcommand{\encoding}[2]{\ensuremath{\llbracket #2 \rrbracket^{#1}}}
\newcommand{\undef}{\mbox{$\bot$}}
\newcommand{\rec}[2]{\mbox{rec}\, #1.#2}

% \def\bbbn{{\rm I\!N}}
\newcommand{\nat}{{\mathbb N}}
\newcommand{\Net}{\mathit{Net}}
\newcommand{\Reach}{\mathit{Reach}}
\newcommand{\TS}{\mathit{TS}}
\newcommand{\fin}{\mathit{fin}}
\newcommand{\dom}{\mathit{dom}}
\newcommand{\IMG}{\mathit{IMG}}
\newcommand{\SMG}{\mathit{SMG}}
\newcommand{\fork}{\mathit{fork}}
\newcommand{\Act}{\mathit{Act}}
\newcommand{\MSync}{\mathit{MSync}}
\newcommand{\GSync}{\mathit{GSync}}
\newcommand{\Sync}{\mathit{Sync}}
\newcommand{\Int}{\mathit{Int}}

\newcommand{\A}[1]{\textcolor{red}{#1}}
\newcommand{\inet}[1]{\ensuremath{INet(#1)}}

\newcommand{\LL}[1]{\mbox{\em{#1}}}



\begin{document}


\title{A Fully-Abstract Semantics for Atomicity}
\author{Roberto Gorrieri\\
\institute{Dipartimento di Scienze dell'Informazione, Universit\`a di
Bologna, \\Mura A. Zamboni, 7,
40127 Bologna, Italy}
\email{gorrieri@cs.unibo.it}
}


\maketitle


\begin{abstract}
Multi-CCS extends conservatively CCS with an operator of strong 
prefixing able to model atomic sequences of actions as well as multiparty synchronization.
We equip this calculus with an interleaving semantics, which makes use of a limited 
structural congruence in order to ensure that parallel composition be associative.
Unfortunately, this semantics is not compositional, as it is not a congruence for parallel composition:
in $p \para q$, an atomic sequence performed by a process $p$ can sense, to some extent, 
the degree of concurrency of the parallel component $q$. Step bisimilarity, however, turns out to be too discriminating:
there are interleaving bisimilar processes, that no parallel context can tell apart, that are not step bisimilar.
Then, linear-step bisimilarity is introduced 
and proved to be the coarsest congruence contained in interleaving bisimilarity. 
A finite sound and complete axiomatization of linear-step bisimilarity is presented for finite Multi-CCS
processes.
\end{abstract}

\section{Introduction}

CCS \cite{Mil89}, even if Turing-complete, is not expressive enough to model some useful
behaviour. For instance, it does not support multiway synchronization, nor atomic transactions.
In this line, it can be proved \cite{LR81,LV10} that a classic solution to the famous dining philosophers
problem \cite{Dij71} that assumes atomicity in the acquisition of the forks (or equivalently, that requires
a three-way synchronization among one philosopher and the two forks), cannot be provided in CCS.

Multi-CCS is a conservative extension to CCS to allow for the description of transactional behaviour,
as well as multiparty synchronization expressed by means of an atomic sequence of binary CCS synchronizations.
Syntactically, this is obtained by adding one prefixing operator $\underline\mu.p$, called {\em strong prefixing} 
(in opposition to normal prefixing $\mu.p$):  in a process $\underline{\mu}.p$, action $\mu$ is the first one 
of a transaction that continues with $p$, provided that $p$ can complete the transaction. 
Therefore, for instance, $\underline{a}.b.p \deriv{ab} p$, i.e., the resulting
transition system is labeled on sequences of actions. There is also a generalization of the 
operational rule for communication
that takes care of the synchronization of two sequences: if $p \deriv{\sigma_1} p'$ and $q \deriv{\sigma_2} q'$, then
$p \para q \deriv{\sigma} p' \para q'$, where sequence $\sigma$ has been obtained by an interleaving (with possible synchronizations)
of $\sigma_1$ and $\sigma_2$ where at least one synchronization has taken place. Sequence $\sigma$ can be 
used for a possible synchronization with an additional process $r$: since $p \para q \deriv{\sigma} p' \para q'$,
if $r \deriv{\sigma'} r'$, then $(p \para q) \para r \deriv{\sigma''} (p' \para q') \para r'$, hence representing a form of
ternary synchronization, realized by means of an atomic sequence of binary synchronizations. In general,
any form of multiway synchronization can be represented.

Strong prefixing was introduced in \cite{GMM,GM90}. The calculus presented there was named A$^2$CCS.
It differs from Multi-CCS for two important features. First, parallel composition is associative for Multi-CCS
thanks to the introduction of a structural congruence $\equiv$ induced by a minimal set of equations. Second,
the synchronization discipline on pairs of sequences is more liberal in Multi-CCS (it simply requires that at least
one synchronization takes place) and less verbose (as intermediate occurrences of $\tau$ are dissolved, so
that labels can be only of two types: either $\tau$ or a non-empty sequence of visibile actions).

Multi-CCS is very expressive and many well-known synchronization problems are easily solvable, e.g., 
dining philosophers \cite{Dij71}, concurrent readers and writers \cite{CHP71}, 
Patil's cigarettes' smokers \cite{Pat71}, etc.. 
In \cite{GV10a}, it is also shown that
such a calculus is expressive enough to represent all finite Place/Transition Petri nets. 

In this paper, we investigate the behavioural semantics for Multi-CCS. Interleaving bisimilarity 
enjoys some expected algebraic laws but it is not compositional, as it is not a congruence for parallel 
composition. The problem is due to the fact that interleaving bisimilar processes may be distinguished 
by a parallel component able to perform an atomic sequence. For instance, 
consider process $p = \overline{a}.\overline{a}.\nil$ and process $q = \overline{a}.\nil \para \overline{a}.\nil$.
Clearly, $p$ is bisimilar to $q$, written $p \sim q$. However, if we consider a context $\mathcal{C}[-] =  - \para \underline{a}.\underline{a}.c.\nil$, we get
that $\mathcal{C}[p] \not\sim \mathcal{C}[q]$, because the latter can perform $c$, i.e., 
$\mathcal{C}[q] \deriv{c} (\nil \para \nil) \para \nil$, 
while $\mathcal{C}[p]$ cannot. The reason for this
difference is that the process $\underline{a}.\underline{a}.c.\nil$ can react with a number of concurrently active components
equal to the length of the trace it can perform. Hence, a congruence semantics 
for parallel composition may need to distinguish $p$ and $q$ on the basis of their different degree of parallelism.

Therefore, one may define a step transition system where transitions are labeled by multisets of
concurrently executable sequences. Step bisimilarity would distinguish 
$p = \overline{a}.\overline{a}.\nil$ and $q = \overline{a}.\nil \para \overline{a}.\nil$, as only $q$ can perform a
step $\{\overline{a}, \overline{a}\}$.
However, step bisimilarity is not fully abstract: it is possible to find processes $p$ and $q$ 
such that for all  $r$, $p \para r \sim q \para r$, but $p \not\sim_{step} q$. 
Simply take $p = (a \para a) + \underline{a}.a.\nil$
and $q = a.a.\nil + \underline{a}.a.\nil$. 
In other words, step
bisimilarity is not the coarsest congruence contained in interleaving bisimilarity.

This example suggests that the contents of a step $\{a, a\}$ 
which is testable by a parallel component is its possible atomic linearisation $aa$. Hence, it seems that
the coarsest congruence should not observe steps, rather linearisation of steps. Following
this intuition, we define a novel semantics, called {\em linear-step} semantics and prove to be the coarsest
congruence contained in interleaving bisimilarity.
Moreover, contrary to non-interleaving equivalences, linear-step bisimilarity
can be axiomatized. For finite Multi-CCS we present a finite sound and complete axiomatization.

The paper is organized as follows. Section 2 presents the syntax of Multi-CCS. Section 3 reports
the interleaving semantics and its algebraic properties. Section 4 introduces the step semantics
for Multi-CCS and discusses that, albeit a congruence, it is not fully asbtract. Section 5
proposes the novel linear-step semantics
and proves that it is the coarsest congruence. Section 6 is devoted to the finite axiomatization.
Some concluding remarks are reported in Section 7.

\section{Multi-CCS: Syntax}

We assume the reader familiar with CCS \cite{Mil89,Mil99}.
We assume to have a denumerable set ${\mathcal L}$ of channel names, its complementary 
set  $\overline{\mathcal L}$ of co-names, the set  ${\mathcal L}\cup\overline{\mathcal L}$ (ranged over by $\alpha, \beta, \ldots$)
of visible actions and the set of all actions $Act = {\mathcal L} \cup \overline{\mathcal L} \cup\{\tau\}$,
such that $\tau\not\in{\mathcal L}\cup\overline{\mathcal L}$, ranged over by $\mu$.

The process terms are generated by the following grammar:

\[p ::= \nil \mid \mu.p \mid  \underline{\mu}.p \mid p+p \mid  p\para p \mid \restr{a}p \mid C \]

Term $\nil$ is the terminated process, $\mu.p$ is a normally prefixed process where action $\mu$ (that
can be either an input $a$, an output  $\bar{a}$ or a silent move $\tau$) is first performed and then $p$ is ready, 
$ \underline{\mu}.p$ is a  strongly prefixed process where $\mu$ is the first action of a transaction 
that continues with $p$
(provided that $p$ can complete the transaction).
With $p + p'$ we denote the process obtained by the alternative composition of 
$p$ and $p'$.
Term $p \para p'$ is the parallel composition of $p$ and $p'$, $\restr{a}p$ is process $p$ where the (input) name $a$ is made private (restriction),
and $C$ is a process constant, equipped with a defining equation $C \eqdef p$.

We denote with ${\mathcal P}$ the set of {\em processes}, containing those terms which are, 
w.r.t. process constants,
closed  and {\em guarded}  (for any defining equation $C \eqdef p$,
any occurrence of a constant in $p$ is  within a {\em normally prefixed} subprocess $\mu.p'$ of $p$). 
${\mathcal P}$ will be ranged over by $p, q, r$, possibly indexed.  



\section{Interleaving Semantics}

The  operational semantics for Multi-CCS is given by the labelled 
transition system $({\mathcal P},{\mathcal A}, \deriv{ })$, where the states are the processes
in ${\mathcal P}$, ${\mathcal A} = (\mathcal{L} \cup \overline{\mathcal L})^+ \cup \{\tau\}$
is the set of labels (ranged over by $\sigma$, possibly indexed),
and $\deriv{ } \subseteq {\mathcal P}\times{\mathcal A}\times{\mathcal P}$ is the minimal 
transition relation generated by the rules listed in Table~\ref{int-rules}. 


Let us comment only the non-standard rules.
Rule (S-Pref$_3$) is responsible for the creation of
transitions labeled by non-empty sequences of actions: in order for $\underline{\alpha}.p$ to make a move, it is
necessary that $p$ can perform a transition, i.e., the rest of the transaction. Hence, if $p \deriv{\sigma} p'$ then 
$\underline{\alpha}.p \deriv{\alpha\sigma} p'$. Note that $\underline \mu.\nil$ 
cannot perform any action, as $\nil$ is stuck. Usually, if a transition is labeled by 
$\sigma = \alpha_1 \ldots \alpha_{n-1} \alpha_n$, then all the
actions $\alpha_1 \ldots \alpha_{n-1}$ are due to strong prefixes, while $\alpha_n$ to a
normal prefix (or $\alpha_n$ is the last strong prefix before a $\tau$).
Rules (S-Pref$_1$) and (S-Pref$_2$) ensures that $\tau$'s are never added in a sequence $\sigma$, hence ensuring that
in a transition $p \deriv{\sigma} p'$ either $\sigma = \tau$ or $\sigma$ is composed only of visible actions,
i.e., $\sigma$ ranges over ${\mathcal A} = (\mathcal{L} \cup \overline{\mathcal L})^+ \cup \{\tau\}$.


\begin{table}[t]
{\renewcommand{\arraystretch}{2.8}
\hrulefill\\[-.8cm]

%{\renewcommand{\arraystretch}{2.8}
\begin{center}
$\begin{array}{lcllcl}
\mbox{(Pref)}  & \mu.p\deriv{\mu}p & & \; \; \;
%\mbox{(S-Pref)}  & \bigfrac{p\deriv{\sigma}p'}{\underline{\mu}.p\deriv{\sigma'}p'} & \Conc(\mu, \sigma, \sigma') \\
\mbox{(S-Pref$_1$)}  & \bigfrac{p\deriv{\sigma}p'}{\underline{\tau}.p\deriv{\sigma}p'} &  \\
\mbox{(S-Pref$_2$)}  & \bigfrac{p\deriv{\tau}p'}{\underline{\alpha}.p\deriv{\alpha}p'} & & \; \; \;
\mbox{(S-Pref$_3$)}  & \bigfrac{p\deriv{\sigma}p' \quad \sigma \neq \tau}{\underline{\alpha}.p\deriv{\alpha\sigma}p'} & \\
\mbox{(Cong)} & \bigfrac{p \equiv p' \deriv{\sigma}q' \equiv q}{p\deriv{\sigma}q}  & & \; \; \;
\mbox{(Sum$_1$)}  & \bigfrac{p\deriv{\sigma}p'}{p+q\deriv{\sigma}p'}  \\
\mbox{(Par$_1$)}  & \bigfrac{p\deriv{\sigma}p'}{p\para q\deriv{\sigma}p'\para q} & & \; \; \;
\mbox{(S-Res)}  & \bigfrac{p\deriv{\sigma}p'}{\restr{a}p\deriv{\sigma}
\restr{a}p'} & a, \bar{a} \not\in n(\sigma) \\
%\mbox{(Cong)} & \bigfrac{p \equiv p' \deriv{\sigma}q' \equiv q}{p\deriv{\sigma}q}  & & \; \; \;
\end{array}$
$\begin{array}{lcllcl}
\mbox{(S-Com)}  & \bigfrac{p\deriv{\sigma_1} p'\spazio q\deriv{\sigma_2}q'}{p
\para q \deriv{\sigma}p'\para q'} & \Sync(\sigma_1,  \sigma_2, \sigma) \\
\end{array}$

\hrulefill
\end{center}}
\caption{Operational semantics (symmetric rules for (Sum$_1$) and (Par$_1$) omitted)}\label{int-rules}
\end{table}



%\begin{table}
%\hrulefill\\[-.5cm]
%\begin{center}
%$\begin{array}{lclcl}
%\Conc(\mu, \tau, \mu) \; \; &  \; \; 
%\Conc(\tau, \sigma, \sigma) \; \; & \; \; 
% \bigfrac{\mu \neq \tau \neq \sigma}{\Conc(\mu, \sigma, \mu\sigma)} \\
%\end{array}$

%\end{center}

%\hrulefill%\\[-.4cm]
%\caption{Concatenation relation $\Conc$.}\label{conc}
%\end{table}


\begin{table}[t]
{\renewcommand{\arraystretch}{1.2}
\hrulefill\\[-.6cm]
\normalsize{

%{\renewcommand{\arraystretch}{2.8}
\begin{center}

\hrulefill

$\begin{array}{llrclllrrl}
{\bf S1} &\; \; &\; \;  (p \para q) \para r & = & p \para (q \para r) \\
{\bf S2} &\; \; &\; \; p \para q & = & q \para p \\
{\bf S3} &\; \; &\; \; A & = & q & \quad \mbox{ if $A \eqdef q$}\\
{\bf S4} &\; \; &\; \; \restr{a} (p \para q)  & = & p \para \restr{a}q &  \quad \mbox{ if $a$ not free in $p$}\\
{\bf S5} &\; \; &\; \; \restr{a} p & = &  \restr{b} (p\sost{b}{a}) & \quad \mbox{ if $b$ does not occur in $p$}\\
\end{array}$


\hrulefill

\end{center}}
}
\caption{Axioms generating the structural congruence $\equiv$.}\label{structural}
\end{table}




\begin{table}
\hrulefill\\[-.5cm]
\begin{center}
$\begin{array}{lclcl}
\Sync(\alpha, \overline{\alpha}, \tau) &  
\bigfrac{\Int(\sigma_1, \sigma_2, \sigma)}{\Sync(\alpha\sigma_1, \overline{\alpha}\sigma_2, \sigma)} &
% \bigfrac{\Sync(\sigma_1, \sigma_2, \sigma) \; \; \Conc(\mu, \sigma, \sigma')}{\Sync(\mu\sigma_1, \sigma_2, \sigma')} &
% \bigfrac{\Sync(\sigma_1, \sigma_2, \sigma) \; \; \Conc(\mu, \sigma, \sigma')}{\Sync(\sigma_1, \mu\sigma_2, \sigma')} \\
 \bigfrac{\Sync(\sigma_1, \sigma_2, \tau)}{\Sync(\alpha\sigma_1, \sigma_2, \alpha)} \\
\bigfrac{\Sync(\sigma_1, \sigma_2, \tau)}{\Sync(\sigma_1, \alpha\sigma_2, \alpha)} &
\bigfrac{\Sync(\sigma_1, \sigma_2, \sigma) \quad \sigma \neq \tau}{\Sync(\alpha\sigma_1, \sigma_2, \alpha\sigma)} &
 \bigfrac{\Sync(\sigma_1, \sigma_2, \sigma) \quad \sigma \neq \tau}{\Sync(\sigma_1, \alpha\sigma_2, \alpha\sigma)} \\
\end{array}$\\

\hrulefill%\\[-.4cm]

$\begin{array}{lllll}
\Int(\alpha, \overline{\alpha}, \tau) \; \; &  \; \; \Int(\alpha, \epsilon, \alpha) \; \; &
\; \;\Int(\epsilon, \alpha, \alpha) \; \; & \; \; \bigfrac{\Int(\sigma_1, \sigma_2, \sigma)}{\Int(\alpha\sigma_1, \overline{\alpha}\sigma_2, \sigma)} \\
\end{array}$

$\begin{array}{lclcl}
% \bigfrac{\Int(\sigma_1, \sigma_2, \sigma) \; \; \Conc(\mu, \sigma, \sigma')}{\Int(\mu\sigma_1, \sigma_2, \sigma')} \; \; & \; \; 
% \bigfrac{\Int(\sigma_1, \sigma_2, \sigma) \; \; \Conc(\mu, \sigma, \sigma')}{\Int(\sigma_1, \mu\sigma_2, \sigma')} \\
 \bigfrac{\Int(\sigma_1, \sigma_2, \tau)}{\Int(\alpha\sigma_1, \sigma_2, \alpha)} & 
\; \; \bigfrac{\Int(\sigma_1, \sigma_2, \tau)}{\Int(\sigma_1, \alpha\sigma_2, \alpha)} &
\; \;  \bigfrac{\Int(\sigma_1, \sigma_2, \sigma) \quad \sigma \neq \tau}{\Int(\alpha\sigma_1, \sigma_2, \alpha\sigma)} & 
\; \; \bigfrac{\Int(\sigma_1, \sigma_2, \sigma) \quad \sigma \neq \tau}{\Int(\sigma_1, \alpha\sigma_2, \alpha\sigma)} \\
\end{array}$

\end{center}

\hrulefill%\\[-.4cm]
\caption{Synchronization relation $\Sync$ and interleaving relation $\Int$.}\label{b-sync}
\end{table}

Rule (S-Com) has a side-condition on the possible synchronizability of sequences $\sigma_1$ and $\sigma_2$,
whose result may be $\sigma$. 
$\Sync(\sigma_1, \sigma_2, \sigma)$ holds
if $\sigma$ is obtained from an interleaving (possibly with synchronizations) of $\sigma_1$ and $\sigma_2$, 
where at least one synchronization has taken place. 
Relation $\Sync$ is defined by the inductive rules of Table~\ref{b-sync}, which make use of the auxiliary
relation $\Int$, which is just as $\Sync$ without requiring that at least one synchronization occurs.\footnote{In the definition 
of $\Sync$ and $\Int$, with abuse of notation, we let $\sigma$ range over $\mathcal{A} \cup \{\epsilon\}$, the empty sequence.}
Note that when the resulting $\sigma$ is not $\tau$,
then it can be used for further synchronization with some additional parallel components, hence allowing for
multiparty synchronization.

Rule (S-Res) is slightly different, as it requires that no action in $\sigma$ can be $a$ or $\bar{a}$.
With $n(\sigma)$ we denote the set of all actions occurring in $\sigma$.

There is one further new rule, called  (Cong), which makes use of a structural congruence $\equiv$,
that is needed to overcome a shortcoming of parallel composition: without rule (Cong), parallel
composition is not associative.
The structural congruence $\equiv$ on process terms is  induced by the 
five axioms (S1-S5) in Table \ref{structural}.


%$
%(p \para q) \para r \; = \; p \para (q \para r)
%$

%$
%\restr{a} (p \para q) = \restr{a}p \para q \; \; \mbox{ if $a$ is not free in $q$}.
%p \para q \; = \; q \para p
%$

%$
%\restr{a} (p \para q) = p \para \restr{a}q \; \; \mbox{ if $a$ is not free in $p$}.
%$

%$
%\restr{a} p =  \restr{b} (p\sost{b}{a}) \; \; \mbox{ if $b$ does not occur in $p$}.
%$

\noindent
The first equation is for associativity of the parallel operator; the second law is for commutativity of parallel composition.
The third equation explains why we have no explicit operational rule for handling constants: the transitions derivable from $A$
are those transitions derivable from the structurally congruent term $p$, if $A \eqdef p$. 
The fourth law allows for enlargement of the scope of restriction;
the last equation is the so-called law of {\em alpha-conversion}, which makes use of syntactic substitution. 
The intuition is that, with this structural congruence, all the parallel components are to be considered 
as in a bunch where they can freely interact independently of their actual association or ordering.
Indeed, rule (Cong) enlarges the set of transitions derivable from $p$, as the following four examples show. 

\begin{example}\label{multi-synch}{\bf (Multi-party synchronization)} Assume that three processes want to synchronize. 
This can be easily expressed in Multi-CCS by means of a transaction. E.g., consider $p = \underline{a}.a.p'$,
$q = \bar{a}.q'$ and $r = \bar{a}.r'$ and the whole system $P = \restr{a}((p \para q) \para r)$. It is easy to see
that $P \deriv{\tau} \restr{a}((p' \para q') \para r')$ (and this can be proved in two ways), so the three processes 
have synchronized in one single atomic transition.
It is interesting to observe that $P' = \restr{a}(p \para (q \para r))$ could not
perform the multiway synchronization if  rule (Cong) were not allowed.
\fine
\end{example}

\begin{example}\label{comm-needed}
In order to see that also the commutativity axiom may be useful, consider process $p = 
 (\underline{a}.c.0 | b.0) | (\overline{a}.0 | \underline{\overline{b}}.\overline{c}.0)$. Such a process
 can do a four-way synchronization $\tau$ to $q = (\nil \para \nil) \para (\nil \para \nil)$, because $p' = 
 (\underline{a}.c.0 | \overline{a}.0) | (b.0 | \underline{\overline{b}}.\overline{c}.0)$, which is structurally congruent to $p$,
 can perform $\tau$ reaching $q$.
Without rule (Cong), process $p$ could not perform such a multiway synchronization.
\fine
\end{example}

\begin{example}
Consider $Q = p_1 \para \restr{a}(p_2 \para p_3)$, where $p_1 = \underline{b}.c.p'_1$, $p_2 = \overline{b}.p'_2$ 
and $p_3 = \overline{c}.p'_3$. Without rule (Cong), the three-way synchronization $Q \deriv{\tau} Q''$, 
where $Q'' =  p'_1 \para \restr{a}(p'_2 \para p'_3)$, could not take place. However, by taking a new name $d$
not occurring free in $Q$, we can prove that
$Q$ is structurally congruent to $Q' = \restr{d}((p_1 \para p_2\sost{d}{a}) p_3\sost{d}{a})$, because
of scope enlargement and alpha-conversion, and $Q' \deriv{\tau} Q''' \equiv Q''$, where $Q''' = 
\restr{d}((p'_1 \para p'_2\sost{d}{a}) p'_3\sost{d}{a})$.
\fine
\end{example}

\begin{example}
Consider $R = \underline{a}.c.0 | A$, where $A \eqdef \overline{a}.0 \para \overline{c}.0$. Without rule (Cong) 
(and axiom (S3) of Table \ref{structural}),
it is not possible to derive $R \deriv{\tau} 0 \para (0 \para 0)$. 
\fine
\end{example}

\begin{example}\label{guarded-rec}{\bf (Guardedness)} 
We assume that each process constant in a defining equation occurs
inside a normally prefixed subprocess $\mu.q$. This will prevent infinitely branching sequential processes.
E.g, consider the non legal process $A  \eqdef  \underline{a}.A + b.\nil$.
According to the operational rules, $A$ has infinitely many transitions leading to $\nil$, each of the form $a^nb$, for 
$n = 0, 1, ...$. 
In fact, under guardedness, the set of terms generated by 
\[ p ::= \nil \mid  \mu.p \mid  \underline{\mu}.p \mid p+p  \mid  C\]
defines, up to isomorphism, the set of transition systems labeled on ${\mathcal A} = 
(\mathcal{L} \cup \overline{\mathcal L})^+ \cup \{\tau\}$ with finitely many states and transitions.
\fine
\end{example}


\begin{example}\label{ex-dining}
{\bf(Dining Philosophers)} This famous problem, defined by Dijkstra in \cite{Dij71}, can be solved in Multi-CCS.
Five philosophers sit at a round table, with a private plate and where each of the five forks is shared by two neighbors.
Philosophers can think and eat; in order to eat, a philosopher has to acquire both forks that he shares with his neighbors,
starting from the fork at his left and then the one at his right. All philosophers behave the same, so the problem is intrinsically symmetric.
Clearly a na\"ive solution would cause deadlock exactly when all five philosophers take the fork at their left at the same time 
and are waiting for the fork at their right.
A simple solution is to force atomicity on the acquisition of the two forks so that either both are taken or none. 
In order to have a small model, we consider the case of two philosophers only.
The forks can be defined by the constants $fork_i$: 
\begin{eqnarray*}
fork_i \eqdef  \overline{up_i}.\overline{dn_i}.fork_i \; \;  \mbox{   for  }i = 0, 1 
\end{eqnarray*}
The two philosophers can be described as 
\begin{eqnarray*}
phil_i \eqdef think.phil_i + \underline{up_i}.up_{i+1}.eat.\underline{dn_i}.dn_{i+1}.phil_i  \; \;  \mbox{   for  }i = 0, 1 
\end{eqnarray*}
where $i+1$ is computed modulo $2$ and  the atomic sequence $up_i up_{i+1}$ ensures the atomic 
acquisition of the two forks. For simplicity,
we assume also that the release of the two forks is atomic, but this is not necessary for correctness.
The whole system is  
\begin{eqnarray*}
DP \;  \eqdef \;  \restr{L}(phil_0\para phil_1 \para fork_0 \para fork_1) 
\end{eqnarray*}
where $L = \{up_0, up_1, dn_0, dn_1\}$. 
Note that the operational semantics generates a finite-state 
lts for $DP$, depicted in Figure \ref{ltsdining}. 
\fine
\end{example}


\begin{figure}[t]
\centering
% \includegraphics[width=106mm]{philinter}
\input{philinter}
\caption{The labeled transition system for $DP$.}
\label{ltsdining}
\end{figure}




\subsection{Properties of the semantics}

Two terms $p$ and $q$ are {\em interleaving bisimilar}, written
$p\sim q$, if there exists a strong bisimulation \cite{Mil89} $R$ such that $(p,q)\in R$.
Interleaving bisimulation equivalence enjoys some expected algebraic properties. 

\begin{proposition}\label{prop2}
Let $p, q, r \in {\mathcal P}$ be processes. Then the following holds:\\

$\begin{array}{llll}
(1) & (p + q) + r \; \sim \; p + (q +r)  &
(2) & p + q\; \sim \; q + p  \\
(3) & p + \nil \; \sim \;  p  &
(4) & p + p \; \sim \; p \\
(5) & p \para (q \para r) \; \sim \; (p \para q) \para r  &
(6) & p \para q\; \sim \; q \para p \\
(7) & p \para \nil \; \sim \;  p &
(8) & \restr{x}(p \para q) \; \sim \;  p \para \restr{x}q \mbox{  if } x \not\in fn(p) \\
(9) & \restr{x}\restr{y}p \; \sim \;  \restr{y}\restr{x}p &
(10) & \restr{x}p \; \sim \;  \restr{y} (p\sost{y}{x}) \mbox{  if } y \not\in fn(p) \\
(11) & \restr{x} \nil \; \sim \; \nil\\
\end{array}$

\proof 
The proof is standard.
\fine
\end{proposition}

\begin{corollary}\label{cong>int}
$p \equiv q$ implies $p \sim q$.

\proof Because of (5), (6), (8) and (10) of Proposition \ref{prop2}.
\fine
\end{corollary}

A few properties of strong prefixing are as follows:

\begin{proposition}\label{prop}
Let $p, q \in {\mathcal P}$ be processes. Then the following hold:\\

$\begin{array}{lllllr}
(1) & \underline\mu.(p + q)  \; \sim \;   \underline\mu.p +  \underline\mu.q & \; \;
(2) &  \underline\mu.\nil  \; \sim \;   \nil &\\
(3) &  \underline\tau.p   \; \sim \;  p  &  \; \;
(4) &  \underline\mu.\tau.p   \; \sim \;  \mu.p \\
\end{array}
$\\[-.8cm]

\fine
\end{proposition}


Interleaving bisimulation is a congruence for almost all the operators of Multi-CCS, in particular for strong prefixing.

\begin{proposition}
If $p \sim q$, then the following hold:
\begin{enumerate}
\item  $\mu.p \sim \mu.q \quad$ for all $\mu \in Act$,
\item $\underline{\mu}.p \sim \underline{\mu}.q \quad$ for all $\mu \in Act$
\item $p + r \sim q + r \quad $ for all  $r \in \mathcal{P}$,
\item $\restr{a}p \sim \restr{a}q \quad$ for all $a \in \mathcal{L}$.
\end{enumerate}

\proof
Standard.
\fine
\end{proposition}

Unfortunately, $\sim$ is not a congruence for parallel composition, as the following example shows.

\begin{example}\label{sim-no-cong}{\bf (No congruence for parallel composition)}
Consider process $p = \overline{a}.\overline{a}.\nil$ and process $q = \overline{a}.\nil \para \overline{a}.\nil$.
Clearly, $p \sim q$. However, if we consider a context $\mathcal{C}[-] =  - \para \underline{a}.\underline{a}.c.\nil$, we get
that $\mathcal{C}[p] \not\sim \mathcal{C}[q]$, because the latter can perform $c$, i.e., 
$\mathcal{C}[q] \deriv{c} (\nil \para \nil) \para \nil$, 
while $\mathcal{C}[p]$ cannot. The reason for this
difference is that the process $\underline{a}.\underline{a}.c.\nil$ can react with a number of concurrently active components
equal to the length of the trace it can perform. Hence, a congruence semantics 
for parallel composition must distinguish $p$ and $q$ on the basis of their different degree of parallelism.
\fine
\end{example}




\section{Step Semantics}


Multi-CCS can be equipped with a step semantics, i.e., a semantics where each transition is 
labeled by a (multi-)set of 
sequences that concurrent subprocesses can perform at the same time. This equivalence was originally introduced 
over Petri nets \cite{NT84} and for a process algebra in \cite{Mil85}.

The  step operational semantics for Multi-CCS is given by the lts 
$({\mathcal P},{\mathcal B}, \derivas{ })$, where the states are the processes
in ${\mathcal P}$, ${\mathcal B} = {\mathcal M}_{fin}({\mathcal A})$ 
is the set of labels (ranged over by $M$),
and $\derivas{ } \subseteq {\mathcal P}\times{\mathcal B}\times{\mathcal P}$ is the minimal 
transition relation generated by the rules listed in Table~\ref{step-rules}. 
Note that rules (S-Pref$_1^s$), (S-Pref$_2^s$) and (S-Pref$_3^s$) assume that the transition in the premise 
is sequential, i.e., composed of one single sequence. Moreover, it can be proved that in rule (Sum$_1^s$),
the derivable label $M$ is always a singleton. Note also that rule (S-Com$^s$)
uses an additional auxiliary relation $\MSync$, defined in Table~\ref{Msync}, where $\oplus$ denotes multiset union.
The intuition behind the definition of rule (S-Com$^s$) and $\MSync$ is that, whenever two parallel processes 
$p$ and $q$ perform steps $M_1$ and $M_2$, then we can put all the sequences together -- $M_1 \oplus M_2$ --
and see if $\MSync(M_1 \oplus M_2, \overline{M})$ holds. The resulting $\overline{M}$ may be just 
$M_1 \oplus M_2$ (hence no synchronization takes place), according to axiom $\MSync(M, M)$, 
or the $M'$ we obtain from the application of the rule:
select two sequences $\sigma_1$ and $\sigma_2$ from $M_1 \oplus M_2$, synchronize them producing $\sigma$,
then recursively apply $\MSync$ to $M_1 \oplus M_2 \setminus \{\sigma_1, \sigma_2\} \cup \{\sigma\}$ to obtain $M'$.
This procedure of synchronizing sequences may go on until pairs of synchronizable sequences can be found, 
but may also stop at any moment due to the axiom $\MSync(M, M)$.



\begin{table}[t]
\hrulefill\\[-.6cm]

{\renewcommand{\arraystretch}{2.5}
\begin{center}
$\begin{array}{lcllllcl}
\mbox{(Pref$^s$)}  & \mu.p\derivas{\{\mu\}}p & \qquad &
\mbox{(S-Pref$_1^s$)}  & \bigfrac{p\derivas{\{\sigma\}}p'}{\underline{\tau}.p\derivas{\{\sigma\}}p'} & \\
\mbox{(S-pref$_2^s$)}  & \bigfrac{p\derivas{\{\tau\}}p'}{\underline{\alpha}.p\derivas{\{\alpha\}}p'} & \qquad &
\mbox{(S-Pref$_3^s$)}  & \bigfrac{p\derivas{\{\sigma\}}p' \quad \sigma \neq \tau}{\underline{\alpha}.p\derivas{\{\alpha\sigma\}}p'} & \\
\mbox{(Sum$_1^s$)}  & \bigfrac{p\derivas{M}p'}{p+q\derivas{M}p'} & \qquad &
%\mbox{(Sum$_2^s$)}  & \bigfrac{q\derivas{M}q'}{p+q\derivas{M}q'}  & \\
\mbox{(Con$^s$)} & \bigfrac{p\derivas{M}p'}{C \derivas{M}p'}& C \eqdef p &  \\
%\mbox{(Par$_2^s$)}  & \bigfrac{q\derivas{M}q'}{p\para q\derivas{M}p\para q'} &  \\
\mbox{(Par$_1^s$)}  & \bigfrac{p\derivas{M}p'}{p\para q\derivas{M}p'\para q} & \qquad &
\mbox{(Res$^s$)}  & \bigfrac{p\derivas{M}p'}{\restr{a}p\derivas{M}
\restr{a}p'} & \forall \sigma \in M\, \; a, \bar{a} \not\in n(\sigma)\;  &\\
\end{array}$

$\begin{array}{lcllcl}
\mbox{(S-Com$^s$)}  & \bigfrac{p\derivas{M_1} p'\spazio q\derivas{M_2}q'}{p
\para q \deriv{M}p'\para q'} & \MSync(M_1 \oplus M_2, M) &\\
\end{array}$

\hrulefill
\end{center}}
\caption{Step operational semantics (symmetric rules for (Sum$_1^s$) and (Par$_1^s$) omitted).}\label{step-rules}
\end{table}


\begin{table}[t]
\hrulefill\\[-.4cm]
\begin{center}

$\begin{array}{lcr}
\MSync(M, M) & \qquad \quad & \bigfrac{\Sync(\sigma_1, \sigma_2, \sigma) \; \;  \; \MSync(M\oplus\{\sigma\}, M')}
{\MSync(M \oplus \{\sigma_1, \sigma_2\}, M' )}  
\end{array}$

\end{center}
\hrulefill
\caption{Step synchronization relation}\label{Msync}
\end{table}



It is interesting to observe that these step operational rules do not make use of structural congruence $\equiv$. The same
operational effect of rule (Cong) is here ensured by relation $\MSync$ that allows for multiple synchronization of concurrently
active subprocesses. 

\begin{lemma}\label{step>int}
Let $p, q \in {\mathcal P}$ be processes. Then the following hold:
\begin{enumerate}
\item If $p \derivas{\{\sigma\}} q$, then $p \deriv{\sigma} q$.
\item If $p \deriv{\sigma} q$, then $\exists q' \equiv q$ such that $p \derivas{\{\sigma\}} q'$.
\end{enumerate}

\proof The proof of (1) is by induction on the proof of $p \derivas{\{\sigma\}} q$. In the case of rule (S-Com$^s$),
the actual proof of relation $\MSync$ tells in which order the parallel subcomponents are to be arranged by means of the structural congruence.

The proof of (2) is by induction on the proof of $p \deriv{\sigma} q$. We cannot prove the
stronger result $p \derivas{\{\sigma\}} q$, because of the free use of structural congruence; e.g., 
$\mu.(p \para (q \para r)) \deriv{\mu} ((p \para q) \para r)$ (due to (Cong)), 
while $\mu.(p \para (q \para r))$ cannot reach $((p \para q) \para r)$ in the step transition system.
\fine
\end{lemma}

We call {\em step equivalence}, denoted $\sim_{step}$, the bisimulation equivalence on the step transition 
system of Multi-CCS. The following obvious fact follows.

\begin{lemma}\label{cong>step}
Let $p, q \in {\mathcal P}$ be processes. If $p \equiv q$ then $p  \sim_{step} q$.

\proof One has to show that for each equation $p = q$ generating $\equiv$, we have that $p  \sim_{step} q$.
The only non-trivial case is for associativity $(p \para q) \para r \; = \; p \para (q \para r)$
where one has to prove the following auxiliary lemma: if $p  \derivas{M_1} p'$, $q  \derivas{M_2} q'$
and $r \derivas{M_3} r'$, then for all $M, M'$ such that $\Sync(M_1 \oplus M_2, M')$ and $\Sync(M' \oplus M_2, M)$,
there exists $N$ such that $\Sync(N_2 \oplus M_3, N)$ and $\Sync(M_1 \oplus N, M)$. The thesis of this lemma follows by observing that such $M$ can be obtained as $\Sync(M_1 \oplus M_2 \oplus M_3, M)$.
\fine
\end{lemma}

\begin{proposition}\label{step->int}
For any pair of processes $p, q \in \mathcal{P}$, if $p \sim_{step} q$ then $p \sim q$.

\proof
Let $R$ be a step bisimulation (i.e., a bisimulation over the step lts) such that $(p, q) \in R$.
Then, $\equiv \circ \, R \, \circ \equiv$ is an interleaving bisimulation containing the pair $(p, q)$ (where $\circ$ is the operation of relation composition). To prove this, we can resort to 
Lemma \ref{step>int} and Lemma \ref{cong>step}.
\fine
\end{proposition}



\begin{theorem}\label{cong-step}{\bf (Congruence)}
If $p \sim_{step} q$, then the following hold:
\begin{enumerate}
\item  $\mu.p \sim_{step} \mu.q \quad$ for all $\mu \in Act$,
\item $\underline{\mu}.p \sim_{step} \underline{\mu}.q \quad$ for all $\mu \in Act$
\item $p + r \sim_{step} q + r \quad $ for all  $r \in \mathcal{P}$,
\item $p \para r \sim_{step} q \para r \quad  $ for all  $r \in \mathcal{P}$,
\item $\restr{a}p \sim_{step} \restr{a}q \quad$ for all $a \in \mathcal{L}$.
\end{enumerate}

\proof
Standard.
\fine
\end{theorem}

Theorem \ref{cong-step}(4) and Proposition \ref{step->int} ensure that 
for any pair of processes $p, q \in \mathcal{P}$, if $p \sim_{step} q$ then,  for all  $r \in \mathcal{P}$, $p \para r \sim q \para r$.
One may wonder if the reverse hold, i.e., if for all  $r \in \mathcal{P}$, $p \para r \sim q \para r$ can we conclude that 
$p \sim_{step} q$? If this is the case, we can say that step equivalence is the coarsest congruence contained in
interleaving bisimulation. The answer to this question is negative, as the following examples show.

\begin{example}\label{no-tau-coarsest}
Take processes $p = \tau.\tau.\nil$ and $q = \tau \para \tau$.
It is not difficult to see that for all  $r \in \mathcal{P}$, $p \para r \sim q \para r$; 
however, $p \not\sim_{step} q$ 
as only the latter can perform the step $\{\tau, \tau\}$. 
\fine
\end{example}

\begin{example}\label{no-coarsest}
Take $p = (a \para a) + \underline{a}.a.\nil$
and $q = a.a.\nil + \underline{a}.a.\nil$. It is easy to see that $p \not\sim_{step} q$, even if
for all  $r \in \mathcal{P}$, $p \para r \sim q \para r$.
\fine
\end{example}


\section{Coarsest Congruence: Linear-step bisimilarity}

We may wonder which is the coarsest (i.e., as abstract as possible) 
congruence contained in interleaving bisimulation. We observed above that step bisimilarity, albeit a congruence,
is not the coarsest one, as it distinguishes processes that no parallel context can tell apart.
Exemple \ref{no-coarsest} suggests that the contents of a step $\{a, a\}$ 
which is testable by a parallel component is its possible atomic linearisation $aa$. Hence, it seems that
the coarsest congruence should not observe steps, rather linearisation of steps.
Example \ref{no-tau-coarsest} also suggests that the coarsest congruence should not be able
to observe multiple occurrences of concurrent $\tau$'s. This requirement is rather intuitive: 
if some activity is unobservable, 
we cannot say how many concurrent internal activities it is composed of.


\begin{table}[t]
{\renewcommand{\arraystretch}{2.8}
\hrulefill\\[-.4cm]

%{\renewcommand{\arraystretch}{2.8}
\begin{center}
$\begin{array}{lcllcl}
\mbox{(Pref$^{ls}$)}  & \mu.p\derivls{\mu}p & & \; \; \;
\mbox{(S-Pref$^{ls}_1$)}  & \bigfrac{p\derivls{\sigma}p'}{\underline{\tau}.p\derivls{\sigma}p'} & \\
\mbox{(S-Pref$^{ls}_2$)}  & \bigfrac{p\derivls{\tau}p'}{\underline{\alpha}.p\derivls{\alpha}p'} & & \; \; \;
\mbox{(S-Pref$^{ls}_3$)}  & \bigfrac{p\derivls{\sigma}p' \quad \sigma \neq \tau}{\underline{\alpha}.p\derivls{\alpha\sigma}p'} & \\
\mbox{(Sum$^{ls}_1$)}  & \bigfrac{p\derivls{\sigma}p'}{p+q\derivls{\sigma}p'} &  & \; \; \;
\mbox{(Cons$^{ls}$)} & \bigfrac{p\derivls{\sigma}p'}{C\derivls{\sigma}p'}& C \eqdef p  \\
\mbox{(Par$^{ls}_1$)}  & \bigfrac{p\derivls{\sigma}p'}{p\para q\derivls{\sigma}p'\para q} & & \; \; \;
\mbox{(S-Res$^{ls}$)}  & \bigfrac{p\derivls{\sigma}p'}{\restr{a}p\derivls{\sigma}
\restr{a}p'} & a, \bar{a} \not\in n(\sigma) \\
\end{array}$

$\begin{array}{lcllcl}
\mbox{(S-Com$^{ls}$)}  & \bigfrac{p\derivls{\sigma_1} p'\spazio q\derivls{\sigma_2}q'}{p
\para q \derivls{\sigma}p'\para q'} & AInt(\sigma_1,  \sigma_2, \sigma) \\ 
\end{array}$

\hrulefill
\end{center}}
\caption{Operational semantics (symmetric rules for (Sum$^{ls}_1$) and (Par$^{ls}_1$) omitted)}\label{linear-rules}
\end{table}


\begin{table}[t]
\hrulefill\\[-.5cm]
\begin{center}

$\begin{array}{lllll}
AInt(\tau, \sigma, \sigma) &  \; \; AInt(\sigma, \tau, \sigma) &
& \bigfrac{\sigma_1 \neq \tau \quad \sigma_2 \neq \tau \quad \Int(\sigma_1, \sigma_2, \sigma)}{AInt(\sigma_1, \sigma_2, \sigma)} \\
\end{array}$

\end{center}

\hrulefill%\\[-.4cm]
\caption{Abstract interleaving relation}\label{c-sync}
\end{table}


We define a third operational semantics for Multi-CCS. The  linear-step operational semantics
is given by the lts 
$({\mathcal P},{\mathcal A}, \derivls{ })$, where 
and $\derivls{ } \subseteq {\mathcal P}\times{\mathcal A}\times{\mathcal P}$ is the minimal 
transition relation generated by the rules listed in Table~\ref{linear-rules}. 
Note also that rule (S-Com$^{ls}$)
uses an additional auxiliary relation $AInt$, defined in Table~\ref{c-sync},
which is an extension of relation $\Int$ of Table \ref{b-sync} to consider also the case
in which the arguments may be $\tau$. Note that if $AInt(\sigma_1, \sigma_2, \sigma)$ holds,
then $\sigma$ is either $\tau$ or a sequence composed only of observable actions.
Observe that the only real difference between the interleaving semantics and the linear-step semantics is 
that $AInt(\sigma_1, \sigma_2, \sigma)$ allows for the free interleaving of $\sigma_1$ and $\sigma_2$,
while $\Sync(\sigma_1, \sigma_2, \sigma)$ holds only if at least one synchronization takes place.
Therefore, $a \para a \derivls{aa} \nil \para \nil$, while $a \para a \nonderiv{aa}$. This extra possibility
is also compensating the omission of rule (Cong), as already observed for the step semantics.


\begin{lemma}\label{int-lin-step}
Let $p, q \in {\mathcal P}$.  
If $p \deriv{\sigma} q$, then $\exists q' \equiv q$ such that $p \derivls{\sigma} q'$.
\fine
%\proof By induction on the proof of transition $p \deriv{\sigma} p'$.
\end{lemma}

Ordinary bisimulation equivalence on the linear-step transition system is
 called {\em linear step equivalence} and denoted with $\sim_{ls}$. Of course,
 the structural congruence $\equiv$ is a finer equivalence relation.
 
\begin{lemma}\label{cong>lin}
Let $p, q \in {\mathcal P}$ be processes. If $p \equiv q$ then $p  \sim_{ls} q$.
\fine
\end{lemma}

Linear step bisimilarity $\sim_{ls}$ is indeed finer than interleaving bisimilarity $\sim$ because 
of Lemma \ref{int-lin-step} and Lemma \ref{cong>lin}. The reverse implication does not hold:
for instance, $(a.\nil \para b.\nil) \sim a.b.\nil + b.a.\nil$ but the two are not linear-step bisimilar as only
$(a.\nil \para b.\nil)$
can perform the atomic sequence $ab$.

Moreover, linear-step bisimilarity $\sim_{ls}$ is coarser than step bisimilarity $\sim_{step}$. 
This can be proved by showing that if $p \derivls{\sigma} p'$ then %either $p \derivas{\{\sigma\}} p'$ or
$p \derivas{M} p'$, where $\sigma$ is the result of the interleaving (according to relation $AInt$)
of all the sequences in $M$. Hence, If $p \not\sim_{ls} q$ then there is a pair of reachable states $p'$ and $q'$
and a sequence $\sigma$ such that $p' \derivls{\sigma} p''$ while $q'$ cannot perform $\sigma$; as a consequence,
this means that $p' \derivas{M} p''$ with $\sigma$ as the result of the linearisation of the sequences in $M$, while
$q'$ cannot perform $M$, i.e., $p \not\sim_{step}q$.
%The proof of this fact can be done by induction on the proof of $p \derivls{\sigma} p'$.
The reverse implication does not hold: for instance, the two processes $p$ and $q$ discussed in Exercise \ref{no-coarsest}
are linear-step bisimilar, but not step bisimilar, as only $p$ can perform a transition labeled by $\{a, a\}$.
These observations justify the following:

\begin{proposition}\label{comparison}
$p \equiv q \Rightarrow p \sim_{step} q  \Rightarrow p \sim_{ls} q  \Rightarrow p \sim q$.
\fine
\end{proposition}

 
 
One can prove that 
$\sim_{ls}$ is a congruence for Multi-CCS. 


\begin{theorem}\label{cong-lin-step}
If $p \sim_{ls} q$, then the following hold:
\begin{enumerate}
\item  $\mu.p \sim_{ls} \mu.q \quad$ for all $\mu \in Act$,
\item $\underline{\mu}.p \sim_{ls} \underline{\mu}.q \quad$ for all $\mu \in Act$
\item $p + r \sim_{ls} q + r \quad $ for all  $r \in \mathcal{P}$,
\item $p \para r \sim_{ls} q \para r \quad  $ for all  $r \in \mathcal{P}$,
\item $\restr{a}p \sim_{ls} \restr{a}q \quad$ for all $a \in \mathcal{L}$.
\end{enumerate}

\proof
Standard.
\fine
\end{theorem}


We are now ready to state the coarsest congruence theorem. We first need an auxiliary lemma.

\begin{lemma}\label{tau-tau-int}
If $p \derivls{\tau} p'$, then $p \Deriv{\tau} p'$.

\proof
By induction on the proof of $p \derivls{\tau} p'$. The only non-trivial case is when $p = p_1 \para p_2$
and $p_1 \para p_2 \derivls{\tau}p'_1 \para p'_2$ because 
$p_1 \derivls{\sigma_1} p'_1$, $p_2 \derivls{\sigma_2} p'_2$ and $AInt(\sigma_1,  \sigma_2, \tau)$.
By definition of relation $AInt$, the result may be $\tau$ only if either $\sigma_1 = \sigma_2 = \tau$
or $\sigma_1 \neq \tau \neq \sigma_2$ and $\Int(\sigma_1, \sigma_2, \tau)$. In the former case,
by induction we know that $p_1 \Deriv{\tau} p'_1$ and $p_2 \Deriv{\tau} p'_2$, hence by rule 
(Par$_1$) and (Par$_2$) we have $p_1 \para p_2 \Deriv{\tau}p'_1 \para p_2 \Deriv{\tau}p'_1 \para p'_2$, as required.
In the latter case,  $\Int(\sigma_1, \sigma_2, \tau)$ can be derived only if at least one synchronization takes place,
hence also $\Sync(\sigma_1, \sigma_2, \tau)$ holds. Now, by rule (S-Com) we have 
$p_1 \para p_2 \deriv{\tau} p'_1 \para p'_2$, as required.
\fine
\end{lemma}

\begin{theorem}\label{coarsest-cong-step} {\bf (Coarsest congruence)}
Assume that $fn(p) \cup fn(q) \neq \mathcal{L} \cup \overline{\mathcal{L}}$. Then, $p \sim_{ls} q$ if and only if,  
for all  $r \in \mathcal{P}$, $p \para r \sim q \para r$.

\proof
The implication from left to right is due to the fact that $\sim_{ls}$ is a congruence for parallel composition 
(by Theorem \ref{cong-lin-step}(4))
and is also finer than interleaving bisimilarity (by Proposition \ref{comparison}). 
The implication from right to left is proved by contradiction:
assuming that $p \not\sim_{ls} q$, we prove that there exists $r$ such that $p \para r \not\sim q \para r$.
As $p \not\sim_{ls} q$, there exist $p_i, q_i, i = 1, \ldots n (n \geq 0)$ such that 
$p \derivls{\sigma_1} p_1 \derivls{\sigma_2} p_2 \ldots \derivls{\sigma_n} p_n$ and 
$q \derivls{\sigma_1} q_1 \derivls{\sigma_2} q_2 \ldots \derivls{\sigma_n} q_n$, but $p_n \derivls{\gamma} p_{n+1}$ 
and $q_n \nonderivls{\gamma}$. If $\gamma = \tau$, then by Lemma \ref{tau-tau-int}, we have that
$p_n \Deriv{\tau} p_{n+1}$, while by Lemma \ref{int-lin-step} we have that $q_n \nonderiv{\tau}$.
As a consequence, $p \not\sim q$ and so for, e.g., $r = \nil$, we have that  $p \para r \not \sim q \para r$.
If instead $\gamma = \alpha_1 \ldots \alpha_n$ for $n > 0$, let $r = \underline{\overline{\alpha_1}}. \ldots .\underline{\overline{\alpha_n}}.d.\nil$,
where $d$ is an action not occurring in $fn(p) \cup fn(q)$.
It is the case that $p_n \para r \deriv{d} p_{n+1} \para \nil$, while $q_n \para r \nonderiv{d}$. Hence, 
$p \para r \not\sim q \para r$.
\fine
\end{theorem}


\section{Axiomatization}

With the help of the auxiliary operators of left-merge $- \lfloor -$ and communication-merge 
$- \Vert -$ of ACP \cite{BK84}, we can provide
a finite, sound and complete axiomatization to linear-step bisimilarity over finite Multi-CCS processes.
The operational SOS rules for these two operators are as follows:

\begin{center}
$\begin{array}{lcllcl}
\mbox{(Left)}  & \bigfrac{p\derivls{\sigma}p'}{p\lfloor q\derivls{\sigma}p' \para q} & \qquad & 
\mbox{(S-Merge)}  & \bigfrac{p\derivls{\sigma_1} p'\spazio q\derivls{\sigma_2}q'}{p \Vert q \derivls{\sigma}p' \para q'}  & AInt(\sigma_1,  \sigma_2, \sigma) \\
\end{array}$
\end{center}


It is possible to prove that $\sim_{ls}$ is preserved by the left merge and the communication merge. 
So, it is a congruence not only for the operators of finite Multi-CCS,
but also for these auxiliary operators. Hence, it can be axiomatized.

The axioms for the operators of choice, strong prefixing and restriction are reported in
Table \ref{axiom-multi-+}. These reflect the intuition that the normal forms for finite Multi-CCS are terms of the form
$\Sigma_{i\in I}\sigma_i.p_i$, where each $p_i$ is a normal form and $\sigma_i$ is either $\tau$ or a sequence of visible actions;
to be more precise, with $\sigma_i.p_i$ we actually mean $\tau.p_i$ if $\sigma = \tau$, or 
$\underline{\alpha}_1. \ldots .\underline{\alpha}_{n-1}.\alpha_n.p_i$ if $\sigma = \alpha_1 \ldots \alpha_n$.


\begin{table}[t]
{\renewcommand{\arraystretch}{1.2}
\hrulefill\\[-.6cm]
\normalsize{

%{\renewcommand{\arraystretch}{2.8}
\begin{center}
$\begin{array}{lrlllllllll}
{\bf A1}  &\; \;  x + (y + z) & = & (x + y) + z & \; \; {\bf A2}  &\; \;  x + y & = & y + x\\
{\bf A3}  &\; \;  x + \nil & = & x & \; \; {\bf A4}  &\; \;  x + x & = & x\\
\end{array}$

\hrulefill

$\begin{array}{lrlllrlllllcl}
{\bf P1}  &\; \;  \underline{\mu}.\nil & = & \nil & \quad \;\;{\bf P2}  &\; \;  \underline{\tau}.x & = &  x\\
{\bf P3}  &\; \;  \underline{\mu}.\tau.x & = & \mu.x & \quad \; \; {\bf P4}  &\; \;  \underline{\mu}.(x + y) & = & \underline{\mu}.x + \underline{\mu}.y\\
\end{array}$

\hrulefill

$\begin{array}{llrcl}
{\bf R1} &\; \; &\; \;  \restr{a}\nil & = & \nil\\
{\bf R2} &\; \;  \mbox{if $\mu \not\in \{a, \bar{a}\}$} &\; \; \restr{a}\mu.x & = & \mu.\restr{a}x\\
{\bf R3} &\; \;  \mbox{if $\mu \in \{a, \bar{a}\}$} &\; \; \restr{a}\mu.x & = & \nil \\
{\bf R4} &\; \;  \mbox{if $\mu \not\in \{a, \bar{a}\}$} &\; \; \restr{a}\underline{\mu}.x & = & \underline{\mu}.\restr{a}x\\
{\bf R5} &\; \;  \mbox{if $\mu \in \{a, \bar{a}\}$} &\; \; \restr{a}\underline{\mu}.x & = & \nil \\
{\bf R6} &\; \; &\; \; \restr{a}(x + y) & = & \restr{a}x + \restr{a}y\\
\end{array}$

\hrulefill

\end{center}}
}
\caption{Axioms for choice, strong prefixing and restriction.}\label{axiom-multi-+}
\end{table}


The axiomatization of the parallel composition operator is given by axiom {\bf Par} of 
Table \ref{axiom-multi-aux}, together with the axioms
{\bf L1-L4} for left merge and {\bf C1-C9} for communication merge. Let  $\mathcal{MSB}$ be the set of axioms reported in Table \ref{axiom-multi-+} and Table \ref{axiom-multi-aux}.
It is possible to prove that the equational theory $\mathcal{MSB}$ is a sound and complete {\em finite} axiomatization 
of $\sim_{ls}$.

\begin{theorem}{\bf (Soundness)}\label{soundness}
If $\mathcal{MSB} \vdash p = q$, then $p \sim_{ls} q$.

\proof Because all the axioms in Tables \ref{axiom-multi-+} 
and \ref{axiom-multi-aux} are sound for $\sim_{ls}$. 
\fine
\end{theorem}

About completeness, we first prove the statement for normal forms. 

\begin{proposition}\label{compl-nf}
Let $p$, $q$ be normal forms. If $p \sim_{ls} q$, then $\mathcal{MSB} \vdash p = q$.

\proof By induction on the sum of depths\footnote{The depth of $p$ is the maximal number of nested 
prefixes in its syntax.} of $p$ and $q$.
\fine
\end{proposition}

In order to prove completeness for any pair of equivalent processes, we first need to
show that any process can be reduced to normal form.

\begin{lemma}\label{aux-lemma}
If $p$ and $q$ are normal forms, then the following hold:
\begin{itemize}
\item
there exists a normal form $r$ such that $\mathcal{MSB} \vdash \underline{\mu}.p =  r$.
\item 
there exists a normal form $r$ such that $\mathcal{MSB} \vdash \restr{a}p =  r$.
\item 
there exists a normal form $r$ such that $\mathcal{MSB} \vdash p \para q = r$.
\end{itemize}

\proof By induction on the sum of depths of $p$ and $q$.
\fine
\end{lemma}

\begin{proposition}\label{prop-reduc}
For any process $p$ there exists a normal form $q$ such that $\mathcal{MSB} \vdash p = q$.

\proof By structural induction on $p$, using Lemma \ref{aux-lemma}.
\fine
\end{proposition}

\begin{theorem}{\bf (Completeness)}
If $p \sim_{ls} q$, then $\mathcal{MSB} \vdash p = q$.

\proof First, by Proposition \ref{prop-reduc}, we have normal forms $p'$ and $q'$ such that  
$\mathcal{MSB} \vdash p = p'$ and $\mathcal{MSB} \vdash q = q'$. By Theorem \ref{soundness}, 
$p \sim_{ls} p'$ and $q \sim_{ls} q'$, hence, by transitivity, also $p' \sim_{ls} q'$.
By Proposition \ref{compl-nf}, we get that $\mathcal{MSB} \vdash p' = q'$ and so, by transitivity,
also $\mathcal{MSB} \vdash p = q$.
\fine
\end{theorem}




\begin{table}[t]
{\renewcommand{\arraystretch}{1.2}
\hrulefill\\[-.6cm]
\normalsize{

%{\renewcommand{\arraystretch}{2.8}
\begin{center}

$\begin{array}{llllll}
{\bf Par} &\; \;   x \para y \; = \;  x \lfloor y \; + \; y \lfloor x \; + \; x \Vert y\\
\end{array}$


\hrulefill

$\begin{array}{llrclllrrl}
{\bf L1} &\; \; &\; \;  \nil \lfloor y & = & \nil & \quad
{\bf L2} &\; \; &\; \; (\mu.x) \lfloor y & = & \mu.(x \para y)\\
{\bf L3} &\; \; &\; \; (\underline{\mu}.x) \lfloor y & = & \underline{\mu}.(x \lfloor y) & \quad
{\bf L4} &\; \; &\; \; (x + y) \lfloor z & = & x \lfloor z \; + \; y \lfloor z\\
\end{array}$

\hrulefill

$\begin{array}{llrclllllrrlllllll}
{\bf C1} &\; \; &\; \;  x \Vert y & = & y \Vert x & \quad
{\bf C2} &\; \; &\; \;  \nil \Vert y & = & \nil & \quad 
{\bf C3} &\; \; &\; \; (x + y) \Vert z & = & x \Vert z \; + \; y \Vert z\\
\end{array}$
$\begin{array}{llrcl}
{\bf C4} &\; \;  \mbox{if $\mu_1 = \overline{\mu_2}$} &\; \; (\mu_1.x) \Vert (\mu_2.y) & = & \tau.(x \para y) + \underline{\mu}_1.(\mu_2.y \lfloor x) + \underline{\mu}_2.(\mu_1.x \lfloor y)\\
{\bf C5} &\; \;  \mbox{if $\mu_1 \neq \overline{\mu_2}$} &\; \; (\mu_1.x) \Vert (\mu_2.y) & = & \underline{\mu}_1.(\mu_2.y \lfloor x) + \underline{\mu}_2.(\mu_1.x \lfloor y)\\
{\bf C6} &\; \;  \mbox{if $\mu_1 = \overline{\mu_2}$} &\; \; (\underline{\mu}_1.x) \Vert (\mu_2.y) & = & (x \lfloor y) + \underline{\mu}_1.(x \Vert \mu_2.y) + \underline{\mu}_2.(\underline{\mu}_1.x \lfloor y)\\
{\bf C7} &\; \;  \mbox{if $\mu_1 \neq \overline{\mu_2}$} &\; \; (\underline{\mu}_1.x) \Vert (\mu_2.y) & = &\underline{\mu}_1.(x \Vert \mu_2.y) + \underline{\mu}_2.(\underline{\mu}_1.x \lfloor y)\\
{\bf C8} &\; \;  \mbox{if $\mu_1 = \overline{\mu_2}$} &\; \; (\underline{\mu}_1.x) \Vert (\underline{\mu}_2.y) & = & (x \Vert y) + 
\underline{\mu}_1.(x \Vert \underline{\mu}_2.y) + \underline{\mu}_2.(\underline{\mu}_1.x \Vert y)\\
{\bf C9} &\; \;  \mbox{if $\mu_1 \neq \overline{\mu_2}$} &\; \; (\underline{\mu}_1.x) \Vert (\underline{\mu}_2.y) & = &\underline{\mu}_1.(x \Vert \underline{\mu}_2.y) + \underline{\mu}_2.(\underline{\mu}_1.x \Vert y)\\

\end{array}$

\hrulefill

\end{center}}
}
\caption{Axioms for the auxiliary operators}\label{axiom-multi-aux}
\end{table}



\section{Conclusion}

We have studied the problem of finding the correct semantics for atomicity. To this aim, we
have chosen a simple operator for atomicity (strong prefixing) and a discipline of interaction
among atomic behaviors (relation $\Sync$), yielding Multi-CCS. 
Then, we have identified the intuitively correct model for Multi-CCS:  the interleaving model 
identifies all the actual (i.e., physical) atomic transitions of processes. However,
we have seen that interleaving bisimulation is not a congruence for parallel composition.
Hence, it turns out to be necessary to enrich the model with additional transitions. The first attempt
has been to model steps of interleaving transitions; the resulting step bisimilarity is a congruence,
but not the coarsest congruence. To this aim, a linear-step transitions system has been defined 
whose transitions are labeled with linearisation of steps. Linear-step bisimilarity, which is a new behavioural semantics,
turns out to be the coarsest congruence contained in interleaving bisimilarity.

Linear-step bisimilarity is interesting also because, although finer than interleaving bisimilarity, it is still
equipped with an axiomatization in interleaving style, i .e., with an expansion theorem. 
This is not the case, for instance, for step
bisimilarity, for which no axiomatization is available. 

We conjecture that linear step bisimilarity can be a very appropriate bisimu-lation-based 
behavioural equivalence also for calculi with joint input, such as those in \cite{FG96,LV10}, as Multi-CCS subsumes such
a kind of synchronization pattern. For instance, the joint input construct $[a_1, \ldots a_{n-1}, a_n].p$ of \cite{LV10}
can be easily encoded in Multi-CCS as $\underline{a_1}. \ldots \underline{a_{n-1}}.a_n.p$.


\section*{Acknowledgment}

The first author would like to thank Massimo Morara for useful comments and suggestions.



\begin{thebibliography}{11}

\bibitem{BK84}
J.A.~Bergstra, J.W.~Klop,
\newblock ``Process Algebra for Synchronous Communication'',
\newblock {\em Information and Control} 60:109-137, 1984.


\bibitem{CHP71}
P. Courtois, F. Heymans, D. Parnas,
\newblock ``Concurrent control with Readers and Writers'',
\newblock {\em Communications of the ACM} 14(10):667-668, 1971.

\bibitem{Dij71}
E.W. Dijkstra, ``Hierarchical ordering of sequential processes'',
\newblock {\em Acta Informatica} 1(2):115-138, 1971.

\bibitem{FG96}
C. Fournet, G. Gonthier,
\newblock `The reflexive CHAM and the join-calculus'', in 
Procs POPL'96, pages 372� 385, ACM Press, 1996.

\bibitem{GM90} 
R. Gorrieri, U. Montanari, ``Towards Hierarchical Specification of Systems:
A Proof System for Strong Prefixing'',
{\em Int. Journal of Foundations of Computer Science},
1(3):277-293, 1990.

\bibitem{GMM}
R.~Gorrieri, S.~Marchetti, U.~Montanari,
\newblock ``A$^2$CCS: Atomic Actions for CCS'',
\newblock {\em Theoretical Computer Science} 72(2-3): 203-223, 1990.

\bibitem{GV10a}
R. Gorrieri, C. Versari,
``A Process Calculus for Expressing Finite Place/Transition Petri Nets'',
in Procs. EXPRESS'10, EPTCS, 2010. DOI 10.4204/EPTCS.41.6,
available at arXiv:1011.6433v1.

\bibitem{Hoare}
C.A.R. Hoare,
\newblock {\it Communicating Sequential Processes}
\newblock Prentice-Hall, 1985.

\bibitem{LR81}
D.J. Lehmann. M.l O. Rabin.,
\newblock ``On the advantages of free choice: 
A symmetric and fully distributed solution to the dining philosophers problem'',
\newblock In Procs. POPL'81, pages 133-138,  ACM Press, 1981. 

\bibitem{LV10}
C.~Laneve, A.~Vitale,
\newblock ``The Expressive Power of Synchronizations'',
\newblock  in Procs LICS'10, pages 382-391, IEEE-CS Press, 2010.

\bibitem{Mil85}
G. J. Milne,
\newblock  ``Circal and the Representation of Communication, Concurrency, and Time'',
\newblock {\em ACM Trans. Program. Lang. Syst.} 7(2): 270-298, 1985.

\bibitem{Mil89} R. Milner. {\it Communication and Concurrency},
Prentice-Hall, 1989.

\bibitem{Mil89b} R. Milner.
\newblock A Complete Axiomatisation for Observational Congruence of Finite-State Behaviors.
\newblock {\em Inf. Comput.}  81(2): 227-247, 1989.

\bibitem{Mil99} R. Milner. {\it Communicating and mobile systems: the
$\pi$-calculus},
Cambridge University Press, 1999.

\bibitem{NT84}
M. Nielsen and P. S. Thiagarajan,
\newblock  ``Degrees of Non-Determinism and Concurrency: A Petri Net View'',
\newblock in Procs. of the Fourth Conference on Foundations of Software Technology and Theoretical Computer Science (FSTTCS'84),
Lecture Notes in Computer Science vol. 181, Springer, Berlin, pp. 89-117, 1984.

\bibitem{Pat71}
S. Patil
\newblock ``Limitations and Capabilities of Dijkstra's Semaphore 
Primitives for Coordination Among Processes'',
\newblock Computation Structures Group Memo 57, Project MAC, MIT, 1971.

\bibitem{Pet81}
J.L. Peterson,
\newblock {\em Petri Net Theory and the Modeling of Systems}, Prentice-Hall, 1981.


\bibitem{VBG09}
C. Versari, N. Busi, R. Gorrieri.
\newblock An expressiveness study of priority in process calculi. 
\newblock {\em Mathematical Structures in Computer Science} 19(6): 1161-1189, 2009.

\end{thebibliography}



\end{document}



