

\begin{definition}
  Let $\sigma$ be a permutation of names. $\beta,\sigma$ equivalence is the smallest binary relation on processes that satisfies the laws in table \ref{betaequivalence}. In a process $P$ we can assume that all bound names are different. 
  \begin{table}
    \begin{tabular}{ll}
      \multicolumn{2}{l}{\line(1,0){415}}\\\\
	  $\inferrule*[left=BetaOut]{
	      P\equiv_{\beta}Q
	    \\
	      \sigma
	  }{
	      \overline{x}y.P\equiv_{\beta}\overline{z}w.Q
	    \\
	      \sigma (xz) (yw)
	  }$
	&
	  $\inferrule*[left=BetaTau]{
	      P\equiv_{\beta}Q
	    \\
	      \sigma
	  }{
	      \tau.P\equiv_{\beta}\tau.Q
	    \\
	      \sigma
	  }$
	\\\\
	  $\inferrule*[left=BetaInp]{
	      P\equiv_{\beta}Q
	    \\
	      \sigma
	    \\
	      y \sigma y
	  }{
	      x(y).P\equiv_{\beta}z(y).Q
	    \\
	      \sigma (xz)
	  }$
	&
	  $\inferrule*[left=BetaRes]{
	      P\equiv_{\beta}Q
	    \\
	      \sigma
	    \\
	      x \sigma x
	  }{
	    (\nu x)P\equiv_{\beta}(\nu x)Q
	  }$
      \\
    \end{tabular}
    \\
    \begin{tabular}{lll}
      \\
	  $\inferrule*[left=BetaIde]{
	  }{
	      A(\tilde{x})\equiv_{\beta}A(\tilde{x})
	    \\
	      id
	  }$
	&
	  $\inferrule*[left=BetaZero]{
	  }{
	      0\equiv_{\beta}0
	    \\
	      id
	  }$
	&

      \\
    \end{tabular}
    \\
    \begin{tabular}{l}
      \\
	  $\inferrule*[left=BetaPar]{
	      P_{1}\equiv_{\beta}Q_{1}
	    \\
	      \sigma_{1}
	    \\
	      P_{2}\equiv_{\beta}Q_{2}
	    \\
	      \sigma_{2}
	  }{
	      P_{1}|P_{2}\equiv_{\beta}Q_{1}|Q_{2}
	    \\
	      \sigma_{1} \cdot \sigma_{2}
	  }$
      \\\\
	  $\inferrule*[left=BetaSum]{
	      P_{1}\equiv_{\beta}Q_{1}
	    \\
	      \sigma_{1}
	    \\
	      P_{2}\equiv_{\beta}Q_{2}
	    \\
	      \sigma_{2}
	  }{
	      P_{1}+P_{2}\equiv_{\beta}Q_{1}+Q_{2}
	    \\
	      \sigma_{1} \cdot \sigma_{2}
	  }$
      \\
    \end{tabular}
    \\
    \begin{tabular}{l}
      \\
	  $\inferrule*[left=BetaRes1]{
	      P\equiv_{\beta}Q
	    \\
	      \sigma
	    \\
	      ((xy)\in \sigma) \vee ((xx)(yy)\in \sigma)
	  }{
	      (\nu x)P\equiv_{\beta}(\nu y)Q
	    \\
	      \sigma - (xy)
	  }$
      \\\\
	  $\inferrule*[left=BetaInp1]{
	      P\equiv_{\beta}Q
	    \\
	      \sigma
	    \\
	      ((xw)\in \sigma) \vee ((xx)(ww)\in \sigma)
	  }{
	      x(y).P\equiv_{\beta}z(w).Q
	    \\
	      (\sigma (xz))- (yw)
	  }$
    \\\\\multicolumn{1}{l}{\line(1,0){415}}
    \end{tabular}
    \caption{$\beta$ equivalence laws}
    \label{betaequivalence}
  \end{table}
\end{definition}


\begin{definition}
  Processes $P$ and $Q$ are \emph{$\alpha$ convertible} or \emph{$\alpha$ equivalent} if they are $\beta$ equivalent with respect to identity.
\end{definition}

\begin{lemma}
  If $P\xrightarrow{\lambda}_{2} P^{'}$ then $P\equiv_{\alpha} Q$, $Q\stackrel{\lambda}{\twoheadrightarrow}_{2} Q^{'}$ and $P^{'}\equiv_{\alpha} Q^{'}$
\end{lemma}


\begin{lemma}
  If $P\equiv_{\beta}^{\sigma} Q$, $Q\stackrel{\lambda}{\twoheadrightarrow}_{2} Q^{'}$ then $P\stackrel{\lambda\sigma}{\twoheadrightarrow}_{2} P^{'}$, $P^{'}\equiv_{\beta}^{\sigma^{'}} Q^{'}$ and $\sigma^{'}\subseteq \sigma$.
\end{lemma}

\begin{lemma}
  If $P\equiv_{\alpha} Q$, $Q\stackrel{\lambda}{\twoheadrightarrow}_{2} Q^{'}$ then $P\stackrel{\lambda}{\twoheadrightarrow}_{2} P^{'}$ and $P^{'}\equiv_{\alpha} Q^{'}$.
\end{lemma}

in questo modo si puo' dimostrare che una bisimulazione e' conservata da un operatore senza trattare i casi dati dalla regola per l'alfa conversione
