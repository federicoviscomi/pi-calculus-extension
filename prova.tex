

\begin{definition}
  Let $\sigma$ be a permutation of names. $\beta,\sigma$ equivalence is the smallest binary relation on processes that satisfies the laws in table \ref{betaequivalence}. In a process $P$ we can assume that all bound names are different. 
  \begin{table}
    \begin{tabular}{ll}
      \multicolumn{2}{l}{\line(1,0){415}}\\\\
	  $\inferrule*[left=BetaOut]{
	      P\equiv_{\beta}Q
	    \\
	      \sigma
	  }{
	      \overline{x}y.P\equiv_{\beta}\overline{z}w.Q
	    \\
	      \sigma \{x\mapsto z\} \{y\mapsto w\}
	  }$
	&
	  $\inferrule*[left=BetaTau]{
	      P\equiv_{\beta}Q
	    \\
	      \sigma
	  }{
	      \tau.P\equiv_{\beta}\tau.Q
	    \\
	      \sigma
	  }$
	\\\\
	  $\inferrule*[left=BetaInp]{
	      P\equiv_{\beta}Q
	    \\
	      \sigma
	    \\
	      y \sigma y
	  }{
	      x(y).P\equiv_{\beta}z(y).Q
	    \\
	      \sigma \{x\mapsto z\}
	  }$
	&
	  $\inferrule*[left=BetaRes]{
	      P\equiv_{\beta}Q
	    \\
	      \sigma
	    \\
	      x \sigma x
	  }{
	      (\nu x)P\equiv_{\beta}(\nu x)Q
	    \\
	      \sigma
	  }$
      \\
    \end{tabular}
    \\
    \begin{tabular}{lll}
      \\
	  $\inferrule*[left=BetaIde]{
	  }{
	      A(\tilde{x})\equiv_{\beta}A(\tilde{x})
	    \\
	      id
	  }$
	&
	  $\inferrule*[left=BetaZero]{
	  }{
	      0\equiv_{\beta}0
	    \\
	      id
	  }$
	&

      \\
    \end{tabular}
    \\
    \begin{tabular}{l}
      \\
	  $\inferrule*[left=BetaPar]{
	      P_{1}\equiv_{\beta}Q_{1}
	    \\
	      \sigma_{1}
	    \\
	      P_{2}\equiv_{\beta}Q_{2}
	    \\
	      \sigma_{2}
	  }{
	      P_{1}|P_{2}\equiv_{\beta}Q_{1}|Q_{2}
	    \\
	      \sigma_{1} \cdot \sigma_{2}
	  }$
      \\\\
	  $\inferrule*[left=BetaSum]{
	      P_{1}\equiv_{\beta}Q_{1}
	    \\
	      \sigma_{1}
	    \\
	      P_{2}\equiv_{\beta}Q_{2}
	    \\
	      \sigma_{2}
	  }{
	      P_{1}+P_{2}\equiv_{\beta}Q_{1}+Q_{2}
	    \\
	      \sigma_{1} \cdot \sigma_{2}
	  }$
      \\
    \end{tabular}
    \\
    \begin{tabular}{l}
      \\
	  $\inferrule*[left=BetaRes1]{
	      P\equiv_{\beta}Q
	    \\
	      \sigma
	    \\
	      (x\sigma y) \vee (x\sigma x \wedge y\sigma y)
	    \\
	      x \neq y
	  }{
	      (\nu x)P\equiv_{\beta}(\nu y)Q
	    \\
	      \sigma - \{x\mapsto y\}
	  }$
      \\\\
	  $\inferrule*[left=BetaInp1]{
	      P\equiv_{\beta}Q
	    \\
	      \sigma
	    \\
	      (y\sigma w) \vee (y \sigma y \wedge w \sigma w)
	    \\
	      y \neq w
	  }{
	      x(y).P\equiv_{\beta}z(w).Q
	    \\
	      (\sigma \{x\mapsto z\})- \{y\mapsto w\}
	  }$
    \\\\\multicolumn{1}{l}{\line(1,0){415}}
    \end{tabular}
    \caption{$\beta$ equivalence laws}
    \label{betaequivalence}
  \end{table}
\end{definition}

\begin{definition}
  Processes $P$ and $Q$ are \emph{$\alpha$ convertible} or \emph{$\alpha$ equivalent} if they are $\beta$ equivalent with respect to identity.
\end{definition}


\begin{lemma}
  $\alpha$ equivalence is reflexive.
  \begin{proof}
    We prove that $P\equiv_{\beta}^{id} P$. The proof is a structural induction on $P$:
	\begin{center}
	  \begin{tabular}{ll}
	    $\inferrule*[left=BetaOut]{
		P\equiv_{\beta}P
	      \\
		id
	    }{
		\overline{x}y.P\equiv_{\beta}\overline{x}y.Q
	      \\
		id \{x\mapsto x\} \{y\mapsto y\}
	    }$
	  &
	    $\inferrule*[left=BetaTau]{
	      P\equiv_{\beta}Q
	    \\
	      id
	    }{
		\tau.P\equiv_{\beta}\tau.Q
	      \\
		id
	    }$
	\\\\
	  $\inferrule*[left=BetaInp]{
	      P\equiv_{\beta}Q
	    \\
	      id
	    \\
	      y id y
	  }{
	      x(y).P\equiv_{\beta}x(y).Q
	    \\
	      id \{x\mapsto x\}
	  }$
	&
	  $\inferrule*[left=BetaRes]{
	      P\equiv_{\beta}Q
	    \\
	      id
	    \\
	      x id x
	  }{
	      (\nu x)P\equiv_{\beta}(\nu x)Q
	    \\
	      id \{x\mapsto x\}
	  }$
      \\\\
	  $\inferrule*[left=BetaIde]{
	  }{
	      A(\tilde{x})\equiv_{\beta}A(\tilde{x})
	    \\
	      id
	  }$
	&
	  $\inferrule*[left=BetaZero]{
	  }{
	      0\equiv_{\beta}0
	    \\
	      id
	  }$
      \\
    \end{tabular}
    \\
    \begin{tabular}{l}
      \\
	  $\inferrule*[left=BetaPar]{
	      P_{1}\equiv_{\beta}Q_{1}
	    \\
	      id 
	    \\
	      P_{2}\equiv_{\beta}Q_{2}
	    \\
	      id
	  }{
	      P_{1}|P_{2}\equiv_{\beta}Q_{1}|Q_{2}
	    \\
	      id \cdot id
	  }$
      \\\\
	  $\inferrule*[left=BetaSum]{
	      P_{1}\equiv_{\beta}Q_{1}
	    \\
	      id
	    \\
	      P_{2}\equiv_{\beta}Q_{2}
	    \\
	      id
	  }{
	      P_{1}+P_{2}\equiv_{\beta}Q_{1}+Q_{2}
	    \\
	      id \cdot id
	  }$
	\\
	\end{tabular}
      \end{center}
  \end{proof}
\end{lemma}

\begin{lemma}
  $\alpha$ equivalence is symmetric.
  \begin{proof}
    We prove that $P\equiv_{\beta}^{\sigma}Q$ imply $Q\equiv_{\beta}^{\sigma^{-1}}P$.
  \end{proof}
\end{lemma}


\begin{lemma}
  If $P\equiv_{\beta}^{\sigma_{1}} Q$ and $Q\equiv_{\beta}^{\sigma_{2}} R$ then $P \equiv_{\beta}^{\sigma_{3}} R$ and $\sigma_{3}= \sigma_{1} \cdot \sigma_{2}$.
  \begin{proof}
    We have to prove that $P\equiv_{\beta}^{\sigma_{3}} R$. We go by induction on the derivation of both $P\equiv_{\beta}^{\sigma_{1}} Q$ and $Q\equiv_{\beta}^{\sigma_{2}} R$:
    \begin{description}
      \item[$(BetaIde, BetaIde)$]:
	In this case $P$, $Q$ and $R$ are some identifier $A(\tilde{x})$ and $\sigma_{1}=\sigma_{2}=\sigma_{3}=id$.
      \item[$(BetaZero, BetaZero)$]
	in this case $P$, $Q$ and $R$ are $0$ and $\sigma_{1}=\sigma_{2}=\sigma_{3}=id$.
      \item[$(BetaTau, BetaTau)$]:
	\begin{center}
	  \begin{tabular}{ll}
	      $\inferrule*[left=BetaTau]{
		  P \equiv_{\beta} Q
		\\
		   \sigma_{1}
	      }{
		  \tau.P \equiv_{\beta} \tau.Q
		\\
		  \sigma_{1}
	      }$	      
	    &
	      $\inferrule*[left=BetaTau]{
		  Q \equiv_{\beta} R
		\\
		  \sigma_{2}
	      }{
		  \tau.Q \equiv_{\beta} \tau.R
		\\
		  \sigma_{2}
	      }$
	  \end{tabular}
	\end{center}
	for inductive hypothesis $P\equiv_{\beta}^{\sigma_{3}} R$ and $\sigma_{3}\subseteq (\sigma_{1}\cup \sigma_{2})$. For rule $BetaTau$: $\tau.P\equiv_{\beta}^{\sigma_{3}} \tau.R$
      \item[$(BetaSum, BetaSum)$]
      \item[$(BetaPar, BetaPar)$]
      \item[$(BetaRes, BetaRes)$]
      \item[$(BetaInp, BetaInp)$]
      \item[$(BetaRes1, BetaRes1)(1)$]:
	\begin{center}
	  \begin{tabular}{l}
	      $\inferrule*[left=BetaRes1]{
		  P \equiv_{\beta} Q
		\\
		  \sigma_{1}
		\\
		  ((xy)\in \sigma_{1})) \vee (x\sigma_{1}x \wedge y\sigma_{1}y)
		\\
		  x \neq y
	      }{
		  (\nu x)P \equiv_{\beta} (\nu y)Q
		\\
		  \sigma_{1} - (xy)
	      }$	      
	    \\\\
	      $\inferrule*[left=BetaRes1]{
		  Q \equiv_{\beta}^{\sigma_{2}} R
		\\
		  \sigma_{2}
		\\
		  ((yz)\in \sigma_{2})) \vee (y\sigma_{2}y \wedge z\sigma_{2}z)
		\\
		  y \neq z
	      }{
		  (\nu y)Q \equiv_{\beta} (\nu z)R
		\\
		  \sigma_{2} - (yz)
	      }$
	  \end{tabular}
	\end{center}
	for inductive hypothesis $P\equiv_{\beta}^{\sigma_{3}} R$ and $\sigma_{3} \subseteq (\sigma_{1}\cup \sigma_{2})$
      \item[$(BetaRes1, BetaRes)$]:
	\begin{center}
	  \begin{tabular}{ll}
	      $\inferrule*[left=BetaRes1]{
		  P \equiv_{\beta}^{id} Q
		\\
		  \{x\mapsto x\}\{y\mapsto y\}\in id
		\\
		  x \neq y
	      }{
		  (\nu x)P \equiv_{\beta}^{id} (\nu y)Q
	      }$	      
	    &
	      $\inferrule*[left=BetaRes]{
		  Q \equiv_{\beta}^{id} R
		\\
		  \{y\mapsto y\}\in id
	      }{
		  (\nu y)Q \equiv_{\beta}^{id} (\nu y)R
	      }$
	  \end{tabular}
	\end{center}
	for inductive hypothesis $P\equiv_{\beta}^{id} R$. Clearly $\{x\mapsto x\}\{y\mapsto y\}\in id$ so for rule $BetaRes1$: $(\nu x)P \equiv_{\beta}^{id} (\nu y)R$
      \item[$(BetaRes, BetaRes1)$]:
      \item[$(BetaInp1, BetaInp1)$]
    \end{description}
  \end{proof}
\end{lemma}


\begin{lemma}
  If $P\xrightarrow{\lambda}_{2} P^{'}$ then $P\equiv_{\alpha} Q$, $Q\stackrel{\lambda}{\twoheadrightarrow}_{2} Q^{'}$ and $P^{'}\equiv_{\alpha} Q^{'}$
\end{lemma}


\begin{lemma}
  If $P\equiv_{\beta}^{\sigma} Q$, $Q\stackrel{\lambda}{\twoheadrightarrow}_{2} Q^{'}$ then $P\stackrel{\lambda\sigma}{\twoheadrightarrow}_{2} P^{'}$, $P^{'}\equiv_{\beta}^{\sigma^{'}} Q^{'}$ and $\sigma^{'}\subseteq \sigma$.
\end{lemma}

\begin{lemma}
  If $P\equiv_{\alpha} Q$, $Q\stackrel{\lambda}{\twoheadrightarrow}_{2} Q^{'}$ then $P\stackrel{\lambda}{\twoheadrightarrow}_{2} P^{'}$ and $P^{'}\equiv_{\alpha} Q^{'}$.
\end{lemma}

in questo modo si puo' dimostrare che una bisimulazione e' conservata da un operatore senza trattare i casi dati dalla regola per l'alfa conversione
