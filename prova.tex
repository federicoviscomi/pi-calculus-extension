

\begin{definition}
  Let $\sigma$ be a function of names. $\beta,\sigma$ equivalence is the smallest binary relation on processes that satisfies the laws in table \ref{betaequivalence}. In a process $P$ we can assume that all bound names are different. 
  \begin{table}
    \begin{tabular}{l}
      \multicolumn{1}{l}{\line(1,0){415}}\\\\
	  $\inferrule*[left=BetaOut]{
	      P\equiv_{\beta}^{\sigma}Q
	    \\
	      (x\sigma z) \vee (x,z \notin n(\sigma)) \vee (x \sigma x \wedge z \sigma z)
	    \\
	      (y\sigma w) \vee (y,w \notin n(\sigma)) \vee (y \sigma y \wedge w \sigma w)
	  }{
	      \overline{x}y.P\equiv_{\beta}\overline{z}w.Q
	    \\
	      \sigma \cup \{x\mapsto z\} \cup \{y\mapsto w\}
	  }$
	\\
      \end{tabular}
      \\
      \begin{tabular}{ll}
      \\
	  $\inferrule*[left=BetaTau]{
	      P\equiv_{\beta}Q
	    \\
	      \sigma
	  }{
	      \tau.P\equiv_{\beta}\tau.Q
	    \\
	      \sigma
	  }$
	&
	  $\inferrule*[left=BetaZero]{
	  }{
	      0\equiv_{\beta}0
	    \\
	      \emptyset
	  }$
    \\
    \end{tabular}
      \\
    \begin{tabular}{l}
    \\
	  $\inferrule*[left=BetaInp]{
	      P\equiv_{\beta}Q
	    \\
	      \sigma
	    \\
	      (y \sigma y) \vee (y\notin n(\sigma))
	    \\
	      (x\sigma z) \vee (x,z\notin n(\sigma)) \vee (x\sigma x \wedge z\sigma z)
	  }{
	      x(y).P\equiv_{\beta}z(y).Q
	    \\
	      \sigma \{x\mapsto z\}
	  }$
	\\\\
	  $\inferrule*[left=BetaRes]{
	      P\equiv_{\beta}Q
	    \\
	      \sigma
	    \\
	      (x \sigma x) \vee (x\notin n(\sigma))
	  }{
	      (\nu x)P\equiv_{\beta}(\nu x)Q
	    \\
	      \sigma
	  }$
      \\
    \end{tabular}
    \\
    \begin{tabular}{ll}
      \\
	  $\inferrule*[left=BetaIde]{
	  }{
	      A(\tilde{x})\equiv_{\beta}A(\tilde{x})
	    \\
	      \emptyset
	  }$
	&
	  $\inferrule*[left=BetaIde1]{
	    \tilde{x} \neq \tilde{y}
	  }{
	      A(\tilde{x})\equiv_{\beta}A(\tilde{y})
	    \\
	      \{x_{1}\mapsto y_{1}\}\cdot\{x_{n}\mapsto y_{n}\}
	  }$
      \\
    \end{tabular}
    \\
    \begin{tabular}{l}
      \\
	  $\inferrule*[left=BetaPar]{
	      P_{1}\equiv_{\beta}Q_{1}
	    \\
	      \sigma_{1}
	    \\
	      P_{2}\equiv_{\beta}Q_{2}
	    \\
	      \sigma_{2}
	    \\
	      (\sigma_{1}\subseteq \sigma_{2}) \vee (\sigma_{2}\subseteq \sigma_{1})
	  }{
	      P_{1}|P_{2}\equiv_{\beta}Q_{1}|Q_{2}
	    \\
	      \sigma_{1} \cup \sigma_{2}
	  }$
      \\\\
	  $\inferrule*[left=BetaSum]{
	      P_{1}\equiv_{\beta} Q_{1}
	    \\
	      \sigma_{1}
	    \\
	      P_{2}\equiv_{\beta} Q_{2}
	    \\
	      \sigma_{2}
	    \\
	      (\sigma_{1}\subseteq \sigma_{2}) \vee (\sigma_{2}\subseteq \sigma_{1})
	  }{
	      P_{1}+P_{2}\equiv_{\beta}Q_{1}+Q_{2}
	    \\
	      \sigma_{1} \cup \sigma_{2}
	  }$
      \\
    \end{tabular}
    \\
    \begin{tabular}{l}
      \\
	  $\inferrule*[left=BetaRes1]{
	      P\equiv_{\beta}Q
	    \\
	      \sigma
	    \\
	      (x\sigma y) \vee (x\sigma x \wedge y\sigma y) \vee (x,y\notin n(\sigma))
	    \\
	      x \neq y
	  }{
	      (\nu x)P\equiv_{\beta}(\nu y)Q
	    \\
	      \sigma - \{x\mapsto y\}
	  }$
      \\\\
	  $\inferrule*[left=BetaInp1]{
	      P\equiv_{\beta}Q
	    \\
	      \sigma
	    \\
	      (y\sigma w) \vee (y \sigma y \wedge w \sigma w) \vee (y,w\notin n(\sigma))
	    \\
	      y \neq w
	    \\
	      (x\sigma z) \vee (x\sigma x \wedge z\sigma z) \vee (x,z\notin n(\sigma))
	  }{
	      x(y).P\equiv_{\beta}z(w).Q
	    \\
	      (\sigma \{x\mapsto z\})- \{y\mapsto w\}
	  }$
    \\\\\multicolumn{1}{l}{\line(1,0){415}}
    \end{tabular}
    \caption{$\beta$ equivalence laws}
    \label{betaequivalence}
  \end{table}
\end{definition}

\begin{definition}
  Processes $P$ and $Q$ are \emph{$\alpha$ convertible} or \emph{$\alpha$ equivalent} if they are $\beta$ equivalent with respect to a substitution which is contained in the identity function on names.
\end{definition}


\begin{lemma}
  $\alpha$ equivalence is reflexive.
  \begin{proof}
    We prove that $P\equiv_{\beta}^{\emptyset} P$. The proof is a structural induction on $P$:
	\begin{center}
	  \begin{tabular}{ll}
	    $\inferrule*[left=BetaOut]{
		P\equiv_{\beta}P
	      \\
		\emptyset
	    }{
		\overline{x}y.P\equiv_{\beta}\overline{x}y.Q
	      \\
		\emptyset \{x\mapsto x\} \{y\mapsto y\}
	    }$
	  &
	    $\inferrule*[left=BetaTau]{
	      P\equiv_{\beta}Q
	    \\
	      \emptyset
	    }{
		\tau.P\equiv_{\beta}\tau.Q
	      \\
		\emptyset
	    }$
	\\\\
	  $\inferrule*[left=BetaInp]{
	      P\equiv_{\beta}Q
	    \\
	      \emptyset
	  }{
	      x(y).P\equiv_{\beta}x(y).Q
	    \\
	      \emptyset \{x\mapsto x\}
	  }$
	&
	  $\inferrule*[left=BetaRes]{
	      P\equiv_{\beta}Q
	    \\
	      \emptyset
	  }{
	      (\nu x)P\equiv_{\beta}(\nu x)Q
	    \\
	      \emptyset \{x\mapsto x\}
	  }$
      \\\\
	  $\inferrule*[left=BetaIde]{
	  }{
	      A(\tilde{x})\equiv_{\beta}A(\tilde{x})
	    \\
	      \emptyset
	  }$
	&
	  $\inferrule*[left=BetaZero]{
	  }{
	      0\equiv_{\beta}0
	    \\
	      \emptyset
	  }$
      \\
    \end{tabular}
    \\
    \begin{tabular}{l}
      \\
	  $\inferrule*[left=BetaPar]{
	      P_{1}\equiv_{\beta}Q_{1}
	    \\
	      \emptyset 
	    \\
	      P_{2}\equiv_{\beta}Q_{2}
	    \\
	      \emptyset
	  }{
	      P_{1}|P_{2}\equiv_{\beta}Q_{1}|Q_{2}
	    \\
	      \emptyset \cup \emptyset
	  }$
      \\\\
	  $\inferrule*[left=BetaSum]{
	      P_{1}\equiv_{\beta}Q_{1}
	    \\
	      \emptyset
	    \\
	      P_{2}\equiv_{\beta}Q_{2}
	    \\
	      \emptyset
	  }{
	      P_{1}+P_{2}\equiv_{\beta}Q_{1}+Q_{2}
	    \\
	      \emptyset \cup \emptyset
	  }$
	\\
	\end{tabular}
      \end{center}
  \end{proof}
\end{lemma}

\begin{lemma}
  $\alpha$ equivalence is symmetric.
  \begin{proof}
    We prove that $P\equiv_{\beta}^{\sigma}Q$ imply $Q\equiv_{\beta}^{\sigma^{-1}}P$ by induction on the rules:
    \begin{description}
      \item[$BetaOut$]:
	\begin{center}
	  $\inferrule*[left=BetaOut]{
	      P\equiv_{\beta}Q
	    \\
	      \sigma
	    \\
	      (x\sigma z) \vee (x\sigma x \wedge z\sigma z)
	    \\
	      (y\sigma w) \vee (y\sigma y \wedge w\sigma w)
	  }{
	      \overline{x}y.P\equiv_{\beta}\overline{z}w.Q
	    \\
	      \sigma \{x\mapsto z\} \{y\mapsto w\}
	  }$
	\end{center}
	for inductive hypothesis: $Q\equiv_{\beta}^{\sigma^{-1}} P$. $(x\sigma z) \vee (x\sigma x \wedge z\sigma z)$ imply $(z\sigma^{-1} x) \vee (x\sigma^{-1} x \wedge z\sigma^{-1} z)$. $(y\sigma w) \vee (y\sigma y \wedge w\sigma w)$ imply $(w\sigma^{-1} y) \vee (y\sigma^{-1} y \wedge w\sigma^{-1} w)$. And rule $BetaOut$ imply $\overline{z}w.Q\equiv_{\beta}\overline{x}y.P$ with respect to $\sigma^{-1} \{z\mapsto x\} \{w\mapsto y\}$.
      \item[$BetaTau$]:
	\begin{center}
	  $\inferrule*[left=BetaTau]{
	      P\equiv_{\beta}Q
	    \\
	      \sigma
	  }{
	      \tau.P\equiv_{\beta}\tau.Q
	    \\
	      \sigma
	  }$
	\end{center}
	for inductive hypothesis $Q\equiv_{\beta}^{\sigma^{-1}}P$ and for rule $BetaTau$: $\tau.Q\equiv_{\beta}\tau.P$.
      \item[$BetaInp$]:
	\begin{center}
	  $\inferrule*[left=BetaInp]{
	      P\equiv_{\beta}Q
	    \\
	      \sigma
	    \\
	      y \sigma y
	    \\
	      (x\sigma z) \vee (x\sigma x \wedge z\sigma z)
	  }{
	      x(y).P\equiv_{\beta}z(y).Q
	    \\
	      \sigma \{x\mapsto z\}
	  }$	  
	\end{center}
	for inductive hypothesis $Q\equiv_{\beta}^{\sigma^{-1}}P$. $(x\sigma z) \vee (x\sigma x \wedge z\sigma z)$ imply $(z\sigma^{-1} x) \vee (x\sigma^{-1} x \wedge z\sigma^{-1} z)$. And rule $BetaInp$ imply $z(y).Q\equiv_{\beta} x(y).P$ with respect to $\sigma^{-1} \{z\mapsto x\}$.
      \item[$BetaRes$]:
	\begin{center}
	  $\inferrule*[left=BetaRes]{
	      P\equiv_{\beta}Q
	    \\
	      \sigma
	    \\
	      x \sigma x
	  }{
	      (\nu x)P\equiv_{\beta}(\nu x)Q
	    \\
	      \sigma
	  }$	  
	\end{center}
	for inductive hypothesis $Q\equiv_{\beta}^{\sigma^{-1}}P$. $x \sigma x$ imply $x \sigma^{-1} x$. For rule $BetaRes$: $(\nu x)Q\equiv_{\beta}^{\sigma^{-1}}(\nu x)P$
      \item[$BetaIde, BetaZero, BetaIde1$] it easy to see that the lemma holds in these cases
	
    \end{description}
DA CONTINUARE
    \begin{tabular}{l}
      \\
	  $\inferrule*[left=BetaPar]{
	      P_{1}\equiv_{\beta}Q_{1}
	    \\
	      \sigma_{1}
	    \\
	      P_{2}\equiv_{\beta}Q_{2}
	    \\
	      \sigma_{2}
	    \\
	      (\sigma_{1}\subseteq \sigma_{2}) \vee (\sigma_{2}\subseteq \sigma_{1})
	  }{
	      P_{1}|P_{2}\equiv_{\beta}Q_{1}|Q_{2}
	    \\
	      \sigma_{1} \cup \sigma_{2}
	  }$
      \\\\
	  $\inferrule*[left=BetaSum]{
	      P_{1}\equiv_{\beta} Q_{1}
	    \\
	      \sigma_{1}
	    \\
	      P_{2}\equiv_{\beta} Q_{2}
	    \\
	      \sigma_{2}
	    \\
	      (\sigma_{1}\subseteq \sigma_{2}) \vee (\sigma_{2}\subseteq \sigma_{1})
	  }{
	      P_{1}+P_{2}\equiv_{\beta}Q_{1}+Q_{2}
	    \\
	      \sigma_{1} \cup \sigma_{2}
	  }$
      \\
    \end{tabular}
    \\
    \begin{tabular}{l}
      \\
	  $\inferrule*[left=BetaRes1]{
	      P\equiv_{\beta}Q
	    \\
	      \sigma
	    \\
	      (x\sigma y) \vee (x\sigma x \wedge y\sigma y)
	    \\
	      x \neq y
	  }{
	      (\nu x)P\equiv_{\beta}(\nu y)Q
	    \\
	      \sigma - \{x\mapsto y\}
	  }$
      \\\\
	  $\inferrule*[left=BetaInp1]{
	      P\equiv_{\beta}Q
	    \\
	      \sigma
	    \\
	      (y\sigma w) \vee (y \sigma y \wedge w \sigma w)
	    \\
	      y \neq w
	    \\
	      (x\sigma z) \vee (x\sigma x \wedge z\sigma z)
	  }{
	      x(y).P\equiv_{\beta}z(w).Q
	    \\
	      (\sigma \{x\mapsto z\})- \{y\mapsto w\}
	  }$
    \\
    \end{tabular}    
  \end{proof}
\end{lemma}

\begin{lemma}
  $P\equiv_{\beta}^{\sigma}Q$ and $x\in fn(P)$ then one and only one of the following case holds: 
  \begin{itemize}
    \item 
      $x\in fn(Q)$ and $(x\sigma x \vee x\notin n(\sigma))$ 
    \item
      $x\notin fn(Q)$ and there exists $y$ such that $y\in fn(Q)$ and $x\sigma y$
  \end{itemize}
\end{lemma}

\begin{lemma}
  $P\equiv_{\beta}^{\sigma}Q$ and $x\sigma y$ imply $x\in fn(P)$ and $y\in fn(Q)$
\end{lemma}



\begin{lemma}
  If $P\equiv_{\beta}^{\sigma_{1}} Q$ and $Q\equiv_{\beta}^{\sigma_{2}} R$ then $P \equiv_{\beta}^{\sigma_{3}} R$ and $\sigma_{3}= \sigma_{1} \cdot \sigma_{2}$.
  \begin{proof}
    We have to prove that $P\equiv_{\beta}^{\sigma_{3}} R$. We go by induction on the derivation of both $P\equiv_{\beta}^{\sigma_{1}} Q$ and $Q\equiv_{\beta}^{\sigma_{2}} R$:
    \begin{description}
      \item[$(BetaIde, BetaIde)$]:
	In this case $P$, $Q$ and $R$ are some identifier $A(\tilde{x})$ and $\sigma_{1}=\sigma_{2}=\sigma_{3}=\emptyset$.
      \item[$(BetaZero, BetaZero)$]
	in this case $P$, $Q$ and $R$ are $0$ and $\sigma_{1}=\sigma_{2}=\sigma_{3}=\emptyset$.
      \item[$(BetaTau, BetaTau)$]:
	\begin{center}
	  \begin{tabular}{ll}
	      $\inferrule*[left=BetaTau]{
		  P \equiv_{\beta} Q
		\\
		   \sigma_{1}
	      }{
		  \tau.P \equiv_{\beta} \tau.Q
		\\
		  \sigma_{1}
	      }$	      
	    &
	      $\inferrule*[left=BetaTau]{
		  Q \equiv_{\beta} R
		\\
		  \sigma_{2}
	      }{
		  \tau.Q \equiv_{\beta} \tau.R
		\\
		  \sigma_{2}
	      }$
	  \end{tabular}
	\end{center}
	for inductive hypothesis $P\equiv_{\beta}^{\sigma_{3}} R$ and $\sigma_{3}=\sigma_{1}\cdot \sigma_{2}$. For rule $BetaTau$: $\tau.P\equiv_{\beta}^{\sigma_{3}} \tau.R$
      \item[$(BetaSum, BetaSum)$]
      \item[$(BetaPar, BetaPar)$]
      \item[$(BetaRes, BetaRes)$]
      \item[$(BetaInp, BetaInp)$]
      \item[$(BetaRes1, BetaRes1)(1)$]:
	\begin{center}
	  \begin{tabular}{l}
	      $\inferrule*[left=BetaRes1]{
		  P \equiv_{\beta} Q
		\\
		  \sigma_{1}
		\\
		  (x\sigma_{1}y) \vee (x\sigma_{1}x \wedge y\sigma_{1}y)
		\\
		  x \neq y
	      }{
		  (\nu x)P \equiv_{\beta} (\nu y)Q
		\\
		  \sigma_{1} - (xy)
	      }$	      
	    \\\\
	      $\inferrule*[left=BetaRes1]{
		  Q \equiv_{\beta}^{\sigma_{2}} R
		\\
		  \sigma_{2}
		\\
		  (y\sigma_{2}z) \vee (y\sigma_{2}y \wedge z\sigma_{2}z)
		\\
		  y \neq z
	      }{
		  (\nu y)Q \equiv_{\beta} (\nu z)R
		\\
		  \sigma_{2} - (yz)
	      }$
	  \end{tabular}
	\end{center}
	there can only happen two cases:
	\begin{itemize}
	  \item 
	    $(x\sigma_{1}y)$ and $(y\sigma_{2}z)$
	  \item
	    or $(x\sigma_{1}x \wedge y\sigma_{1}y)$ and $(y\sigma_{2}y \wedge z\sigma_{2}z)$
	\end{itemize}
	in both cases we can apply the inductive hypothesis and get $P\equiv_{\beta}^{\sigma_{3}} R$ and $\sigma_{3} = \sigma_{1} \cdot \sigma_{2}$. Then the conclusion follows for rule $BetaRes1$.
      \item[$(BetaRes1, BetaRes)$]
      \item[$(BetaRes, BetaRes1)$]:
      \item[$(BetaInp1, BetaInp1)$]
    \end{description}
  \end{proof}
\end{lemma}

\begin{corollary}
  $\alpha$ equivalence is transitive.
\end{corollary}

$\rightarrow_{2}$ is the $\pi$ calculus early semantics without structural congruence but with explicit $\alpha$ conversion. Whether $\twoheadrightarrow_{2}$ is the same of $\rightarrow_{2}$ but without explicit $\alpha$ conversion.

\begin{lemma}
  If $P\xrightarrow{\lambda}_{2} P^{'}$ then $P\equiv_{\alpha} Q$, $Q\stackrel{\lambda}{\twoheadrightarrow}_{2} Q^{'}$ and $P^{'}\equiv_{\alpha} Q^{'}$
\end{lemma}


\begin{lemma}
  $P\equiv_{\alpha}Q$ and $x\notin bn(P,Q)$ imply $P\{x/y\} \equiv_{\alpha} Q\{x/y\}$.
  \begin{proof}
    We prove that $P\equiv_{\beta}^{\sigma}Q$ and $x\notin bn(P,Q)$ imply $P\{x/y\} \equiv_{\beta}^{\sigma^{'}} Q\{x/y\}$ where $\sigma^{'}$ is such that:
    $\{a\mapsto b\} \in \sigma$ imply $\{a\{x/y\}\mapsto b\{x/y\}\} \in \sigma^{'}$. Then the conclusion follows because if $\sigma\subseteq id$ then also $\sigma^{'}\subseteq id$.
  \end{proof}
\end{lemma}


\begin{lemma}
  $P\equiv_{\alpha}Q$ imply $P\{x/y\} \equiv_{\alpha} Q\{x/y\}$.
  \begin{proof}
    We prove that $P\equiv_{\beta}^{\sigma}Q$ imply $P\{x/y\} \equiv_{\beta}^{\sigma^{'}} Q\{x/y\}$ where $\sigma^{'}$ is such that:
    $\{a\mapsto b\} \in \sigma$ imply $\{a\{x/y\}\mapsto b\{x/y\}\} \in \sigma^{'}$. Then the conclusion follows because if $\sigma\subseteq id$ then also $\sigma^{'}\subseteq id$.
  \end{proof}
\end{lemma}


\begin{lemma}
  If $P\equiv_{\alpha} Q$ and $Q\stackrel{\lambda}{\twoheadrightarrow}_{2} Q^{'}$ then $P\stackrel{\lambda}{\twoheadrightarrow}_{2} P^{'}$ and $P^{'}\equiv_{\alpha}^{\sigma^{'}} Q^{'}$.
  \begin{proof}
    The proof is an induction on both the derivation of $P\equiv_{\beta}^{\sigma} Q$ and $Q\stackrel{\lambda}{\twoheadrightarrow}_{2} Q^{'}$. The last pair of rules used can be:
    \begin{description}
      \item[$(BetaIde, Ide)$]:
	\begin{center}
	  \begin{tabular}{ll}
	      $\inferrule*[left=BetaIde]{
	      }{
		  A(\tilde{y})\equiv_{\beta}A(\tilde{y})
		\\
		  \emptyset
	      }$	      
	    &
	      $\inferrule*[left=Ide]{
		  A(\tilde{x}) \stackrel{def}{=} P
		\\
		  P\{\tilde{y}/\tilde{x}\} \stackrel{\lambda}{\twoheadrightarrow}_{2} P^{'}
	      }{
		  A(\tilde{y}) \stackrel{\lambda}{\twoheadrightarrow}_{2} P^{'}
	      }$	      
	  \\
	  \end{tabular}
	\end{center}
      \item[$(BetaInp, EInp)$]:
	\begin{center}
	  \begin{tabular}{l}
	      $\inferrule*[left=BetaInp]{
		  P\equiv_{\beta}^{\sigma}Q
		\\
		  (y \sigma y) \vee (y\notin n(\sigma))
		\\
		  (x \sigma x) \vee (x\notin n(\sigma))
	      }{
		  x(y).P\equiv_{\beta}x(y).Q
		\\
		  \sigma \{x\mapsto x\}
	      }$
	    \\
	      $\inferrule*[left=EInp]{
	      }{
		  x(y).Q \stackrel{xw}{\longtwoheadrightarrow}_{2} Q\{w/y\}
	      }$	      
	  \\
	  \end{tabular}
	\end{center}
	where $\sigma \subseteq id$. For rule $EInp$: $x(y).P \stackrel{xw}{\longtwoheadrightarrow}_{2} P\{w/y\}$. $P\equiv_{\beta}^{\sigma}Q$ and $\sigma\subseteq id$ imply $P\equiv_{\alpha}Q$ which in turn imply $P\{w/y\} \equiv_{\alpha} Q\{w/y\}$.
      \item[$(BetaInp1, EInp)$]:
	\begin{center}
	  \begin{tabular}{l}
	      $\inferrule*[left=BetaInp1]{
		  P\equiv_{\beta}^{\sigma}Q
		\\
		  (y \sigma w) \vee (y,w\notin n(\sigma))
		\\
		  (x \sigma x) \vee (x\notin n(\sigma))
	      }{
		  x(y).P\equiv_{\beta}x(w).Q
		\\
		  \sigma \{x\mapsto x\}
	      }$
	    \\
	      $\inferrule*[left=EInp]{
	      }{
		  x(w).Q \stackrel{xa}{\longtwoheadrightarrow}_{2} Q\{a/w\}
	      }$	      
	  \\
	  \end{tabular}
	\end{center}
	where $\sigma \subseteq id$. For rule $EInp$: $x(y).P \stackrel{xa}{\longtwoheadrightarrow}_{2} P\{a/y\}$. The premises of rule $BetaInp$ imply $P\{a/y\} \equiv_{\alpha} Q\{a/w\}$. DA DIMOSTRARE!
      \item[$(BetaOut, Out)$]:
	\begin{center}
	  \begin{tabular}{l}
	      $\inferrule*[left=BetaOut]{
		  P\equiv_{\beta}^{\sigma}Q
		\\
		  (y \sigma y) \vee (y\notin n(\sigma))
		\\
		  (x \sigma x) \vee (x\notin n(\sigma))
	      }{
		  \overline{x}y.P\equiv_{\beta} \overline{x}y.Q
		\\
		  \sigma \cup \{x\mapsto x\} \cup \{y\mapsto y\}
	      }$
	    \\
	      $\inferrule*[left=Out]{
	      }{
		  \overline{x}y.Q \stackrel{\overline{x}y}{\longtwoheadrightarrow}_{2} Q
	      }$	      
	  \\
	  \end{tabular}
	\end{center}
	where $\sigma \subseteq id$. For rule $Out$: $\overline{x}y.P \stackrel{\overline{x}y}{\longtwoheadrightarrow}_{2} P$ 
      \item[$(BetaPar, Par)$]
      \item[$(BetaPar, ECom)$]
      \item[$(BetaRes, Res)$]
      \item[$(BetaRes, Opn)$]
      \item[$(BetaRes1, Res)$]
      \item[$(BetaRes1, Opn)$]
      \item[$(BetaSum, Sum)$]
      \item[$(BetaTau, Tau)$]
    \end{description}
  \end{proof}
\end{lemma}

\begin{corollary}
  If $P\equiv_{\alpha} Q$, $Q\stackrel{\lambda}{\twoheadrightarrow}_{2} Q^{'}$ then $P\stackrel{\lambda}{\twoheadrightarrow}_{2} P^{'}$ and $P^{'}\equiv_{\alpha} Q^{'}$.
\end{corollary}

\begin{lemma}
  If $P\xrightarrow{\lambda}_{2} P^{'}$ then $P\stackrel{\lambda}{\twoheadrightarrow}_{2} P^{''}$ and $P^{'}\equiv_{\alpha} P^{''}$.
\end{lemma}


in questo modo si puo' dimostrare che una bisimulazione e' conservata da un operatore senza trattare i casi dati dalla regola per l'alfa conversione
