
\section{syntax}

As we did whit multi $\pi$ calculus, we suppose that we have a countable set of names $\mathbb{N}$, ranged over by lower case letters $a,b, \cdots, z$. This names are used for communication channels and values. Furthermore we have a set of identifiers, ranged over by $A$. We represent the agents or processes by upper case letters $P,Q, \cdots $. A multi $\pi$ process, in addiction to the same actions of a $\pi$ process, can perform also a strong prefix:
\begin{center}
  $\pi$ ::= $\overline{x}y$ | $x(z)$ | $\underline{x(y)}$ | $\underline{\overline{x}y}$ |$\tau$ 
\end{center}
The process are defined, just as original $\pi$ calculus, by the following grammar:
\begin{center}
  \begin{tabular}{l}
    $P,Q$ ::= $0$ | $\pi.P$ | $P|Q$ | $P+Q$ | $(\nu x) P$ | $A(y_{1}, \cdots, y_{n})$
  \end{tabular}
\end{center}
and they have the same intuitive meaning as for the $\pi$ calculus. The strong prefix input allows a process to make an atomic sequence of actions, so that more than one process can synchronize on this sequence. 

We have to extend the following definition to deal with the strong prefix:
\begin{center}
  \begin{tabular}{ll}
	$bn(\underline{x(y)})\; =\; \{y\}$
      &
	$fn(\underline{x(y)})\; =\; \{x\}$
    \\
	$bn(\underline{\overline{x}y})\; =\; \emptyset$
      &
	$fn(\underline{\overline{x}y})\; =\; \{x,y\}$
    \\
  \end{tabular}
\end{center}


\section{early operational semantic with structural congruence}

The semantic of a multi $\pi$ process is labeled transition system such that
\begin{itemize}
  \item 
    the nodes are multi $\pi$ calculus process. The set of node is $\mathbb{P}_{m}$
  \item
    the actions are multi $\pi$ calculus actions plus two new kind of action:
    \begin{center}
      \begin{tabular}{ll}
	  strong input 
	&
	  strong output
      \\
	  $\underline{xy}$
	&
	  $\underline{\overline{x}y}$
      \end{tabular}
    \end{center}
    The set of actions is $\mathbb{A}_{m}$, we use $\alpha, \alpha_{1}, \alpha_{2},\cdots $ to range over the set of actions, we use $\sigma, \sigma_{1}, \sigma_{2}, \cdots $ to range over the set $\mathbb{A}_{m}^{+} \cup \{\tau\}$. 
  \item
    the transition relations is $\rightarrow\subseteq \mathbb{P}_{m}\times (\mathbb{A}_{m}^{+} \cup \{\tau\})\times \mathbb{P}_{m}$
\end{itemize}

In this case, a label can be a sequence of prefixes, whether in the original $\pi$ calculus a label can be only a prefix. We use the symbol $\cdot$ to denote the concatenation operator.

\begin{definition}\index{transition relation! multipi! early! with structural congruence}
  The \emph{early transition relation with structural congruence} is the smallest relation induced by the following rules:
  \begin{center}
    \begin{tabular}{ll}
	  $\inferrule* [left=\bf{Out}]{  
	  }{
	    \overline{x}y.P \;\xrightarrow{\overline{x}y} P
	  }$
	&
	  $\inferrule* [left=\bf{EInp}]{
	    z\notin fn(P)
	  }{
	    x(y).P \;\xrightarrow{xz} P\{z/y\}
	  }$
      \\\\
	  $\inferrule* [left=\bf{SOut}]{	      
	  }{
	    \underline{\overline{x}y}.P \;\xrightarrow{\underline{\overline{x}y}} P
	  }$
	&
      \\\\
	  $\inferrule* [left=\bf{Tau}]{ 
	  }{
	    \tau.P \;\xrightarrow{\tau} P
	  }$
	&
	  $\inferrule* [left=\bf{SInp}]{
	  }{
	    \underline{x(y)}.P \;\xrightarrow{\underline{xz}} P\{z/y\}
	  }$
      \\\\
	  $\inferrule* [left=\bf{Sum}]{
	    P \;\xrightarrow{\sigma} P^{'}
	  }{
	    P+Q \;\xrightarrow{\sigma} P^{'}
	  }$
	&
	$\inferrule* [left=\bf{Str}]{
	    P\equiv P^{'}
	  \\
	    P^{'}\; \;\xrightarrow{\alpha}\; Q^{'}
	  \\
	    Q\equiv Q^{'}
	}{
	    P\; \;\xrightarrow{\alpha}\; Q
	}$
      \\\\
	  $\inferrule* [left=\bf{Par}]{
	    P \;\xrightarrow{\sigma} P^{'}\;\; bn(\sigma)\cap fn(Q)=\emptyset
	  }{
	    P|Q \;\xrightarrow{\sigma} P^{'}|Q
	  }$
	&
	  $\inferrule* [left=\bf{Com}]{
	      P \;\xrightarrow{\sigma_{1}} P^{'}
	    \\
	      Q\;\xrightarrow{\sigma_{2}} Q^{'}
	    \\
	      Sync(\sigma_{1}, \sigma_{2}, \sigma_{3})
	  }{
	    P|Q \;\xrightarrow{\sigma_{3}} P^{'}|Q^{'}
	  }$
      \\\\
	  $\inferrule* [left=\bf{Res}]{
	    P \;\xrightarrow{\sigma} P^{'}\;\; z\notin n(\alpha)
	  }{
	    (\nu) z P \;\xrightarrow{\sigma} (\nu) z P^{'}
	  }$
	&
	  $\inferrule* [left=\bf{Seq}]{
	    P \;\xrightarrow{\underline{\alpha}} P^{'}\;\; P^{'} \;\xrightarrow{\sigma} P^{''}
	  }{
	    P \;\xrightarrow{\alpha \cdot \sigma} P^{''}
	  }$
      \\
    \end{tabular}
  \end{center}
\end{definition}


In this semantic we cannot have
\[
  \underline{x(a)}.\overline{a}z.P\; \xrightarrow{xb\cdot \overline{a}z}\; P\{b/a\}
\]
nor
\[
  \underline{x(a)}.\underline{\overline{a}z}.P\; \xrightarrow{xb\cdot \overline{a}z}\; P\{b/a\}
\]
but we have
\begin{center}
  $
    \inferrule* [left=Seq]{
	\inferrule* [left=SOut]{
	}{
	  \underline{\overline{b}z}.P\{b/a\}\; \xrightarrow{\underline{\overline{b}z}}\; P\{b/a\}
	}
      \\
	\inferrule* [left=SInp]{
	}{
	  \underline{x(a)}.\underline{\overline{a}z}.P\; \xrightarrow{\underline{xb}}\; (\underline{\overline{a}z}.P)\{b/a\}
	}
    }{
      \underline{x(a)}.\underline{\overline{a}z}.P\; \xrightarrow{xb\cdot \overline{b}z}\; P\{b/a\}
    }
  $
\end{center}


\begin{definition}\index{sync! multipi! early}
  We define the synchronization relation in the following way:
  \begin{center}
    \begin{tabular}{ccc}
	  $\inferrule* [left=Com1L]{
	   }{
	    Sync(xy,\overline{x}y,\tau)
	  }$
	&
	  $\inferrule* [left=Com2L]{
	   }{
	    Sync(xy\cdot \sigma,\overline{x}y,\sigma)
	  }$	  
	&
	  $\inferrule* [left=Com3L]{
	    Sync(\sigma_{1},\sigma_{2},\sigma_{3})
	   }{
	    Sync(xy\cdot \sigma_{1},\overline{x}y\cdot \sigma_{2},\sigma_{3})
	  }$
      \\\\
	  $\inferrule* [left=Com1R]{
	   }{
	    Sync(\overline{x}y,xy,\tau)
	  }$
	&
	  $\inferrule* [left=Com2R]{
	   }{
	    Sync(\overline{x}y\cdot \sigma,xy,\sigma)
	  }$	  
	&
	  $\inferrule* [left=Com3R]{
	    Sync(\sigma_{1},\sigma_{2},\sigma_{3})
	   }{
	    Sync(\overline{x}y\cdot \sigma_{1},xy\cdot \sigma_{2},\sigma_{3})
	  }$
      \\
    \end{tabular}
  \end{center}
\end{definition}

NON FUNZIONA!!


\section{late operational semantic with structural congruence}

\begin{definition}\index{transition relation! multipi! late! with structural congruence}
  The \emph{late transition relation with structural congruence} is the smallest relation induced by the following rules:
  \begin{center}
    \begin{tabular}{ll}
 	  \bf{Pref}
 	  \begin{tabular}{c}
 	      $\alpha\; not\; a\; strong\; prefix$
 	    \\\hline
 	      $\alpha.P \;\xrightarrow{\alpha} P$
 	  \end{tabular}
	&
	  \bf{Par}
	  \begin{tabular}{c}
	      $P \;\xrightarrow{\sigma} P^{'}\;\; bn(\sigma)\cap fn(Q)=\emptyset$
	    \\\hline
	      $P|Q \;\xrightarrow{\sigma} P^{'}|Q$
	  \end{tabular}
      \\\\
	  \bf{SOut}
	  \begin{tabular}{c}
	      $P \;\xrightarrow{\sigma} P^{'}\;\; \sigma\neq \tau$
	    \\\hline
	      $\underline{\overline{x}y}.P \;\xrightarrow{\overline{x}y \cdot \sigma} P^{'}$
	  \end{tabular}
	&
	  $\inferrule* [left=\bf{LComSeq1}]{
	      P \;\xrightarrow{x(y)}\; P^{'}
	    \\
	      Q\;\xrightarrow{\overline{x}z\cdot \sigma} Q^{'}
	    \\
	      z\notin fn(P)
	  }{
	    P|Q \;\xrightarrow{\sigma} P^{'}\{z/y\}|Q^{'}
	  }$
      \\\\
	  \bf{Sum}
	  \begin{tabular}{c}
	      $P \;\xrightarrow{\sigma} P^{'}$
	    \\\hline
	      $P+Q \;\xrightarrow{\sigma} P^{'}$
	  \end{tabular}
	&
	$\inferrule* [left=\bf{Str}]{
	    P\equiv P^{'}
	  \\
	    P^{'}\; \;\xrightarrow{\alpha}\; Q^{'}
	  \\
	    Q\equiv Q^{'}
	}{
	    P\; \;\xrightarrow{\alpha}\; Q
	}$
      \\\\
	  \bf{Res}
	  \begin{tabular}{c}
	      $P \;\xrightarrow{\sigma} P^{'}\;\; z\notin n(\alpha)$
	    \\\hline
	      $(\nu) z P \;\xrightarrow{\sigma} (\nu) z P^{'}$
	  \end{tabular}
	&
	  $\inferrule* [left=\bf{LComSng}]{
	      P \;\xrightarrow{x(y)} P^{'}
	    \\
	      Q\;\xrightarrow{\overline{x}z} Q^{'}
	    \\
	      z\notin fn(P)
	  }{
	    P|Q \;\xrightarrow{\tau} P^{'}\{z/y\}|Q^{'}
	  }$
      \\\\
	  \bf{SInp}
	  \begin{tabular}{c}
	      $P \;\xrightarrow{\sigma} P^{'}\;\; \sigma\neq \tau$
	    \\\hline
	      $\underline{x(y)}.P \;\xrightarrow{x(y) \cdot \sigma} P^{'}$
	  \end{tabular}
	&
	  $\inferrule* [left=\bf{LComSeq2}]{
	      P \;\xrightarrow{\overline{x}z}\; P^{'}
	    \\
	      Q\;\xrightarrow{x(y)\cdot \sigma}\; Q^{'}
	    \\
	      z\notin fn(P)
	  }{
	    P|Q \;\xrightarrow{\sigma\{z/y\}}\; P^{'}|Q^{'}\{z/y\}
	  }$
      \\
    \end{tabular}
  \end{center}
\end{definition}
