
\section{Syntax}

As we did whit multi $\pi$ calculus, we suppose that we have a countable set of names $\mathbb{N}$, ranged over by lower case letters $a,b, \cdots, z$. This names are used for communication channels and values. Furthermore we have a set of identifiers, ranged over by $A$. We represent the agents or processes by upper case letters $P,Q, \cdots $. A multi $\pi$ process, in addiction to the same actions of a $\pi$ process, can perform also a strong prefix:
\begin{center}
  $\pi$ ::= $\overline{x}y$ | $x(z)$ | $\underline{x(y)}$ | $\underline{\overline{x}y}$ |$\tau$ 
\end{center}
The process are defined, just as original $\pi$ calculus, by the following grammar:
\begin{center}
  \begin{tabular}{l}
    $P,Q$ ::= $0$ | $\pi.P$ | $P|Q$ | $P+Q$ | $(\nu x) P$ | $A(y_{1}, \cdots, y_{n})$
  \end{tabular}
\end{center}
and they have the same intuitive meaning as for the $\pi$ calculus. The strong prefix input allows a process to make an atomic sequence of actions, so that more than one process can synchronize on this sequence. 

We have to extend the following definition to deal with the strong prefix:
\begin{center}
  \begin{tabular}{ll}
	$B(\underline{x(y)}.Q, I)\; =\; \{y,\overline{y}\}\cup B(Q, I)$
      &
	$F(\underline{x(y)}.Q, I)\; =\; \{x,\overline{x}\}\cup (F(Q, I)-\{y,\overline{y}\})$
    \\
	$B(\underline{\overline{x}y}.Q, I)\; =\; B(Q,I)$
      &
	$F(\underline{\overline{x}y}.Q, I)\; =\; \{x,\overline{x},y,\overline{y}\}\cup F(Q, I)$
    \\
  \end{tabular}
\end{center}

\section{Operational semantic}
\subsection{Early operational semantic with structural congruence}

\subsection{Late operational semantic with structural congruence}

The semantic of a multi $\pi$ process is labeled transition system such that
\begin{itemize}
  \item 
    the nodes are multi $\pi$ calculus process. The set of node is $\mathbb{P}_{m}$
  \item
    The set of actions is $\mathbb{A}_{m}$ and can contain
    \begin{itemize}
      \item 
	bound output $\overline{x}(y)$
      \item
	unbound output $\overline{x}y$ 
      \item
	bound input $x(z)$
    \end{itemize}
    We use $\alpha, \alpha_{1}, \alpha_{2},\cdots $ to range over the set of actions, we use $\sigma, \sigma_{1}, \sigma_{2}, \cdots $ to range over the set $\mathbb{A}_{m}^{+} \cup \{\tau\}$. 
  \item
    the transition relations is $\rightarrow\subseteq \mathbb{P}_{m}\times (\mathbb{A}_{m}^{+} \cup \{\tau\})\times \mathbb{P}_{m}$
\end{itemize}

In this case, a label can be a sequence of prefixes, whether in the original $\pi$ calculus a label can be only a prefix. We use the symbol $\cdot$ to denote the concatenation operator.

\begin{definition}\index{transition relation! multipi! late! with structural congruence}
  The \emph{late transition relation with structural congruence} is the smallest relation induced by the following rules:
  \begin{center}
    \begin{tabular}{ll}
 	  \bf{Pref}
 	  \begin{tabular}{c}
 	      $\alpha\; not\; a\; strong\; prefix$
 	    \\\hline
 	      $\alpha.P \;\xrightarrow{\alpha} P$
 	  \end{tabular}
	&
	  \bf{Par}
	  \begin{tabular}{c}
	      $P \;\xrightarrow{\sigma} P^{'}\;\; bn(\sigma)\cap fn(Q)=\emptyset$
	    \\\hline
	      $P|Q \;\xrightarrow{\sigma} P^{'}|Q$
	  \end{tabular}
      \\\\
	  \bf{SOut}
	  \begin{tabular}{c}
	      $P \;\xrightarrow{\sigma} P^{'}\;\; \sigma\neq \tau$
	    \\\hline
	      $\underline{\overline{x}y}.P \;\xrightarrow{\overline{x}y \cdot \sigma} P^{'}$
	  \end{tabular}
	&
	  $\inferrule* [left=\bf{LComSeq1}]{
	      P \;\xrightarrow{x(y)}\; P^{'}
	    \\
	      Q\;\xrightarrow{\overline{x}z\cdot \sigma} Q^{'}
	    \\
	      z\notin fn(P)
	  }{
	    P|Q \;\xrightarrow{\sigma} P^{'}\{z/y\}|Q^{'}
	  }$
      \\\\
	  \bf{Sum}
	  \begin{tabular}{c}
	      $P \;\xrightarrow{\sigma} P^{'}$
	    \\\hline
	      $P+Q \;\xrightarrow{\sigma} P^{'}$
	  \end{tabular}
	&
	$\inferrule* [left=\bf{Str}]{
	    P\equiv P^{'}
	  \\
	    P^{'}\; \;\xrightarrow{\alpha}\; Q^{'}
	  \\
	    Q\equiv Q^{'}
	}{
	    P\; \;\xrightarrow{\alpha}\; Q
	}$
      \\\\
	  \bf{Res}
	  \begin{tabular}{c}
	      $P \;\xrightarrow{\sigma} P^{'}\;\; z\notin n(\alpha)$
	    \\\hline
	      $(\nu z) P \;\xrightarrow{\sigma} (\nu z) P^{'}$
	  \end{tabular}
	&
	  $\inferrule* [left=\bf{LComSng}]{
	      P \;\xrightarrow{x(y)} P^{'}
	    \\
	      Q\;\xrightarrow{\overline{x}z} Q^{'}
	    \\
	      z\notin fn(P)
	  }{
	    P|Q \;\xrightarrow{\tau} P^{'}\{z/y\}|Q^{'}
	  }$
      \\\\
	  \bf{SInp}
	  \begin{tabular}{c}
	      $P \;\xrightarrow{\sigma} P^{'}\;\; \sigma\neq \tau$
	    \\\hline
	      $\underline{x(y)}.P \;\xrightarrow{x(y) \cdot \sigma} P^{'}$
	  \end{tabular}
	&
	  $\inferrule* [left=\bf{LComSeq2}]{
	      P \;\xrightarrow{\overline{x}z}\; P^{'}
	    \\
	      Q\;\xrightarrow{x(y)\cdot \sigma}\; Q^{'}
	    \\
	      z\notin fn(P)
	  }{
	    P|Q \;\xrightarrow{\sigma\{z/y\}}\; P^{'}|Q^{'}\{z/y\}
	  }$
      \\
    \end{tabular}
  \end{center}
\end{definition}


\subsection{Another attemp to late operational semantic with structural congruence}

\begin{definition}\index{transition relation! multipi! late! with structural congruence}
  The \emph{late transition relation with structural congruence} is the smallest relation induced by the following rules:
  \begin{center}
    \begin{tabular}{ll}
 	  \bf{Pref}
 	  \begin{tabular}{c}
 	      $\alpha\; not\; a\; strong\; prefix$
 	    \\\hline
 	      $\alpha.P \;\xrightarrow{\alpha} P$
 	  \end{tabular}
	&
	  \bf{Par}
	  \begin{tabular}{c}
	      $P \;\xrightarrow{\sigma} P^{'}\;\; bn(\sigma)\cap fn(Q)=\emptyset$
	    \\\hline
	      $P|Q \;\xrightarrow{\sigma} P^{'}|Q$
	  \end{tabular}
      \\\\
	  \bf{SOut}
	  \begin{tabular}{c}
	      $P \;\xrightarrow{\sigma} P^{'}\;\; \sigma\neq \tau$
	    \\\hline
	      $\underline{\overline{x}y}.P \;\xrightarrow{\overline{x}y \cdot \sigma} P^{'}$
	  \end{tabular}
	&
	  $\inferrule* [left=\bf{LCom}]{
	      P \;\xrightarrow{\sigma_{1}}\; P^{'}
	    \\
	      Q\;\xrightarrow{\sigma_{2}} Q^{'}
	    \\
	      Sync(\sigma_{1}, \sigma_{2}, \sigma_{3}, \delta_{1}, \delta_{2})
	  }{
	    P|Q \;\xrightarrow{\sigma_{3}} P^{'}\delta_{1}|Q^{'}\delta_{2}
	  }$
      \\\\
	  \bf{Sum}
	  \begin{tabular}{c}
	      $P \;\xrightarrow{\sigma} P^{'}$
	    \\\hline
	      $P+Q \;\xrightarrow{\sigma} P^{'}$
	  \end{tabular}
	&
	$\inferrule* [left=\bf{Str}]{
	    P\equiv P^{'}
	  \\
	    P^{'}\; \;\xrightarrow{\alpha}\; Q^{'}
	  \\
	    Q\equiv Q^{'}
	}{
	    P\; \;\xrightarrow{\alpha}\; Q
	}$
      \\\\
	  \bf{Res}
	  \begin{tabular}{c}
	      $P \;\xrightarrow{\sigma} P^{'}\;\; z\notin n(\alpha)$
	    \\\hline
	      $(\nu z) P \;\xrightarrow{\sigma} (\nu z) P^{'}$
	  \end{tabular}
	&
      \\\\
	  \bf{SInp}
	  \begin{tabular}{c}
	      $P \;\xrightarrow{\sigma} P^{'}\;\; \sigma\neq \tau$
	    \\\hline
	      $\underline{x(y)}.P \;\xrightarrow{x(y) \cdot \sigma} P^{'}$
	  \end{tabular}
	&
      \\
    \end{tabular}
  \end{center}
\end{definition}

where the relation $Sync$ is defined in the following way:
\begin{center}
  \begin{tabular}{ll}
	$\inferrule* [left=S1]{
	}{
	  Sync(x(y),\overline{x}z,\tau,\{z/y\},\{\})
	}$
      &
	$\inferrule* [left=S2]{
	}{
	  Sync(x(y),\overline{x}z\cdot \sigma,\sigma,\{z/y\},\{\})
	}$
    \\\\  
	$\inferrule* [left=S3]{
	}{
	  Sync(\overline{x}z,x(y)\cdot \sigma,\sigma\{z/y\},\{\},\{z/y\})
	}$	
      &
	$\inferrule* [left=S4]{
	  Sync(\sigma_{1}, \sigma_{2}\{z/y\}, \sigma_{3}, \delta_{1}, \delta_{2})
	}{
	  Sync(\overline{x}z\cdot\sigma_{1},x(y)\cdot \sigma_{2}, \sigma_{3}, \delta_{1}, \{z/y\}\delta_{2})
	}$		
    \\\\
	$\inferrule* [left=I1]{
	  Sync(\sigma_{1}, \sigma_{2}, \tau, \delta_{1}, \delta_{2})
	}{
	  Sync(\alpha \cdot \sigma_{1}, \sigma_{2}, \alpha, \delta_{1}, \delta_{2})
	}$		
      &
	$\inferrule* [left=I2]{
	  Sync(\sigma_{1}, \sigma_{2}, \sigma_{3}, \delta_{1}, \delta_{2})
	}{
	  Sync(\alpha \cdot \sigma_{1}, \sigma_{2}, \alpha \cdot \sigma_{3}, \delta_{1}, \delta_{2})
	}$			
    \\
  \end{tabular}
\end{center}


\begin{example}
  We want to prove that:
  \[
    \underline{\overline{a}x}.\overline{a}x.0|\underline{a(y)}.a(z).0\; \xrightarrow{\tau} 0|0
  \]
\end{example}


\begin{example}
  We want to prove that:
  \[
    (\underline{\overline{a}x}.\overline{b}x.0|\underline{b(y)}.\overline{a}(z).0)|\underline{a}(x).a(z).0\; \xrightarrow{\tau} (0|0)|0
  \]
\end{example}


\subsection{Step semantic}
%verificare se è una congruenza per il parallelo (come mi auguro sia).



