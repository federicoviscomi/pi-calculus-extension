
\section{Syntax}

As we did with $\pi$ calculus, we suppose that we have a countable set of names $\mathbf{N}$, ranged over by lower case letters $a,b, \cdots, z$. This names are used for communication channels and values. Furthermore we have a set of identifiers, ranged over by $A$. We represent the agents or processes by upper case letters $P,Q, \cdots $. A multi $\pi$ process, in addiction to the same actions of a $\pi$ process, can perform also a strong prefix input:
\begin{center}
  $\pi$ ::= $\overline{x}y$ | $x(z)$ | $\underline{x(y)}$ | $\tau$ 
\end{center}
The process are defined, just as original $\pi$ calculus, by the following grammar:
\begin{center}
  \begin{tabular}{l}
    $P,Q$ ::= $0$ | $\pi.P$ | $P|Q$ | $P+Q$ | $(\nu x) P$ | $A(y_{1}, \cdots, y_{n})$
  \end{tabular}
\end{center}
and they have the same intuitive meaning as for the $\pi$ calculus. The strong prefix input allows a process to make an atomic sequence of actions, so that more than one process can synchronize on this sequence. For the moment we allow the strong prefix to be on input names only. Also one can use the strong prefix only as an action prefixing for processes that can make at least a further action. 

Multi $\pi$ calculus is a conservative extension of the $\pi$ calculus in the sense that: any $\pi$ calculus process $p$ is also a multi $\pi$ calculus process and the semantic of $p$ according to the SOS rules of $\pi$ calculus is the same as the semantic of $p$ according to the SOS rules of multi $\pi$ calculus. 
We have to extend the following definition to deal with the strong prefix:
\begin{center}
  \begin{tabular}{ll}
	$B(\underline{x(y)}.Q, I) = \{y,\overline{y}\}\cup B(Q, I)$
      &
	$F(\underline{x(y)}.Q, I) = \{x,\overline{x}\}\cup (F(Q, I)-\{y,\overline{y}\})$
    \\
  \end{tabular}
\end{center}
The scope of the object of a strong input is the process that follows the strong input. For example the scope of a name $x$ in a process $\underline{y(x)}.x(b).P$ is $x(b).P$.

In this setting two process cannot synchronize on a sequence of actions with length greater than one so we cannot have transactional synchronization but we can have multi-party synchronization.




\section{Operational semantic}

\subsection{Early operational semantic with structural congruence}

The semantic of a multi $\pi$ process is labeled transition system such that
\begin{itemize}
  \item 
    the nodes are multi $\pi$ calculus process. The set of node is $\mathbf{P}_{m}$
  \item
    the actions are multi $\pi$ calculus actions. The set of actions is $\mathbf{A}_{m}$, we use $\alpha, \alpha_{1}, \alpha_{2},\cdots $ to range over the set of actions, we use $\sigma, \sigma_{1}, \sigma_{2}, \cdots $ to range over the set $\mathbf{A}_{m}^{+} \cup \{\tau\}$.
  \item
    the transition relations is $\rightarrow\subseteq \mathbf{P}_{m}\times (\mathbf{A}_{m}^{+} \cup \{\tau\})\times \mathbf{P}_{m}$
\end{itemize}

In this case, a label can be a sequence of prefixes, whether in the original $\pi$ calculus a label can be only a prefix. We use the symbol $\cdot$ to denote the concatenation operator.

\begin{definition}
  The \emph{early transition relation with structural congruence} is the smallest relation induced by the rules in table \ref{multipisoloinputearlywith} where $inpSeq$ is a non empty sequence of input actions and $\sigma$ is a sequence of any action.
  \begin{table}
    \begin{tabular}{lll}
	\multicolumn{3}{l}{\line(1,0){415}}\\
	  $\inferrule* [left=\bf{Out}]{
	  }{
	    \overline{x}y.P \xrightarrow{\overline{x}y} P
	  }$
	&
	  $\inferrule* [left=\bf{EInp}]{
	  }{
	    x(y).P \xrightarrow{xz} P\{z/y\}
	  }$
	&
	  $\inferrule* [left=\bf{Tau}]{
	  }{
	    \tau.P \xrightarrow{\tau} P
	  }$
      \\
      \end{tabular}
	\\
      \begin{tabular}{lll}
      \\
	  $\inferrule* [left=\bf{SInpTau}]{
	      P\{y/z\} \xrightarrow{\tau} P^{'}
	  }{
	    \underline{x(z)}.P \xrightarrow{xy} P^{'}
	  }$
	&
	  $\inferrule* [left=\bf{SInp}]{
	      P\{y/z\} \xrightarrow{ab} P^{'}
% 	    \\
% 	      y\notin fn((\nu z) P)
	  }{
	    \underline{x(z)}.P \xrightarrow{xy \cdot ab} P^{'}
	  }$
	&
	  $\inferrule* [left=\bf{SInpSeq}]{
	      P\{y/z\} \xrightarrow{\sigma} P^{'}
	    \\
	      |\sigma|>1
% 	    \\
% 	      y\notin fn((\nu z) P)
	  }{
	    \underline{x(z)}.P \xrightarrow{xy \cdot \sigma} P^{'}
	  }$
      \\
      \end{tabular}
	\\
      \begin{tabular}{lll}
      \\
	  $\inferrule* [left=\bf{Sum}]{
	    P \xrightarrow{\sigma} P^{'}
	  }{
	    P+Q \xrightarrow{\sigma} P^{'}
	  }$
	&
	  $\inferrule* [left=\bf{Cong}]{
	      P\equiv P^{'}
	    \\
	      P^{'} \xrightarrow{\alpha} Q
	  }{
	      P \xrightarrow{\alpha} Q
	  }$
	&
	  $\inferrule* [left=\bf{Res}]{
	      P \xrightarrow{\sigma} P^{'}
	    \\
	      z\notin n(\sigma)
	  }{
	    (\nu z) P \xrightarrow{\sigma} (\nu z) P^{'}
	  }$
      \\
      \end{tabular}
	\\
      \begin{tabular}{ll}
      \\
	  $\inferrule* [left=\bf{Par}]{
	      P \xrightarrow{\sigma} P^{'}
	  }{
	      P|Q \xrightarrow{\sigma} P^{'}|Q
	  }$
	&
	  $\inferrule* [left=\bf{Opn}]{
	      P \xrightarrow{\overline{x}z} P^{'}
	    \\ 
	      z\neq x
	  }{
	      (\nu z)P \xrightarrow{\overline{x}(z)} P^{'}
	  }$
      \\\\
	  $\inferrule* [left=\bf{ECom}]{
	      P \xrightarrow{xy} P^{'}
	    \\
	      Q \xrightarrow{\overline{x}y} Q^{'}
	  }{
	    P|Q \xrightarrow{\tau} P^{'}|Q^{'}
	  }$
	&
	  $\inferrule* [left=\bf{EComSeq}]{
	      P \xrightarrow{xy\cdot \sigma} P^{'}
	    \\
	      Q \xrightarrow{\overline{x}y} Q^{'}
	  }{
	    P|Q \xrightarrow{\sigma} P^{'}|Q^{'}
	  }$
      \\\\\multicolumn{2}{l}{\line(1,0){415}}\\
    \end{tabular}
    \caption{Multi $\pi$ early semantic with structural congruence}
    \label{multipisoloinputearlywith}
  \end{table}
\end{definition}



\begin{example}Multi-party synchronization
  We show an example of a derivation of three processes that synchronize.

  \begin{center}
  $\inferrule* [left=\bf{EComSeq}]{
      \inferrule* [left=\bf{SInp}]{
	\inferrule* [left=\bf{EInp}]{
	}{
	  (x(b).P)\{y/a\} 
	    \xrightarrow{xz} 
	      P\{y/a\}\{z/b\}
	}
      }{
	\underline{x(a)}.(x(b).P) 
	  \xrightarrow{xy \cdot xz} 
	    P\{y/a\}\{z/b\}
      }
    \\
      \inferrule* [left=\bf{Out}]{
      }{
	\overline{x}y.Q 
	  \xrightarrow{\overline{x}y} 
	    Q
      }
  }{
	\underline{x(a)}.x(b).P|\overline{x}y.Q
	  \xrightarrow{xz}
	    P\{y/a\}\{z/b\}|Q
  }$
  \end{center}


  \begin{center}
  $
      \inferrule* [left=\bf{EComSng}]{
	\underline{x(a)}.x(b).P|\overline{x}y.Q
	  \xrightarrow{xz}
	    P\{y/a\}\{z/b\}|Q
	\\
	  \inferrule* [left=\bf{Out}]{
	  }{
	    \overline{x}z.R	
	      \xrightarrow{\overline{x}z} 
		R
	  }
      }{
	(\underline{x(a)}.x(b).P|\overline{x}y.Q)|\overline{x}z.R
	  \xrightarrow{\tau}
	    (P\{y/a\}\{z/b\}|Q)|R
      }
  $
  \end{center}
 
\end{example}

\begin{lemma}\label{lemmastrongsequence}
  If $P\xrightarrow{\sigma} Q$ then only one of the following cases hold: 
  \begin{itemize}
    \item 
      $|\sigma|=1$
    \item
      $|\sigma|>1$, the actions in $\sigma$ are input.
  \end{itemize}
\end{lemma}



\subsection{Late operational semantic with structural congruence}

\begin{definition}
  The \emph{late transition relation with structural congruence} is the smallest relation induced by the rules in table \ref{multipisoloinputlatewith}.
  \begin{table}
    \begin{tabular}{lll}
	\multicolumn{3}{l}{\line(1,0){415}}
	\\\\
	  $\inferrule* [left=\bf{Out}]{
	  }{
	    \overline{x}y.P \xrightarrow{\overline{x}y} P
	  }$
	&
	  $\inferrule* [left=\bf{LInp}]{
% 	    w \notin fn(x(y).P)
	  }{
% 	    x(y).P \xrightarrow{x(w)} P\{w/y\}
	    x(y).P \xrightarrow{x(y)} P
	  }$
	&
	  $\inferrule* [left=\bf{Tau}]{
	  }{
	    \tau.P \xrightarrow{\tau} P
	  }$
      \\
      \end{tabular}
	\\
      \begin{tabular}{ll}
      \\
	  $\inferrule* [left=\bf{SInp}]{
	      P \xrightarrow{\gamma} P^{'}
% 	    \\
% 	      y\notin fn((\nu z) P)
	  }{
	    \underline{x(z)}.P \xrightarrow{x(z) \cdot \gamma} P^{'}
	  }$
	&
	  $\gamma$ is a non empty sequence of inputs
      \\
      \end{tabular}
	\\
      \begin{tabular}{ll}
      \\
	  $\inferrule* [left=\bf{LComSeq}]{
	      P \xrightarrow{x(y)\cdot \sigma} P^{'}
	    \\
	      Q\xrightarrow{\overline{x}z} Q^{'}
  	    \\
   	      bn(\sigma)\cap fn(Q) = \emptyset
	  }{
	    P|Q \xrightarrow{\sigma\{z/y\}} P^{'}\{z/y\}|Q^{'}
	  }$
	&
	  $\inferrule* [left=\bf{LCom}]{
	      P \xrightarrow{x(y)} P^{'}
	    \\
	      Q\xrightarrow{\overline{x}z} Q^{'}
% 	    \\
% 	      z\notin fn(P)
	  }{
	    P|Q \xrightarrow{\tau} P^{'}\{z/y\}|Q^{'}
	  }$
      \\
      \end{tabular}
	\\
      \begin{tabular}{lll}
      \\
	  $\inferrule* [left=\bf{Sum}]{
	    P \xrightarrow{\sigma} P^{'}
	  }{
	    P+Q \xrightarrow{\sigma} P^{'}
	  }$
	&
	  $\inferrule* [left=\bf{Cong}]{
	      P\equiv P^{'}
	    \\
	      P^{'} \xrightarrow{\sigma} Q
	  }{
	      P \xrightarrow{\sigma} Q
	  }$
	&
	  $\inferrule* [left=\bf{Opn}]{
	      P \xrightarrow{\overline{x}z} P^{'}
	    \\ 
	      z\neq x
	  }{
	      (\nu z)P \xrightarrow{\overline{x}(z)} P^{'}
	  }$
      \\\\
	  $\inferrule* [left=\bf{Res}]{
	      P \xrightarrow{\sigma} P^{'}
	    \\
	      z\notin n(\alpha)
	  }{
	    (\nu z) P \xrightarrow{\sigma} (\nu z) P^{'}
	  }$
	&
	  $\inferrule* [left=\bf{Par}]{
	      P \xrightarrow{\sigma} P^{'}
	    \\
	      bn(\sigma)\cap fn(Q)=\emptyset
	  }{
	    P|Q \xrightarrow{\sigma} P^{'}|Q
	  }$
	&
	\\\\\multicolumn{3}{l}{\line(1,0){415}}
    \end{tabular}
    \caption{Multi $\pi$ late semantic with structural congruence}
    \label{multipisoloinputlatewith}
  \end{table}
\end{definition}

\begin{example}Multi-party synchronization
  We show an example of a derivation of three processes that synchronize with the late semantic. The three processes are $\underline{x(a)}.x(b).P$, $\overline{x}y.Q$ and $\overline{x}z.R$. We assume modulo $\alpha$ conversion that:
  \begin{center}
      $a\notin fn(x(b))\cup fn (\underline{x(a)}.x(b).P)$
  \end{center}
  and
  \begin{center}
      $c\notin fn(\overline{x}y.Q)$
  \end{center}

   \begin{center}
  $\inferrule* [left=\bf{LComSeq}]{
      \inferrule* [left=\bf{SInp}]{
	\inferrule* [left=\bf{LInp}]{
	}{
	  x(b).P \xrightarrow{x(b)} P
	}
      }{
	\underline{x(a)}.x(b).P
	  \xrightarrow{x(a) \cdot x(b)} 
	    P
      }
    \\
      \inferrule* [left=\bf{Out}]{
      }{
	\overline{x}y.Q \xrightarrow{\overline{x}y} Q
      }
  }{
	\underline{x(a)}.x(b).P|\overline{x}y.Q
	  \xrightarrow{x(b)}
	    P\{y/a\}|Q
  }$
  \end{center}

  \begin{center}
  $
      \inferrule* [left=\bf{LCom}]{
	\underline{x(a)}.x(b).P|\overline{x}y.Q
	  \xrightarrow{x(b)}
	    P\{y/a\}|Q
	\\
	  \inferrule* [left=\bf{Out}]{
	  }{
	    \overline{x}z.R	
	      \xrightarrow{\overline{x}z} 
		R
	  }
      }{
	(\underline{x(a)}.x(b).P|\overline{x}y.Q)|\overline{x}z.R
	  \xrightarrow{\tau}
	    (P\{y/a\}|Q)\{z/b\}|R=(P\{y/a\}\{z/b\}|Q)|R
      }
  $
  \end{center}
\end{example}




\begin{definition}
  $\twoheadrightarrow$ is the smallest relation induced by the all the rules in table \ref{multipisoloinputlatewith} except $Cong$. 
\end{definition}

\begin{proposition}\label{moveCongDown}
  If $P\xrightarrow{\sigma} Q$ then there exists a process $R$ such that: $R\stackrel{\sigma}{\twoheadrightarrow}Q$ and $P\equiv R$
  \begin{proof}
    We show that we can move the rule $Cong$ down the inference tree of $P\xrightarrow{\sigma} Q$. So a derivation of $P\xrightarrow{\sigma} Q$ can translate into a derivation of $P\xrightarrow{\sigma} Q$ which uses the rule $Cong$ only as its last rule.
    \begin{description}
      \item[$SInp$]\hfill \\
	\begin{center}
	  $\inferrule* [left=\bf{SInp}]{
	    \inferrule* [left=\bf{Cong}]{
		P \equiv R
	      \\
		R \xrightarrow{\gamma} Q
	    }{
	      P \xrightarrow{\gamma} Q
	    }
	  }{
	    \underline{x(z)}.P \xrightarrow{x(z) \cdot \gamma} Q
	  }$
	\end{center}
	become
	\begin{center}
	  $\inferrule* [left=\bf{Cong}]{
	    \inferrule* {
	      P \equiv R
	    }{
	      \underline{x(z)}.P \equiv \underline{x(z)}.R
	    }
	  \\
	    \inferrule* [left=\bf{SInp}]{
	      R \xrightarrow{\gamma} Q
	    }{
	      \underline{x(z)}.R \xrightarrow{x(z) \cdot \gamma} Q
	    }
	  }{
	    \underline{x(z)}.P \xrightarrow{x(z) \cdot \gamma} Q
	  }$
	\end{center}
      \item[$Sum$]\hfill \\
	\begin{center}
	  $\inferrule* [left=\bf{Sum}]{
	    \inferrule* [left=\bf{Cong}]{
		P \equiv R
	      \\
		R \xrightarrow{\gamma} Q
	    }{
	      P \xrightarrow{\gamma} Q
	    }
	  }{
	    P+S \xrightarrow{\gamma} Q
	  }$
	\end{center}
	become
	\begin{center}
	  $\inferrule* [left=\bf{Cong}]{
	    \inferrule* {
	      P \equiv R
	    }{
	      P+S \equiv R+S
	    }
	  \\
	    \inferrule* [left=\bf{Sum}]{
	      R \xrightarrow{\gamma} Q
	    }{
	      R+S \xrightarrow{\gamma} Q
	    }
	  }{
	    P+S \xrightarrow{\gamma} Q
	  }$
	\end{center}
      \item[$Cong$]\hfill \\
	\begin{center}
	  $\inferrule* [left=\bf{Cong}]{
	      P \equiv R
	    \\
	      \inferrule* [left=\bf{Cong}]{
		  R \equiv S
		\\
		  S \xrightarrow{\gamma} Q
	      }{
		R \xrightarrow{\gamma} Q
	      }
	  }{
	    P \xrightarrow{\gamma} Q
	  }$
	\end{center}
	become
	\begin{center}
	  $\inferrule* [left=\bf{Cong}]{
	    \inferrule* {
		P \equiv R
	      \\
		R \equiv S
	    }{
	      P \equiv S
	    }
	  \\
	    S \xrightarrow{\gamma} Q
	  }{
	    P \xrightarrow{\gamma} Q
	  }$
	\end{center}
      \item[$Par$]\hfill \\
	\begin{center}
	  $\inferrule* [left=\bf{Par}]{
	      \inferrule* [left=\bf{Cong}]{
		  P \equiv R
		\\
		  R \xrightarrow{\gamma} Q
	      }{
		P \xrightarrow{\gamma} Q
	      }
	    \\
	      bn(\gamma)\cap fn(S)=\emptyset
	  }{
	    P|S \xrightarrow{\gamma} Q
	  }$
	\end{center}
	become
	\begin{center}
	  $\inferrule* [left=\bf{Cong}]{
	      \inferrule* {
		P \equiv R
	      }{
		P|S \equiv R|S
	      }
	    \\
	      \inferrule* [left=\bf{Par}]{
		  R \xrightarrow{\gamma} Q
		\\
		  bn(\gamma)\cap fn(S)=\emptyset
	      }{
		R|S \xrightarrow{\gamma} Q
	      }
	  }{
	    P|S \xrightarrow{\gamma} Q
	  }$
	\end{center}
      \item[$LComSeq$]\hfill \\
	\begin{center}
	  $\inferrule* [left=\bf{LComSeq}]{
	      \inferrule* [left=\bf{Cong}]{
		  P_{1} \equiv R_{1}
		\\
		  R_{1} \xrightarrow{x(y) \cdot \sigma} Q_{1}
	      }{
		P_{1} \xrightarrow{x(y) \cdot \sigma} Q_{1}
	      }
	    \\
	      \inferrule* [left=\bf{Cong}]{
		  P_{2} \equiv R_{2}
		\\
		  R_{2} \xrightarrow{\overline{x}z} Q_{2}
	      }{
		P_{2} \xrightarrow{\overline{x}z} Q_{2}
	      }
% 	    \\
% 	      bn(\sigma) \cap fn(P_{2}) = \emptyset
	  }{
	    P_{1}|P_{2} \xrightarrow{\gamma\{z/y\}} Q_{1}\{z/y\}|Q_{2}
	  }$
	\end{center}
	become
	\begin{center}
	  $\inferrule* [left=\bf{Cong}]{
	      \inferrule* {
		  P_{1} \equiv R_{1}
		\\
		  P_{2} \equiv R_{2}
	      }{
		P_{1}|P_{2} \equiv R_{1}|R_{2}
	      }
	    \\
	      \inferrule* [left=\bf{LComSeq}]{
		  R_{1} \xrightarrow{x(y) \cdot \sigma} Q_{1}
		\\
		  R_{2} \xrightarrow{\overline{x}z} Q_{2}
	      }{
		R_{1}|R_{2} \xrightarrow{\sigma\{z/y\}} Q_{1}\{z/y\}|Q_{2}
	      }
	  }{
	    P_{1}|P_{2} \xrightarrow{\gamma\{z/y\}} Q_{1}\{z/y\}|Q_{2}
	  }$
	\end{center}
      \item[$LCom$]\hfill \\
	\begin{center}
	  $\inferrule* [left=\bf{LCom}]{
	      \inferrule* [left=\bf{Cong}]{
		  P_{1} \equiv R_{1}
		\\
		  R_{1} \xrightarrow{x(y)} Q_{1}
	      }{
		P_{1} \xrightarrow{x(y)} Q_{1}
	      }
	    \\
	      \inferrule* [left=\bf{Cong}]{
		  P_{2} \equiv R_{2}
		\\
		  R_{2} \xrightarrow{\overline{x}z} Q_{2}
	      }{
		P_{2} \xrightarrow{\overline{x}z} Q_{2}
	      }
	  }{
	    P_{1}|P_{2} \xrightarrow{\tau} Q_{1}\{z/y\}|Q_{2}
	  }$
	\end{center}
	become
	\begin{center}
	  $\inferrule* [left=\bf{Cong}]{
	      \inferrule* {
		  P_{1} \equiv R_{1}
		\\
		  P_{2} \equiv R_{2}
	      }{
		P_{1}|P_{2} \equiv R_{1}|R_{2}
	      }
	    \\
	      \inferrule* [left=\bf{LCom}]{
		  R_{1} \xrightarrow{x(y)} Q_{1}
		\\
		  R_{2} \xrightarrow{\overline{x}z} Q_{2}
	      }{
		R_{1}|R_{2} \xrightarrow{\tau} Q_{1}\{z/y\}|Q_{2}
	      }
	  }{
	    P_{1}|P_{2} \xrightarrow{\tau} Q_{1}\{z/y\}|Q_{2}
	  }$
	\end{center}
      \item[$Res$]\hfill \\
	\begin{center}
	  $\inferrule* [left=\bf{Res}]{
	      \inferrule* [left=\bf{Cong}]{
		  P \equiv R
		\\
		  R \xrightarrow{\gamma} Q
	      }{
		P \xrightarrow{\gamma} Q
	      }
	    \\
	      z\notin n(\gamma)
	  }{
	    (\nu z)P \xrightarrow{\gamma} (\nu z)Q
	  }$
	\end{center}
	become
	\begin{center}
	  $\inferrule* [left=\bf{Cong}]{
	      \inferrule* {
		P \equiv R
	      }{
		(\nu z)P \equiv (\nu z)R
	      }
	    \\
	      \inferrule* [left=\bf{Res}]{
		  R \xrightarrow{\gamma} Q
		\\
		  z\notin n(\gamma)
	      }{
		(\nu z)R \xrightarrow{\gamma} (\nu z)Q
	      }
	  }{
	    (\nu z)P \xrightarrow{\gamma} (\nu z)Q
	  }$
	\end{center}
      \item[$Opn$]\hfill \\
	\begin{center}
	  $\inferrule* [left=\bf{Opn}]{
	      \inferrule* [left=\bf{Cong}]{
		  P \equiv R
		\\
		  R \xrightarrow{\overline{x}y} Q
	      }{
		P \xrightarrow{\overline{x}y} Q
	      }
	    \\
	      y\neq x
	  }{
	    (\nu y)P \xrightarrow{\overline{x}(y)} Q
	  }$
	\end{center}
	become
	\begin{center}
	  $\inferrule* [left=\bf{Cong}]{
	      \inferrule* {
		P \equiv R
	      }{
		(\nu y)P \equiv (\nu y)R
	      }
	    \\
	      \inferrule* [left=\bf{Opn}]{
		  R \xrightarrow{\overline{x}y} Q
		\\
		  y\neq x
	      }{
		(\nu y)R \xrightarrow{\overline{x}(y)} Q
	      }
	  }{
	    (\nu y)P \xrightarrow{\overline{x}(y)} Q
	  }$
	\end{center}
    \end{description}
  \end{proof}
\end{proposition}

In the following lemma we call $\rightarrow_{3}$ the semantic defined in table \ref{multipisoloinputlatewith} and we call $\rightarrow_{2}$ the semantic defined in table \ref{multipisoloinputlatewith} but without rule $Cong$ and with rules: $Ide$ and the commutativity counterpart for rules $Par$, $Sum$, $ECom$, $EComSeq$. 
\begin{lemma}\label{structuralCongruenceElimination}
  If $P\xrightarrow{\gamma}_{3} P^{'}$ then there exist processes $N,N^{'}$ such that $N$ is a normal form, $P\equiv N$, $P^{'} \equiv N^{'}$ and $N\xrightarrow{\gamma}_{2} N^{'}$ without using the rules $Cls$. Also we can assume that the rules $ScpExtPar1$, $ScpExtSum1$ and the rules for commutativity are not used in the derivation of $P\equiv N$ because this rule can be emulated by $ScpExtPar2$, $ScpExtSum2$ and commutativity in $\rightarrow_{2}$.
  \begin{proof}
    The proof of this lemma is similar to those in section \ref{equivalenceOfEarlySemantics}
  \end{proof}
\end{lemma}

In the following section the symbol $\rightarrow$ will refer to the late semantic with structural congruence of multi $\pi$ calculus with strong input which is illustrated in table \ref{multipisoloinputlatewith}. Also we consider a structural congruence without the rules $P|0\equiv 0$ and $P+0\equiv 0$. For the purpose of clarity the rule of structural congruence are repeated in this secition.


\subsection{Low level semantic}
This section contains the definition of an alternative semantic for multi $\pi$. First we define a low level version of the multi $\pi$ calculus(here with strong prefixing on input only), we call this language low multi $\pi$. The low multi $\pi$ is the multi $\pi$ enriched with a marked or intermediate process $*P$:
\begin{center}
   \begin{tabular}{l}
     $P,Q$ ::= $0$ | $\pi.P$ | $P|Q$ | $P+Q$ | $(\nu x) P$ | $A$ | $*P$
   \\\\
     $\pi$ ::= $\overline{x}y$ | $x(y)$ | $\underline{x(y)}$ | $\tau$ 
   \end{tabular}
\end{center}
\begin{definition}
  The low level transition relation is the smallest relation induced by the rules in table \ref{lowleveltransitionrelationinput} in which $P$ stands for a process without mark, $L$ stands for a process with mark and $S$ can stand for both. 
  \begin{table}
    \begin{tabular}{lll}
      	\multicolumn{3}{l}{\line(1,0){415}}\\
	  $\inferrule* [left=\bf{Out}]{
	  }{
	    \overline{x}y.P \stackrel{\overline{x}y}{\longmapsto} P
	  }$
	  &
	  $\inferrule* [left=\bf{EInp}]{
	  }{
	    x(y).P \stackrel{xz}{\longmapsto} P\{z/y\}
	  }$
	  &
	  $\inferrule* [left=\bf{Tau}]{
	  }{
	    \tau.P \stackrel{\tau}{\longmapsto} P
	  }$
      \\\\
	  $\inferrule* [left=\bf{StarInp}]{
	      P \stackrel{xy}{\longmapsto} S^{'}
	  }{
	      *P \stackrel{xy}{\longmapsto} S^{'}
	  }$
	  &
	  $\inferrule* [left=\bf{SInpLow}]{
%	      y\notin fn(P)-\{z\}
	  }{
	    \underline{x(z)}.P \stackrel{xy}{\longmapsto} * P\{y/z\}
	  }$
	  &
	  $\inferrule* [left=\bf{StarEps}]{
	      P \stackrel{\epsilon}{\longmapsto} S^{'}
	  }{
	      *P \stackrel{\epsilon}{\longmapsto} S^{'}
	  }$
      \\
      \end{tabular}
	\\
      \begin{tabular}{lll}
      \\
	  $\inferrule* [left=\bf{Com1}]{
	      P \stackrel{\overline{x}y}{\longmapsto} P^{'}
	    \\
	      Q \stackrel{xy}{\longmapsto} Q^{'}
	  }{
	    P|Q \stackrel{\tau}{\longmapsto} P^{'}|Q^{'}
	  }$
	  &
	  &
      \\\\
	  $\inferrule* [left=\bf{Com2L}]{
	      L_{1} \stackrel{xy}{\longmapsto} L_{2}
	    \\
	      P \stackrel{\overline{x}y}{\longmapsto} Q
	  }{
	    L_{1}|P \stackrel{\epsilon}{\longmapsto} L_{2}|Q
	  }$
	&
	  $\inferrule* [left=\bf{Com2R}]{
	      P \stackrel{\overline{x}y}{\longmapsto} Q
	    \\
	      L_{1} \stackrel{xy}{\longmapsto} L_{2}
	  }{
	    P|L_{1} \stackrel{\epsilon}{\longmapsto} Q|L_{2}
	  }$
	  &
      \\\\
	  $\inferrule* [left=\bf{Com3L}]{
	      P \stackrel{xy}{\longmapsto} L
	    \\
	      Q \stackrel{\overline{x}y}{\longmapsto} Q^{'}
	  }{
	    P|Q \stackrel{\epsilon}{\longmapsto} L|Q^{'}
	  }$
	&
	  $\inferrule* [left=\bf{Com3R}]{
	      Q \stackrel{\overline{x}y}{\longmapsto} Q^{'}	      
	    \\
	      P \stackrel{xy}{\longmapsto} L
	  }{
	    Q|P \stackrel{\epsilon}{\longmapsto} Q^{'}|L
	  }$
	  &
      \\\\
	  $\inferrule* [left=\bf{Com4L}]{
	      L \stackrel{xy}{\longmapsto} P
	    \\
	      Q \stackrel{\overline{x}y}{\longmapsto} Q^{'}
	  }{
	    L|Q \stackrel{\tau}{\longmapsto} P|Q^{'}
	  }$
	  &
	  $\inferrule* [left=\bf{Com4R}]{
	      Q \stackrel{\overline{x}y}{\longmapsto} Q^{'}
	    \\
	      L \stackrel{xy}{\longmapsto} P
	  }{
	    L|Q \stackrel{\tau}{\longmapsto} P|Q^{'}
	  }$
	  &
      \\
      \end{tabular}
	\\
      \begin{tabular}{lll}
      \\
	  $\inferrule* [left=\bf{Res}]{
	      S \stackrel{\gamma}{\longmapsto} S^{'}
	    \\
	      y\notin n(\gamma)
	  }{
	    (\nu y) S \stackrel{\gamma}{\longmapsto} (\nu y) S^{'}
	  }$
	  &
	  $\inferrule* [left=\bf{Opn}]{
	      P \stackrel{\overline{x}y}{\longmapsto} Q
	    \\ 
	      y\neq x
	  }{
	      (\nu y)P \stackrel{\overline{x}(y)}{\longmapsto} Q
	  }$
	  &
	  $\inferrule* [left=\bf{Cong}]{
	      P\equiv P^{'}
	    \\
	      P^{'} \stackrel{\gamma}{\longmapsto} S
	  }{
	      P \stackrel{\gamma}{\longmapsto} S
	  }$
      \\
      \end{tabular}
	\\
      \begin{tabular}{lll}
      \\
	  $\inferrule* [left=\bf{Par1L}]{
	      S \stackrel{\gamma}{\longmapsto} S^{'}
% 	    \\ 
% 	      bn(\gamma)\cap fn(Q)=\emptyset
	  }{
	      S|Q \stackrel{\gamma}{\longmapsto} S^{'}|Q
	  }$
	&
	  $\inferrule* [left=\bf{Par1R}]{
	      S \stackrel{\gamma}{\longmapsto} S^{'}
% 	    \\ 
% 	      bn(\gamma)\cap fn(Q)=\emptyset
	  }{
	      Q|S \stackrel{\gamma}{\longmapsto} Q|S^{'}
	  }$
	  &
	  $\inferrule* [left=\bf{Sum}]{
	    P \stackrel{\gamma}{\longmapsto} S
	  }{
	    P+Q \stackrel{\gamma}{\longmapsto} S
	  }$

      \\\\	\multicolumn{3}{l}{\line(1,0){415}}
    \end{tabular}
    \caption{Low multi $\pi$ early semantic with structural congruence}
    \label{lowleveltransitionrelationinput}
  \end{table}
\end{definition}



\begin{lemma}\label{multiinpconstraintswithmarked}
  For all unmarked processes $P,Q$ and marked processes $L_{1}, L_{2}$.
  \begin{itemize}
    \item
      if $P\stackrel{\alpha}{\longmapsto}L_{1}$ or $L_{1}\stackrel{\alpha}{\longmapsto}L_{2}$ then $\alpha$ can only be an input or an $\epsilon$
    \item
      if $L_{1}\stackrel{\alpha}{\longmapsto}P$ then $\alpha$ is an input or a $\tau$
    \item
      if $P\stackrel{\alpha}{\longmapsto}Q$ then $\alpha$ is not an $\epsilon$
  \end{itemize}
\end{lemma}


  
\begin{definition}\label{low}
  Let $P, Q$ be unmarked processes and $L_{1}, \cdots, L_{k-1}$ marked processes. We define the derivation relation $\rightarrow_{s}$ in the following way:
  \begin{center}
    $\inferrule* [left=\bf{Low}]{
	P \stackrel{\gamma_{1}}{\longmapsto} L_{1} \stackrel{\gamma_{2}}{\longmapsto} L_{2} \cdots L_{k-1} \stackrel{\gamma_{k}}{\longmapsto} Q
      \\
	k\geq 1
    }{
      P \xrightarrow{\gamma_{1} \cdots \gamma_{k}}_{s}  Q
    }$
  \end{center}
  We need to be precise about the concatenation operator $\cdot$ since we have introduced the new label $\epsilon$. Let $a$ be an action such that $a\neq \tau$ and $a\neq \epsilon$ then the following rules hold:
  \begin{center}
      \begin{tabular}{lll}
	  $\epsilon \cdot a = a \cdot \epsilon = a$
	&
	  $\epsilon \cdot \epsilon = \epsilon$
	&
	  $\tau \cdot \epsilon = \epsilon \cdot \tau = \tau$
	\\
	  $\tau \cdot a = a \cdot \tau = a$
	&
	  $\tau \cdot \tau = \tau$
	&
      \end{tabular}
  \end{center}
\end{definition}

\begin{example}Multi-party synchronization
  We show an example of a derivation of three processes that synchronize.
 
  \begin{center}$
    \inferrule* [left=\bf{Par1L}]{
      \inferrule* [left=\bf{Com3L}]{
	\inferrule* [left=\bf{SInpLow}]{
	}{
	  \underline{x(a)}.x(b).P
	    \stackrel{xy}{\longmapsto}
	      *(x(b).P\{y/a\})
	}
      \\
	\inferrule* [left=\bf{Out}]{
	}{
	  \overline{x}y.Q \stackrel{\overline{x}y}{\longmapsto} Q
	}
      }{
	\underline{x(a)}.x(b).P|\overline{x}y.Q
	  \stackrel{\epsilon}{\longmapsto}
	    *(x(b).P\{y/a\})|Q
      }
  }{
	(\underline{x(a)}.x(b).P|\overline{x}y.Q) | \overline{x}z.R
	  \stackrel{\epsilon}{\longmapsto}
	    (*(x(b).P\{y/a\})|Q)|\overline{x}z.R
  }
  $\end{center}

  \begin{center}$
    \inferrule* [left=\bf{Par1L}]{
      \inferrule*[left=\bf{Star}]{
	\inferrule* [left=\bf{EInp}]{
	}{
	  x(b).P\{y/a\} \stackrel{xz}{\longmapsto} P\{y/a\}\{z/b\}
	}
      }{
	*(x(b).P\{y/a\}) \stackrel{xz}{\longmapsto} P\{y/a\}\{z/b\}      
      }
    }{
      *(x(b).P\{y/a\}) | Q \stackrel{xz}{\longmapsto} P\{y/a\}\{z/b\} | Q
    }
  $\end{center}

  \begin{center}$
    \inferrule* [left=\bf{Com4L}]{
      *(x(b).P\{y/a\}) | Q \stackrel{xz}{\longmapsto} P\{y/a\}\{z/b\} | Q
    \\
      \inferrule* [left=\bf{Out}]{
      }{
	\overline{x}z.R	
	  \stackrel{\overline{x}z}{\longmapsto}
	    R
      }
    }{
	(\underline{x(a)}.x(b).P|\overline{x}y.Q)|\overline{x}z.R
	  \stackrel{\tau}{\longmapsto}
	    (P\{y/a\}\{z/b\}|Q)|R
    }
  $\end{center}

\end{example}










\begin{proposition}\label{equivalencehightolowinput}
  Let $\rightarrow$ be the relation defined in table \ref{multipisoloinputearlywith}. If $P\xrightarrow{\sigma} Q$ then there exist $L_{1}, \cdots, L_{k}$ and $\gamma_{1}, \cdots, \gamma_{k+1}$ with $k\geq 0$ such that 
  \begin{center}
    \begin{tabular}{lll}
      $P \stackrel{\gamma_{1}}{\longmapsto} L_{1}  \stackrel{\gamma_{2}}{\longmapsto} L_{2} \cdots L_{k-1} \stackrel{\gamma_{k}}{\longmapsto} L_{k} \stackrel{\gamma_{k+1}}{\longmapsto} Q$ 
    &
      and
    &
      $\gamma_{1} \cdot \ldots \cdot \gamma_{k+1} = \sigma$  
    \end{tabular}
  \end{center}
  \begin{proof}
    The proof is by induction on the depth of the derivation tree of $P\xrightarrow{\sigma} Q$ and by cases on the last rule used in the derivation:
    \begin{description}
      \item[$EInp, Out, Tau$]
	These rules are also in table \ref{lowleveltransitionrelationinput} so we can derive $P \stackrel{\sigma}{\longmapsto}Q$.
      \item[$SInpSeq$]
	    the last part of the derivation tree looks like this:
	    \begin{center}
	      $\inferrule* [left=\bf{SInpSeq}]{
		  P_{1}\{y/z\} \xrightarrow{\sigma} Q
		\\
		  |\sigma|>1
	      }{
		\underline{x(z)}.P_{1} \xrightarrow{xy \cdot \sigma} Q
	      }$	      
	    \end{center}
	    for inductive hypothesis there exist $L_{1}, \cdots, L_{k}$ and $\gamma_{1}, \cdots, \gamma_{k+1}$ with $k\geq 0$ such that 
	    \begin{center}
	      \begin{tabular}{lll}
		$P_{1}\{y/z\} \stackrel{\gamma_{1}}{\longmapsto} L_{1} \stackrel{\gamma_{2}}{\longmapsto} L_{2} \cdots L_{k-1} \stackrel{\gamma_{k}}{\longmapsto} L_{k} \stackrel{\gamma_{k+1}}{\longmapsto} Q$ 
	      &
		and
	      &
		$\gamma_{1} \cdot \ldots \cdot \gamma_{k+1} = \sigma$
	      \end{tabular}
	    \end{center}
	    then a proof of the conclusion follows from:
	    \begin{center}
	      \begin{tabular}{ll}
		$\inferrule* [left=\bf{SInpLow}]{
 		}{
 		  \underline{x(z)}.P_{1} \stackrel{xy}{\longmapsto} *P_{1}\{y/z\}
 		}$
	      &
		$\inferrule* [left=\bf{Star}]{
 		  P_{1}\{y/z\} \stackrel{\gamma_{1}}{\longmapsto} L_{1}
 		}{
 		  *P_{1}\{y/z\} \stackrel{\gamma_{1}}{\longmapsto} L_{1}
 		}$
	      \end{tabular}
	    \end{center}
	    where $Star$ means $StarInp$ or $StarEps$, note that $\gamma_{1}$ is an input or an $epsilon$ because of \ref{lemmastrongsequence}.
	  \item[$SInp$] this case is similar to the previous.
	  \item[$SInpTau$] this case is similar to the previous observing that $xy \cdot \tau = xy$.
	  \item[$Sum$]
	the last part of the derivation tree looks like this:
	\begin{center}
	  $\inferrule* [left=\bf{Sum}]{
	    P_{1} \xrightarrow{\sigma} Q
	  }{
	    P_{1}+P_{2} \xrightarrow{\sigma} Q
	  }$
	\end{center}
	for the inductive hypothesis there exist $L_{1}$, $\cdots$, $L_{k}$ and $\gamma_{1}$, $\cdots$, $\gamma_{k+1}$ with $k\geq 0$ such that 
	\begin{center}
	  \begin{tabular}{lll}
	    $P_{1} \stackrel{\gamma_{1}}{\longmapsto} L_{1}  \stackrel{\gamma_{2}}{\longmapsto} L_{2} \cdots L_{k-1} \stackrel{\gamma_{k}}{\longmapsto} L_{k} \stackrel{\gamma_{k+1}}{\longmapsto} Q$ 
	  &
	    and
	  &
	    $\gamma_{1} \cdot \ldots \cdot \gamma_{k+1} = \sigma$  
	  \end{tabular}
	\end{center}
	A proof of the conclusion is:
	\begin{center}
	  $\inferrule* [left=\bf{Sum}]{
	      P_{1} \stackrel{\gamma_{1}}{\longmapsto} L_{1}
	    }{
	      P_{1}+P_{2} \stackrel{\gamma_{1}}{\longmapsto} L_{1}
	    }
	  $
	\end{center}
      \item[$Cong$] this case is similar to the previous.
      \item[$ECom$] 
	the last part of the derivation tree looks like this:
	\begin{center}
	  $\inferrule* [left=\bf{ECom}]{
	      P_{1} \xrightarrow{xy} P_{1}^{'}
	    \\
	      Q_{1} \xrightarrow{\overline{x}y} Q_{1}^{'}
	  }{
	    P_{1}|Q_{1} \xrightarrow{\tau} P_{1}^{'}|Q_{1}^{'}
	  }$
	\end{center}
	for inductive hypothesis there exist $L_{1}, \cdots, L_{k}$ and $\gamma_{1}, \cdots, \gamma_{k+1}$ with $k\geq 0$ such that 
	\begin{center}
	  \begin{tabular}{lll}
	    $P_{1} \stackrel{\gamma_{1}}{\longmapsto} L_{1}  \stackrel{\gamma_{2}}{\longmapsto} L_{2} \cdots L_{k-1} \stackrel{\gamma_{k}}{\longmapsto} L_{k} \stackrel{\gamma_{k+1}}{\longmapsto} P_{1}^{'}$ 
	  &
	    and
	  &
	    $\gamma_{1} \cdot \ldots \cdot \gamma_{k+1} = xy$
	  \end{tabular}
	\end{center}
	and there exist $R_{1}, \cdots, R_{h}$ and $\delta_{1}, \cdots, \delta_{h+1}$ with $h\geq 0$ such that 
	\begin{center}
	  \begin{tabular}{lll}
	    $Q_{1} \stackrel{\delta_{1}}{\longmapsto} R_{1}  \stackrel{\delta_{2}}{\longmapsto} R_{2} \cdots R_{h-1} \stackrel{\delta_{h}}{\longmapsto} R_{h} \stackrel{\delta_{h+1}}{\longmapsto} Q_{1}^{'}$ 
	  &
	    and
	  &
	    $\delta_{1} \cdot \ldots \cdot \delta_{h+1} = \overline{x}y$
	  \end{tabular}
	\end{center}
	For lemma \ref{multiinpconstraintswithmarked} there cannot be an output action in a transition involving marked processes so $h$ must be $0$ and $Q_{1} \stackrel{\delta_{1}}{\longmapsto} Q_{1}^{'}$ with $\delta_{1}=\overline{x}y$. We can have three different cases now: 
	\begin{description}
	  \item[$\gamma_{1}=xy$]
	    A proof of the conclusion is:
	    \begin{center}
	      $P_{1}|Q_{1} \stackrel{\epsilon}{\longmapsto} L_{1}|Q_{1}^{'}
			      \stackrel{\epsilon}{\longmapsto} L_{2}|Q_{1}^{'}
		  \cdots
				\stackrel{\epsilon}{\longmapsto} L_{k}|Q_{1}^{'}
				\stackrel{\tau}{\longmapsto} P_{1}^{'}|Q_{1}^{'}$	  
	    \end{center}
	    we derive the first transition with rule $Com3L$, whether for the other transition we use the rule $Par1L$.
	  \item[$\gamma_{i}=xy$]
	    A proof of the conclusion is:
	    \begin{center}
	      $
		  P_{1}|Q_{1} \stackrel{\epsilon}{\longmapsto} L_{1}|Q_{1} 
		  \cdots
			      \stackrel{\epsilon}{\longmapsto} L_{i-1}|Q_{1} 
			      \stackrel{\epsilon}{\longmapsto} L_{i}|Q_{1}^{'}
			      \stackrel{\epsilon}{\longmapsto} L_{i+1}|Q_{1}^{'}
		  \cdots 
			      \stackrel{\epsilon}{\longmapsto} L_{k}|Q_{1}^{'}
			      \stackrel{\tau}{\longmapsto} P_{1}^{'}|Q_{1}^{'}$	  
	    \end{center}
	    we derive the transaction $ L_{i-1}|Q_{1} \stackrel{\epsilon}{\longmapsto} L_{i}|Q_{1}^{'}$ with rule $Com2L$, whether for the other transactions  we use the rule $Par1L$.
	  \item[$\gamma_{k+1}=xy$] similar.
	\end{description}
      \item[$Res$]
	the last part of the derivation tree looks like this:
	\begin{center}
	  $\inferrule* [left=\bf{Res}]{
	      P_{1} \xrightarrow{\sigma} Q_{1}
	    \\
	      z\notin n(\sigma)
	  }{
	    (\nu z) P_{1} \xrightarrow{\sigma} (\nu z) Q_{1}
	  }$
	\end{center}
	for the inductive hypothesis there exist $L_{1}, \cdots, L_{k}$ and $\gamma_{1}, \cdots, \gamma_{k+1}$ with $k\geq 0$ such that 
	\begin{center}
	  \begin{tabular}{lll}
	    $P_{1} \stackrel{\gamma_{1}}{\longmapsto} L_{1}  \stackrel{\gamma_{2}}{\longmapsto} L_{2} \cdots L_{k-1} \stackrel{\gamma_{k}}{\longmapsto} L_{k} \stackrel{\gamma_{k+1}}{\longmapsto} Q_{1}$ 
	  &
	    and
	  &
	    $\gamma_{1} \cdot \ldots \cdot \gamma_{k+1} =  \sigma$
	  \end{tabular}
	\end{center}
	We can apply the rule $Res$ to each of the previous transitions because 
	\begin{center}
	  $z\notin n(\sigma)$ implies $z\notin n(\gamma_{i})$ for each $i$
	\end{center}
	and then get a proof of the conclusion:
	\begin{center}
	  $(\nu z)P_{1} \stackrel{\gamma_{1}}{\longmapsto} (\nu z)L_{1}  \stackrel{\gamma_{2}}{\longmapsto} (\nu z)L_{2} \cdots (\nu z)L_{k-1} \stackrel{\gamma_{k}}{\longmapsto} (\nu z)L_{k} \stackrel{\gamma_{k+1}}{\longmapsto} (\nu z)Q_{1}$
	\end{center}
      \item[$Par$] this case is similar to the previous.
      \item[$EComSeq$] 
	the last part of the derivation tree looks like this:
	\begin{center}
	  $\inferrule* [left=\bf{EComSeq}]{
	      P_{1} \xrightarrow{xy \cdot \sigma} P_{1}^{'}
	    \\
	      Q_{1} \xrightarrow{\overline{x}y} Q_{1}^{'}
	  }{
	    P_{1}|Q_{1} \xrightarrow{\sigma} P_{1}^{'}|Q_{1}^{'}
	  }$
	\end{center}
	for inductive hypothesis there exist $L_{1}$, $\cdots$, $L_{k}$ and $\gamma_{1}$, $\cdots$, $\gamma_{k+1}$ with $k\geq 0$ such that 
	\begin{center}
	  \begin{tabular}{lll}
	    $P_{1} \stackrel{\gamma_{1}}{\longmapsto} L_{1}  \stackrel{\gamma_{2}}{\longmapsto} L_{2} \cdots L_{k-1} \stackrel{\gamma_{k}}{\longmapsto} L_{k} \stackrel{\gamma_{k+1}}{\longmapsto} P_{1}^{'}$ 
	  &
	    and
	  &
	    $\gamma_{1} \cdot \ldots \cdot \gamma_{k+1} = xy \cdot \sigma$  
	  \end{tabular}
	\end{center}
	For inductive hypothesis and lemma \ref{multiinpconstraintswithmarked} $Q_{1} \stackrel{\overline{x}y}{\longmapsto} Q_{1}^{'}$. We can have two different cases now depending on where the first $xy$ is:
	\begin{description}
	  \item[$\gamma_{1}=xy$]
	    A proof of the conclusion is:
	    \begin{center}
	      $P_{1}|Q_{1} \stackrel{\epsilon}{\longmapsto} L_{1}|Q_{1}^{'}
			      \stackrel{\gamma_{2}}{\longmapsto} L_{2}|Q_{1}^{'}
		  \cdots
			      \stackrel{\gamma_{k}}{\longmapsto} L_{k}|Q_{1}^{'}
			      \stackrel{\gamma_{k+1}}{\longmapsto} P_{1}^{'}|Q_{1}^{'}$	  
	    \end{center}
	    we derive the first transition with rule $Com3L$, whether for the other transactions we use the rule $Par1L$. Since $\gamma_{1} \cdot \ldots \cdot \gamma_{k+1} = xy \cdot \sigma$ and $\gamma_{1}=xy$ then $\epsilon \cdot \gamma_{2}\cdot \ldots \cdot \gamma_{k+1}=\sigma$
	  \item[$\gamma_{i}=xy$]
	    A proof of the conclusion is:
	    \begin{center}
	      $P_{1}|Q_{1} \stackrel{\epsilon}{\longmapsto} L_{1}|Q_{1} 
		  \cdots
			      \stackrel{\epsilon}{\longmapsto} L_{i-1}|Q_{1} 
			      \stackrel{\epsilon}{\longmapsto} L_{i}|Q_{1}^{'}
			      \stackrel{\gamma_{i+1}}{\longmapsto} L_{i+1}|Q_{1}^{'}
		  \cdots 
			      \stackrel{\gamma_{k}}{\longmapsto} L_{k}|Q_{1}^{'}
			      \stackrel{\gamma_{k+1}}{\longmapsto} P_{1}^{'}|Q_{1}^{'}$	  
	    \end{center}
	    we derive the transition $ L_{i-1}|Q_{1} \stackrel{\epsilon}{\longmapsto} L_{i}|Q_{1}^{'}$ with rule $Com2L$, whether for the other transactions of the premises we use the rule $Par1L$.
	  \item[$\gamma_{k+1}=xy$] cannot happen because $\sigma$ is not empty.
	\end{description}
    \end{description}
  \end{proof}
\end{proposition}






\begin{proposition}
  Let $\rightarrow$ be the relation defined in table \ref{multipisoloinputearlywith}. Let $\alpha$ be an action. If $P \stackrel{\alpha}{\longmapsto} Q$ then $P\xrightarrow{\alpha} Q$.
  \begin{proof}
    The proof is by induction the depth of the derivation of $P \stackrel{\alpha}{\longmapsto} Q$:
    \begin{description}
      \item[$Out, EInp, Tau$]
	These rules are also in table \ref{multipisoloinputearlywith} so we can derive $P\xrightarrow{\alpha} Q$. 
      \item[$Com1$]\hfill \\
	    \begin{center}
	      $\inferrule* [left=\bf{Com1}]{
		  P_{1} \stackrel{xy}{\longmapsto} Q_{1}
		\\
		  P_{2} \stackrel{\overline{x}y}{\longmapsto} Q_{2}
	      }{
		P_{1}|P_{2} \stackrel{\tau}{\longmapsto} Q_{1}|Q_{2}
	      }$ 
	    \end{center}
	    for inductive hypothesis $P_{1} \xrightarrow{xy} Q_{1}$ and $P_{2} \xrightarrow{\overline{x}y} Q_{2}$ so for rule $Com$ $P_{1}|P_{2} \xrightarrow{\tau} Q_{1}|Q_{2}$
      \item[$Sum$]\hfill \\
	    \begin{center}
	      $\inferrule* [left=\bf{Sum}]{
		P_{1} \stackrel{\alpha}{\longmapsto} Q
	      }{
		P_{1}+P_{2} \stackrel{\alpha}{\longmapsto} Q
	      }$ 
	    \end{center}
	    for inductive hypothesis $P_{1} \xrightarrow{\alpha} Q$ and for rule $Sum$ $P_{1}+P_{2} \xrightarrow{\alpha} Q$.
      \item[$Res$] the first transition is:
	    \begin{center}
	      $\inferrule* [left=\bf{Res}]{
		  P_{1} \stackrel{\alpha}{\longmapsto} Q_{1}
		\\
		  z\notin n(\gamma_{1})
	      }{
		(\nu z) P_{1} \stackrel{\alpha}{\longmapsto} (\nu z)Q_{1}
	      }$ 
	    \end{center}		
	    for inductive hypothesis $P_{1} \xrightarrow{\alpha} Q_{1}$ and for rule $Res$ $(\nu z)P_{1} \xrightarrow{\alpha} (\nu z)Q_{1}$.
      \item[$others$] other cases are similar.
    \end{description}	    
  \end{proof}
\end{proposition}


% \subsection{Late operational semantic without structural congruence}
% 
% \begin{definition}
%   The \emph{late transition relation without structural congruence} is the smallest relation induced by the rules in table \ref{multipisoloinputlatewithout}.
%   \begin{table}
%     \begin{tabular}{lll}
% 	\hline\\
%      	  $\inferrule* [left=\bf{Out}]{
% 	  }{
% 	    \overline{x}y.P \xrightarrow{\overline{x}y} P
% 	  }$
% 	&
%      	  $\inferrule* [left=\bf{LInp}]{
% 	    w \notin (fn(P)-\{z\})
% 	  }{
% 	    x(z).P \xrightarrow{x(w)} P\{w/z\}
% 	  }$
% 	&
%      	  $\inferrule* [left=\bf{Tau}]{
% 	  }{
% 	    \tau.P \xrightarrow{\tau} P
% 	  }$
%       \\
%     \end{tabular}
%     \\
%     \begin{tabular}{ll}
%       \\
% 	  $\inferrule* [left=\bf{SInp}]{
% 	      P\{w/z\} \xrightarrow{\sigma} P^{'}
% 	    \\
% 	      w \notin (fn(P)-\{z\})
% 	  }{
% 	    \underline{x(y)}.P \xrightarrow{x(w) \cdot \sigma} P^{'}
% 	  }$
% 	&
% 	  $\sigma$ is a non empty sequence of inputs
%       \\
%     \end{tabular}
%     \\
%     \begin{tabular}{ll}
%       \\
% 	  $\inferrule* [left=\bf{LComSeq}]{
% 	      P \xrightarrow{x(y)\cdot \sigma} P^{'}
% 	    \\
% 	      Q\xrightarrow{\overline{x}z} Q^{'}
% 	    \\
% 	      bn(\sigma)\cap fn(Q)=\emptyset
% 	  }{
% 	    P|Q \xrightarrow{\sigma\{z/y\}} P^{'}\{z/y\}|Q^{'}
% 	  }$
% 	&
% 	  $\inferrule* [left=\bf{LCom}]{
% 	      P \xrightarrow{x(y)} P^{'}
% 	    \\
% 	      Q\xrightarrow{\overline{x}z} Q^{'}
% 	  }{
% 	    P|Q \xrightarrow{\tau} P^{'}\{z/y\}|Q^{'}
% 	  }$
%       \\
%     \end{tabular}
%     \\
%     \begin{tabular}{ll}
%       \\
% 	  $\inferrule* [left=\bf{Sum}]{
% 	    P \xrightarrow{\sigma} P^{'}
% 	  }{
% 	    P+Q \xrightarrow{\sigma} P^{'}
% 	  }$
% 	&
% 	  $\inferrule* [left=\bf{Opn}]{
% 	      P \xrightarrow{\overline{x}z} P^{'}
% 	    \\ 
% 	      z\neq x
% 	    \\
% 	      w\notin fn(P^{'}-\{y\})
% 	  }{
% 	      (\nu z)P \xrightarrow{\overline{x}(z)} P^{'}\{z/w\}
% 	  }$
%       \\\\
% 	  $\inferrule* [left=\bf{Res}]{
% 	      P \xrightarrow{\sigma} P^{'}
% 	    \\
% 	      z\notin n(\alpha)
% 	  }{
% 	    (\nu z) P \xrightarrow{\sigma} (\nu z) P^{'}
% 	  }$
% 	&
% 	  $\inferrule* [left=\bf{Par}]{
% 	      P \xrightarrow{\sigma} P^{'}
% 	    \\
% 	      bn(\sigma)\cap fn(Q)=\emptyset
% 	  }{
% 	    P|Q \xrightarrow{\sigma} P^{'}|Q
% 	  }$
%       \\
%     \end{tabular}
%     \\
%     \begin{tabular}{ll}
%       \\
% 	  $\inferrule* [left=\bf{Close}]{
% 	      P \xrightarrow{x(w)} P^{'}
% 	    \\
% 	      Q \xrightarrow{\overline{x}(w)} Q^{'}
% 	  }{
% 	      P|Q \xrightarrow{\tau} (\nu w)(P^{'}|Q^{'})
% 	  }$
% 	&
% 	  $\inferrule* [left=\bf{CloseSeq}]{
% 	      P \xrightarrow{x(w) \cdot \sigma} P^{'}
% 	    \\
% 	      Q \xrightarrow{\overline{x}(w)} Q^{'}
% 	  }{
% 	      P|Q \xrightarrow{\sigma} (\nu w)(P^{'}|Q^{'})
% 	  }$
%       \\\\\hline
%     \end{tabular}
%     \caption{Multi $\pi$ late semantic without structural congruence}
%     \label{multipisoloinputlatewithout}
%   \end{table}
% \end{definition}
% 
% \begin{example}Multi-party synchronization
%   We show an example of a derivation of three processes that synchronize with the late semantic. The three processes are $\underline{x(a)}.x(b).P$, $\overline{x}y.Q$ and $\overline{x}z.R$. We assume that:
%   \begin{center}
%       $b\notin fn(\underline{x(a)}.x(b).P|\overline{x}y.Q)$
%   \end{center}
% 
%   \begin{center}
%   $
%       \inferrule* [left=\bf{LCom}]{
% 	\underline{x(a)}.x(b).P|\overline{x}y.Q
% 	  \xrightarrow{x(b)}
% 	    P\{y/a\}|Q
% 	\\
% 	  \inferrule* [left=\bf{Out}]{
% 	  }{
% 	    \overline{x}z.R	
% 	      \xrightarrow{\overline{x}z} 
% 		R
% 	  }
%       }{
% 	(\underline{x(a)}.x(b).P|\overline{x}y.Q)|\overline{x}z.R
% 	  \xrightarrow{\tau}
% 	    (P\{y/a\}|Q)\{z/b\}|R
%       }
%   $
%   \end{center}
%   
%   \begin{center}
%   $\inferrule* [left=\bf{LComSeq}]{
%       \inferrule* [left=\bf{SLInp}]{
% 	\inferrule* [left=\bf{LInp}]{
% 	}{
% 	  x(b).P \xrightarrow{x(b)} P
% 	}
%       }{
% 	\underline{x(a)}.x(b).P
% 	  \xrightarrow{x(a) \cdot x(b)} 
% 	    P
%       }
%     \\
%       \inferrule* [left=\bf{Out}]{
%       }{
% 	\overline{x}y.Q 
% 	  \xrightarrow{\overline{x}y} 
% 	    Q
%       }
%   }{
%     \underline{x(a)}.x(b).P
%       | \overline{x}y.Q
%     \xrightarrow{x(b)}
% 	    P\{y/a\}|Q
%   }$
%   \end{center}
% 
% \end{example}
% 
% 
% In the following example we assume that all bound variables have different names.
% 
% 
% \begin{example} Scope intrusion without strong prefixing.
%   \begin{center}
%     $\overline{y}x.P|(\nu x)(y(z).Q) \xrightarrow{\tau} P|(\nu w)(Q\{w/x\}\{x/z\})$
%   \end{center}
%   This cannot be proved without explicit $\alpha$ conversion.
% \end{example}
% 
% \begin{example} Scope intrusion with strong prefixing.
%   \begin{center}
%     $(\nu x)(\nu b)\underline{y(z)}.a(c).P
%       | y(x).Q
%       | a(b).R
%     \xrightarrow{\tau} 
%       ((\nu x^{'})(\nu b^{'})P\{x^{'}/x\}\{x/z\}
% 	|Q)\{b^{'}/b\}\{b/c\} 
% 	| R$
%   \end{center}
%   This cannot be proved without explicit $\alpha$ conversion.
% \end{example}
% 
% \begin{example} Scope extrusion without strong prefixing.
%   $x\notin fn(Q)$  
%   \begin{center}
%   $\inferrule* [left=\bf{Cls}]{
%       \inferrule* [left=\bf{Opn}]{
% 	\inferrule* [left=\bf{Out}]{
% 	}{
% 	  \overline{y}x.P
% 	  \xrightarrow{\overline{y}x} 
% 	    P
% 	}
%       }{
% 	(\nu x)
% 	  (\overline{y}x.P)
% 	  \xrightarrow{\overline{y}(x)} 
% 	    P
%       }
%     \\
%       \inferrule* [left=\bf{LInp}]{
%       }{
% 	y(z).Q
% 	  \xrightarrow{y(x)} 
% 	    Q\{x/z\}
%       }
%   }{
%     (\nu x)
%       (\overline{y}x.P)
%       | y(z).Q 
%     \xrightarrow{\tau} 
%       (\nu x)
%       (	  
% 	P
%       | 
% 	Q\{x/z\}
%       )
%   }$
%   \end{center}
% \end{example}
% 
% \begin{example} Scope extrusion with strong prefixing.
%   \begin{center}
%     $\underline{y(x_{1})}.\underline{a(b)}.y(x_{2}).P
%       | (\nu z_{1})\overline{y}z_{1}.Q 
%       | \overline{a}b.R
%       | (\nu z_{2})\overline{y}z_{2}.S
%     \xrightarrow{\tau} 
%       (\nu z_{2})(
%  	  (\nu z_{1})(
%  	    P\{z_{1}/x_{1}\}\{x_{2}/z_{2}\}
%  	    | Q
%  	  )
%  	  | R
% 	  | S
% 	)$
%   \end{center}
%   \begin{center}
%   $\inferrule* [left=\bf{ClsSeq}]{
%       \inferrule* [left=\bf{SInp}]{
% 	\inferrule* [left=\bf{SInp}]{
% 	  \inferrule* [left=\bf{LInp}]{
% 	  }{
% 	    (y(x_{2}).P)\{x_{1}/z_{1}\}
% 	      \xrightarrow{y(z_{2})} 
% 		P\{x_{1}/z_{1}\}\{x_{2}/z_{2}\}
% 	  }
% 	}{
% 	  (\underline{a(b)}.y(x_{2}).P)\{x_{1}/z_{1}\}
% 	    \xrightarrow{a(b) \cdot y(z_{2})} 
% 	      P\{x_{1}/z_{1}\}\{x_{2}/z_{2}\}
% 	}
%       }{
% 	\underline{y(x_{1})}.\underline{a(b)}.y(x_{2}).P
% 	  \xrightarrow{y(z_{1}) \cdot a(b) \cdot y(z_{2})} 
% 	    P\{x_{1}/z_{1}\}\{x_{2}/z_{2}\}
%       }
%     \\
%       \inferrule* [left=\bf{Opn}]{
%       }{
% 	(\nu z_{1})\overline{y}z_{1}.Q 
% 	  \xrightarrow{y(z_{1})} 
% 	    Q
%       }
%   }{
%     \underline{y(x_{1})}.\underline{a(b)}.y(z_{2}).P
% 	| (\nu z_{1})\overline{y}z_{1}.Q 
%     \xrightarrow{a(b) \cdot y(x_{2})} 
% 	  (\nu z_{1})(
% 	    P\{z_{1}/x_{1}\}\{x_{2}/z_{2}\}
% 	    | Q
% 	  )
%   }$
%   \end{center}
% 
%   \begin{center}
%   $\inferrule* [left=\bf{ClsSeq}]{
%       \underline{y(x_{1})}.\underline{a(b)}.y(x_{2}).P
% 	| (\nu z_{1})\overline{y}z_{1}.Q 
%       \xrightarrow{a(b) \cdot y(z_{2})} 
% 	  (\nu z_{1})(
% 	    P\{z_{1}/x_{1}\}\{x_{2}/z_{2}\}
% 	    | Q
% 	  )
%     \\
%       \inferrule* [left=\bf{Out}]{
%       }{
% 	\overline{a}b.R
% 	  \xrightarrow{\overline{a}b} 
% 	    R
%       }
%   }{
%     \underline{y(x_{1})}.\underline{a(b)}.y(x_{2}).P
% 	| (\nu z_{1})\overline{y}z_{1}.Q 
% 	| \overline{a}b.R
%     \xrightarrow{y(z_{2})} 
%  	  (\nu z_{1})(
%  	    P\{z_{1}/x_{1}\}\{x_{2}/z_{2}\}
%  	    | Q
%  	  )
%  	  | R
% %     \underline{y(x_{1})}.\underline{a(b)}.y(x_{2}).P
% % 	| (\nu z_{1})\overline{y}z_{1}.Q 
% %  	| \overline{a}b.R
% %  	| (\nu z_{2})\overline{y}z_{2}.S
% %     \xrightarrow{\tau} 
% %       (\nu z_{2})[(
% % 	  (\nu z_{1})(
% % 	    P\{z_{1}/x_{1}\}
% % 	    | Q
% % 	  )
% % 	  | R
% % 	)\{z_{2}/x_{2}\}
% % 	| S
% %       ]
%   }$
%   \end{center}
% 
%   \begin{center}
%   $\inferrule* [left=\bf{Cls}]{
%       \underline{y(x_{1})}.\underline{a(b)}.y(x_{2}).P
% 	| (\nu z_{1})\overline{y}z_{1}.Q 
% 	| \overline{a}b.R
%       \xrightarrow{y(z_{2})} 
%  	  (\nu z_{1})(
%  	    P\{z_{1}/x_{1}\}\{x_{2}/z_{2}\}
%  	    | Q
%  	  )
%  	  | R
%     \\
%       \inferrule* [left=\bf{Opn}]{
% 	\inferrule* [left=\bf{Out}]{
% 	}{
% 	  \overline{y}z_{2}.S
% 	    \xrightarrow{\overline{y}z_{2}} 
% 	      S
% 	}
%       }{
% 	(\nu z_{2})\overline{y}z_{2}.S
% 	  \xrightarrow{\overline{y}(z_{2})} 
% 	    S
%       }
%   }{
%     \underline{y(x_{1})}.\underline{a(b)}.y(x_{2}).P
% 	| (\nu z_{1})\overline{y}z_{1}.Q 
% 	| \overline{a}b.R
% 	| (\nu z_{2})\overline{y}z_{2}.S
%     \xrightarrow{\tau} 
% 	(\nu z_{2})(
%  	  (\nu z_{1})(
%  	    P\{z_{1}/x_{1}\}\{x_{2}/z_{2}\}
%  	    | Q
%  	  )
%  	  | R
% 	  | S
% 	)
%   }$
%   \end{center}
% 
% \end{example}

\subsection{Semantic on normal forms}

\begin{definition}
  The \emph{late transition relation for normal forms} is the smallest relation induced by the rules in table \ref{multipiInpNorm}, written $\rightarrow_{n}$. Every process in the head of transition in the premise of a rule in table \ref{multipiInpNorm} is assumed to be in normal form. Also when we write $(\nu \tilde{x})P$ is a normal form, it means that $P$ has no restriction at the top level.
  \begin{table}
    \begin{tabular}{lll}
	\multicolumn{3}{l}{\line(1,0){415}}
	\\\\
	  $\inferrule* [left=\bf{Out}]{
	  }{
	    \overline{x}y.P \xrightarrow{\overline{x}y}_{n} P
	  }$
	&
	  $\inferrule* [left=\bf{Tau}]{
	  }{
	    \tau.P \xrightarrow{\tau}_{n} P
	  }$
	&
	  $\inferrule* [left=\bf{Inp}]{
	      n \geq 0
	  }{
	    \underline{x_{1}(y_{1})}.\; \ldots\; .\underline{x_{n}(y_{n})}.z(w).P
	      \xrightarrow{\widetilde{x(y)}\cdot z(w)}_{n}
		P
	  }$
      \\
      \end{tabular}
	\\\\
      \begin{tabular}{l}
      \\
	  $\inferrule* [left=\bf{LComSeqL}]{
	      (\nu \tilde{a})P \xrightarrow{x(y)\cdot \sigma}_{n} (\nu \tilde{b})P^{'}
	    \\
	      (\nu \tilde{c})Q \xrightarrow{\overline{x}z}_{n} (\nu \tilde{d})Q^{'}
  	    \\
   	      bn(\sigma\{z/y\}) \cap fn(Q) = \emptyset
	  }{
	    (\nu \tilde{a} \tilde{c}) (P|Q) \xrightarrow{\sigma\{z/y\}}_{n} (\nu \tilde{b} \tilde{d})(P^{'}\{z/y\}|Q^{'})
	  }$
      \\\\
	  $\inferrule* [left=\bf{LComL}]{
	      (\nu \tilde{a}) P \xrightarrow{x(y)}_{n} (\nu \tilde{b})P^{'}
	    \\
	      (\nu \tilde{c}) Q\xrightarrow{\overline{x}z}_{n} (\nu \tilde{d})Q^{'}
	  }{
	    (\nu \tilde{a} \tilde{b})(P|Q) \xrightarrow{\tau}_{n} (\nu \tilde{c} \tilde{d})(P^{'}\{z/y\}|Q^{'})
	  }$
%       \\
% 	  $\inferrule* [left=\bf{LComSeq2}]{
% 	      (\nu \tilde{a})P \xrightarrow{\overline{x}z}_{n} (\nu \tilde{b})P^{'}
% 	    \\
% 	      (\nu \tilde{c})Q \xrightarrow{x(y)\cdot \sigma}_{n} (\nu \tilde{d})Q^{'}
%   	    \\
%    	      bn(\sigma)\cap fn(Q) = \emptyset
% 	  }{
% 	    (\nu \tilde{a} \tilde{c}) (P|Q) \xrightarrow{\sigma\{z/y\}}_{n} (\nu \tilde{b} \tilde{d})(P^{'}\{z/y\}|Q^{'})
% 	  }$
% 	\\
% 	  $\inferrule* [left=\bf{LCom2}]{
% 	      (\nu \tilde{a}) P \xrightarrow{\overline{x}z}_{n} (\nu \tilde{b})P^{'}
% 	    \\
% 	      (\nu \tilde{c}) Q\xrightarrow{x(y)}_{n} (\nu \tilde{d})Q^{'}
% 	  }{
% 	    (\nu \tilde{a} \tilde{b})(P|Q) \xrightarrow{\tau}_{n} (\nu \tilde{c} \tilde{d})(P^{'}\{z/y\}|Q^{'})
% 	  }$
      \\
      \end{tabular}
	\\\\
      \begin{tabular}{ll}
      \\
	  $\inferrule* [left=\bf{SumL}]{
	      (\nu \tilde{a}) P \xrightarrow{\sigma}_{n} (\nu \tilde{b})P^{'}
	    \\
	      (\nu \tilde{c})Q\; n.\; f.
	    \\
	      \tilde{c} \cap n(\sigma) = \emptyset
	  }{
	    (\nu \tilde{a} \tilde{c}) (P+Q) \xrightarrow{\sigma}_{n} (\nu \tilde{b} \tilde{c})P^{'}
	  }$
	&
% 	  $\inferrule* [left=\bf{Sum2}]{
% 	      (\nu \tilde{a})P\; n.\; f.
% 	    \\
% 	      (\nu \tilde{b}) Q \xrightarrow{\sigma}_{n} (\nu \tilde{c})Q^{'}
% 	  }{
% 	    (\nu \tilde{a} \tilde{c}) (P+Q) \xrightarrow{\sigma}_{n} (\nu \tilde{b} \tilde{c})Q^{'}
% 	  }$
      \\
      \end{tabular}
	\\
      \begin{tabular}{ll}
      \\
	  $\inferrule* [left=\bf{Res}]{
	      P \xrightarrow{\sigma}_{n} P^{'}
	    \\
	      z\notin n(\sigma)
	  }{
	    (\nu z) P \xrightarrow{\sigma}_{n} (\nu z) P^{'}
	  }$
	  &
	  $\inferrule* [left=\bf{Opn}]{
	      P \xrightarrow{\overline{x}z}_{n} P^{'}
	    \\ 
	      z\neq x
	  }{
	      (\nu z)P \xrightarrow{\overline{x}(z)}_{n} P^{'}
	  }$
      \\
      \end{tabular}
	\\
      \begin{tabular}{l}
      \\
	  $\inferrule* [left=\bf{ParL}]{
	      (\nu \tilde{a}) P \xrightarrow{\sigma}_{n} (\nu \tilde{b})P^{'}
	    \\
	      bn(\sigma)\cap fn(Q)=\emptyset
	    \\
	      (\nu \tilde{c})Q\; n.f.
	    \\
	      \tilde{c} \cap n(\sigma) = \emptyset
	  }{
	    (\nu \tilde{a} \tilde{c})(P|Q) \xrightarrow{\sigma}_{n} (\nu \tilde{b} \tilde{c})(P^{'}|Q)
	  }$
      \\
% 	  $\inferrule* [left=\bf{Par2}]{
% 	      (\nu \tilde{b}) Q \xrightarrow{\sigma}_{n} (\nu \tilde{c})Q^{'}
% 	    \\
% 	      bn(\sigma)\cap fn((\nu \tilde{a})P)=\emptyset
% 	    \\
% 	      (\nu \tilde{a})P\; n.f.
% 	    \\
% 	      \tilde{a} \cap n(\sigma) = \emptyset
% 	  }{
% 	    (\nu \tilde{a} \tilde{c})(P|Q) \xrightarrow{\sigma}_{n} (\nu \tilde{b} \tilde{c})(P|Q^{'})
% 	  }$
	\\\\\multicolumn{1}{l}{\line(1,0){415}}
    \end{tabular}
    \caption{Multi $\pi$ calculus late semantic for normal forms. Every process in the head of a transition in the premise of a rule is in normal form. The restrictions can be empty. When we write $(\nu \tilde{x})P$ n.f. it means that $P$ has no restriction at the top level. Commutative counterpart of rule $ParL$, $SumL$, $EComL$ are omitted. Also in rules for prefix $P$ is a normal form}
    \label{multipiInpNorm}
  \end{table}
\end{definition}


\section{Strong bisimilarity and equivalence}

\subsection{Strong bisimilarity}

In the following $\widetilde{x(y)}=x_{1}(y_{1}) \cdot \ldots \cdot x_{n}(y_{n})$ and $\tilde{x}= x_{1}\cdot \ldots \cdot x_{n}$. In the following section $\rightarrow$ and $\rightarrow_{3}$ are the late semantic with structural congruence of multi $\pi$ calculus with strong prefixing.

\begin{definition}\label{strongLateBisimulation}
  A \emph{strong bisimulation} is a symmetric binary relation $\mathbf{S}$ on multi $\pi$ processes such that for all $P\mathbf{S}Q$:
   \begin{itemize}
     \item 
       $P \xrightarrow{\alpha} P^{'}$, $bn(\alpha)$ is fresh and $\alpha$ is not an input nor a sequence of inputs then there exists some $Q^{'}$ such that $Q \xrightarrow{\alpha} Q^{'}$ and $P^{'}\mathbf{S}Q^{'}$
     \item
       $P \xrightarrow{\widetilde{x(y)}} P^{'}$ where $\gamma$ is a possibly empty sequence of inputs and $\tilde{y}$ is fresh then there exists some $Q^{'}$ such that $Q \xrightarrow{\widetilde{x(y)}} Q^{'}$ and for all $\tilde{w}$, $P^{'}\{\tilde{w}/\tilde{y}\}\mathbf{S}Q^{'}\{\tilde{w}/\tilde{y}\}$
  \end{itemize}
  $P$ and $Q$ are strongly bisimilar, written $P \sim_{L} Q$, if they are related by a strong bisimulation.
\end{definition}


Is this definition a proper extension of the one in \cite{parrow}? The only way to tell is by showing some example of process that we intuitively want to be bisimilar.
\begin{example}:
  \begin{center}
    \begin{tabular}{lll}
	$P=\underline{a(u)}.b(v).0$
      &
	$P \sim_{L} Q$ 
      &
	$\underline{a(x)}.b(v).(\nu y)\overline{y}u.0=Q$
    \end{tabular}
  \end{center}
  This is because for all $u\in \mathbf{N}-\{b\}$ and for all $v\in \mathbf{N}-\{u\}$: $P \xrightarrow{a(u) \cdot b(v)} 0$. For all $x\in \mathbf{N}-\{b,u\}$ and for all $v\in \mathbf{N}-\{u,x,y\}$:$Q \xrightarrow{a(x) \cdot b(v)} 0$. Taking $z,w$ fresh in $P$ and $Q$ means: $z,w\in \mathbf{N}-\{a,b,u\}$, so both $P$ and $Q$ can make the transition $\xrightarrow{a(z) \cdot b(w)}$ and arrive to $0$.
\end{example}

% \begin{definition}
%   We say that $P$ and $Q$ are \emph{strongly bisimular up to structural congruence} written $\sim_{L}^{\equiv}$ if:
%    \begin{itemize}
%      \item 
%        $P \xrightarrow{\alpha} P^{'}$, $bn(\alpha)$ is fresh and $\alpha$ is not an input nor a sequence of inputs then there exist processes $Q^{'},Q^{''},P^{''}$ such that $Q \xrightarrow{\alpha} Q^{'}$ and $P^{'} \equiv P^{''} \sim_{L} Q^{''} \equiv Q^{'}$
%      \item
%        $P \xrightarrow{x_{1}(y_{1}) \cdot \ldots \cdot x_{n}(y_{n})} P^{'}$ where $\gamma$ is a possibly empty sequence of inputs and $y_{1} \cdot \ldots \cdot y_{n}$ is fresh then there exist processes $Q^{'}, Q^{''}, P^{''}$ such that $Q \xrightarrow{x_{1}(y_{1}) \cdot \ldots \cdot x_{n}(y_{n})} Q^{'}$ and for all $w_{1}\cdot \ldots \cdot w_{n}$, $P^{'}\{w_{1}/y_{1}, \ldots, w_{n}/y_{n}\} \equiv P^{''} \sim_{L} Q^{''} \equiv Q^{'}\{w_{1}/y_{1}, \ldots, w_{n}/y_{n}\}$
%   \end{itemize}
% \end{definition}

% \begin{proposition}
%   $P \sim_{L}_{E}^{up} Q$ imply $P \sim_{L}_{E} Q$.
%   \begin{proof}
%     Let $\mathbf{S}$ be a bisimulation up to $\sim_{L}_{E}$ such that $P \mathbf{S} Q$. It can be proved that $\sim_{L}_{E} \mathbf{S} \sim_{L}_{E}$ is a bisimulation: let $A \sim_{L}_{E} B \mathbf{S} C \sim_{L}_{E} D$
%     \begin{center}
%       \begin{tabular}{l}
% 	$A \xrightarrow{\gamma} A^{'}$ $\wedge$ $A \sim_{L}_{E} B$ $\wedge$ definition \ref{strongEarlyBisimulation} 
%       
% 	$\Rightarrow \exists B^{'}:$ $B \xrightarrow{\gamma} B^{'}$ $\wedge$ $A^{'}\sim_{L}_{E} B^{'}$
%       \\
% 	$B \mathbf{S} C$ $\wedge$ definition \ref{strongEarlyBisimulationUpTo}
%       
% 	$\Rightarrow \exists C^{'} C^{''} B^{''}:$ $C \xrightarrow{\gamma} C^{'}$ $\wedge$ $B^{'} \sim_{L}_{E} B^{''} \mathbf{S} C^{''} \sim_{L}_{E} C^{'}$
%       \\
% 	$C \xrightarrow{\gamma} C^{'}$ $\wedge$ $C \sim_{L}_{E} D$ $\wedge$ definition \ref{strongEarlyBisimulation} 
%       
% 	$\Rightarrow \exists D^{'}:$ $D \xrightarrow{\gamma} D^{'}$ $\wedge$ $C^{'}\mathbf{S} D^{'}$
%       \\
% 	$A^{'} \sim_{L}_{E} B^{'} \sim_{L}_{E} B^{''} \mathbf{S} C^{''} \sim_{L}_{E} C^{'} \sim_{L}_{E} D^{'}$ $\wedge$ transitivity of $\sim_{L}_{E}$
% 	$\Rightarrow$ $A^{'} \sim_{L}_{E} B^{''} \mathbf{S} C^{''} \sim_{L}_{E} D^{'}$
%       \end{tabular}
%     \end{center}
%     It is easy to see that the symmetric also holds.
%   \end{proof}
% \end{proposition}

\begin{definition}\label{strongLateBisimulationUpTo}
  Let $\mathbf{R}$ be a strong late bisimulation. A \emph{strong bisimulation up to $\mathbf{R}$} is a symmetric binary relation $\mathbf{S}$ on multi $\pi$ processes such that for all $P\mathbf{S}Q$:
   \begin{itemize}
     \item 
       $P \xrightarrow{\alpha} P^{'}$, $bn(\alpha)$ is fresh and $\alpha$ is not an input nor a sequence of inputs then there exist processes $Q^{'}, Q^{''}, P^{''}$ such that $Q \xrightarrow{\alpha} Q^{'}$ and $P^{'} \mathbf{R} P^{''} \mathbf{S} Q^{''} \mathbf{R} Q^{'}$
     \item
       $P \xrightarrow{x_{1}(y_{1}) \cdot \ldots \cdot x_{n}(y_{n})} P^{'}$ where $\gamma$ is a possibly empty sequence of inputs and $y_{1} \cdot \ldots \cdot y_{n}$ is fresh then there exists some $Q^{'}$ such that $Q \xrightarrow{x_{1}(y_{1}) \cdot \ldots \cdot x_{n}(y_{n})} Q^{'}$ and for all $w_{1}\cdot \ldots \cdot w_{n}$  $P^{'}\{w_{1}/y_{1}, \ldots, w_{n}/y_{n}\} \mathbf{R} \mathbf{S} \mathbf{R} Q^{'}\{w_{1}/y_{1}, \ldots, w_{n}/y_{n}\}$
  \end{itemize}
  $P$ and $Q$ are strongly bisimilar up to $\mathbf{R}$, written $P \sim_{L}^{\mathbf{R}} Q$, if they are related by a strong bisimulation up to $\mathbf{R}$.
\end{definition}


\begin{proposition}
  $P \sim_{L}^{\mathbf{R}} Q$ imply $P \sim_{L} Q$.
   \begin{proof}
     Let $\mathbf{S}$ be a bisimulation up to $\mathbf{R}$ such that $P \mathbf{S} Q$. It can be proved that $\mathbf{R} \mathbf{S} \mathbf{R}$ is a bisimulation: let $A \mathbf{R} B \mathbf{S} C \mathbf{R} D$ and let $\gamma$ be a non input action
     \begin{center}
       \begin{tabular}{l}
 	$A \xrightarrow{\gamma} A^{'}$ $\wedge$ $A \mathbf{R} B$ $\wedge$ definition \ref{strongLateBisimulation}
       
 	$\Rightarrow \exists B^{'}:$ $B \xrightarrow{\gamma} B^{'}$ $\wedge$ $A^{'} \mathbf{R} B^{'}$
       \\
 	$B \mathbf{S} C$ $\wedge$ definition \ref{strongLateBisimulationUpTo}
       
 	$\Rightarrow \exists C^{'} C^{''} B^{''}:$ $C \xrightarrow{\gamma} C^{'}$ $\wedge$ $B^{'} \mathbf{R} B^{''} \mathbf{S} C^{''} \mathbf{R} C^{'}$
       \\
 	$C \xrightarrow{\gamma} C^{'}$ $\wedge$ $C \mathbf{R} D$ $\wedge$ definition \ref{strongLateBisimulation} 
       
	$\Rightarrow \exists D^{'}:$ $D \xrightarrow{\gamma} D^{'}$ $\wedge$ $C^{'}\mathbf{R} D^{'}$
       \\
	$A^{'} \mathbf{R} B^{'} \mathbf{R} B^{''} \mathbf{S} C^{''} \mathbf{R} C^{'} \mathbf{R} D^{'}$ $\wedge$ transitivity of $\mathbf{R}$
	$\Rightarrow$ $A^{'} \mathbf{R} B^{''} \mathbf{S} C^{''} \mathbf{R} D^{'}$
      \end{tabular}
    \end{center}
    It is easy to see that the symmetric also holds. For the other case: let $x_{1}(y_{1}) \cdot \ldots \cdot x_{n}(y_{n})= \tilde{x}(\tilde{y})$
     \begin{center}
       \begin{tabular}{l}
 	$A \xrightarrow{\tilde{x}(\tilde{y})} A^{'}$ $\wedge$ $A \mathbf{R} B$ $\wedge$ definition \ref{strongLateBisimulation}
       
 	$\Rightarrow \exists B^{'}:$ $B \xrightarrow{\tilde{x}(\tilde{y})} B^{'}$ and for all $\tilde{w}:A^{'}\{\tilde{w}/\tilde{y}\} \mathbf{R} B^{'}\{\tilde{w}/\tilde{y}\}$
       \\
 	$B \mathbf{S} C$ $\wedge$ definition \ref{strongLateBisimulationUpTo}
       
 	$\Rightarrow \exists C^{'}:$ $C \xrightarrow{\tilde{x}(\tilde{y})} C^{'}$ $\wedge$ $B^{'}\{\tilde{w}/\tilde{y}\} \mathbf{R} \mathbf{S} \mathbf{R} C^{'}\{\tilde{w}/\tilde{y}\}$
       \\
 	$C \xrightarrow{\tilde{x}(\tilde{y})} C^{'}$ $\wedge$ $C \mathbf{R} D$ $\wedge$ definition \ref{strongLateBisimulation} 
       
	$\Rightarrow \exists D^{'}:$ $D \xrightarrow{\tilde{x}(\tilde{y})} D^{'}$ $\wedge$ $C^{'}\{\tilde{w}/\tilde{y}\} \mathbf{R} D^{'}\{\tilde{w}/\tilde{y}\}$
       \\
	$A^{'}\{\tilde{w}/\tilde{y}\} \mathbf{R} B^{'}\{\tilde{w}/\tilde{y}\} \mathbf{R} \mathbf{S} \mathbf{R} C^{'}\{\tilde{w}/\tilde{y}\} \mathbf{R} D^{'}\{\tilde{w}/\tilde{y}\}$ $\wedge$ transitivity of $\mathbf{R}$
	$\Rightarrow$ $A^{'}\{\tilde{w}/\tilde{y}\} \mathbf{R} \mathbf{S} \mathbf{R} D^{'}\{\tilde{w}/\tilde{y}\}$
      \end{tabular}
    \end{center}
    It is easy to see that the symmetric also holds.
   \end{proof}
\end{proposition}

\begin{lemma}\label{structuralCongruenceIsALateBisimulation}
  Structural congruence is a strong late bisimulation.
  \begin{proof}
    Let $P\equiv Q$. If $P\xrightarrow{\sigma}P^{'}$ then for symmetry of $\equiv$ and rule $Cong$: $Q\xrightarrow{\sigma}P^{'}$. If $Q\xrightarrow{\sigma}Q^{'}$ then for rule $Cong$: $P\xrightarrow{\sigma}Q^{'}$
  \end{proof}
\end{lemma}


\begin{lemma}\label{lateBisimulationAbsorbesStructuralCongruence}
  $P \sim_{L} Q$ and $P \equiv R$ then $R \sim_{L} Q$.
  \begin{proof}
    $P\equiv R$ implies for lemma \ref{structuralCongruenceIsALateBisimulation}: $P \sim_{L} R$. $P \sim_{L} Q$ and $P \sim_{L} R$ imply for transitivity and symmetry that $R\sim_{L}Q$
  \end{proof}
\end{lemma}

% \begin{definition}\label{strongLateBisimulationUpToStructuralCongruence}
%     A \emph{strong bisimulation up to structural congruence} is a symmetric binary relation $\mathbf{S}$ on multi $\pi$ processes such that for all $P\mathbf{S}Q$:
%    \begin{itemize}
%      \item 
%        $P \xrightarrow{\alpha} P^{'}$, $bn(\alpha)$ is fresh and $\alpha$ is not an input nor a sequence of inputs then there exist processes $Q^{'}, Q^{''}, P^{''}$ such that $Q \xrightarrow{\alpha} Q^{'}$ and $P^{'} \mathbf{R} P^{''} \mathbf{S} Q^{''} \mathbf{R} Q^{'}$
%      \item
%        $P \xrightarrow{x_{1}(y_{1}) \cdot \ldots \cdot x_{n}(y_{n})} P^{'}$ where $\gamma$ is a possibly empty sequence of inputs and $y_{1} \cdot \ldots \cdot y_{n}$ is fresh then there exists some $Q^{'}$ such that $Q \xrightarrow{x_{1}(y_{1}) \cdot \ldots \cdot x_{n}(y_{n})} Q^{'}$ and for all $w_{1}\cdot \ldots \cdot w_{n}$  $P^{'}\{w_{1}/y_{1}, \ldots, w_{n}/y_{n}\} \mathbf{R} \mathbf{S} \mathbf{R} Q^{'}\{w_{1}/y_{1}, \ldots, w_{n}/y_{n}\}$
%   \end{itemize}
%   $P$ and $Q$ are strongly bisimilar up to $\mathbf{R}$, written $P \sim_{L}^{\mathbf{R}} Q$, if they are related by a strong bisimulation up to $\mathbf{R}$.
% \end{definition}


\begin{definition}\label{strongEarlyBisimulationUpToRestriction}
  A \emph{strong bisimulation up to restriction} is a symmetric binary relation $\mathbf{S}$ on multi $\pi$ processes such that for all $P$ and $Q$: $P \mathbf{S} Q$ imply
  \begin{itemize}
    \item
      for all $w\notin (fn(P)\cup fn(Q))$: $P\{w/z\} \sim_{L} Q\{w/z\}$
    \item
      if $P \xrightarrow{\overline{x}y} P^{'}$ then there exists $Q^{'}$ such that $Q \xrightarrow{\overline{x}y} Q^{'}$ and $P^{'} \mathbf{S} Q^{'}$
    \item
      if $y\notin n(P,Q)$ and $P \xrightarrow{x(y)} P^{'}$ then there exists $Q^{'}$ such that $Q \xrightarrow{x(y)} Q^{'}$ and for all $v$: $P^{'}\{v/y\} \mathbf{S} Q^{'}\{v/y\}$
    \item
      if $y\notin n(P,Q)$ and $P \xrightarrow{\overline{x}(y)} P^{'}$ then there exists $Q^{'}$ such that $Q \xrightarrow{\overline{x}(y)} Q^{'}$ and $P^{'} \mathbf{S} Q^{'}$
    \item
      if $P \xrightarrow{\tau} P^{'}$ then for some $Q^{'}$: $Q \xrightarrow{\tau} Q^{'}$ and either $P^{'} \mathbf{S} Q^{'}$ or for some $P^{''},Q^{''}$ and $w$: $P^{'}\equiv (\nu w)P^{''}$, $Q^{'}\equiv (\nu w)Q^{''}$ and $P^{''} \mathbf{S} Q^{''}$
  \end{itemize}
  Two processes $P,Q$ are \emph{strongly late bisimilar up to restriction}, written $P \sim_{L}^{\nu} Q$, if they are related by a strong early bisimulation up to restriction.
\end{definition}

\begin{lemma}
  A strong bisimulation up to restriction is a strong bisimulation.
\end{lemma}


\begin{lemma}\label{outPreservesBisimulation}
  $\sim_{L}$ is preserved by output prefix.
  \begin{proof}
    Let $P \sim_{L} Q$. There are two cases:
    \begin{description}
      \item[$Out$]:
	$\overline{x}y.P \xrightarrow{\overline{x}y} P \sim_{L} Q \stackrel{\overline{x}y}{\leftarrow} \overline{x}y.Q$. 
      \item[$Cong$]:
	\begin{center}
	  $\inferrule* [left=\bf{Cong}]{
	      \inferrule* [left=\bf{CongOut}]{
		P \equiv R
	      }{
		\overline{x}y.P \equiv \overline{x}y.R
	      }
	    \\
	      \overline{x}y.R \xrightarrow{\overline{x}y} R
	  }{
	    \overline{x}y.P \xrightarrow{\overline{x}y} R
	  }$
	\end{center}
	for rule $Out$: $\overline{x}y.P \xrightarrow{\overline{x}y} P$. $P\equiv R$ imply $P\sim_{L}R$.
    \end{description}
  \end{proof}
\end{lemma}

\begin{lemma}\label{tauPreservesBisimulation}
  $\sim_{L}$ is preserved by $\tau$ prefix.
  \begin{proof}
    Let $P \sim_{L} Q$. There are two cases:
    \begin{description}
      \item[$Tau$]:
	$\tau.P \xrightarrow{\tau} P \sim_{L} Q \stackrel{\tau}{\leftarrow} \tau.Q$. 
      \item[$Cong$]:
	\begin{center}
	  $\inferrule* [left=\bf{Cong}]{
	      \inferrule* [left=\bf{CongTau}]{
		P \equiv R
	      }{
		\tau.P \equiv \tau.R
	      }
	    \\
	      \tau.R \xrightarrow{\tau} R
	  }{
	    \tau.P \xrightarrow{\tau} R
	  }$
	\end{center}
	for rule $Out$: $\tau.P \xrightarrow{\tau} P$. $P\equiv R$ imply $P\sim_{L}R$.
    \end{description}
  \end{proof}
\end{lemma}

\begin{lemma}\label{bisimulationPreservedByRestrictionSubLemma}
  Let $y$ be fresh in $P,Q$. If for all $w$: $P\{w/y\} \sim_{L} Q\{w/y\}$ then $((\nu x) P)\{x/y\} \equiv_{\alpha} Res(\sim_{L}) \equiv_{\alpha} ((\nu x) Q)\{x/y\}$.
  \begin{proof}
    for all $w$: $P\{w/y\} \sim_{L} Q\{w/y\}$ imply $P\{x/y\} \sim_{L} Q\{x/y\}$ which imply $P\{x/y\} (xy) \sim_{L} Q\{x/y\} (xy)$ where $(xy)$ is the permutation that swaps $x$ with $y$ and does not change other names. $P\{x/y\} (xy) \sim_{L} Q\{x/y\} (xy)$ imply 
    \begin{center}
      $P\{y/x\} \equiv_{\alpha} P\{x/y\} (xy) \sim_{L} Q\{x/y\} (xy) \equiv_{\alpha} Q\{y/x\}$ 
    \end{center}
    which imply $P\{y/x\} \sim_{L}  Q\{y/x\}$. So $P\{y/x\} \sim_{L}  Q\{y/x\}$ and 
    \begin{center}
      $(\nu y)(P\{y/x\}) Res(\sim_{L}) (\nu y)(Q\{y/x\})$ 
    \end{center}
    which imply 
    \begin{center}
      $((\nu y)P\{y/x\})\{x/y\} Res(\sim_{L}) ((\nu y)Q\{y/x\})\{x/y\}$ 
    \end{center}
    and 
    \begin{center}
      $((\nu x)P)\{x/y\} \equiv_{\alpha} ((\nu y)P\{y/x\})\{x/y\} Res(\sim_{L}) ((\nu y)Q\{y/x\})\{x/y\} \equiv_{\alpha} ((\nu x)Q)\{x/y\}$
    \end{center}
  \end{proof}
\end{lemma}


\begin{lemma}\label{restrictionPreservesBisimulation}
  Restriction preserves strong late bisimulation.
  \begin{proof}
    We prove that 
    \begin{center}
      $Res(\sim_{L})=\{((\nu x)P, (\nu x)Q): P\sim_{L} Q\}\cup \sim_{L}$
    \end{center}
    is a late bisimulation up to $\alpha$ equivalence. For lemma \ref{structuralCongruenceElimination} and reflexivity of structural congruence we can assume that every transition that starts from $(\nu x)P$ is in the form: $(\nu x)P \equiv S$, $S \xrightarrow{\gamma}S$ and this derivation does not used the rule $Cong$ but can use the commutative counterpart of the rules for sum, parallel and communication. So we can proceed by induction on the sum of the sizes of the derivations of $(\nu x)P\equiv S$ and $S \xrightarrow{\gamma}S$. Then by cases on the last pair of rules used:
    \begin{description}
      \item[$(CongRes, Res)$]:
	\begin{center}
	  \begin{tabular}{ll}
	      $\inferrule* [left=\bf{CongRes}]{
		  P \equiv P_{1}
	      }{
		(\nu x)P \equiv (\nu x)P_{1}
	      }$
	    &
	      $\inferrule* [left=\bf{Res}]{
		  P_{1} \xrightarrow{\gamma} P_{1}^{'}
		\\
		  x\notin n(\gamma)
	      }{
		(\nu x)P_{1} \xrightarrow{\gamma} (\nu x)P_{1}^{'}
	      }$	      
	  \end{tabular}
	\end{center}
	$P \equiv P_{1}$ and $P_{1} \xrightarrow{\gamma} P_{1}^{'}$ imply $P \xrightarrow{\gamma} P_{1}^{'}$. $P \sim_{L}$ and $P \xrightarrow{\gamma} P_{1}^{'}$ imply $Q \xrightarrow{\gamma} Q_{1}^{'}$ and the conditions of bisimulation on $P_{1}^{'}$ and $Q_{1}^{'}$ are met. For rule $Res$: $(\nu x)Q \xrightarrow{\gamma} (\nu x)Q_{1}^{'}$. There are three cases now:
	\begin{itemize}
	  \item 
	    $\gamma$ is a sequence of bound outputs: $\widetilde{\overline{y}(z)}$. In this case for all $\tilde{w}$ such that $x\notin \tilde{w}$: $P_{1}^{'}\{\tilde{w}/\tilde{z}\}\sim_{L} Q_{1}^{'}\{\tilde{w}/\tilde{z}\}$ which imply $(\nu x)(P_{1}^{'}\{\tilde{w}/\tilde{z}\}) Res(\sim_{L}) (\nu x)(Q_{1}^{'}\{\tilde{w}/\tilde{z}\})$. For definition of substitution: $((\nu x)P_{1}^{'})\{\tilde{w}/\tilde{z}\} Res(\sim_{L}) ((\nu x)Q_{1}^{'})\{\tilde{w}/\tilde{z}\}$.
	  \item 
	    $\gamma$ is a sequence of bound outputs: $\widetilde{\overline{y}(z)}$. In this case for all $\tilde{w}$ such that $x\in \tilde{w}$: $P_{1}^{'}\{\tilde{w}/\tilde{z}\}\sim_{L} Q_{1}^{'}\{\tilde{w}/\tilde{z}\}$ which imply $(\nu x)(P_{1}^{'}\{\tilde{w}/\tilde{z}\}) Res(\sim_{L}) (\nu x)(Q_{1}^{'}\{\tilde{w}/\tilde{z}\})$. For definition of substitution and lemma \ref{bisimulationPreservedByRestrictionSubLemma}:
	    \begin{center}
	      $((\nu x^{'})P_{1}^{'}\{x^{'}/x\})\{\tilde{w}/\tilde{z}\} Res(\sim_{L})^{\equiv_{\alpha}} ((\nu x^{'})Q_{1}^{'}\{x^{'}/x\})\{\tilde{w}/\tilde{z}\}$
	    \end{center}
	  \item
	    $\gamma$ is an action and not a bound output. In this case $P_{1}^{'}\sim_{L} Q_{1}^{'}$ and $(\nu x)(P_{1}^{'}) Res(\sim_{L}) (\nu x)(Q_{1}^{'})$.
	\end{itemize}
      \item[$(CongRes, Opn)$] similar.
      \item[$(Alp, Res)(1)$]:
	\begin{center}
	  \begin{tabular}{ll}
	      $\inferrule* [left=\bf{AlpRes}]{
		  P \equiv_{\alpha} P_{1}
	      }{
		(\nu x)P \equiv_{\alpha} (\nu x)P_{1}
	      }$
	    &
	      $\inferrule* [left=\bf{Res}]{
		  P_{1} \xrightarrow{\gamma} P_{1}^{'}
		\\
		  x\notin n(\gamma)
	      }{
		(\nu x)P_{1} \xrightarrow{\gamma} (\nu x)P_{1}^{'}
	      }$	      
	  \end{tabular}
	\end{center}
	$P \equiv_{\alpha} P_{1}$ imply $P \equiv P_{1}$ and this case is similar to $(CongRes, Res)$.
      \item[$(Alp, Res)(2)$]:
	\begin{center}
	  \begin{tabular}{ll}
	      $\inferrule* [left=\bf{AlpRes}]{
		  P \equiv_{\alpha} P_{1}
		\\
		  y\notin fn(P_{1},Q)
	      }{
		(\nu x)P \equiv_{\alpha} (\nu y)(P_{1}\{y/x\})
	      }$
	    &
	      $\inferrule* [left=\bf{Res}]{
		  P_{1}\{y/x\} \xrightarrow{\gamma} P_{1}^{'}
		\\
		  y\notin n(\gamma)
	      }{
		(\nu y)P_{1}\{y/x\} \xrightarrow{\gamma} (\nu y)P_{1}^{'}
	      }$	      
	  \end{tabular}
	\end{center}
	$y\notin fn(P_{1})$ and $(xy)$ is a permutation that swaps $x$ with $y$ imply that $P_{1}\{y/x\}\equiv_{\alpha} P_{1}(xy)$. $P\sim_{E}Q$ and $P\equiv_{\alpha}P_{1}$ imply $P_{1}\sim_{E} Q$. $P_{1}\sim_{E} Q$ imply $P_{1}(xy) \sim_{E} Q(xy)$. $P_{1}(xy) \sim_{E} Q(xy)$ and $P_{1}\{y/x\}\equiv_{\alpha} P_{1}(xy)$ imply $P_{1}\{y/x\}\sim_{E} Q(xy)$. $y\notin fn(Q)$ imply $Q\sim_{E} Q(xy)$. $Q\sim_{E} Q(xy)$ and $P_{1}\{y/x\}\sim_{E} Q(xy)$ imply $P_{1}\{y/x\}\sim_{E} Q$. So $Q \xrightarrow{\gamma} Q_{1}^{'}$ and $Q_{1}^{'}\sim_{E} P_{1}^{'}$. For rule $Res$: $(\nu x)Q \xrightarrow{\gamma} (\nu x)Q_{1}^{'}$ and $(\nu x)Q_{1}^{'} Res(\sim_{E}) (\nu x)P_{1}^{'}$.
      \item[$(Alp, Opn)$] similar.
      \item[$(ScpExtPar1, \_)$] 
	rule $ScpExtPar1$ is emulated by $ScpExtPar2$ and commutativity.
      \item[$(ScpExtSum1, \_)$] 
	rule $ScpExtSum1$ is emulated by $ScpExtSum2$ and commutativity.
    \end{description}
  \end{proof}
\end{lemma}

\begin{lemma}\label{sumPreservesBisimulation}
  $\sim_{L}$ is preserved by sum.
  \begin{proof}
    For lemma \ref{structuralCongruenceElimination} and reflexivity of structural congruence we can assume that every transition that starts from $P+R$ is in the form: $P+R\equiv S$, $S \xrightarrow{\gamma}S$ and this derivation does not used the rule $Cong$ but can use the commutative counterpart of the rules for sum, parallel and communication. So we can proceed by induction on the sum of the sizes of the derivations of $P+R\equiv S$ and $S \xrightarrow{\gamma}S$. Then by cases on the last pair of rules used:
    \begin{description}
      \item[$(CongSum, SumL)$]:
	\begin{center}
	  \begin{tabular}{ll}
	      $\inferrule* [left=\bf{CongSum}]{
		  P \equiv P_{1}
		\\
		  R \equiv R_{1}
	      }{
		P+R \equiv P_{1}+R_{1}
	      }$
	    &
	      $\inferrule* [left=\bf{SumL}]{
		  P_{1} \xrightarrow{\gamma} P_{1}^{'}
	      }{
		P_{1}+R_{1} \xrightarrow{\gamma} P_{1}^{'}
	      }$	      
	  \end{tabular}
	\end{center}
	$P \equiv P_{1}$ and $P_{1} \xrightarrow{\gamma} P_{1}^{'}$ imply $P \xrightarrow{\gamma} P_{1}^{'}$. $P \sim_{L} Q$ and $P \xrightarrow{\gamma} P_{1}^{'}$ imply $Q \xrightarrow{\gamma} Q_{1}^{'}$ and the condition of late bisimulation on $P_{1}^{'}$ and $Q_{1}^{'}$ are met. For rule $SumL$: $Q+R \xrightarrow{\gamma} Q_{1}^{'}$
      \item[$(CongSum, SumR)$]:
	\begin{center}
	  \begin{tabular}{ll}
	      $\inferrule* [left=\bf{CongSum}]{
		  P \equiv P_{1}
		\\
		  R \equiv R_{1}
	      }{
		P+R \equiv P_{1}+R_{1}
	      }$
	    &
	      $\inferrule* [left=\bf{SumR}]{
		  R_{1} \xrightarrow{\gamma} R_{1}^{'}
	      }{
		P_{1}+R_{1} \xrightarrow{\gamma} R_{1}^{'}
	      }$	      
	  \end{tabular}
	\end{center}
	$R \equiv R_{1}$ and $R_{1} \xrightarrow{\gamma} P_{1}^{'}$ imply $R \xrightarrow{\gamma} R_{1}^{'}$. For rule $SumR$: $Q+R \xrightarrow{\gamma} R_{1}^{'}$
      \item[$(ScpExtSum2, Res)(1)$]:
	\begin{center}
	  \begin{tabular}{ll}
	      $\inferrule* [left=\bf{ScpExtSum2}]{
		  P \equiv P_{1}
		\\
		  R \equiv R_{1}
	      }{
		P+(\nu x)R \equiv (\nu x)(P_{1}+R_{1})
	      }$
	    &
	      $\inferrule* [left=\bf{Res}]{
		  \inferrule* [left=\bf{SumL}]{
		    P_{1} \xrightarrow{\gamma} P_{1}^{'}
		  }{
		    P_{1}+R_{1} \xrightarrow{\gamma} P_{1}^{'}
		  }
		\\
		  x\notin n(\gamma)
	      }{
		(\nu x)(P_{1}+R_{1}) \xrightarrow{\gamma} (\nu x)P_{1}^{'}
	      }$	      
	  \end{tabular}
	\end{center}
	$P\equiv P_{1}$ and $P_{1} \xrightarrow{\gamma} P_{1}^{'}$ imply $P \xrightarrow{\gamma} P_{1}^{'}$. $P \sim_{L} Q$ and $P \xrightarrow{\gamma} P_{1}^{'}$ imply $Q \xrightarrow{\gamma} Q_{1}^{'}$ and the conditions of late bisimulation on $P_{1}^{'}$ and $Q_{1}^{'}$ are met. For rule $SumL$: $Q+(\nu x)R \xrightarrow{\gamma} Q_{1}^{'}$. For rules $Cong$ and $ScpExtSum2$: $(\nu x)(Q+R) \xrightarrow{\gamma} (\nu x)Q_{1}^{'}$. Also the conditions of late bisimulation on $(\nu x)P_{1}^{'}$ and $(\nu x)Q_{1}^{'}$ are met since for lemma \ref{restrictionPreservesBisimulation} restriction preserves late bisimulation.
      \item[$(ScpExtSum2, Res)(2)$]:
	\begin{center}
	  \begin{tabular}{ll}
	      $\inferrule* [left=\bf{ScpExtSum2}]{
		  P \equiv P_{1}
		\\
		  R \equiv R_{1}
	      }{
		P+(\nu x)R \equiv (\nu x)(P_{1}+R_{1})
	      }$
	    &
	      $\inferrule* [left=\bf{Res}]{
		  \inferrule* [left=\bf{SumR}]{
		    R_{1} \xrightarrow{\gamma} R_{1}^{'}
		  }{
		    P_{1}+R_{1} \xrightarrow{\gamma} R_{1}^{'}
		  }
		\\
		  x\notin n(\gamma)
	      }{
		(\nu x)(P_{1}+R_{1}) \xrightarrow{\gamma} (\nu x)R_{1}^{'}
	      }$	      
	  \end{tabular}
	\end{center}
	For rules $SumR$, $Cong$ and $ScpExtSum2$: $(\nu x)(Q+R) \xrightarrow{\gamma} (\nu x)Q_{1}^{'}$ and the conditions of late bisimulation on $(\nu x)P_{1}^{'}$ and $(\nu x)Q_{1}^{'}$ are met.
      \item[$(ScpExtSum2, Opn)$] 
	similar.
      \item[$(ScpExtSum1, \_)$] 
	rule $ScpExtSum1$ is emulated by $ScpExtSum2$ and commutativity.    
      \item[$(Alp, \_)$] 
% 	$P \equiv_{\alpha} P_{1}$ and $P_{1} \xrightarrow{\gamma} P_{1}^{'}$ imply $P \xrightarrow{\gamma} P_{1}^{'}$. $P\sim_{L}$ and $P \xrightarrow{\gamma} P_{1}^{'}$ imply $Q \xrightarrow{\gamma} Q_{1}^{'}$. 
    \end{description}
  \end{proof}
\end{lemma}

\begin{lemma}
  Parallel composition preserves late bisimilarity.
  \begin{proof}
    
  \end{proof}
\end{lemma}


\begin{lemma}
  $\sim_{L}$ is not preserved by input prefix and strong input prefix.
  \begin{proof}
    
  \end{proof}
\end{lemma}


\begin{theorem}
  $\sim_{L}$ is preserved by all operators except input prefix.
  \begin{proof}
    
  \end{proof}
\end{theorem}


\subsection{Open bisimulation}


Let $\rightarrow_{L}$ be the semantic defined in table \ref{multipisoloinputlatewith} the following is an attempt to extend the definition of strong open bisimulation found in \cite{parrow}:
\begin{definition}
  An \emph{strong open $\mathbb{D}$ bisimulation} is $\{S_{D}\}_{D\in \mathbb{D}}$ a family of symmetric binary relations on processes such that for each process $P, Q$, for each distinction $D\in \mathbb{D}$, for each name substitution $\sigma$ which respects $D$ if $P S_{D} Q$ then
   \begin{itemize}
    \item 
      if $P\sigma \xrightarrow{\overline{a}(x)}_{L} P^{'}$ and $x$ is fresh then there exists $Q^{'}$ such that $Q\sigma \xrightarrow{\overline{a}(x)}_{L} Q^{'}$ and $P^{'} S_{D^{'}} Q^{'}$ where $D^{'}=D\sigma \cup \{x\}\times (fn(P\sigma)\cup fn(Q\sigma)) \cup  (fn(P\sigma)\cup fn(Q\sigma))\times\{x\}$
    \item
      if $P\sigma \xrightarrow{\gamma}_{L} P^{'}$ and $bn(\gamma)$ is fresh then there exists $Q^{'}$ such that $Q\sigma \xrightarrow{\gamma}_{L} Q^{'}$ and $P^{'} S_{D\sigma} Q^{'}$
  \end{itemize}
  $P$ and $Q$ are \emph{(strongly) open $\mathbb{D}$ bisimilar}, written $P \sim_{O}^{\mathbb{D}} Q$ if there exists a member $S_{D}$ of an open bisimulation such that $P S_{D} Q$. They are \emph{open bisimilar} if they are open $\mathbb{D}$ bisimilar and $\emptyset \in \mathbb{D}$, written $P \dot{\sim}_{O} D$.
\end{definition}

% \begin{definition}
%   Let $Sem_{0}$ be the set of all process with empty semantic. Then we define $\sim_{O}^{0}$ as $(Sem_{0})^{2}$ plus all the pair $(P,Q)$ of processes such that: $P\xrightarrow{\gamma}P$ imply $Q\xrightarrow{\gamma}Q$ and vice versa. Then we define $\sim_{O}^{i+1}$ as the set of pairs $(P,Q)$ such that
% 
% DOVE VOGLIO ARRIVARE?  
% 
%   An \emph{strong open $\mathbb{D}$ bisimulation} is $\{S_{D}\}_{D\in \mathbb{D}}$ a family of symmetric binary relations on processes such that for each process $P, Q$, for each distinction $D\in \mathbb{D}$, for each name substitution $\sigma$ which respects $D$ if $P S_{D} Q$ then
%    \begin{itemize}
%     \item 
%       if $P\sigma \xrightarrow{\overline{a}(x)}_{L} P^{'}$ and $x$ is fresh then there exists $Q^{'}$ such that $Q\sigma \xrightarrow{\overline{a}(x)}_{L} Q^{'}$ and $P^{'} S_{D^{'}} Q^{'}$ where $D^{'}=D\sigma \cup \{x\}\times (fn(P\sigma)\cup fn(Q\sigma)) \cup  (fn(P\sigma)\cup fn(Q\sigma))\times\{x\}$
%     \item
%       if $P\sigma \xrightarrow{\gamma}_{L} P^{'}$ and $bn(\gamma)$ is fresh then there exists $Q^{'}$ such that $Q\sigma \xrightarrow{\gamma}_{L} Q^{'}$ and $P^{'} S_{D\sigma} Q^{'}$
%   \end{itemize}
%   $P$ and $Q$ are \emph{(strongly) open $\mathbb{D}$ bisimilar}, written $P \sim_{O}^{\mathbb{D}} Q$ if there exists a member $S_{D}$ of an open bisimulation such that $P S_{D} Q$. They are \emph{open bisimilar} if they are open $\{\emptyset\}$ bisimilar, written $P \dot{\sim}_{O} D$.
% \end{definition}


\begin{lemma}
  $P\sim_{O} Q$ imply $P\sim_{O}^{\mathbb{D}} Q$.
  \begin{proof}
    For definition $P\sim_{O} Q$ imply that there exists a $\mathbb{D}$ such that $P\sim_{O}^{\mathbb{D}} Q$.
  \end{proof}
\end{lemma}

\begin{lemma}
  Output prefixing preserves open $\mathbb{D}$ bisimulation.
  \begin{proof}
    Let $\mathbb{S}=\{S_{D}\}_{D\in \mathbb{D}}$ be an open $\mathbb{D}$ bisimulation and let $P S_{D} Q$ for some $D\in \mathbb{D}$. Then the goal is to prove that: $(\overline{x}y.P)\sigma S_{D\sigma} (\overline{x}y.Q)\sigma$. There are two cases according to the last rule that can be applied to $(\overline{x}y.P)\sigma$:
    \begin{description}
      \item[$Out$]:
	$(\overline{x}y.P)\sigma \xrightarrow{(\overline{x}y)\sigma} P\sigma ? Q\sigma \stackrel{(\overline{x}y)\sigma}{\longleftarrowA} (\overline{x}y.Q)\sigma$
	E POI?
      \item[$Cong$] 
	the last part of the derivation look like this:
	\begin{center}
	  $\inferrule* [left=\bf{Cong}]{
	      \inferrule* [left=\bf{CongOut}]{
		P \equiv R
	      }{
		\overline{x}y.P \equiv \overline{x}y.R
	      }
	    \\
	      \inferrule* [left=\bf{Out}]{
	      }{
		\overline{x}y.R \xrightarrow{\overline{x}y} R
	      }
	  }{
	    \overline{x}y.P \xrightarrow{\overline{x}y} R
	  }$
	\end{center}
	For rule $Out$: $\overline{x}y.Q \xrightarrow{\overline{x}y} Q$. $P\equiv R$ imply $P \sim_{O}$
    \end{description}
  \end{proof}
\end{lemma}



\begin{theorem}
  $P \sim_{O}^{D} Q$ is a congruence.
  \begin{proof}
    
  \end{proof}
\end{theorem}


% 
% \subsection{Strong equivalence and distinctions}
% \subsection{Properties of strong bisimilarity}
% \subsection{Properties of strong $D-$ equivalence}



