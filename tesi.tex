\documentclass[14pt, a4paper, draft]{book}


\usepackage[latin1]{inputenc}
\usepackage[T1]{fontenc}
\usepackage{a4wide}
\usepackage{makeidx}
\usepackage{multicol}
\usepackage{amsthm}
\usepackage{amsmath}
\usepackage{amssymb}
\usepackage{algorithm2e}
\usepackage{vmargin}
\usepackage[pdftex]{graphicx}
%\usepackage{synttree}
\usepackage{tikz}
\usetikzlibrary{automata,positioning}
\usepackage{proof}
\usepackage{mathpartir}


\newcommand{\openrigthchapter}[1]{\chapter{#1}}
\newcommand{\HRule} {\rule{\linewidth}{0.5mm}}

\newtheorem{theorem}{Theorem}[section]
\newtheorem{lemma}[theorem]{Lemma}
\newtheorem{proposition}[theorem]{Proposition}
\newtheorem{corollary}[theorem]{Corollary}


%\newenvironment{proof}[1][Proof]{\begin{trivlist}
%\item[\hskip \labelsep {\bfseries #1}]}{\end{trivlist}}
\newenvironment{abstract}{
  \cleardoublepage  \begin{center}  \end{center}
}{\vfill \null}
\newenvironment{definition}[1][Definition]{\begin{trivlist}
\item[\hskip \labelsep {\bfseries #1}]}{\end{trivlist}}
\newenvironment{example}[1][Example]{\begin{trivlist}
\item[\hskip \labelsep {\bfseries #1}]}{\end{trivlist}}
\newenvironment{remark}[1][Remark]{\begin{trivlist}
\item[\hskip \labelsep {\bfseries #1}]}{\end{trivlist}}

%\newcommand{\qed}{\nobreak \ifvmode \relax \else
%      \ifdim\lastskip<1.5em \hskip-\lastskip
%      \hskip1.5em plus0em minus0.5em \fi \nobreak
%      \vrule height0.75em width0.5em depth0.25em\fi}


\title{Multi Pi Calculus\\[4mm]\small{boh}}

\author{Federico Viscomi}

\makeindex

\begin{document}

\pagestyle{empty}


\begin{titlepage}

\begin{center}
% Upper part of the page
\textsc{\LARGE Universita' di Bologna}\\[1.5cm]
\textsc{\Large Facolta' di scienze matematiche fisiche e naturali}\\[0.6cm]
\textsc{\Large Corso di laurea magistrale in Scienze Informatiche}\\[0.4cm]
{Tesi di laurea}\\[0.2cm]
 
% Title
\HRule \\[0.4cm]
{ \huge \bfseries Multi $\pi$ calcolo}\\[0.4cm]
\HRule \\[1.5cm]
\begin{minipage}{0.4\textwidth}
\begin{center} \large
\emph{Candidato:}\\
Federico \textsc{Viscomi}
{\large}
\end{center}
\end{minipage}

\vfill

% Tutors 
\begin{minipage}{0.4\textwidth}
\begin{flushleft} \large
\emph{Tutore}\\
Prof. Roberto \textsc{Gorrieri}\\ 
\emph{\newline}\\
\dotfill \\
\emph{\newline\newline\newline} \\ 
\end{flushleft}
\end{minipage}


%\includegraphics[width=0.15\textwidth]{./pi}\\[1cm]
\includegraphics[width=0.15\textwidth]{./pi}
 
% Bottom of the page
\HRule \\[1.5cm]
\textsc{\Large Anno accademico 20011/2012}\\[0.5cm]
\end{center}

\end{titlepage}

\begin{abstract} \section{Introduzione} 


%LA TESI PUO' ESSERE IN INGLESE MA IL SOMMARIO(ABSTRACT) DEVE ESSERE IN ITALIANO!


% Qual e' lo stato dell'arte? 
% Stato dell'arte: pi-calcolo, mancanza di multi-party synch mobile, per ccs e' stato proposto multi-ccs per multy-party synch statica.
% \newline
% Cosa ho fatto? multi-pi
% \newline
% Perche'?
% \newline


\end{abstract}


\tableofcontents

%\input{ringraziamenti}

\openrigthchapter{Multi ccs}

%guardare l'introduzione del libro per TLP
%leggere per bene il libro del martini e immergere qualcosa nella tesi

\section{syntax}

Multi ccs is introduced in \cite{gorrieriMCCS}. Let $\mathbb{L}$ be a numerable set of channel names, let $\overline{\mathbb{L}}$ be the set of the complementary of channel names $\overline{\mathbb{L}}=\{\overline{l}: l\in \mathbb{L}\}$, let $Act=\mathbb{L}\cup\overline{\mathbb{L}}\cup\{\tau\}$ be the set of all action, where $\tau \notin \mathbb{L}\cup \overline{\mathbb{L}}$. The process terms are generated by the following grammar:
\begin{center}
  $p$ ::= $0$ | $\mu.p$ | $\underline{\mu}.p$ | $p+p$ | $p|p$ | $(\nu a)p$ | $C$
\end{center}
and we have also expression in the form
\begin{center}
  $C\; =\; p$ 
\end{center}
in which the constant $C$ is defined as the process $p$.
The meaning of each production is the following:
\begin{description}
  \item[$0$] is terminated process
  \item[$\mu.p$] is action prefixing, $\mu$ is an action
  \item[$\underline{\mu}.p$] is strong action prefixing
  \item[$p+p$] is non deterministic sum of two processes
  \item[$p|p$] is parallel composition of two processes
  \item[$(\nu a)p$] is restriction: the name $a$ is private in p. $\nu a$ is a binder for the name $a$ in the process $p$
  \item[$C$] is a constant
\end{description}

\begin{definition}
\label{guarded process}
A \emph{guarded process} is 
\end{definition}


We denote with $\mathbb{P}$ the set of all process that are guarded and closed with respect to constant.



\section{semantic}

\begin{definition}
\label{labeled transition system} 
A \emph{labeled transition system} is a labeled graph, that is a triple 
\begin{center}
  $(N, L, LE)$ 
\end{center}
such that:
\begin{description}
  \item[$N$] is a set of nodes
  \item[$L$] is a set of labels
  \item[$LE$] is a set of labeled edges(also called transition relation): $LE\subseteq N\times L\times N$
\end{description}
\end{definition}
\begin{definition}
The \emph{semantic of a multi-CCS process} is a labeled transition system 
\begin{center}
  $(\mathbb{P}, \mathbb{A}, \rightarrow)$
\end{center}
such that
\begin{description}
  \item[$\mathbb{P}$] is the set of all multi ccs processes
  \item[$\mathbb{A}$] is the set $(\mathbb{L}\cup \overline{\mathbb{L}})^{+}\cup \{\tau\}$
  \item[$\rightarrow$] is the minimal relation contained in $\mathbb{P}\times \mathbb{A} \times \mathbb{P}$ that is generated by the rules of the operational semantic of multi-CCS
\end{description}
\end{definition}

\begin{definition}
\label{operational semantic of multi-CCS}
The \emph{operational semantic of multi-CCS} is given by the following rules:
\begin{description}
  \item[$Pref$]:
    \begin{center}
      $\mu.p {\rightarrow}^{\mu} p $
    \end{center}
  \item[$S-Pref_{1}$]:
    \[
      \frac{p\rightarrow^{\sigma}p^{'}}{\underline{\tau}.p\rightarrow^{\sigma}p^{'}}
    \]
\end{description}

\end{definition}


\begin{example}
  We show an example of three processes that synchronize. In particular we prove that
  \[
	    \inferrule* [left=Com]{
		\inferrule* [left=SPref3]{
		    \inferrule* [left=Inp]{
		    }{
		      a.0\; \xrightarrow{a}\; 0
		    }
		  \\
		    a\neq \tau
		}{
		  \underline{a}.a.0\; \xrightarrow{aa}\; 0
		}
	      \\
		\inferrule* [left=Out]{
		}{
		  \overline{a}.0\; \xrightarrow{\overline{a}}\; 0
		}
	      \\
		Sync(aa,\overline{a}, a)
	    }{
	      \underline{a}.a.0|\overline{a}.0\; \xrightarrow{a}\; 0|0
	    }   
  \]

  \[
    \inferrule* [left=Res]{
	\inferrule* [left=Com]{
	    \underline{a}.a.0|\overline{a}.0\; \xrightarrow{a}\; 0|0
	  \\
	    \inferrule* [left=Out]{
	    }{
	      \overline{a}.0\; \xrightarrow{\overline{a}}\; 0
	    }
	  \\
	    Sync(a,\overline{a},\tau)
	}{
	  ((\underline{a}.a.0|\overline{a}.0)|\overline{a}.0)\; \xrightarrow{\tau}\; ((0|0)|0)\;
	}
      \\
	a\notin n(\tau)
    }{
      (\nu a)((\underline{a}.a.0|\overline{a}.0)|\overline{a}.0)\; \xrightarrow{\tau}\; (\nu a)((0|0)|0)\;
    }
  \]

\end{example}



\openrigthchapter{$\Pi$ calculus}

The $\pi$ calculus is a mathematical model of processes whose interconnections change as they interact. The basic computational step is the transfer of a communications link between two processes. The idea that the names of the links belong to the same category as the transferred objects is one of the cornerstone of the calculus. The $\pi$ calculus allows channel names to be communicated along the channels themselves, and in this way it is able to describe concurrent computations whose network configuration may change during the computation.


\section{syntax}
We suppose that we have a countable set of names $\mathbb{N}$, ranged over by lower case letters $a,b, \cdots, z$. This names are used for communication channels and values. Furthermore we have a set of identifiers, ranged over by $A$. We represent the agents or processes by upper case letters $P,Q, \cdots $. A process can perform the following actions:
\begin{center}
  $\pi$ ::= $\overline{x}y$ | $x(z)$ | $\tau$ 
\end{center}
The process are defined by the following grammar:
\begin{center}
  \begin{tabular}{l}
    $P,Q$ ::= $0$ | $\pi.P$ | $P|Q$ | $P+Q$ | $(\nu x) P$ | $A(y_{1}, \cdots, y_{n})$ 
  \end{tabular}
\end{center}
and they have the following intuitive meaning:
\begin{description}
  \item[$0$] 
    is the empty process, which cannot perform any actions
  \item[$\pi.P$] 
    is an action prefixing, this process can perform action $\pi$ e then behave like $P$, the action can be:
    \begin{description}
      \item[$\overline{x}y$] 
	is an output action, this sends the name $y$ along the name $x$. We can think about $x$ as a channel or a port, and about $y$ as an output datum sent over the channel
      \item[$x(z)$] 
	is an input action, this receives a name along the name $x$. $z$ is a variable which stores the received data.
      \item[$\tau$] 
	is a silent or invisible action, this means that a process can evolve to $P$ without interaction with the environment 
    \end{description}
  \item[$P+Q$] 
    is the sum, this process can enact either $P$ or $Q$
  \item[$P|Q$] 
    is the parallel composition, $P$ and $Q$ can execute concurrently and also syncronize with each other
  \item[$(\nu z) P$] 
    is the scope restriction. This process behave as $P$ but the name $z$ is local. This process cannot use the name $z$ to interact with other process but it can for communication within it.
  \item[$A(y_{1}, \cdots, y_{n})$] 
    is an identifier whose arity is $n$. Every identifier has a definition
    \begin{center}
      $A(x_{1}, \cdots, x_{n}) = P$
    \end{center}
    where the $x_{i}$ must be pairwise disjoint. The intuition is that if the $y_{i}$ replace the $x_{i}$ then $A(y_{1}, \cdots, y_{n})$ behave as $P$. 
\end{description}

To resolve ambiguity we can use parentheses and observe the conventions that prefixing and restriction bind more tightly than composition and prefixing binds more tightly than sum. 

\begin{definition} \index{binder} \index{bind} \index{name occurrence! bound} \index{scope}
  We say that the input prefix $x(z).P$ \emph{binds} $z$ in $P$ or is a \emph{binder} for $z$ in $P$. We also say that $P$ is the \emph{scope} of the binder and that any occurrence of $z$ in $P$ are \emph{bound} by the binder. There are two other binders: the restriction operator $(\nu z)P$ is a binder for $z$ in $P$ and the definition of an identifier $A(x_{1}, \cdots, x_{n}) = P$ is a binder, specifically the names $x_{1}, \cdots, x_{n}$ are bound in the process $P$.
\end{definition}

\begin{definition} \index{$bn$}
  $bn(P)$ is the set of names that have a bound occurrence in $P$ and is defined as $B(P, \emptyset)$, where $B(P, I)$, with $I$ a set of process constants, is defined as follows: 
  \begin{center}
    \begin{tabular}{l}
	$B(0, I)\; =\; \emptyset$
      \\\\
	$B(\overline{x}y.Q, I)\; =\; B(Q, I)$
      \\\\
	$B(x(y).Q, I)\; =\; \{y,\overline{y}\}\cup B(Q, I)$
      \\\\
	$B(\tau.Q, I)\; =\; B(Q, I)$
      \\\\
	$B(A(x_{1},\cdots, x_{n}), I)=\left\{
	  \begin{array}{ll}
		\{x_{1},\overline{x_{1}},\cdots, x_{n},\overline{x_{n}}\}\cup B(Q, I\cup \{A\})
	      &
		if\; A\stackrel{def}{=}Q\; and\; A\notin I
	    \\
		\emptyset
	      &
		if\; A\in I
	  \end{array}\right.$
      \\\\
	$B(Q+R,I)\; =\; B(Q,I)\cup B(R,I)$
      \\\\
	$B(Q|R,I)\; =\; B(Q,I)\cup B(R,I)$
      \\\\
	$B((\nu x)Q, I)\; =\; \{x, \overline{x}\}\cup B(Q, I)$
    \end{tabular}
  \end{center}
\end{definition}



\begin{definition} \index{name occurrence! free}
  We say that a name $x$ is \emph{free} in $P$ if $P$ contains a non bound occurrence of $x$. We write $fn(P)$ for the set of names with a free occurrence in $P$. $fn(P)$ is defined as $fn(P,\emptyset)$ where $fn(P, I)$, with $I$ a set of process constants, is defined as follows:
  \begin{center}
    \begin{tabular}{l}
	$F(0, I)\; =\; \emptyset$
      \\\\
	$F(\overline{x}y.Q, I)\; =\; \{x,\overline{x},y,\overline{y}\}\cup F(Q, I)$
      \\\\
	$F(x(y).Q, I)\; =\; \{x,\overline{x}\}\cup (F(Q, I)-\{y,\overline{y}\})$
      \\\\
	$F(\tau.Q, I)\; =\; F(Q, I)$
      \\\\
	$F(A(x_{1},\cdots, x_{n}), I)=\left\{
	  \begin{array}{ll}
		F(Q, I\cup \{A\})-\{x_{1},\overline{x_{1}},\cdots, x_{n},\overline{x_{n}}\}
	      &
		if\; A\stackrel{def}{=}Q\; and\; A\notin I
	    \\
		\emptyset
	      &
		if\; A\in I
	  \end{array}\right.$
      \\\\
	$F(Q+R,I)\; =\; F(Q,I)\cup F(R,I)$
      \\\\
	$F(Q|R,I)\; =\; F(Q,I)\cup F(R,I)$
      \\\\
	$F((\nu x)Q, I)\; =\; F(Q, I)-\{x,\overline{x}\}$
    \end{tabular}
  \end{center}
\end{definition}


\begin{definition} \index{n}
  $n(P)$ which is the set of all names in $P$ and is defined in the following way:
  \begin{center}
    $n(P)\; =\; fn(P)\cup bn(P)$
  \end{center}
\end{definition}



In a definition $A(x_{1}, \cdots, x_{n})=P$ we assume that $fn(P)\subseteq \{x_{1}, \cdots, x_{n}\}$. 


\begin{definition}\index{syntactic substitution}
  $P\{b/a\}$ is the syntactic substitution of name $b$ for a different name $a$ inside a $\pi$ calculus process, and it consist in replacing every free occurrences of $a$ with $b$. If $b$ is a bound name in $P$, in order to avoid name capture we perform an appropriate $\alpha$ conversion. $P\{b/a\}$ is defined as follows:
  \begin{center}
    \begin{tabular}{l}
	$0\{b/a\}\; =\; 0$
      \\\\
	$(\overline{x}y.Q)\{b/a\}\; =\; \overline{x}\{b/a\}y\{b/a\}.Q\{b/a\}$
      \\\\
	$(x(y).Q)\{b/a\}\; =\; x\{b/a\}(y).Q\{b/a\}$ if $y\neq a$ and $y\neq b$
      \\\\
	$(x(a).Q)\{b/a\}\; =\; x\{b/a\}(a).Q$
      \\\\
	$(x(b).Q)\{b/a\}\; =\; x\{b/a\}(c).((Q\{c/b\})\{b/a\})$ where $c\notin n(Q)$
      \\\\
	$(\tau.Q)\{b/a\}\; =\; \tau.Q\{b/a\}$
      \\\\
	$(A(x_{1},\cdots, x_{n}))\{b/a\}=\left\{
	  \begin{array}{ll}
		A_{\{b/a\}}
	      &
		where\; A_{\{b/a\}}=q\{b/a\}\; if\; A\stackrel{def}{=}Q
	    \\
		A
	      &
		if\; a\notin fn(A)
	  \end{array}\right.$
      \\\\
	$(Q+R)\{b/a\}\; =\; Q\{b/a\} + R\{b/a\}$
      \\\\
	$(Q|R)\{b/a\}\; =\; Q\{b/a\} | R\{b/a\}$
      \\\\
	$((\nu y)Q)\{b/a\}\; =\;(\nu y)Q\{b/a\}$ if $y\neq a$ and $y\neq b$
      \\\\
	$((\nu a)Q)\{b/a\}\; =\;(\nu a)Q$
      \\\\
	$((\nu b)Q)\{b/a\}\; =\;(\nu c)((Q\{c/b\})\{b/a\})$ where $c\notin n(Q)$
    \end{tabular}
  \end{center}
\end{definition}



\section{structural congruence}

Structural congruences are a set of equations defining equality and congruence relations on process. They can be used in combination with an SOS semantic for languages. In some cases structural congruences help simplifying the SOS rules: for example they can capture inherent properties of composition operators(e.g. commutativity, associativity and zero element). Also, in process calculi, structural congruences let processes interact even in case they are not adjacent in the syntax. There is a possible trade off between what to include in the structural congruence and what to include in the transition rules: for example in the case of the commutativity of the sum operator. It is worth noticing that in most process calculi every structurally congruent processes should never be distinguished and thus any semantic must assign them the same behaviour.

\begin{definition}\index{context}
  A \emph{context} $C[\cdot]$ is a process with a placeholder. If $C[\cdot]$ is a context and we replace the placeholder with $P$, than we obtain $C[P]$. In doing so, we make no $\alpha$ conversions.
\end{definition}


\begin{definition}\index{congruence}
  A \emph{congruence} is a binary relation on processes such that:
  \begin{itemize}
    \item 
      $S$ is an equivalence relation
    \item 
      $S$ is preserved by substitution in contexts: for each pair of processes $(P, Q)$ and for each context $C[\cdot]$
      \begin{center}
	$(P,Q)\in S\; \Rightarrow\; (C[P], C[Q])\in S$
      \end{center}
  \end{itemize}
\end{definition}

% \begin{definition}\label{congruenza strutturale}
%   As stated in \cite{mousavireniers}, a \emph{structural congruence} is the minimal congruence on processes that satisify the following property:
% \end{definition}

\begin{definition}\index{structural congruence}
  We define a \emph{structural congruence $\equiv$} as the smallest congruence on processes that satisfies the following axioms 
  \begin{center}
    \begin{tabular}{lll}
      \hline\\
	SC-ALP&$\begin{array}{c}P \stackrel{\alpha}{=} Q\\\overline{P\equiv Q}\end{array}$&$\alpha$ conversion
      \\\\
	\multicolumn{3}{l}{abelian monoid laws for sum:}
      \\
	SC-SUM-ASC& $M_{1}+(M_{2}+M_{3})\equiv (M_{1}+M_{2})+M_{3}$ &associativity
      \\
	SC-SUM-COM& $M_{1}+M_{2}\equiv M_{2}+M_{1}$ &commutativity
      \\
	SC-SUM-INC& $M+0\equiv M$&zero element
      \\\\
	\multicolumn{3}{l}{abelian monoid laws for parallel:}
      \\
	SC-COM-ASC& $P_{1}|(P_{2}|P_{3})\equiv (P_{1}|P_{2})|P_{3}$ &associativity
      \\
	SC-COM-COM& $P_{1}|P_{2}\equiv P_{2}|P_{1}$ &commutativity
      \\
	SC-COM-INC& $P|0\equiv P$&zero element
      \\\\
	\multicolumn{3}{l}{scope extension laws:}
      \\
	SC-RES& $(\nu z) (\nu w) P \equiv (\nu w) (\nu z) P$ &
      \\
	SC-RES-INC& $(\nu z) 0 \equiv 0$ &
      \\
	SC-RES-COM& $(\nu z) (P_{1}|P_{2}) \equiv P_{1}|(\nu z) P_{2}$ if $z\notin fn(P_{1})$&
      \\
	SC-RES-SUM& $(\nu z) (P_{1}+P_{2}) \equiv P_{1}+(\nu z) P_{2}$ if $z\notin fn(P_{1})$&
      \\\\
	\multicolumn{3}{l}{unfolding law:}
      \\
	SC-IDE&$A(\tilde{y})\equiv P\{\tilde{y}/\tilde{x}\}$&if $A(\tilde{x})\stackrel{def}{=}P$
      \\\hline
    \end{tabular}
  \end{center}
\end{definition}

We can make some clarification on the axioms of the structural congruence:
\begin{description}
  \item[$unfolding$] 
    this just helps replace an identifier by its definition, with the appropriate parameter instantiation. The alternative is to use an appropriate SOS rule: 
    \begin{center}
	  \bf{Cns}
	  \begin{tabular}{c}
	      $A(\tilde{x}) \stackrel{def}{=} P\; P\{\tilde{y}/\tilde{x}\} \xrightarrow{\alpha} P^{'}$
	    \\\hline
	      $A(\tilde{y}) \xrightarrow{\alpha} P^{'}$
	  \end{tabular}
    \end{center}
  \item[$\alpha\; conversion$]
    is the $\alpha$ conversion, i.e., the choice of bound names, it identifies agents like $x(y).\overline{z}y$ and $x(w).\overline{z}w$. In the semantic of pi calculus we can use the structural congruence with the rule SC-ALP or the SOS rule
    \begin{center}
      \bf{Alpha}
      \begin{tabular}{c}
	  $P\xrightarrow{\alpha}P^{'}\;\; P\stackrel{\alpha}{\equiv}Q$
	\\\hline
	  $Q \xrightarrow{\alpha} P^{'}$
      \end{tabular}
    \end{center}
  \item[$abelian\; monoidal\; properties\; of\; some\; operators$]
    We can deal with associativity and commutativity properties of sum and parallel composition by using SOS rules or by axiom of the structural congruence. For example the commutativity of the sum can be expressed by the following two rules:
    \begin{center}
      \begin{tabular}{cc}
	\bf{Sum-L}
	  \begin{tabular}{c}
	    $P \xrightarrow{\alpha} P^{'}$\\
	    \hline
	    $P+Q \xrightarrow{\alpha} P^{'}$
	  \end{tabular}
	 &
	 \bf{Sum-R}
	   \begin{tabular}{c}
	    $Q \xrightarrow{\alpha} Q^{'}$\\
	    \hline
	    $P+Q \xrightarrow{\alpha} Q^{'}$
	   \end{tabular}
      \end{tabular}
    \end{center}
  or by the following rule and axiom:
    \begin{center}
      \begin{tabular}{cc}
	\bf{Sum}
	  \begin{tabular}{c}
	    $P \xrightarrow{\alpha} P^{'}$\\
	    \hline
	    $P+Q \xrightarrow{\alpha} P^{'}$
	  \end{tabular}
	 &
	 \bf{SC-SUM}
	   \begin{tabular}{c}
	    $P+Q \equiv Q+P$
	   \end{tabular}
      \end{tabular}
    \end{center}
    and the rule $Str$
  \item[$scope\; extension\; laws$]
    We can use this scope extension laws or the rules $Opn$ and $Cls$ to deal with the scope extension.
\end{description}

\section{operational semantic}
\subsection{early semantic without structural congruence}

% L’idea è che nella early quello che ricevi (cioe’ z) è un valore e come tale non ha vincoli rispetto ai free name esistenti, mentre nella late z sarebbe un place holder e quindi devi tenerlo distinto da tutti gli altri nomi free (onde evitare che quando andrai a fare la sostituzione in comunicazione, tu non vada a legare altri nomi che non hanno partecipato alla comunicazione). 

The semantic of a $\pi$ calculus process is a labeled transition system such that:
\begin{itemize}
  \item 
    the nodes are $\pi$ calculus process. The set of node is $\mathbb{P}$
  \item
    the actions can be:
    \begin{itemize}
      \item unbound input $xy$
      \item unbound output $\overline{x}y$
      \item the silent action $\tau$
      \item bound output $\overline{x}(y)$
      \item serve anche il bound input? non sono sicuro dell'utilita' delle regole CloseLIn, CloseRIn e OpenIn
    \end{itemize}
    The set of actions is $\mathbb{A}$, we use $\alpha$ to range over the set of actions.
  \item
    the transition relations is $\rightarrow\subseteq \mathbb{P}\times \mathbb{A}\times \mathbb{P}$
\end{itemize}
In the following section we present the early semantic without structural congruence and without $alpha$ conversion. We call this semantic early because in the rule $ECom$
\begin{center}
  \begin{tabular}{c}
    $P \xrightarrow{xy} P^{'}\;\; Q\xrightarrow{\overline{x}y} Q^{'}$\\
    \hline
    $P|Q \xrightarrow{\tau} P^{'}|Q^{'}$
  \end{tabular}
\end{center}
there is no substitution, instead the substitution occurres at an early point in the inference of this translation, namely during the inference of the input action. 

 
\begin{definition}\index{transition relation! early! without structural congruence}
  The \emph{early transition relation} $\rightarrow\subseteq \mathbb{P}\times \mathbb{A} \times \mathbb{P}$ is the smallest relation induced by the following rules:

  \begin{center}
    \begin{tabular}{ll}  
	  \bf{Out}
	  \begin{tabular}{c}
	      $\;\;$
	    \\\hline
	      $\overline{x}y.P \xrightarrow{\overline{x}y} P$
	  \end{tabular}
	&
	  \bf{EInp}
	  \begin{tabular}{c}
	    \\\hline
	      $x(y).P \xrightarrow{xz} P\{z/y\}$
	  \end{tabular}
      \\\\
	  \bf{Par-L}
	  \begin{tabular}{c}
	      $P \xrightarrow{\alpha} P^{'}\;\; bn(\alpha)\cap fn(Q)=\emptyset$
	    \\\hline
	      $P|Q \xrightarrow{\alpha} P^{'}|Q$
	  \end{tabular}
	&
	  \bf{Par-R}
	  \begin{tabular}{c}
	      $Q \xrightarrow{\alpha} Q^{'}\;\; bn(\alpha)\cap fn(Q)=\emptyset$
	    \\\hline
	      $P|Q \xrightarrow{\alpha} P|Q^{'}$
	  \end{tabular}
      \\\\
	  \bf{Sum-L}
	  \begin{tabular}{c}
	      $P \xrightarrow{\alpha} P^{'}$
	    \\\hline
	      $P+Q \xrightarrow{\alpha} P^{'}$
	  \end{tabular}
	&
	  \bf{Sum-R}
	  \begin{tabular}{c}
	      $Q \xrightarrow{\alpha} Q^{'}$
	    \\\hline
	      $P+Q \xrightarrow{\alpha} Q^{'}$
	    \end{tabular}
      \\\\
	  \bf{Tau}
	  \begin{tabular}{c}
	      $\;\;$
	    \\\hline
	      $\tau.P \xrightarrow{\tau} P$
	  \end{tabular}
	&
	  \bf{Res}
	  \begin{tabular}{c}
	      $P \xrightarrow{\alpha} P^{'}\;\; z\notin n(\alpha)$
	    \\\hline
	      $(\nu z) P \xrightarrow{\alpha} (\nu z) P^{'}$
	  \end{tabular}
      \\\\
	  \bf{EComR}
	  \begin{tabular}{c}
	      $P \xrightarrow{\overline{x}y} P^{'}\;\; Q\xrightarrow{xy} Q^{'}$
	    \\\hline
	      $P|Q \xrightarrow{\tau} P^{'}|Q^{'}$
	  \end{tabular}
	&
	  \bf{EComL}
	  \begin{tabular}{c}
	      $P \xrightarrow{xy} P^{'}\;\; Q\xrightarrow{\overline{x}y} Q^{'}$
	    \\\hline
	      $P|Q \xrightarrow{\tau} P^{'}|Q^{'}$
	  \end{tabular}
      \\\\
	    \bf{ClsLOut}
	    \begin{tabular}{c}
		$P \xrightarrow{\overline{x}(z)} P^{'}$  
		$Q \xrightarrow{xz} Q^{'}$ 
		$z\notin fn(Q)$
	      \\\hline
		$P|Q \xrightarrow{\tau} (\nu z)(P^{'}|Q^{'})$
	    \end{tabular}
	&
	    \bf{ClsROut}
	    \begin{tabular}{c}
		$P \xrightarrow{xz} P^{'}$  
		$Q \xrightarrow{\overline{x}(z)} Q^{'}$ 
		$z\notin fn(P)$
	      \\\hline
		$P|Q \xrightarrow{\tau} (\nu z)(P^{'}|Q^{'})$
	    \end{tabular}
      \\\\
	    \bf{ClsLIn}
	    \begin{tabular}{c}
		$P \xrightarrow{\overline{x}z} P^{'}$  
		$Q \xrightarrow{x(z)} Q^{'}$ 
		$z\notin fn(Q)$
	      \\\hline
		$P|Q \xrightarrow{\tau} (\nu z)(P^{'}|Q^{'})$
	    \end{tabular}
	&
	    \bf{ClsRIn}
	    \begin{tabular}{c}
		$P \xrightarrow{x(z)} P^{'}$  
		$Q \xrightarrow{\overline{x}z} Q^{'}$ 
		$z\notin fn(P)$
	      \\\hline
		$P|Q \xrightarrow{\tau} (\nu z)(P^{'}|Q^{'})$
	    \end{tabular}
      \\\\
	  \bf{Cns}
	  \begin{tabular}{c}
	      $A(\tilde{x}) \stackrel{def}{=} P\; P\{\tilde{y}/\tilde{x}\} \xrightarrow{\alpha} P^{'}$
	    \\\hline
	      $A(\tilde{y}) \xrightarrow{\alpha} P^{'}$
	  \end{tabular}
	&
      \\\\
	  \bf{OpnOut}
	  \begin{tabular}{c}
	      $P \;\xrightarrow{\overline{x}z} P^{'}\;\; z\neq x$
	    \\\hline
	      $(\nu z) P \;\xrightarrow{\overline{x}(z)} P^{'}$
	  \end{tabular}
	&
	  \bf{OpnIn}
	  \begin{tabular}{c}
	      $P \;\xrightarrow{xz} P^{'}\;\; z\neq x$
	    \\\hline
	      $(\nu z) P \;\xrightarrow{x(z)} P^{'}\;\; z\neq x$
	  \end{tabular}
    \end{tabular}
  \end{center}
\end{definition}

\begin{example}
  We show now an example of the so called scope extrusion, in particular we prove that
  \begin{center}
    $a(x).P\; |\; (\nu b)\overline{a}b.Q\; \xrightarrow{\tau}\; (\nu b) (P\{b/x\}\; |\; Q)$
  \end{center}
  where we suppose that $b\notin fn(P)$. In this example the scope of $(\nu b)$ moves from the right hand component to the left hand.
  \[
    \inferrule* [left=CloseR] {
	\inferrule* [left=Einp] {
	}{
	  a(x).P\; \xrightarrow{ab} P\{b/x\}
	}
      \\
	\inferrule* [left=Opn] {
	    \inferrule* [left=Out]{
	    }{
	      \overline{a}b.Q\; \xrightarrow{\overline{a}b} Q
	    }
	  \\
	    a\neq b
	}{
	  (\nu b)\overline{a}b.Q\; \xrightarrow{\overline{a}(b)} Q
	}
      \\
	b\notin fn((\nu b)\overline{a}b.Q)
    }{
      a(x).P\; |\; (\nu b)\overline{a}b.Q\; \xrightarrow{\tau}\; (\nu b) (P\{b/x\}\; |\; Q)
    }
  \]

\end{example}

\begin{example}
    We want to prove now that:
    \begin{center}
      $((\nu b) a(x).P)\; |\; \overline{a}b.Q\; \xrightarrow{\tau}\; (\nu c) (P\{c/b\}\{b/x\}\; |\; Q)$
    \end{center}
    where $c\notin n(P)$
%      \[
%  	\inferrule* [left=ECloseL] {
%  	    \inferrule* [left=Res] {
%  	      \inferrule* [left=EInp] {
%  	      }{
%  		a(x).P\; \xrightarrow{a ?}\;  P\{\}
%  	      }
%  	    }{
%  	      (\nu b) a(x).P\; \xrightarrow{a ?}\; (\nu b) P\{\}
%  	    }
%  	  \\
%  	    \inferrule* [left=Out] {
%  	    }{
%  	      \overline{a}b.Q\; \xrightarrow{\overline{a}b}\; Q
%  	    }
%  	  \\
%  	    b\notin fn(P)
%  	}{
%  	  ((\nu b) a(x).P)\; |\; \overline{a}b.Q\; \xrightarrow{\tau}\; (\nu c) (P\{c/b\}\{b/x\}\; |\; Q)
%  	}
%      \]
  come faccio? devo aggiungere la regola $\bf{Alpha}$?
\end{example}


% \begin{example}
%   Now we prove that
%   \[
%     \inferrule* [left=Com]{
% 	\overline{a}x.c(x).0|b(x).0\;
% 	  \xrightarrow{\overline{a}x}\;
% 	      c(x).0|b(x).0
%       \\
% 	a(x).0|\overline{b}x.\overline{c}x.0\;
% 	  \xrightarrow{ax}\;
% 	    0|\overline{b}x.\overline{c}x.0
%     }{
%        (\overline{a}x.c(x).0|b(x).0)|(a(x).0|\overline{b}x.\overline{c}x.0)\; 
%  	\xrightarrow{\tau}\; 
%  	  (c(x).0|b(x).0)|(0|\overline{b}x.\overline{c}x.0)
% %        (\overline{a}x.c(x).0|b(x).0)|(a(x).0|\overline{b}x.\overline{c}x.0)\; 
% %  	\xrightarrow{\tau}\; 
% %  	  (0|0)|(0|0)
%     }
%   \]
%   
% \end{example}



\subsection{early semantic with structural congruence}

\begin{definition}\index{transition relation! pi! early! with structural congruence}
  The \emph{early transition relation with structural congruence} $\rightarrow\subseteq \mathbb{P}\times \mathbb{A} \times \mathbb{P}$ is the smallest relation induced by the following rules:

  \begin{center}
    \begin{tabular}{lll}
	  \bf{Out}
	  \begin{tabular}{c}
	    \hline
	    $\overline{x}y.P \xrightarrow{\overline{x}y} P$
	  \end{tabular}
	  &
	  \bf{EInp}
	  \begin{tabular}{c}
	    \hline
	    $x(z).P \xrightarrow{xy} P\{y/z\}$
	  \end{tabular}
	  &
	  \bf{Par}
	  \begin{tabular}{c}
	    $P \xrightarrow{\alpha} P^{'}\;\; bn(\alpha)\cap fn(Q)=\emptyset$\\
	    \hline
	    $P|Q \xrightarrow{\alpha} P^{'}|Q$
	  \end{tabular}
      \\\\
	  \bf{Sum}
	  \begin{tabular}{c}
	    $P \xrightarrow{\alpha} P^{'}$\\
	    \hline
	    $P+Q \xrightarrow{\alpha} P^{'}$
	  \end{tabular}
	  &
	    \bf{ECom}
	    \begin{tabular}{c}
	      $P \xrightarrow{xy} P^{'}\;\; Q\xrightarrow{\overline{x}y} Q^{'}$\\
	      \hline
	      $P|Q \xrightarrow{\tau} P^{'}|Q^{'}$
	    \end{tabular}
	  &
	  \bf{Res}
	  \begin{tabular}{c}
	    $P \xrightarrow{\alpha} P^{'}\;\; z\notin n(\alpha)$\\
	    \hline
	    $(\nu z) P \xrightarrow{\alpha} (\nu z) P^{'}$
	  \end{tabular}
      \\\\
	  \bf{Tau}
	  \begin{tabular}{c}
	    \hline
	    $\tau.P \xrightarrow{\tau} P$
	  \end{tabular}
	  &
	  \bf{Opn}
	  \begin{tabular}{c}
	    $P \xrightarrow{\overline{x}z} P^{'}\;\; z\neq x$\\
	    \hline
	    $(\nu z) P \xrightarrow{\overline{x}(z)} P^{'}$
	  \end{tabular}
	  &
	  \bf{Str}
	  \begin{tabular}{c}
	    $P\equiv P^{'}\;\; P\xrightarrow{\alpha} Q\;\; Q\equiv Q^{'}$\\
	    \hline
	    $P^{'} \xrightarrow{\alpha} Q^{'}$
	  \end{tabular}
    \end{tabular}
  \end{center}
\end{definition}

\begin{example}
  We prove now that
  \begin{center}
    $a(x).P\; |\; (\nu b)\overline{a}b.Q\; \xrightarrow{\tau} P\{b/x\}\; |\; Q$
  \end{center}
  This follows from
  \[
    a(x).P\; |\; (\nu b)\overline{a}b.Q\; \equiv\; (\nu b)(a(x).P\; |\; \overline{a}b.Q)
  \]
  and
  \[
    (\nu b)(P\{b/x\}\; |\; Q) \equiv (P\{b/x\}\; |\; Q)
  \]
  and 
  \[
    (\nu b)(a(x).P\; |\; \overline{a}b.Q) \xrightarrow{\tau} (\nu b)(P\{b/x\}\; |\; Q)
  \]
  with the rule $Str$. We can prove the last transition in the following way:
  \[
    \inferrule* [left=Res] {
      \inferrule* [left=Com] {
	  \inferrule* [left=EInp] {
	  }{
	    a(x).P\; \xrightarrow{ab}\; P\{b/x\}
	  }
	\\
	  \inferrule* [left=Out] {
	  }{
	    \overline{a}b.Q\; \xrightarrow{\overline{a}b}\; Q
	  }
      }{
	a(x).P\; |\; \overline{a}b.Q\; \xrightarrow{\tau}\; P\{b/x\}\; |\; Q
      }
    }{
      (\nu b)(a(x).P\; |\; \overline{a}b.Q)\; \xrightarrow{\tau}\; (\nu b)(P\{b/x\}\; |\; Q)
    }
  \]

\end{example}

\begin{example}
    We want to prove now that:
    \begin{center}
      $((\nu b) a(x).P)\; |\; \overline{a}b.Q\; \xrightarrow{\tau}\; (\nu c) (P\{c/b\}\{b/x\}\; |\; Q)$
    \end{center}
    where the name $c$ is not in the free names of $Q$. We can exploit the structural congruence and get that
    \[
      ((\nu b) a(x).P) | \overline{a}b.Q\; \equiv\; (\nu c) (a(x).(P\{c/b\}) | \overline{a}b.Q)     
    \]
    then we have
    \[
	\inferrule* [left=Res] {
	  \inferrule* [left=Com]{
	      \inferrule* [left=EInp]{
		b\notin fn(P\{c/b\})
	      }{
		a(x).P\{c/b\}\; \xrightarrow{ab}\; P\{c/b\}\{b/x\}
	      }
	    \\
	      \inferrule* [left=Out]{
	      }{
		\overline{a}b.Q\; \xrightarrow{\overline{a}b}\; Q
	      }
	  }{
	      (a(x).(P\{c/b\}) | \overline{a}b.Q)\; \xrightarrow{\tau}\; (P\{c/b\}\{b/x\} | Q)
	  }
	}{
	  (\nu c) (a(x).(P\{c/b\}) | \overline{a}b.Q)\; \xrightarrow{\tau}\; (\nu c) (P\{c/b\}\{b/x\} | Q)
	}
    \]
    Now we just apply the rule $Str$ to prove the thesis.
\end{example}


\subsection{late semantic without structural congruence}

\begin{definition}\index{transition relation! pi! late! without structural congruence}
  The \emph{late transition relation without structural congruence} $\rightarrow\subseteq \mathbb{P}\times \mathbb{A} \times \mathbb{P}$ is the smallest relation induced by the following rules:
  \begin{center}
    \begin{tabular}{ll}
	  \bf{LInp}
	  \begin{tabular}{c}
% 	    $z\notin fn(P)$
	    ?
	    \\\hline
	    ?
% 	    $x(z).P \xrightarrow{xz} P$
	  \end{tabular}
	&
	  \bf{Res}
	  \begin{tabular}{c}
	    $P \xrightarrow{\alpha} P^{'}$ $z\notin n(\alpha)$
	      \\\hline
	    $(\nu z) P \xrightarrow{\alpha} (\nu z) P^{'}$
	  \end{tabular}    
      \\\\
	  \bf{Sum-L}
	  \begin{tabular}{c}
	      $P \xrightarrow{\alpha} P^{'}$
	    \\\hline
	      $P+Q \xrightarrow{\alpha} P^{'}$
	  \end{tabular}
	&
	  \bf{Sum-R}
	  \begin{tabular}{c}
	      $Q \xrightarrow{\alpha} Q^{'}$
	    \\\hline
	      $P+Q \xrightarrow{\alpha} Q^{'}$
	  \end{tabular}
      \\\\
	  \bf{Par-L}
	  \begin{tabular}{c}
	      $P \xrightarrow{\alpha} P^{'}\;\; bn(\alpha)\cap fn(Q)=\emptyset$
	    \\\hline
	      $P|Q \xrightarrow{\alpha} P^{'}|Q$
	  \end{tabular}
	&
	  \bf{Par-R}
	  \begin{tabular}{c}
	      $Q \xrightarrow{\alpha} Q^{'}\;\; bn(\alpha)\cap fn(Q)=\emptyset$
	    \\\hline
	      $P|Q \xrightarrow{\alpha} P|Q^{'}$
	  \end{tabular}
      \\\\
	  \bf{LCom}
	  \begin{tabular}{c}
	      $P \xrightarrow{x(y)} P^{'}\;\; Q\xrightarrow{\overline{x}z} Q^{'}\;\; z\notin fn(P)$
	    \\\hline
	      $P|Q \xrightarrow{\tau} P^{'}\{z/y\}|Q^{'}$
	  \end{tabular}	
	&
	  \bf{RCom}
	  \begin{tabular}{c}
	      $P \xrightarrow{\overline{x}z} P^{'}\;\; Q\xrightarrow{x(y)} Q^{'}\;\; z\notin fn(P)$
	    \\\hline
	      $P|Q \xrightarrow{\tau} P^{'}|Q^{'}\{z/y\}$
	  \end{tabular}	
      \\\\
	  \bf{Opn}
	  \begin{tabular}{c}
	      $P \xrightarrow{\overline{x}z} P^{'}\;\; z\neq x$
	    \\\hline
	      $(\nu z) P \xrightarrow{\overline{x}(z)} P^{'}$
	  \end{tabular}
	&
	  \bf{Out}
	  \begin{tabular}{c}
	    \hline
	    $\overline{x}y.P \xrightarrow{\overline{x}y} P$
	  \end{tabular}
      \\\\
	  \bf{CloseL}
	  \begin{tabular}{c}
	      $P\; \xrightarrow{\overline{x}(z)}\; P^{'}$  $Q \xrightarrow{xz} Q^{'}$ $z\notin fn(Q)$
	    \\\hline
	      $P|Q\; \xrightarrow{\tau}\; (\nu z)(P^{'}|Q^{'})$
	  \end{tabular}
	&
	  \bf{CloseR}
	  \begin{tabular}{c}
	      $P \xrightarrow{xz} P^{'}$  $Q \xrightarrow{\overline{x}(z)} Q^{'}$ $z\notin fn(P)$
	    \\\hline
	      $P|Q \xrightarrow{\tau} (\nu z)(P^{'}|Q^{'})$
	  \end{tabular}
      \\\\
	  \bf{Tau}
	  \begin{tabular}{c}
	    \hline
	      $\tau.P \xrightarrow{\tau} P$
	  \end{tabular}
	&
	  \bf{Cns}
	  \begin{tabular}{c}
	    $A(\tilde{x}) \stackrel{def}{=} P\; P\{\tilde{y}/\tilde{x}\} \xrightarrow{\alpha} P^{'}$
	      \\\hline
	    $A(\tilde{y}) \xrightarrow{\alpha} P^{'}$
	  \end{tabular}
    \end{tabular}
  \end{center}
\end{definition}



\subsection{late semantic with structural congruence}
In this case the set of actions $\mathbb{A}$ contains
\begin{itemize}
      \item bound input $x(y)$
      \item unbound output $\overline{x}y$
      \item the silent action $\tau$
      \item bound output $\overline{x}(y)$
\end{itemize}

\begin{definition}\index{transition relation! pi! late! with structural congruence}
  The \emph{late transition relation with structural congruence} $\rightarrow\subseteq \mathbb{P}\times \mathbb{A} \times \mathbb{P}$ is the smallest relation induced by the following rules:
  \begin{center}
    \begin{tabular}{ll}
	\bf{Prf}
	\begin{tabular}{c}
	  \hline
	  $\alpha.P \xrightarrow{\alpha} P$
	\end{tabular}
      &
	\bf{Sum}
	\begin{tabular}{c}
	    $P \xrightarrow{\alpha} P^{'}$
	  \\\hline
	    $P+Q \xrightarrow{\alpha} P^{'}$
	\end{tabular}
    \\\\
	\bf{Par}
	\begin{tabular}{c}
	    $P \xrightarrow{\alpha} P^{'}\;\; bn(\alpha)\cap fn(Q)=\emptyset$
	  \\\hline
	    $P|Q \xrightarrow{\alpha} P^{'}|Q$
	\end{tabular}
      &
	\bf{Res}
	\begin{tabular}{c}
	    $P \xrightarrow{\alpha} P^{'}$ $z\notin n(\alpha)$
	  \\\hline
	    $(\nu z) P \xrightarrow{\alpha} (\nu z) P^{'}$
	\end{tabular}
    \\\\    
	\bf{LCom}
	\begin{tabular}{c}
	    $P \xrightarrow{x(y)} P^{'}\;\; Q\xrightarrow{\overline{x}z} Q^{'}$
	  \\\hline
	  $P|Q \xrightarrow{\tau} P^{'}\{z/y\}|Q^{'}$
	\end{tabular}
      &
	\bf{Str}
	\begin{tabular}{c}
	    $P\equiv P^{'}\;\; P\xrightarrow{\alpha} Q\;\; Q\equiv Q^{'}$
	  \\\hline
	    $P^{'} \xrightarrow{\alpha} Q^{'}$
	\end{tabular}
    \\\\
	\bf{Opn}
	\begin{tabular}{c}
	    $P \xrightarrow{\overline{x}z} P^{'}\;\; z\neq x$
	  \\\hline
	    $(\nu z) P \xrightarrow{\overline{x}(z)} P^{'}$
	\end{tabular}
      &
    \end{tabular}
  \end{center}
\end{definition}


\begin{example}
  We prove now that
  \begin{center}
    $a(x).P\; |\; (\nu b)\overline{a}b.Q\; \xrightarrow{\tau} P\{b/x\}\; |\; Q$
  \end{center}
  This follows from
  \[
    a(x).P\; |\; (\nu b)\overline{a}b.Q\; \equiv\; (\nu b)(a(x).P\; |\; \overline{a}b.Q)
  \]
  and
  \[
    (\nu b)(P\{b/x\}\; |\; Q) \equiv (P\{b/x\}\; |\; Q)
  \]
  and 
  \[
    (\nu b)(a(x).P\; |\; \overline{a}b.Q) \xrightarrow{\tau} (\nu b)(P\{b/x\}\; |\; Q)
  \]
  with the rule $Str$. We can prove the last transition in the following way:
  \[
    \inferrule* [left=Res] {
      \inferrule* [left=LCom] {
	  \inferrule* [left=Inp] {
	  }{
	    a(x).P\; \xrightarrow{ax}\; P
	  }
	\\
	  \inferrule* [left=Out] {
	  }{
	    \overline{a}b.Q\; \xrightarrow{\overline{a}b}\; Q
	  }
      }{
	a(x).P\; |\; \overline{a}b.Q\; \xrightarrow{\tau}\; P\{b/x\}\; |\; Q
      }
    }{
      (\nu b)(a(x).P\; |\; \overline{a}b.Q)\; \xrightarrow{\tau}\; (\nu b)(P\{b/x\}\; |\; Q)
    }
  \]

\end{example}

\begin{example}
    We want to prove now that:
    \begin{center}
      $((\nu b) a(x).P)\; |\; \overline{a}b.Q\; \xrightarrow{\tau}\; (\nu c) (P\{c/b\}\{b/x\}\; |\; Q)$
    \end{center}
    where the name $c$ is not in the free names of $Q$. We can exploit the structural congruence and get that
    \[
      ((\nu b) a(x).P) | \overline{a}b.Q\; \equiv\; (\nu c) (a(x).(P\{c/b\}) | \overline{a}b.Q)     
    \]
    then we have
    \[
	\inferrule* [left=Res] {
	  \inferrule* [left=LCom]{
	      \inferrule* [left=Inp]{
		b\notin fn(P\{c/b\})
	      }{
		a(x).P\{c/b\}\; \xrightarrow{ax}\; P\{c/b\}
	      }
	    \\
	      \inferrule* [left=Out]{
	      }{
		\overline{a}b.Q\; \xrightarrow{\overline{a}b}\; Q
	      }
	  }{
	      (a(x).(P\{c/b\}) | \overline{a}b.Q)\; \xrightarrow{\tau}\; (P\{c/b\}\{b/x\} | Q)
	  }
	}{
	  (\nu c) (a(x).(P\{c/b\}) | \overline{a}b.Q)\; \xrightarrow{\tau}\; (\nu c) (P\{c/b\}\{b/x\} | Q)
	}
    \]
    Now we just apply the rule $Str$ to prove the thesis.
\end{example}


\subsection{behavioural semantic}

\begin{definition}\index{congruence! strong}
  We say that two agents $P$ and $Q$ are \emph{strongly congruent}, written $P\sim Q$ if
  \begin{center}
    $P\sigma \dot{\sim} Q\sigma$ for all substitution $\sigma$    
  \end{center}
\end{definition}

We define a bisimulation for the early and the late semantic with structural congruence, for clarity when referring to the early semantic we index the transition with $ _{E}$. In the following we will use the phrase $bn(\alpha)$ is fresh in a definition to mean that the name in $bn(\alpha)$, if any, is different from any free name occurring in any of the agents in the definition.
\begin{definition}\index{bisimulation! strong! early! with early semantic}
  A \emph{strong early bisimulation with early semantic} is a symmetric binary relation $\mathbb{R}$ on agents satisfying the following: $P\mathbb{R} Q$ and $P\; \xrightarrow{\alpha}_{E}\; P^{'}$ where $bn(\alpha)$ is fresh implies that
  \begin{center}
    $\exists Q^{'}: Q\xrightarrow{\alpha}Q^{'}\; \wedge\; P^{'}\mathbb{R}Q^{'}$
  \end{center}
  $P$ and $Q$ are strongly early bisimilar, written $P\; \dot{\sim}_{E}\; Q$, if they are related by an early bisimulation.
\end{definition}

\begin{definition}\index{bisimulation! strong! early! with late semantic}
  A \emph{strong early bisimulation with late semantic} is a symmetric binary relation $\mathbb{R}$ on agents satisfying the following: $P\mathbb{R} Q$ and $P\; \xrightarrow{\alpha}\; P^{'}$ where $bn(\alpha)$ is fresh implies that
  \begin{itemize}
    \item
      if $\alpha=a(x)$ then $\forall u\; \exists Q^{'}:\;\; Q\xrightarrow{a(x)}Q^{'}\; \wedge\; P^{'}\{u/x\}\mathbb{R}Q^{'}\{u/x\}$
    \item
      if $\alpha$ is not an input then $\exists Q^{'}:\;\; Q\xrightarrow{\alpha}Q^{'}\; \wedge\; P^{'}\mathbb{R}Q^{'}$
  \end{itemize}
\end{definition}

Early bisimulation is preserved by all operators except input prefix.

\begin{definition}\index{congruence! early}
  The \emph{early congruence} $\sim_{E}$ is defined by
  \begin{center}
    $P\sim_{E} Q$ is $\forall \sigma\; P\sigma \dot{\sim}_{E} Q\sigma$
  \end{center}
  where $\sigma$ is a substitution.
\end{definition}

The early congruence is the largest congruence in $\dot{\sim}_{E}$. 

In the following definition we consider a subcalculus without restriction. 
\begin{definition}\index{bisimulation! strong! open! early}
  A \emph{strong open bisimulation} is a symmetric binary relation $\mathbb{R}$ on agents satisfying the following for all substitutions $\sigma$: $P\mathbb{R} Q$ and $P\sigma\; \xrightarrow{\alpha}\; P^{'}$ where $bn(\alpha)$ is fresh implies that
  \begin{center}
    $\exists Q^{'}:\;\; Q\sigma\xrightarrow{\alpha}Q^{'}\; \wedge\; P^{'}\mathbb{R}Q^{'}$
  \end{center}
  $P$ and $Q$ are strongly open bisimilar, written $P\; \dot{\sim}_{O}\; Q$ if they are related by an open bisimulation.
\end{definition}





\openrigthchapter{Multi $\pi$ calculus solo output}

\section{Syntax}
As we did whit $\pi$ calculus, we suppose that we have a countable set of names $\mathbb{N}$, ranged over by lower case letters $a,b, \cdots, z$. This names are used for communication channels and values. Furthermore we have a set of identifiers, ranged over by $A$. We represent the agents or processes by upper case letters $P,Q, \cdots $. A multi $\pi$ process, in addiction to the same actions of a $\pi$ process, can perform also a strong prefix output:
\begin{center}
  $\pi$ ::= $\overline{x}y$ | $x(z)$ | $\underline{\overline{x}y}$ | $\tau$ 
\end{center}
The process are defined, just as original $\pi$ calculus, by the following grammar:
\begin{center}
  \begin{tabular}{l}
    $P,Q$ ::= $0$ | $\pi.P$ | $P|Q$ | $P+Q$ | $(\nu x) P$ | $A(y_{1}, \cdots, y_{n})$
  \end{tabular}
\end{center}
and they have the same intuitive meaning as for the $\pi$ calculus. The strong prefix output allows a process to make an atomic sequence of actions, so that more than one process can synchronize on this sequence. For the moment we allow the strong prefix to be on output names only. Also one can use the strong prefix only as an action prefixing for processes that can make at least a further action. Since the strong prefix can be on output names only, the only synchronization possible is between a process that executes a sequence of $n$ actions(only the last action can be an input) with $n\geq 1$ and $n$ other processes each executing one single action(at least $n-1$ process execute an output and at most one executes an input).

Multi $\pi$ calculus is a conservative extension of the $\pi$ calculus in the sense that: any $\pi$ calculus process $p$ is also a multi $\pi$ calculus process and the semantic of $p$ according to the SOS rules of $\pi$ calculus is the same as the semantic of $p$ according to the SOS rules of multi $\pi$ calculus. 

We have to extend the following definition to deal with the strong prefix:
\begin{center}
  \begin{tabular}{ll}
	$B(\underline{\overline{x}y}.Q, I)\; =\; B(Q,I)$
      &
	$F(\underline{\overline{x}y}.Q, I)\; =\; \{x,\overline{x},y,\overline{y}\}\cup F(Q, I)$
    \\
  \end{tabular}
\end{center}


\section{Operational semantic}
\subsection{Early operational semantic with structural congruence}

The semantic of a multi $\pi$ process is labeled transition system such that
\begin{itemize}
  \item 
    the nodes are multi $\pi$ calculus process. The set of node is $\mathbb{P}_{m}$
  \item
    the actions are multi $\pi$ calculus actions. The set of actions is $\mathbb{A}_{m}$, we use $\alpha, \alpha_{1}, \alpha_{2},\cdots $ to range over the set of actions, we use $\sigma, \sigma_{1}, \sigma_{2}, \cdots $ to range over the set $\mathbb{A}_{m}^{+} \cup \{\tau\}$. Note that $\sigma$ is a non empty sequence of actions.
  \item
    the transition relations is $\rightarrow\subseteq \mathbb{P}_{m}\times (\mathbb{A}_{m}^{+} \cup \{\tau\})\times \mathbb{P}_{m}$
\end{itemize}

In this case, a label can be a sequence of prefixes, whether in the original $\pi$ calculus a label can be only a prefix. We use the symbol $\cdot$ to denote the concatenation operator.

\begin{definition}\index{transition relation! multipi! output only! early! with structural congruence}
  The \emph{early transition relation without structural congruence} is the smallest relation induced by the rules in table \ref{multipiearlywith}. The relation $opn$ is defined in table \ref{opn}.
  \begin{table}
    \begin{tabular}{ll}
      \hline\\
	  $\inferrule* [left=\bf{Out}]{
	  }{
	    \overline{x}y.P \;\xrightarrow{\overline{x}y} P
	  }$
	&
	  $\inferrule* [left=\bf{EInp}]{
	  }{
	    x(y).P \;\xrightarrow{xz} P\{z/y\}
	  }$
      \\\\
	  $\inferrule* [left=\bf{Tau}]{
	  }{
	    \tau.P \;\xrightarrow{\tau} P
	  }$
	&
	  $\inferrule* [left=\bf{SOut}]{
	    P \;\xrightarrow{\sigma} P^{'}\;\; \sigma\neq \tau
	  }{
	    \underline{\overline{x}y}.P \;\xrightarrow{\overline{x}y \cdot \sigma} P^{'}
	  }$
      \\\\
	  $\inferrule* [left=\bf{Sum}]{
	    P \;\xrightarrow{\sigma} P^{'}
	  }{
	    P+Q \;\xrightarrow{\sigma} P^{'}
	  }$
	&
	  $\inferrule* [left=\bf{Str}]{
	      P\equiv P^{'}
	    \\
	      P^{'}\; \;\xrightarrow{\alpha}\; Q^{'}
	    \\
	      Q\equiv Q^{'}
	  }{
	      P\; \;\xrightarrow{\alpha}\; Q
	  }$
      \\\\
	  $\inferrule* [left=\bf{Par}]{
	    P \;\xrightarrow{\sigma} P^{'}\;\; bn(\sigma)\cap fn(Q)=\emptyset
	  }{
	    P|Q \;\xrightarrow{\sigma} P^{'}|Q
	  }$
	&
	  $\inferrule* [left=\bf{EComSng}]{
	      P\; \xrightarrow{xy}\; P^{'}
	    \\
	      Q\; \xrightarrow{\overline{x}y}\; Q^{'}
	  }{
	    P|Q\; \xrightarrow{\tau}\; P^{'}|Q^{'}
	  }$
      \\\\
	  $\inferrule* [left=\bf{Res}]{
	    P\; \xrightarrow{\sigma} P^{'}\; z\notin n(\alpha)
	  }{
	    (\nu) z P\; \xrightarrow{\sigma}\; (\nu) z P^{'}
	  }$
	&
	  $\inferrule* [left=\bf{EComSeq}]{
	      P \;\xrightarrow{xy}\; P^{'}
	    \\
	      Q\;\xrightarrow{\overline{x}y\cdot \sigma}\; Q^{'}
	  }{
	    P|Q\; \xrightarrow{\sigma}\; P^{'}|Q^{'}
	  }$
      \\\\
	  $\inferrule* [left=\bf{SOutTau}]{
	    P \;\xrightarrow{\tau} P^{'}
	  }{
	    \underline{\overline{x}y}.P \;\xrightarrow{\overline{x}y} P^{'}
	  }$
	&
	  $\inferrule* [left=\bf{OpnSeq}]{
	      P \xrightarrow{\sigma}\; P^{'}
	    \\ 
	      \exists \overline{x}z\in \sigma:\; x\neq z
	  }{
	      (\nu z)P \xrightarrow{opn(\sigma,z)}\; P^{'}
	  }$
      \\\\\hline
    \end{tabular}
    \caption{Multi $\pi$ early semantic with structural congruence}
    \label{multipiearlywith}
  \end{table}


  \begin{table}
    \begin{tabular}{lll}
      \hline\\
	  $\inferrule* {
	      x\neq z
	  }{
	      opn(\overline{x}z,z)=\overline{x}(z)
	  }$
	&
	  $\inferrule* {
	      x\neq z
	  }{
	      opn(\overline{x}z\cdot \sigma,z)=\overline{x}(z)\cdot opn(\sigma,z)
	  }$
	&
	  $\inferrule* {
	  }{
	      opn(xy,z)=xy
	  }$
      \\\\
	  $\inferrule* {
	  }{
	      opn(\overline{x}y,z)=\overline{x}y
	  }$
	&
	  $\inferrule* {
	  }{
	      opn(\overline{x}y\cdot \sigma,z)=\overline{x}y\cdot opn(\sigma,z)
	  }$
	&
      \\\\\hline
    \end{tabular}
    \caption{relation $opn$}
    \label{opn}
  \end{table}


\end{definition}




\begin{example}[Multi-party synchronization]
  We show an example of a derivation of three processes that synchronize.
\begin{center}
$\inferrule* [left=\bf{Res}]{
    x\notin n(\tau)
   \\
    \inferrule* [left=\bf{EComSeq}]{
	  \underline{\overline{x}y}.\overline{x}y.0|x(y).0
	    \xrightarrow{\overline{x}y}
	      0|0
      \\
	\inferrule* [left=\bf{Inp}]{
	}{
	  x(y).0	
	    \xrightarrow{xy} 
	      0
	}
    }{
      ((\underline{\overline{x}y}.\overline{x}y.0|x(y).0)|x(y).0)
	\xrightarrow{\tau}
	  ((0|0)|0)
    }
  }{
    (\nu x)((\underline{\overline{x}y}.\overline{x}y.0|x(y).0)|x(y).0)
      \xrightarrow{\tau}
	(\nu x)((0|0)|0)
}$
\end{center}

\begin{center}
$\inferrule* [left=\bf{EComSng}]{
  \inferrule* [left=\bf{SOut}]{
    \inferrule* [left=\bf{Out}]{
    }{
      \overline{x}y.0
	\xrightarrow{\overline{x}y}
	  0
    }
  }{
    \underline{\overline{x}y}.\overline{x}y.0
      \xrightarrow{\overline{x}y\cdot \overline{x}y}
	0
  }
  \\
    x(y).0\; \;\xrightarrow{xy}\; 0
}{
  \underline{\overline{x}y}.\overline{x}y.0|x(y).0
    \xrightarrow{\overline{x}y}
      0|0
}$
\end{center}

\end{example}

\begin{example}[Transactional synchronization]
  In this setting two process cannot synchronize on a sequence of actions with length greater than one. This is because of the rules $EComSng$ and $EComSeq$.
\end{example}



% \begin{example}
%   We want to prove that 
%   \[
%     (\underline{\overline{a}x}.c(x).0|b(x).0)|(a(x).0|\underline{\overline{b}x}.\overline{c}x.0)\;
%       \;\xrightarrow{\tau}\; 
% 	(0|0)|(0|0)
%   \]
% 
% 
% 
%   \begin{description}
%     \item[Str]
%       $(\underline{\overline{a}x}.c(x).0|b(x).0)|(a(x).0|\underline{\overline{b}x}.\overline{c}x.0)\;
% 	\;\xrightarrow{\tau}\; 
% 	  (0|0)|(0|0)$
%       \begin{description}
% 	\item[EComSng] 
% 	  $(\underline{\overline{a}x}.c(x).0|a(x).0)|(b(x).0|\underline{\overline{b}x}.\overline{c}x.0)\;
% 	    \;\xrightarrow{\tau}\;
% 	      (0|0)|(0|0)$
% 	  \begin{description}
% 	    \item[EComSeq]
% 	      $b(x).0|\underline{\overline{b}x}.\overline{c}x.0\;
% 		\;\xrightarrow{\overline{c}x}\;
% 		  0|0$
% 	      \begin{description}
% 		\item[EInp]
% 		  $b(x).0
% 		    \;\xrightarrow{bx}\;
% 		      0$
% 		\item[SOut]
% 		  $\underline{\overline{b}x}.\overline{c}x.0\;
% 		    \;\xrightarrow{\overline{b}x\cdot \overline{c}x}\;
% 		      0$
% 		  \begin{description}
% 		    \item[Out]
% 		      $\overline{c}x.0\;
% 			\;\xrightarrow{\overline{c}x}\;
% 			  0$
% 		  \end{description}
% 	      \end{description}
% 	    \item[EComSeq]
% 	      $\underline{\overline{a}x}.c(x).0|a(x).0\;
% 		\;\xrightarrow{cx}\;
% 		  0|0$
% 	      \begin{description}
% 		\item[SOut]
% 		  $\underline{\overline{a}x}.c(x).0\;
% 		    \;\xrightarrow{\overline{a}x\cdot cx}\;
% 		      0$
% 		  \begin{description}
% 		    \item[Inp]
% 		      $c(x).0\;
% 			\;\xrightarrow{cx}\;
% 			  0$
% 		  \end{description}
% 		\item[Inp]
% 		  $a(x).0\;
% 		    \;\xrightarrow{ax}\;
% 		      0$
% 	      \end{description}
% 	  \end{description}
% 	\item[]
% 	  $(\underline{\overline{a}x}.c(x).0|b(x).0)|(a(x).0|\underline{\overline{b}x}.\overline{c}x.0)\;
% 	    \equiv\;
% 	      (\underline{\overline{a}x}.c(x).0|a(x).0)|(b(x).0|\underline{\overline{b}x}.\overline{c}x.0)$
%       \end{description}
%   \end{description}
% \end{example}


% \begin{example}
%   The \emph{dining philosophers} problem, originally proposed by Dijkstra in \cite{djkstra}, is defined in the following way: Five silent philosophers sit at a round table. There is one fork between each pair of adjacent philosophers. Each philosopher must alternately think and eat. However, a philosopher can only eat while holding both the fork to the left and the fork to the right. Each philosopher can pick up an adjacent fork, when available, and put it down, when holding it. The problem is to design an algorithm such that no philosopher will starve, i.e. can forever continue to alternate between eating and thinking. We present one solution which uses only two forks and two philosophers:
%   \begin{itemize}
%     \item
%       we define two constants for the forks:
%       \begin{center}
% 	\begin{tabular}{ll}
% 	    $fork_{1}\; \stackrel{def}{=}\; up_{1}(x).dn_{1}(x).fork_{1}$
% 	  &
% 	    $fork_{0}\; \stackrel{def}{=}\; up_{0}(x).dn_{0}(x).fork_{0}$
% 	\end{tabular}
%       \end{center}
%       the input name $x$ is not important and can be anything else.
%     \item
%       we define two constants for the philosophers:
%       \begin{center}
% 	\begin{tabular}{l}
% 	    $phil_{1}\; \stackrel{def}{=}\; think(x).phil_{1}+\underline{\overline{up_{1}}x}.\overline{up_{0}}(x).eat(x).\underline{\overline{dn_{1}}x}.dn_{0}(x).phil_{1}$
% 	  \\
% 	    $phil_{0}\; \stackrel{def}{=}\; think(x).phil_{0}+\underline{\overline{up_{0}}x}.\overline{up_{1}}(x).eat(x).\underline{\overline{dn_{0}}x}.dn_{1}(x).phil_{0}$
% 	\end{tabular}
%       \end{center}
%       also in this case the name $x$ is not relevant.
%     \item
%       the following definition describe the whole system with philosophers and forks:
%       \begin{center}
% 	$DP\; \stackrel{def}{=}\; (\nu \{up_{0}, up_{1}, down_{0}, down_{1}\})(phil_{0}|phil_{1}|fork_{0}|fork_{1})$
%       \end{center}
%       where with $(\nu \{up_{0}, up_{1}, down_{0}, down_{1}\})$ we mean $(\nu\; up_{0}) (\nu\; up_{1}) (\nu\; down_{0}) (\nu\; down_{1})$
%     \item
%       the operational semantic of $DP$ is the following lts:
%       \begin{center}
% 	\begin{tikzpicture}[%
% 	    ->,
% 	    >=stealth,
% 	    shorten >=1pt,
% 	    node distance=2.8cm,
% 	    %on grid,
% 	    auto,
% 	    state/.append style={minimum size=2em},
% 	    thick
% 	  ]
% 
% 
%   %\tikzstyle{every state}=[fill=red,draw=none,text=white]
% 
% 
% 	  \node[state] (A) {$P_{1}eaten$};
% 	  \node[state] (B) [below right of=A] {$P_{1}F_{0,1}$};
% 	  \node[state] (DP)[above right of=B] {$DP$};
% 	  \node[state] (C) [below right of=DP] {$P_{0}F_{0,1}$};
% 	  \node[state] (D) [above right of=C] {$P_{0}eaten$};
%   
% 	  \path[->] 
% 	      
%               (DP)       edge [loop above] node {$think\;x$} (DP)
% 			 edge              node {$\tau$} (C)
% 			 edge              node {$\tau$} (B)
% 	      (A)        edge [loop above] node {$think\;x$} (A)
% 			 edge              node {$\tau$} (DP)
% 	      (B)        edge [loop below] node {$think\;x$} (B)
% 			 edge   	   node {$eat\;x$} (A)
% 	      (C)        edge [loop below] node {$think\;x$} (C)
% 			 edge              node {$eat\;x$} (D)
% 	      (D)        edge [loop above] node {$think\;x$} (D)
% 			 edge              node {$\tau$} (DP);
%          
%       \end{tikzpicture}
%     \end{center}
%   \end{itemize}
%   Now we need to prove every transition in the semantic of $DP$. Let $L=\{up_{0}, up_{1}, down_{0}, down_{1}\}$ we start with $DP\xrightarrow{\tau}DP$:
% 
% 
% 
% \end{example}

% \begin{example}
%   We want to show now an example of synchronization between four processes:
%   \begin{description}
%     \item[Res]
%       $(\nu\; a)((((\underline{\overline{a}x}.\underline{\overline{a}x}.\overline{a}x.0| a(x).0)| a(x).0)| a(x).0)\;
% 	\xrightarrow{\tau}\;
% 	  (\nu\; a)(((0|0)|0)|0))$
%       \begin{description}
% 	\item
% 	  $a\notin n(\tau)$
% 	\item[EComSng]
% 	  $(((\underline{\overline{a}x}.\underline{\overline{a}x}.\overline{a}x.0| a(x).0)| a(x).0)| a(x).0)\;
% 	    \xrightarrow{\tau}\;
% 	      ((0|0)|0)|0)$
% 	  \begin{description}
% 	    \item[EComSeq]
% 	      $(\underline{\overline{a}x}.\underline{\overline{a}x}.\overline{a}x.0| a(x).0)| a(x).0\;
% 		\xrightarrow{\overline{a}x}\;
% 		  (0|0)|0$
% 	      \begin{description}
% 		\item[EComSeq]
% 		  $\underline{\overline{a}x}.\underline{\overline{a}x}.\overline{a}x.0| a(x).0)\;
% 		    \xrightarrow{\overline{a}x\cdot \overline{a}x}\;
% 		      0|0$
% 		  \begin{description}
% 		    \item[SOut]
% 		      $\underline{\overline{a}x}.\underline{\overline{a}x}.\overline{a}x.0\;
% 			\xrightarrow{\overline{a}x\cdot \overline{a}x\cdot\overline{a}x}\;
% 			  0$
% 		      \begin{description}
% 			\item[SOut]
% 			  $\underline{\overline{a}x}.\overline{a}x.0\;\;\xrightarrow{\overline{a}x\cdot\overline{a}x}\;0$\newline
%  			  %\begin{description}
%  			  %   \item[SOut]
%  				$\;\;\;\; {\bf SOut}\; \underline{\overline{a}x}.\overline{a}x.0\;\;\xrightarrow{\overline{a}x\cdot\overline{a}x}\;0$\newline
% 	     			%\begin{description}
% 				%   \item[SOut]
% 				      .\hspace{4 mm}${\bf Out}\; \overline{a}x.0\;\;\xrightarrow{\overline{a}x}\;0$
% 				%\end{description}
%  			  %\end{description}
% 		      \end{description}
% 		    \item[Inp]
% 		      $a(x).0\;\;\xrightarrow{ax}\;0$
% 		  \end{description}
% 		\item[Inp]
% 		  $a(x).0\;\;\xrightarrow{ax}\;0$	 
% 	      \end{description}
% 	    \item[Inp]
% 	      $a(x).0\;\;\xrightarrow{ax}\;0$	      
% 	  \end{description}
%     \end{description}
%   \end{description}
% 
% \end{example}


% \begin{center}
% $\inferrule* [left=\bf{EComSeq}]{
%       \underline{\overline{x}y}.\overline{x}y.0
% 	\xrightarrow{\overline{x}y \cdot \overline{x}y}
% 	  0
%     \\
%     \inferrule* [left=\bf{EComSeq}]{
% 	  \underline{\overline{x}y}.\overline{x}y.0|x(y).0
% 	    \xrightarrow{\overline{x}y}
% 	      0|0
%       \\
% 	\inferrule* [left=\bf{Inp}]{
% 	}{
% 	  x(y).0	
% 	    \xrightarrow{xy} 
% 	      0
% 	}
%     }{
%       ((\underline{\overline{x}y}.\overline{x}y.0|x(y).0)|x(y).0)
% 	\xrightarrow{\tau}
% 	  ((0|0)|0)
%     }
%   }{
%     \underline{\overline{x}y}.\overline{x}y.0 | x(y).x(y).0
%       \xrightarrow{\tau}
% 	0|0
% }$
% \end{center}
% 
% \begin{center}
% $\inferrule* [left=\bf{EComSng}]{
%   \inferrule* [left=\bf{SOut}]{
%     \inferrule* [left=\bf{Out}]{
%     }{
%       \overline{x}y.0
% 	\xrightarrow{\overline{x}y}
% 	  0
%     }
%   }{
%     \underline{\overline{x}y}.\overline{x}y.0
%       \xrightarrow{\overline{x}y\cdot \overline{x}y}
% 	0
%   }
%   \\
%     x(y).0\; \;\xrightarrow{xy}\; 0
% }{
%   \underline{\overline{x}y}.\overline{x}y.0|x(y).0
%     \xrightarrow{\overline{x}y}
%       0|0
% }$
% \end{center}






\subsection{Late operational semantic with structural congruence}
% Definisci le regole late per Multi-pi. In questo caso credo sia indispensabile la restrizione sintattica che ti suggerivo(solo input nello strong prefixing e sincronizzazione solo con la prima), mentre mi pare che nel caso early dovrebbe funzionare anche il caso generale con Sync. 


\begin{definition}\index{transition relation! multipi! output only! late! with structural congruence}
  The \emph{late transition relation with structural congruence} is the smallest relation induced by the rules in table \ref{multipilatewith}.
  \begin{table}
    \begin{tabular}{ll}
	  \hline\\
	    $\inferrule* [left=\bf{Pref}]{
	    \alpha\; not\; a\; strong\; prefix
	  }{
	    \alpha.P \;\xrightarrow{\alpha} P
	  }$
	&
	  $\inferrule* [left=\bf{Par}]{
	      P\; \xrightarrow{\sigma} P^{'}
	    \\
	      bn(\sigma)\cap fn(Q)=\emptyset	  
	  }{
	    P|Q\; \xrightarrow{\sigma}\; P^{'}|Q
	  }$
      \\\\
	  $\inferrule* [left=\bf{SOut}]{
	      P \;\xrightarrow{\sigma} P^{'}
	    \\
	      \sigma\neq \tau
	  }{
	      \underline{\overline{x}y}.P \;\xrightarrow{\overline{x}y \cdot \sigma} P^{'}
	  }$
	&
	  $\inferrule* [left=\bf{LComSeq}]{
	      P \;\xrightarrow{xy} P^{'}
	    \\
	      Q\;\xrightarrow{\overline{x}z\cdot \sigma} Q^{'}
	    \\
	      z\notin fn(P)
	  }{
	    P|Q \;\xrightarrow{\sigma} P^{'}\{z/y\}|Q^{'}
	  }$
      \\\\
	  $\inferrule* [left=\bf{Sum}]{
	    P \;\xrightarrow{\sigma} P^{'}
	  }{
	    P+Q \;\xrightarrow{\sigma} P^{'}
	  }$
	&
	  $\inferrule* [left=\bf{Str}]{
	      P\equiv P^{'}
	    \\
	      P^{'}\; \;\xrightarrow{\alpha}\; Q^{'}
	    \\
	      Q\equiv Q^{'}
	  }{
	      P\; \;\xrightarrow{\alpha}\; Q
	  }$
      \\\\
	  $\inferrule* [left=\bf{Res}]{
	    P \;\xrightarrow{\sigma} P^{'}\;\; z\notin n(\alpha)
	  }{
	    (\nu) z P \;\xrightarrow{\sigma} (\nu) z P^{'}
	  }$
	&
	  $\inferrule* [left=\bf{LComSng}]{
	      P \;\xrightarrow{xy} P^{'}
	    \\
	      Q\;\xrightarrow{\overline{x}z} Q^{'}
	    \\
	      z\notin fn(P)
	  }{
	    P|Q \;\xrightarrow{\tau} P^{'}\{z/y\}|Q^{'}
	  }$
      \\\hline
    \end{tabular}
    \caption{Multi$\pi$ late semantic with structural congruence}
    \label{multipilatewith}
  \end{table}
\end{definition}

\begin{example}[Multi-party synchronization]
  We show an example of a derivation of three processes that synchronize in the late semantic.
\begin{center}
$\inferrule* [left=\bf{Res}]{
    x\notin n(\tau)
   \\
    \inferrule* [left=\bf{LComSeq}]{
	  \underline{\overline{x}y}.\overline{x}y.0|x(y).0
	    \xrightarrow{\overline{x}y}
	      0|0
      \\
	\inferrule* [left=\bf{Pref}]{
	}{
	  x(y).0	
	    \xrightarrow{x(y)} 
	      0
	}
    }{
      ((\underline{\overline{x}y}.\overline{x}y.0|x(y).0)|x(y).0)
	\xrightarrow{\tau}
	  ((0|0)|0)
    }
  }{
    (\nu x)((\underline{\overline{x}y}.\overline{x}y.0|x(y).0)|x(y).0)
      \xrightarrow{\tau}
	(\nu x)((0|0)|0)
}$
\end{center}

\begin{center}
$\inferrule* [left=\bf{LComSng}]{
  \inferrule* [left=\bf{SOut}]{
    \inferrule* [left=\bf{Pref}]{
    }{
      \overline{x}y.0
	\xrightarrow{\overline{x}y}
	  0
    }
  }{
    \underline{\overline{x}y}.\overline{x}y.0
      \xrightarrow{\overline{x}y\cdot \overline{x}y}
	0
  }
  \\
    \inferrule* [left=\bf{Pref}]{
    }{
      x(y).0\; \;\xrightarrow{x(y)}\; 0
    }
}{
  \underline{\overline{x}y}.\overline{x}y.0|x(y).0
    \xrightarrow{\overline{x}y}
      0|0
}$
\end{center}

\end{example}

















\subsection{Low level semantic}
This section contains the definition of an alternative semantic for multi $\pi$. First we define a low level version of the multi $\pi$ calculus(here with strong prefixing on output only), we call this language low multi $\pi$. The low multi $\pi$ is the multi $\pi$ enriched with a marked or intermediate process $*P$:
\begin{center}
   \begin{tabular}{l}
     $P,Q$ ::= $0$ | $\pi.P$ | $P|Q$ | $P+Q$ | $(\nu x) P$ | $A(x_{1}, \cdots, x_{n})$ | $*P$
   \\\\
     $\pi$ ::= $\overline{x}y$ | $x(y)$ | $\underline{\overline{x}(y)}$ | $\tau$ 
   \end{tabular}
\end{center}
\begin{definition}
  The low level transition relation is the smallest relation induced by the rules in table \ref{lowleveltransitionrelation} in which $P$ stands for a process without mark, $L$ stands for a process with mark and $S$ can stand for both. 
  \begin{table}
    \begin{tabular}{ll}
      \hline\\
	  $\inferrule* [left=\bf{Out}]{
	  }{
	    \overline{x}y.P \stackrel{\overline{x}y}{\longmapsto} P
	  }$
	  &
	  $\inferrule* [left=\bf{EInp}]{
	  }{
	    x(y).P \stackrel{xz}{\longmapsto} P\{z/y\}
	  }$
      \\\\
	  $\inferrule* [left=\bf{SOutLow}]{
	  }{
	    \underline{\overline{x}y}.P \stackrel{xy}{\longmapsto} * P
	  }$
	  &
	  $\inferrule* [left=\bf{Tau}]{
	  }{
	    \tau.P \stackrel{\tau}{\longmapsto} P
	  }$
      \\\\
	  $\inferrule* [left=\bf{Sum}]{
	    P \stackrel{\gamma}{\longmapsto} S
	  }{
	    P+Q \stackrel{\gamma}{\longmapsto} S
	  }$
	  &
	  $\inferrule* [left=\bf{Star}]{
	      S \stackrel{\gamma}{\longmapsto} S^{'}
	  }{
	      *S \stackrel{\gamma}{\longmapsto} S^{'}
	  }$
      \\\\\\
	  $\inferrule* [left=\bf{Com1}]{
	      P \stackrel{\overline{x}y}{\longmapsto} P^{'}
	    \\
	      Q \stackrel{xy}{\longmapsto} Q^{'}
	  }{
	    P|Q \stackrel{\tau}{\longmapsto} P^{'}|Q^{'}
	  }$
	&
      \\\\
	  $\inferrule* [left=\bf{Com2L}]{
	      L_{1} \stackrel{xy}{\longmapsto} L_{1}^{'}
	    \\
	      L_{2} \stackrel{\overline{x}y}{\longmapsto} S
	  }{
	    L_{1}|L_{2} \stackrel{\epsilon}{\longmapsto} L_{1}^{'}|S
	  }$
	&
	  $\inferrule* [left=\bf{Com2R}]{
	      L_{1} \stackrel{\overline{x}y}{\longmapsto} L_{1}^{'}
	    \\
	      L_{2} \stackrel{xy}{\longmapsto} S
	  }{
	    L_{1}|L_{2} \stackrel{\epsilon}{\longmapsto} L_{1}^{'}|S
	  }$
      \\\\
	  $\inferrule* [left=\bf{Com3L}]{
	      P \stackrel{xy}{\longmapsto} L
	    \\
	      Q \stackrel{\overline{x}y}{\longmapsto} S
	  }{
	    P|Q \stackrel{\epsilon}{\longmapsto} L|S
	  }$
	&
	  $\inferrule* [left=\bf{Com3R}]{
	      P \stackrel{\overline{x}y}{\longmapsto} L
	    \\
	      Q \stackrel{xy}{\longmapsto} S
	  }{
	    P|Q \stackrel{\epsilon}{\longmapsto} L|S
	  }$
      \\\\
	  $\inferrule* [left=\bf{Com4L}]{
	      L_{1} \stackrel{\overline{x}y}{\longmapsto} P
	    \\
	      L_{2} \stackrel{xy}{\longmapsto} Q
	  }{
	    L_{1}|L_{2} \stackrel{\tau}{\longmapsto} P|Q
	  }$
	&
	  $\inferrule* [left=\bf{Com4L}]{
	      L_{1} \stackrel{xy}{\longmapsto} P
	    \\
	      L_{2} \stackrel{\overline{x}y}{\longmapsto} Q
	  }{
	    L_{1}|L_{2} \stackrel{\tau}{\longmapsto} P|Q
	  }$
      \\\\\\
	  $\inferrule* [left=\bf{Res}]{
	      S \stackrel{\gamma}{\longmapsto} S^{'}
	    \\
	      y\notin n(\gamma)
	  }{
	    (\nu y) S \stackrel{\gamma}{\longmapsto} (\nu y) S^{'}
	  }$
	&
	  $\inferrule* [left=\bf{OpnSeq}]{
	      P \xrightarrow{\sigma}\; P^{'}
	    \\ 
	      \exists \overline{x}z\in \sigma:\; x\neq z
	  }{
	      (\nu z)P \xrightarrow{opn(\sigma,z)}\; P^{'}
	  }$
      \\\\\\
	  $\inferrule* [left=\bf{Par1L}]{
	      S \stackrel{\gamma}{\longmapsto} S^{'}
% 	    \\ 
% 	      bn(\gamma)\cap fn(Q)=\emptyset
	  }{
	      S|Q \stackrel{\gamma}{\longmapsto} S^{'}|Q
	  }$
	&
	  $\inferrule* [left=\bf{Par1R}]{
	      S \stackrel{\gamma}{\longmapsto} S^{'}
% 	    \\ 
% 	      bn(\gamma)\cap fn(Q)=\emptyset
	  }{
	      Q|S \stackrel{\gamma}{\longmapsto} Q|S^{'}
	  }$
      \\\\\\
	  $\inferrule* [left=\bf{Par2L}]{
	     P \stackrel{\gamma}{\longmapsto} L
% 	    \\ 
% 	      bn(\gamma)\cap fn(Q)=\emptyset
	  }{
	      P|Q \stackrel{\gamma}{\longmapsto} L|*Q
	  }$
	&
	  $\inferrule* [left=\bf{Par2R}]{
	     P \stackrel{\gamma}{\longmapsto} L
% 	    \\ 
% 	      bn(\gamma)\cap fn(Q)=\emptyset
	  }{
	      Q|P \stackrel{\gamma}{\longmapsto} *Q|L
	  }$
      \\\\
	  $\inferrule* [left=\bf{Par3L}]{
	      L_{1} \stackrel{\gamma}{\longmapsto} L_{1}^{'}
% 	    \\ 
% 	      bn(\gamma)\cap fn(L_{2})=\emptyset
	  }{
	      L_{1}|L_{2} \stackrel{\gamma}{\longmapsto} L_{1}^{'}|L_{2}
	  }$
	&
	  $\inferrule* [left=\bf{Par3R}]{
	      L_{2} \stackrel{\gamma}{\longmapsto} L_{2}^{'}
% 	    \\ 
% 	      bn(\gamma)\cap fn(L_{2})=\emptyset
	  }{
	      L_{1}|L_{2} \stackrel{\gamma}{\longmapsto} L_{1}|L_{2}^{'}
	  }$
      \\\\\\
	&
	  $\inferrule* [left=\bf{Cong}]{
	      P\equiv P^{'}
	    \\
	      P^{'} \stackrel{\gamma}{\longmapsto} S
	  }{
	      P \stackrel{\gamma}{\longmapsto} S
	  }$
      \\\\\hline
    \end{tabular}
    \caption{Low multi $\pi$ early semantic with structural congruence}
    \label{lowleveltransitionrelation}
  \end{table}
\end{definition}

\begin{lemma}\label{lemmacom3}
  :\begin{itemize}
    \item
      It cannot happen that there exist an unmarked process $P$, a marked process $L$ and an action $\overline{x}y$ such that: $P\stackrel{\overline{x}y}{\longmapsto}L$. 
    \item
      Also it cannot be the case that there exist two unmarked processes $L_{1}, L_{2}$ and an action $\overline{x}y$ such that: $L_{1}\stackrel{\overline{x}y}{\longmapsto}L_{2}$
  \end{itemize}
  \begin{proof}
    DA FARE
  \end{proof}
\end{lemma}


\openrigthchapter{Multi $\pi$ calculus solo input}

\section{Syntax}

As we did whit $\pi$ calculus, we suppose that we have a countable set of names $\mathbb{N}$, ranged over by lower case letters $a,b, \cdots, z$. This names are used for communication channels and values. Furthermore we have a set of identifiers, ranged over by $A$. We represent the agents or processes by upper case letters $P,Q, \cdots $. A multi $\pi$ process, in addiction to the same actions of a $\pi$ process, can perform also a strong prefix input:
\begin{center}
  $\pi$ ::= $\overline{x}y$ | $x(z)$ | $\underline{x(y)}$ | $\tau$ 
\end{center}
The process are defined, just as original $\pi$ calculus, by the following grammar:
\begin{center}
  \begin{tabular}{l}
    $P,Q$ ::= $0$ | $\pi.P$ | $P|Q$ | $P+Q$ | $(\nu x) P$ | $A(y_{1}, \cdots, y_{n})$
  \end{tabular}
\end{center}
and they have the same intuitive meaning as for the $\pi$ calculus. The strong prefix input allows a process to make an atomic sequence of actions, so that more than one process can synchronize on this sequence. For the moment we allow the strong prefix to be on input names only. Also one can use the strong prefix only as an action prefixing for processes that can make at least a further action. 

Multi $\pi$ calculus is a conservative extension of the $\pi$ calculus in the sense that: any $\pi$ calculus process $p$ is also a multi $\pi$ calculus process and the semantic of $p$ according to the SOS rules of $\pi$ calculus is the same as the semantic of $p$ according to the SOS rules of multi $\pi$ calculus. 
We have to extend the following definition to deal with the strong prefix:
\begin{center}
  \begin{tabular}{ll}
	$B(\underline{x(y)}.Q, I) = \{y,\overline{y}\}\cup B(Q, I)$
      &
	$F(\underline{x(y)}.Q, I) = \{x,\overline{x}\}\cup (F(Q, I)-\{y,\overline{y}\})$
    \\
  \end{tabular}
\end{center}


In this setting two process cannot synchronize on a sequence of actions with length greater than one so we cannot have transactional synchronization but we can have multi-party synchronization.


\section{Operational semantic}

\subsection{Early operational semantic with structural congruence}

The semantic of a multi $\pi$ process is labeled transition system such that
\begin{itemize}
  \item 
    the nodes are multi $\pi$ calculus process. The set of node is $\mathbb{P}_{m}$
  \item
    the actions are multi $\pi$ calculus actions. The set of actions is $\mathbb{A}_{m}$, we use $\alpha, \alpha_{1}, \alpha_{2},\cdots $ to range over the set of actions, we use $\sigma, \sigma_{1}, \sigma_{2}, \cdots $ to range over the set $\mathbb{A}_{m}^{+} \cup \{\tau\}$.
  \item
    the transition relations is $\rightarrow\subseteq \mathbb{P}_{m}\times (\mathbb{A}_{m}^{+} \cup \{\tau\})\times \mathbb{P}_{m}$
\end{itemize}

In this case, a label can be a sequence of prefixes, whether in the original $\pi$ calculus a label can be only a prefix. We use the symbol $\cdot$ to denote the concatenation operator.

\begin{definition}
  The \emph{early transition relation with structural congruence} is the smallest relation induced by the rules in table \ref{multipisoloinputearlywith} where $inpSeq$ is a non empty sequence of input actions and $\sigma$ is a sequence of any action.
  \begin{table}
    \begin{tabular}{lll}
	  \hline\\
	  $\inferrule* [left=\bf{Out}]{
	  }{
	    \overline{x}y.P \xrightarrow{\overline{x}y} P
	  }$
	&
	  $\inferrule* [left=\bf{EInp}]{
	  }{
	    x(y).P \xrightarrow{xz} P\{z/y\}
	  }$
	&
	  $\inferrule* [left=\bf{Tau}]{
	  }{
	    \tau.P \xrightarrow{\tau} P
	  }$
      \\\\
	  $\inferrule* [left=\bf{SInpTau}]{
	      P\{y/z\} \xrightarrow{\tau} P^{'}
	  }{
	    \underline{x(z)}.P \xrightarrow{xy} P^{'}
	  }$
	&
	  $\inferrule* [left=\bf{SInp}]{
	      P\{y/z\} \xrightarrow{ab} P^{'}
% 	    \\
% 	      y\notin fn((\nu z) P)
	  }{
	    \underline{x(z)}.P \xrightarrow{xy \cdot ab} P^{'}
	  }$
	&
	  $\inferrule* [left=\bf{SInpSeq}]{
	      P\{y/z\} \xrightarrow{\sigma} P^{'}
	    \\
	      |\sigma|>1
% 	    \\
% 	      y\notin fn((\nu z) P)
	  }{
	    \underline{x(z)}.P \xrightarrow{xy \cdot \sigma} P^{'}
	  }$
      \\\\
	  $\inferrule* [left=\bf{Sum}]{
	    P \xrightarrow{\sigma} P^{'}
	  }{
	    P+Q \xrightarrow{\sigma} P^{'}
	  }$
	&
	  $\inferrule* [left=\bf{Cong}]{
	      P\equiv P^{'}
	    \\
	      P^{'} \xrightarrow{\alpha} Q
	  }{
	      P \xrightarrow{\alpha} Q
	  }$
	&
	  $\inferrule* [left=\bf{Res}]{
	      P \xrightarrow{\sigma} P^{'}
	    \\
	      z\notin n(\sigma)
	  }{
	    (\nu z) P \xrightarrow{\sigma} (\nu z) P^{'}
	  }$
      \\\\
	  $\inferrule* [left=\bf{Par}]{
	      P \xrightarrow{\sigma} P^{'}
	  }{
	      P|Q \xrightarrow{\sigma} P^{'}|Q
	  }$
	&
	  $\inferrule* [left=\bf{Opn}]{
	      P \xrightarrow{\overline{x}z} P^{'}
	    \\ 
	      z\neq x
	  }{
	      (\nu z)P \xrightarrow{\overline{x}(z)} P^{'}
	  }$
	&
	  $\inferrule* [left=\bf{ECom}]{
	      P \xrightarrow{xy} P^{'}
	    \\
	      Q \xrightarrow{\overline{x}y} Q^{'}
	  }{
	    P|Q \xrightarrow{\tau} P^{'}|Q^{'}
	  }$
      \\\\
	&
	&
	  $\inferrule* [left=\bf{EComSeq}]{
	      P \xrightarrow{xy\cdot \sigma} P^{'}
	    \\
	      Q \xrightarrow{\overline{x}y} Q^{'}
	  }{
	    P|Q \xrightarrow{\sigma} P^{'}|Q^{'}
	  }$
      \\\\
	&
	&
      \\\\\hline
    \end{tabular}
    \caption{Multi $\pi$ early semantic with structural congruence}
    \label{multipisoloinputearlywith}
  \end{table}
\end{definition}



\begin{example}Multi-party synchronization
  We show an example of a derivation of three processes that synchronize.
  \begin{center}
  $
      \inferrule* [left=\bf{EComSng}]{
	\underline{x(a)}.x(b).P|\overline{x}y.Q)
	  \xrightarrow{xz}
	    P\{y/a\}\{z/b\}|Q
	\\
	  \inferrule* [left=\bf{Out}]{
	  }{
	    \overline{x}z.R	
	      \xrightarrow{\overline{x}z} 
		R
	  }
      }{
	(\underline{x(a)}.x(b).P|\overline{x}y.Q)|\overline{x}z.R
	  \xrightarrow{\tau}
	    (P\{y/a\}\{z/b\}|Q)|R
      }
  $
  \end{center}
  
  \begin{center}
  $\inferrule* [left=\bf{EComSeq}]{
      \inferrule* [left=\bf{SInp}]{
	\inferrule* [left=\bf{EInp}]{
	}{
	  (x(b).P)\{y/a\} \xrightarrow{xz} P\{y/a\}\{z/b\}
	}
      }{
	\underline{x(a)}.(x(b).P) 
	  \xrightarrow{xy \cdot xz} 
	    P\{y/a\}\{z/b\}
      }
    \\
      \inferrule* [left=\bf{Out}]{
      }{
	\overline{x}y.Q \xrightarrow{\overline{x}y} Q
      }
  }{
	\underline{x(a)}.x(b).P|\overline{x}y.Q)
	  \xrightarrow{xz}
	    P\{y/a\}\{z/b\}|Q
  }$
  \end{center}

\end{example}

\begin{lemma}\label{lemmastrongsequence}
  If $P\xrightarrow{\sigma} Q$ then only one of the following cases hold: 
  \begin{itemize}
    \item 
      $|\sigma|=1$
    \item
      $|\sigma|>1$, the actions in $\sigma$ are input.
  \end{itemize}
\end{lemma}



\subsection{Late operational semantic with structural congruence}

\begin{definition}
  The \emph{late transition relation with structural congruence} is the smallest relation induced by the rules in table \ref{multipisoloinputlateywith}.
  \begin{table}
    \begin{tabular}{ll}
	\hline\\
     	  $\inferrule* [left=\bf{Pref}]{
	    \alpha not a strong prefix
	  }{
	    \alpha.P \xrightarrow{\alpha} P
	  }$
	&
	  $\inferrule* [left=\bf{LComSeq}]{
	      P \xrightarrow{x(y)\cdot \sigma} P^{'}
	    \\
	      Q\xrightarrow{\overline{x}z} Q^{'}
	    \\
	      z\notin fn(\sigma)\cup fn(P)
	  }{
	    P|Q \xrightarrow{\sigma\{z/y\}} P^{'}\{z/y\}|Q^{'}
	  }$
      \\\\
	  $\inferrule* [left=\bf{SInp}]{
	      P \xrightarrow{a(b)} P^{'}
	  }{
	    \underline{x(y)}.P \xrightarrow{x(y) \cdot a(b)} P^{'}
	  }$
	&
	  $\inferrule* [left=\bf{LCom}]{
	      P \xrightarrow{x(y)} P^{'}
	    \\
	      Q\xrightarrow{\overline{x}z} Q^{'}
	    \\
	      z\notin fn(P)
	  }{
	    P|Q \xrightarrow{\tau} P^{'}\{z/y\}|Q^{'}
	  }$
      \\\\
	  $\inferrule* [left=\bf{Sum}]{
	    P \xrightarrow{\sigma} P^{'}
	  }{
	    P+Q \xrightarrow{\sigma} P^{'}
	  }$
	&
	  $\inferrule* [left=\bf{Cong}]{
	      P\equiv P^{'}
	    \\
	      P^{'} \xrightarrow{\sigma} Q
	  }{
	      P \xrightarrow{\sigma} Q
	  }$
      \\\\
	  $\inferrule* [left=\bf{Res}]{
	      P \xrightarrow{\sigma} P^{'}
	    \\
	      z\notin n(\alpha)
	  }{
	    (\nu z) P \xrightarrow{\sigma} (\nu z) P^{'}
	  }$
	&
	  $\inferrule* [left=\bf{Par}]{
	      P \xrightarrow{\sigma} P^{'}
	    \\
	      bn(\sigma)\cup fn(Q)=\emptyset
	  }{
	    P|Q \xrightarrow{\sigma} P^{'}|Q
	  }$
      \\\\
	  $\inferrule* [left=\bf{Opn}]{
	      P \xrightarrow{\overline{x}z} P^{'}
	    \\ 
	      z\neq x
	  }{
	      (\nu z)P \xrightarrow{\overline{x}(z)} P^{'}
	  }$
	&
	  $\inferrule* [left=\bf{SInpSeq}]{
	      P \xrightarrow{\sigma} P^{'}
	    \\
	      |\sigma|>1
	  }{
	    \underline{x(y)}.P \xrightarrow{x(y) \cdot \sigma} P^{'}
	  }$
      \\\\
	  $\inferrule* [left=\bf{SInpTau}]{
	      P \xrightarrow{\tau} P^{'}
	  }{
	    \underline{x(y)}.P \xrightarrow{x(y)} P^{'}
	  }$
	&
      \\\hline
    \end{tabular}
    \caption{Multi $\pi$ late semantic with structural congruence}
    \label{multipisoloinputlateywith}
  \end{table}
\end{definition}

\begin{example}Multi-party synchronization
  We show an example of a derivation of three processes that synchronize with the late semantic. The three processes are $\underline{x(a)}.x(b).P$, $\overline{x}y.Q$ and $\overline{x}z.R$. We assume that:
  \begin{center}
      $a\notin fn(x(b))\cup fn (\underline{x(a)}.x(b).P)$
  \end{center}
  and
  \begin{center}
      $b\notin fn(\underline{x(a)}.x(b).P|\overline{x}y.Q)$
  \end{center}

  \begin{center}
  $
      \inferrule* [left=\bf{LCom}]{
	\underline{x(a)}.x(b).P|\overline{x}y.Q
	  \xrightarrow{x(b)}
	    P\{y/a\}|Q
	\\
	  \inferrule* [left=\bf{Pref}]{
	  }{
	    \overline{x}z.R	
	      \xrightarrow{\overline{x}z} 
		R
	  }
      }{
	(\underline{x(a)}.x(b).P|\overline{x}y.Q)|\overline{x}z.R
	  \xrightarrow{\tau}
	    (P\{y/a\}|Q)\{z/b\}|R
      }
  $
  \end{center}
  
  \begin{center}
  $\inferrule* [left=\bf{LComSeq}]{
      \inferrule* [left=\bf{SInp}]{
	\inferrule* [left=\bf{Pref}]{
	}{
	  x(b).P \xrightarrow{x(b)} P
	}
      }{
	\underline{x(a)}.x(b).P
	  \xrightarrow{x(a) \cdot x(b)} 
	    P
      }
    \\
      \inferrule* [left=\bf{Out}]{
      }{
	\overline{x}y.Q \xrightarrow{\overline{x}y} Q
      }
  }{
	\underline{x(a)}.x(b).P|\overline{x}y.Q)
	  \xrightarrow{x(b)}
	    P\{y/a\}|Q
  }$
  \end{center}

\end{example}



\subsection{Low level semantic}
This section contains the definition of an alternative semantic for multi $\pi$. First we define a low level version of the multi $\pi$ calculus(here with strong prefixing on input only), we call this language low multi $\pi$. The low multi $\pi$ is the multi $\pi$ enriched with a marked or intermediate process $*P$:
\begin{center}
   \begin{tabular}{l}
     $P,Q$ ::= $0$ | $\pi.P$ | $P|Q$ | $P+Q$ | $(\nu x) P$ | $A(x_{1}, \cdots, x_{n})$ | $*P$
   \\\\
     $\pi$ ::= $\overline{x}y$ | $x(z)$ | $\underline{x(y)}$ | $\tau$ 
   \end{tabular}
\end{center}
\begin{definition}
  The low level transition relation is the smallest relation induced by the rules in table \ref{lowleveltransitionrelationinput} in which $P$ stands for a process without mark, $L$ stands for a process with mark and $S$ can stand for both. 
  \begin{table}
    \begin{tabular}{lll}
      \hline\\
	  $\inferrule* [left=\bf{Out}]{
	  }{
	    \overline{x}y.P \stackrel{\overline{x}y}{\longmapsto} P
	  }$
	  &
	  $\inferrule* [left=\bf{EInp}]{
	  }{
	    x(y).P \stackrel{xz}{\longmapsto} P\{z/y\}
	  }$
	  &
	  $\inferrule* [left=\bf{Tau}]{
	  }{
	    \tau.P \stackrel{\tau}{\longmapsto} P
	  }$
      \\\\
	  $\inferrule* [left=\bf{StarInp}]{
	      S \stackrel{xy}{\longmapsto} S^{'}
	  }{
	      *S \stackrel{xy}{\longmapsto} S^{'}
	  }$
	  &
	  $\inferrule* [left=\bf{SInpLow}]{
%	      y\notin fn(P)-\{z\}
	  }{
	    \underline{x(z)}.P \stackrel{xy}{\longmapsto} * P\{y/z\}
	  }$
	  &
	  $\inferrule* [left=\bf{Sum}]{
	    P \stackrel{\gamma}{\longmapsto} S
	  }{
	    P+Q \stackrel{\gamma}{\longmapsto} S
	  }$
      \\\\
	  $\inferrule* [left=\bf{StarEps}]{
	      S \stackrel{\epsilon}{\longmapsto} S^{'}
	  }{
	      *S \stackrel{\epsilon}{\longmapsto} S^{'}
	  }$
	  &
	  &
      \\\\
      \end{tabular}
      \begin{tabular}{lll}
      \\\\
	  $\inferrule* [left=\bf{Com1}]{
	      P \stackrel{\overline{x}y}{\longmapsto} P^{'}
	    \\
	      Q \stackrel{xy}{\longmapsto} Q^{'}
	  }{
	    P|Q \stackrel{\tau}{\longmapsto} P^{'}|Q^{'}
	  }$
	  &
	  &
      \\\\
	  $\inferrule* [left=\bf{Com2L}]{
	      L_{1} \stackrel{xy}{\longmapsto} L_{2}
	    \\
	      P \stackrel{\overline{x}y}{\longmapsto} Q
	  }{
	    L_{1}|P \stackrel{\epsilon}{\longmapsto} L_{2}|Q
	  }$
	&
	  $\inferrule* [left=\bf{Com2R}]{
	      P \stackrel{\overline{x}y}{\longmapsto} Q
	    \\
	      L_{1} \stackrel{xy}{\longmapsto} L_{2}
	  }{
	    P|L_{1} \stackrel{\epsilon}{\longmapsto} Q|L_{2}
	  }$
	  &
      \\\\
	  $\inferrule* [left=\bf{Com3L}]{
	      P \stackrel{xy}{\longmapsto} L
	    \\
	      Q \stackrel{\overline{x}y}{\longmapsto} Q^{'}
	  }{
	    P|Q \stackrel{\epsilon}{\longmapsto} L|Q^{'}
	  }$
	&
	  $\inferrule* [left=\bf{Com3R}]{
	      Q \stackrel{\overline{x}y}{\longmapsto} Q^{'}	      
	    \\
	      P \stackrel{xy}{\longmapsto} L
	  }{
	    Q|P \stackrel{\epsilon}{\longmapsto} Q^{'}|L
	  }$
	  &
      \\\\
	  $\inferrule* [left=\bf{Com4L}]{
	      L \stackrel{xy}{\longmapsto} P
	    \\
	      Q \stackrel{\overline{x}y}{\longmapsto} Q^{'}
	  }{
	    L|Q \stackrel{\tau}{\longmapsto} P|Q^{'}
	  }$
	  &
	  $\inferrule* [left=\bf{Com4R}]{
	      Q \stackrel{\overline{x}y}{\longmapsto} Q^{'}
	    \\
	      L \stackrel{xy}{\longmapsto} P
	  }{
	    L|Q \stackrel{\tau}{\longmapsto} P|Q^{'}
	  }$
	  &
      \\\\
      \end{tabular}
      \begin{tabular}{lll}
      \\\\
	  $\inferrule* [left=\bf{Res}]{
	      S \stackrel{\gamma}{\longmapsto} S^{'}
	    \\
	      y\notin n(\gamma)
	  }{
	    (\nu y) S \stackrel{\gamma}{\longmapsto} (\nu y) S^{'}
	  }$
	  &
	  $\inferrule* [left=\bf{Opn}]{
	      P \stackrel{\overline{x}y}{\longmapsto} Q
	    \\ 
	      y\neq x
	  }{
	      (\nu y)P \stackrel{\overline{x}(y)}{\longmapsto} Q
	  }$
	  &
	  $\inferrule* [left=\bf{Cong}]{
	      P\equiv P^{'}
	    \\
	      P^{'} \stackrel{\gamma}{\longmapsto} S
	  }{
	      P \stackrel{\gamma}{\longmapsto} S
	  }$
      \\\\
      \end{tabular}
      \begin{tabular}{lll}
      \\\\
	  $\inferrule* [left=\bf{Par1L}]{
	      S \stackrel{\gamma}{\longmapsto} S^{'}
% 	    \\ 
% 	      bn(\gamma)\cap fn(Q)=\emptyset
	  }{
	      S|Q \stackrel{\gamma}{\longmapsto} S^{'}|Q
	  }$
	&
	  $\inferrule* [left=\bf{Par1R}]{
	      S \stackrel{\gamma}{\longmapsto} S^{'}
% 	    \\ 
% 	      bn(\gamma)\cap fn(Q)=\emptyset
	  }{
	      Q|S \stackrel{\gamma}{\longmapsto} Q|S^{'}
	  }$
	  &
      \\\\\hline
    \end{tabular}
    \caption{Low multi $\pi$ early semantic with structural congruence}
    \label{lowleveltransitionrelationinput}
  \end{table}
\end{definition}



\begin{lemma}\label{lemmacom3}
  For all unmarked processes $P,Q$ and marked processes $L_{1}, L_{2}$.
  \begin{itemize}
    \item
      if $P\stackrel{\alpha}{\longmapsto}L_{1}$ or $L_{1}\stackrel{\alpha}{\longmapsto}L_{2}$ then $\alpha$ can only be an input or an $\epsilon$
    \item
      if $L_{1}\stackrel{\alpha}{\longmapsto}P$ then $\alpha$ is an input or a $\tau$
    \item
      if $P\stackrel{\alpha}{\longmapsto}Q$ then $\alpha$ is not an $\epsilon$
  \end{itemize}
  \begin{proof}
    DA FARE
  \end{proof}
\end{lemma}


  
\begin{definition}\label{low}
  Let $P, Q$ be unmarked processes and $L_{1}, \cdots, L_{k-1}$ marked processes. We define the derivation relation $\rightarrow_{s}$ in the following way:
  \begin{center}
    $\inferrule* [left=\bf{Low}]{
	P \stackrel{\gamma_{1}}{\longmapsto} L_{1} \stackrel{\gamma_{2}}{\longmapsto} L_{2} \cdots L_{k-1} \stackrel{\gamma_{k}}{\longmapsto} Q
      \\
	k\geq 1
    }{
      P \xrightarrow{\gamma_{1} \cdots \gamma_{k}}_{s}  Q
    }$
  \end{center}
  We need to be precise about the concatenation operator $\cdot$ since we have introduced the new label $\epsilon$. Let $a$ be an action such that $a\neq \tau$ and $a\neq \epsilon$ then the following rules hold:
  \begin{center}
      \begin{tabular}{lll}
	  $\epsilon \cdot a = a \cdot \epsilon = a$
	&
	  $\epsilon \cdot \epsilon = \epsilon$
	&
	  $\tau \cdot \epsilon = \epsilon \cdot \tau = \tau$
	\\
	  $\tau \cdot a = a \cdot \tau = a$
	&
	  $\tau \cdot \tau = \tau$
	&
      \end{tabular}
  \end{center}
\end{definition}

\begin{example}Multi-party synchronization
  We show an example of a derivation of three processes that synchronize.
 
  \begin{center}$
    \inferrule* [left=\bf{Par1L}]{
      \inferrule* [left=\bf{Com3}]{
	\inferrule* [left=\bf{SInpLow}]{
	}{
	  \underline{x(a)}.x(b).P
	    \stackrel{xy}{\longmapsto}
	      *(x(b).P\{y/a\})
	}
      \\
	\inferrule* [left=\bf{Out}]{
	}{
	  \overline{x}y.Q \stackrel{\overline{x}y}{\longmapsto} Q
	}
      }{
	\underline{x(a)}.x(b).P|\overline{x}y.Q
	  \stackrel{\epsilon}{\longmapsto}
	    *(x(b).P\{y/a\})|Q
      }
  }{
	(\underline{x(a)}.x(b).P|\overline{x}y.Q) | \overline{x}z.R
	  \stackrel{\epsilon}{\longmapsto}
	    (*(x(b).P\{y/a\})|Q)|\overline{x}z.R
  }
  $\end{center}

  \begin{center}$
    \inferrule* [left=\bf{Par1}]{
      \inferrule*[left=\bf{Star}]{
	\inferrule* [left=\bf{EInp}]{
	}{
	  x(b).P\{y/a\} \stackrel{xz}{\longmapsto} P\{y/a\}\{z/b\}
	}
      }{
	*(x(b).P\{y/a\}) \stackrel{xz}{\longmapsto} P\{y/a\}\{z/b\}      
      }
    }{
      *(x(b).P\{y/a\}) | Q \stackrel{xz}{\longmapsto} P\{y/a\}\{z/b\} | Q
    }
  $\end{center}

  \begin{center}$
    \inferrule* [left=\bf{Com4}]{
      *(x(b).P\{y/a\}) | Q \stackrel{xz}{\longmapsto} P\{y/a\}\{z/b\} | Q
    \\
      \inferrule* [left=\bf{Out}]{
      }{
	\overline{x}z.R	
	  \stackrel{\overline{x}z}{\longmapsto}
	    R
      }
    }{
	(\underline{x(a)}.x(b).P|\overline{x}y.Q)|\overline{x}z.R
	  \stackrel{\tau}{\longmapsto}
	    (P\{y/a\}\{z/b\}|Q)|R
    }
  $\end{center}

  \begin{center}
  $
      \inferrule* [left=\bf{Low}]{
	(\underline{x(a)}.x(b).P|\overline{x}y.Q) | \overline{x}z.R
	  \stackrel{\epsilon}{\longmapsto}
	    (*(x(b).P\{y/a\})|Q)|\overline{x}z.R
	      \xrightarrow{\tau}
		(P\{y/a\}\{z/b\}|Q)|R
      }{
	(\underline{x(a)}.x(b).P|\overline{x}y.Q)|\overline{x}z.R
	  \xrightarrow{\tau}_{s}
	    (P\{y/a\}\{z/b\}|Q)|R
      }
  $
  \end{center}

\end{example}










\begin{proposition}\label{equivalencehightolowinput}
  Let $\rightarrow$ be the relation defined in table \ref{multipisoloinputearlywith}. If $P\xrightarrow{\sigma} Q$ then there exist $L_{1}, \cdots, L_{k}$ and $\gamma_{1}, \cdots, \gamma_{k+1}$ with $k\geq 0$ such that 
  \begin{center}
    \begin{tabular}{lll}
      $P \stackrel{\gamma_{1}}{\longmapsto} L_{1}  \stackrel{\gamma_{2}}{\longmapsto} L_{2} \cdots L_{k-1} \stackrel{\gamma_{k}}{\longmapsto} L_{k} \stackrel{\gamma_{k+1}}{\longmapsto} Q$ 
    &
      and
    &
      $\gamma_{1} \cdot \ldots \cdot \gamma_{k+1} = \sigma$  
    \end{tabular}
  \end{center}
  \begin{proof}
    The proof is by induction on the depth of the derivation tree of $P\xrightarrow{\sigma} Q$:
    \begin{description}
      \item[base case]
    \end{description}
	If the depth is one then the rule used have to be one of: $EInp$, $Out$, $Tau$. These rules are also in table \ref{lowleveltransitionrelationinput} so we can derive $P \stackrel{\sigma}{\longmapsto}Q$.
    \begin{description}
      \item[inductive case]
    \end{description}
	If the depth is greater than one then the last rule used in the derivation can be:
	\begin{description}
	  \item[$SInpSeq$]: 
	    the last part of the derivation tree looks like this:
	    \begin{center}
	      $\inferrule* [left=\bf{SInpSeq}]{
		  P_{1} \xrightarrow{\sigma} Q
		\\
		  |\sigma|>1
	      }{
		\underline{x(y)}.P_{1} \xrightarrow{xy \cdot \sigma} Q
	      }$	      
	    \end{center}
	    for inductive hypothesis there exist $L_{1}, \cdots, L_{k}$ and $\gamma_{1}, \cdots, \gamma_{k+1}$ with $k\geq 0$ such that 
	    \begin{center}
	      \begin{tabular}{lll}
		$P_{1} \stackrel{\gamma_{1}}{\longmapsto} L_{1} \stackrel{\gamma_{2}}{\longmapsto} L_{2} \cdots L_{k-1} \stackrel{\gamma_{k}}{\longmapsto} L_{k} \stackrel{\gamma_{k+1}}{\longmapsto} Q$ 
	      &
		and
	      &
		$\gamma_{1} \cdot \ldots \cdot \gamma_{k+1} = \sigma$
	      \end{tabular}
	    \end{center}
	    then a proof of the conclusion follows from:
	    \begin{center}
	      \begin{tabular}{ll}
		$\inferrule* [left=\bf{SInpLow}]{
 		}{
 		  \underline{x(y)}.P_{1} \stackrel{xy}{\longmapsto} *P_{1}
 		}$
	      &
		$\inferrule* [left=\bf{Star}]{
 		  P_{1} \stackrel{\gamma_{1}}{\longmapsto} L_{1}
 		}{
 		  *P_{1} \stackrel{\gamma_{1}}{\longmapsto} L_{1}
 		}$
	      \end{tabular}
	    \end{center}
	    where $Star$ means $StarInp$ or $StarEps$, note that $\gamma_{1}$ is an input or an $epsilon$ because of \ref{lemmastrongsequence}.
	  \item[$SInp$]: this case is similar to the previous.
	  \item[$SInpTau$]: this case is similar to the previous observing that $xy \cdot \tau = xy$.
	  \item[$Sum$]: 
	the last part of the derivation tree looks like this:
	\begin{center}
	  $\inferrule* [left=\bf{Sum}]{
	    P_{1} \xrightarrow{\sigma} Q
	  }{
	    P_{1}+P_{2} \xrightarrow{\sigma} Q
	  }$
	\end{center}
	for the inductive hypothesis there exist $L_{1}$, $\cdots$, $L_{k}$ and $\gamma_{1}$, $\cdots$, $\gamma_{k+1}$ with $k\geq 0$ such that 
	\begin{center}
	  \begin{tabular}{lll}
	    $P_{1} \stackrel{\gamma_{1}}{\longmapsto} L_{1}  \stackrel{\gamma_{2}}{\longmapsto} L_{2} \cdots L_{k-1} \stackrel{\gamma_{k}}{\longmapsto} L_{k} \stackrel{\gamma_{k+1}}{\longmapsto} Q$ 
	  &
	    and
	  &
	    $\gamma_{1} \cdot \ldots \cdot \gamma_{k+1} = \sigma$  
	  \end{tabular}
	\end{center}
	A proof of the conclusion is:
	\begin{center}
	  $\inferrule* [left=\bf{Sum}]{
	      P_{1} \stackrel{\gamma_{1}}{\longmapsto} L_{1}
	    }{
	      P_{1}+P_{2} \stackrel{\gamma_{1}}{\longmapsto} L_{1}
	    }
	  $
	\end{center}
      \item[$Cong$]: this case is similar to the previous.
      \item[$ECom$]: 
	the last part of the derivation tree looks like this:
	\begin{center}
	  $\inferrule* [left=\bf{Com}]{
	      P_{1} \xrightarrow{xy} P_{1}^{'}
	    \\
	      Q_{1} \xrightarrow{\overline{x}y} Q_{1}^{'}
	  }{
	    P_{1}|Q_{1} \xrightarrow{\tau} P_{1}^{'}|Q_{1}^{'}
	  }$
	\end{center}
	for inductive hypothesis there exist $L_{1}, \cdots, L_{k}$ and $\gamma_{1}, \cdots, \gamma_{k+1}$ with $k\geq 0$ such that 
	\begin{center}
	  \begin{tabular}{lll}
	    $P_{1} \stackrel{\gamma_{1}}{\longmapsto} L_{1}  \stackrel{\gamma_{2}}{\longmapsto} L_{2} \cdots L_{k-1} \stackrel{\gamma_{k}}{\longmapsto} L_{k} \stackrel{\gamma_{k+1}}{\longmapsto} P_{1}^{'}$ 
	  &
	    and
	  &
	    $\gamma_{1} \cdot \ldots \cdot \gamma_{k+1} = xy$
	  \end{tabular}
	\end{center}
	and there exist $R_{1}, \cdots, R_{h}$ and $\delta_{1}, \cdots, \delta_{h+1}$ with $h\geq 0$ such that 
	\begin{center}
	  \begin{tabular}{lll}
	    $Q_{1} \stackrel{\delta_{1}}{\longmapsto} R_{1}  \stackrel{\delta_{2}}{\longmapsto} R_{2} \cdots R_{h-1} \stackrel{\delta_{h}}{\longmapsto} R_{h} \stackrel{\delta_{h+1}}{\longmapsto} Q_{1}^{'}$ 
	  &
	    and
	  &
	    $\delta_{1} \cdot \ldots \cdot \delta_{h+1} = \overline{x}y$
	  \end{tabular}
	\end{center}
	For lemma \ref{lemmacom3} there cannot be an output action in a transition involving marked processes so $h$ must be $0$ and $Q_{1} \stackrel{\delta_{1}}{\longmapsto} Q_{1}^{'}$ with $\delta_{1}=\overline{x}y$. We can have three different cases now: 
	\begin{description}
	  \item[$\gamma_{1}=xy$]:
	    A proof of the conclusion is:
	    \begin{center}
	      $P_{1}|Q_{1} \stackrel{\tau}{\longmapsto} L_{1}|Q_{1}^{'}
			      \stackrel{\epsilon}{\longmapsto} L_{2}|Q_{1}^{'}
		  \cdots
				\stackrel{\epsilon}{\longmapsto} L_{k}|Q_{1}^{'}
				\stackrel{\epsilon}{\longmapsto} P_{1}^{'}|Q_{1}^{'}$	  
	    \end{center}
	    we derive the first transition with rule $Com3L$, whether for the other transition we use the rule $Par1L$.
	  \item[$\gamma_{i}=xy$]:
	    A proof of the conclusion is:
	    \begin{center}
	      $
		  P_{1}|Q_{1} \stackrel{\epsilon}{\longmapsto} L_{1}|Q_{1} 
		  \cdots
			      \stackrel{\epsilon}{\longmapsto} L_{i-1}|Q_{1} 
			      \stackrel{\tau}{\longmapsto} L_{i}|Q_{1}^{'}
			      \stackrel{\epsilon}{\longmapsto} L_{i+1}|Q_{1}^{'}
		  \cdots 
			      \stackrel{\epsilon}{\longmapsto} L_{k}|Q_{1}^{'}
			      \stackrel{\epsilon}{\longmapsto} P_{1}^{'}|Q_{1}^{'}$	  
	    \end{center}
	    we derive the transaction $ L_{i-1}|Q_{1} \stackrel{\tau}{\longmapsto} L_{i}|Q_{1}^{'}$ with rule $Com2L$, whether for the other transactions  we use the rule $Par1L$.
	  \item[$\gamma_{k+1}=xy$] similar.
	\end{description}
      \item[$Res$]: 
	the last part of the derivation tree looks like this:
	\begin{center}
	  $\inferrule* [left=\bf{Res}]{
	      P_{1} \xrightarrow{\sigma} Q_{1}
	    \\
	      z\notin n(\sigma)
	  }{
	    (\nu z) P_{1} \xrightarrow{\sigma} (\nu z) Q_{1}
	  }$
	\end{center}
	for the inductive hypothesis there exist $L_{1}, \cdots, L_{k}$ and $\gamma_{1}, \cdots, \gamma_{k+1}$ with $k\geq 0$ such that 
	\begin{center}
	  \begin{tabular}{lll}
	    $P_{1} \stackrel{\gamma_{1}}{\longmapsto} L_{1}  \stackrel{\gamma_{2}}{\longmapsto} L_{2} \cdots L_{k-1} \stackrel{\gamma_{k}}{\longmapsto} L_{k} \stackrel{\gamma_{k+1}}{\longmapsto} Q_{1}$ 
	  &
	    and
	  &
	    $\gamma_{1} \cdot \ldots \cdot \gamma_{k+1} =  \sigma$
	  \end{tabular}
	\end{center}
	We can apply the rule $Res$ to each of the previous transitions because 
	\begin{center}
	  $z\notin n(\sigma)$ implies $z\notin n(\gamma_{i})$ for each $i$
	\end{center}
	and then get a proof of the conclusion:
	\begin{center}
	  $(\nu z)P_{1} \stackrel{\gamma_{1}}{\longmapsto} (\nu z)L_{1}  \stackrel{\gamma_{2}}{\longmapsto} (\nu z)L_{2} \cdots (\nu z)L_{k-1} \stackrel{\gamma_{k}}{\longmapsto} (\nu z)L_{k} \stackrel{\gamma_{k+1}}{\longmapsto} (\nu z)Q_{1}$
	\end{center}
      \item[$Par$]: this case is similar to the previous.
      \item[$EComSeq$]: 
	the last part of the derivation tree looks like this:
	\begin{center}
	  $\inferrule* [left=\bf{EComSeq}]{
	      P_{1} \xrightarrow{xy \cdot \sigma} P_{1}^{'}
	    \\
	      Q_{1} \xrightarrow{\overline{x}y} Q_{1}^{'}
	  }{
	    P_{1}|Q_{1} \xrightarrow{\sigma} P_{1}^{'}|Q_{1}^{'}
	  }$
	\end{center}
	for inductive hypothesis there exist $L_{1}$, $\cdots$, $L_{k}$ and $\gamma_{1}$, $\cdots$, $\gamma_{k+1}$ with $k\geq 0$ such that 
	\begin{center}
	  \begin{tabular}{lll}
	    $P_{1} \stackrel{\gamma_{1}}{\longmapsto} L_{1}  \stackrel{\gamma_{2}}{\longmapsto} L_{2} \cdots L_{k-1} \stackrel{\gamma_{k}}{\longmapsto} L_{k} \stackrel{\gamma_{k+1}}{\longmapsto} P_{1}^{'}$ 
	  &
	    and
	  &
	    $\gamma_{1} \cdot \ldots \cdot \gamma_{k+1} = xy \cdot \sigma$  
	  \end{tabular}
	\end{center}
	For inductive hypothesis and lemma \ref{multioutconstraintswithmarked} $Q_{1} \stackrel{\overline{x}y}{\longmapsto} Q_{1}^{'}$. We can have two different cases now depending on where the first $xy$ is:
	\begin{description}
	  \item[$\gamma_{1}=xy$]:
	    A proof of the conclusion is:
	    \begin{center}
	      $P_{1}|Q_{1} \stackrel{\tau}{\longmapsto} L_{1}|Q_{1}^{'}
			      \stackrel{\gamma_{2}}{\longmapsto} L_{2}|Q_{1}^{'}
		  \cdots
			      \stackrel{\gamma_{k}}{\longmapsto} L_{k}|Q_{1}^{'}
			      \stackrel{\gamma_{k+1}}{\longmapsto} P_{1}^{'}|Q_{1}^{'}$	  
	    \end{center}
	    we derive the first transition with rule $Com3L$, whether for the other transactions we use the rule $Par1L$. Since $\gamma_{1} \cdot \ldots \cdot \gamma_{k+1} = xy \cdot \sigma$ and $\gamma_{1}=xy$ then $\tau \cdot \gamma_{2}\cdot \ldots \cdot \gamma_{k+1}\cdot \epsilon \cdot \ldots \epsilon \cdot \tau=\sigma$
	  \item[$\gamma_{i}=xy$]:
	    A proof of the conclusion is:
	    \begin{center}
	      $P_{1}|Q_{1} \stackrel{\epsilon}{\longmapsto} L_{1}|Q_{1} 
		  \cdots
			      \stackrel{\epsilon}{\longmapsto} L_{i-1}|Q_{1} 
			      \stackrel{\tau}{\longmapsto} L_{i}|Q_{1}^{'}
			      \stackrel{\gamma_{i+1}}{\longmapsto} L_{i+1}|Q_{1}^{'}
		  \cdots 
			      \stackrel{\gamma_{k}}{\longmapsto} L_{k}|Q_{1}^{'}
			      \stackrel{\gamma_{k+1}}{\longmapsto} P_{1}^{'}|Q_{1}^{'}$	  
	    \end{center}
	    we derive the transition $ L_{i-1}|Q_{1} \stackrel{\tau}{\longmapsto} L_{i}|Q_{1}^{'}$ with rule $Com2L$, whether for the other transactions of the premises we use the rule $Par1L$.
	  \item[$\gamma_{k+1}=xy$]: cannot happen because $\sigma$ is not empty.
	\end{description}
    \end{description}
  \end{proof}
\end{proposition}






\begin{proposition}
  Let $\rightarrow$ be the relation defined in table \ref{multipisoloinputearlywith}. Let $\alpha$ be an action. If $P \stackrel{\alpha}{\longmapsto} Q$ then $P\xrightarrow{\alpha} Q$.
  \begin{proof}
    The proof is by induction the depth of the derivation of $P \stackrel{\alpha}{\longmapsto} Q$:
    \begin{description}
      \item[base case]
	in this case the derivation of this transition has depth one. The last(and only) rule used can be: $Out$, $EInp$ or $Tau$; these rules are also in table \ref{multipisoloinputearlywith} so we can derive $P\xrightarrow{\alpha} Q$. 
      \item[inductive case]
	in this case the last rule in the derivation can be: $Sum$, $Com1$, $Res$, $Par1L$, $Par1R$, $Cong$, $Opn$:
	\begin{description}
	  \item[$Com1$]:
	    \begin{center}
	      $\inferrule* [left=\bf{Com1}]{
		  P_{1} \stackrel{xy}{\longmapsto} Q_{1}
		\\
		  P_{2} \stackrel{\overline{x}y}{\longmapsto} Q_{2}
	      }{
		P_{1}|P_{2} \stackrel{\tau}{\longmapsto} Q_{1}|Q_{2}
	      }$ 
	    \end{center}
	    for inductive hypothesis $P_{1} \xrightarrow{xy} Q_{1}$ and $P_{2} \xrightarrow{\overline{x}y} Q_{2}$ so for rule $Com$ $P_{1}|P_{2} \xrightarrow{\tau} Q_{1}|Q_{2}$
	  \item[$Sum$]:
	    \begin{center}
	      $\inferrule* [left=\bf{Sum}]{
		P_{1} \stackrel{\alpha}{\longmapsto} Q
	      }{
		P_{1}+P_{2} \stackrel{\alpha}{\longmapsto} Q
	      }$ 
	    \end{center}
	    for inductive hypothesis $P_{1} \xrightarrow{\alpha} Q$ and for rule $Sum$ $P_{1}+P_{2} \xrightarrow{\alpha} Q$.
	  \item[$Res$] the first transition is:
	    \begin{center}
	      $\inferrule* [left=\bf{Res}]{
		  P_{1} \stackrel{\alpha}{\longmapsto} Q_{1}
		\\
		  z\notin n(\gamma_{1})
	      }{
		(\nu z) P_{1} \stackrel{\alpha}{\longmapsto} (\nu z)Q_{1}
	      }$ 
	    \end{center}		
	    for inductive hypothesis $P_{1} \xrightarrow{\alpha} Q_{1}$ and for rule $Res$ $(\nu z)P_{1} \xrightarrow{\alpha} (\nu z)Q_{1}$.
	 \item[$others$]: other cases are similar.
      \end{description}	    
    \end{description}	    
  \end{proof}
\end{proposition}



\subsection{Late operational semantic without structural congruence}

DA FARE

\begin{definition}
  The \emph{late transition relation without structural congruence} is the smallest relation induced by the rules in table \ref{multipisoloinputlateywith}.
  \begin{table}
    \begin{tabular}{ll}
	\hline\\
     	  $\inferrule* [left=\bf{Pref}]{
	    \alpha not a strong prefix
	  }{
	    \alpha.P \xrightarrow{\alpha} P
	  }$
	&
	  $\inferrule* [left=\bf{LComSeq}]{
	      P \xrightarrow{x(y)\cdot \sigma} P^{'}
	    \\
	      Q\xrightarrow{\overline{x}z} Q^{'}
	    \\
	      z\notin fn(\sigma)\cup fn(P)
	  }{
	    P|Q \xrightarrow{\sigma\{z/y\}} P^{'}\{z/y\}|Q^{'}
	  }$
      \\\\
	  $\inferrule* [left=\bf{SInp}]{
	      P \xrightarrow{a(b)} P^{'}
	  }{
	    \underline{x(y)}.P \xrightarrow{x(y) \cdot a(b)} P^{'}
	  }$
	&
	  $\inferrule* [left=\bf{LCom}]{
	      P \xrightarrow{x(y)} P^{'}
	    \\
	      Q\xrightarrow{\overline{x}z} Q^{'}
	    \\
	      z\notin fn(P)
	  }{
	    P|Q \xrightarrow{\tau} P^{'}\{z/y\}|Q^{'}
	  }$
      \\\\
	  $\inferrule* [left=\bf{Sum}]{
	    P \xrightarrow{\sigma} P^{'}
	  }{
	    P+Q \xrightarrow{\sigma} P^{'}
	  }$
	&
	  $\inferrule* [left=\bf{Cong}]{
	      P\equiv P^{'}
	    \\
	      P^{'} \xrightarrow{\sigma} Q
	  }{
	      P \xrightarrow{\sigma} Q
	  }$
      \\\\
	  $\inferrule* [left=\bf{Res}]{
	      P \xrightarrow{\sigma} P^{'}
	    \\
	      z\notin n(\alpha)
	  }{
	    (\nu z) P \xrightarrow{\sigma} (\nu z) P^{'}
	  }$
	&
	  $\inferrule* [left=\bf{Par}]{
	      P \xrightarrow{\sigma} P^{'}
	    \\
	      bn(\sigma)\cup fn(Q)=\emptyset
	  }{
	    P|Q \xrightarrow{\sigma} P^{'}|Q
	  }$
      \\\\
	  $\inferrule* [left=\bf{Opn}]{
	      P \xrightarrow{\overline{x}z} P^{'}
	    \\ 
	      z\neq x
	  }{
	      (\nu z)P \xrightarrow{\overline{x}(z)} P^{'}
	  }$
	&
	  $\inferrule* [left=\bf{SInpSeq}]{
	      P \xrightarrow{\sigma} P^{'}
	    \\
	      |\sigma|>1
	  }{
	    \underline{x(y)}.P \xrightarrow{x(y) \cdot \sigma} P^{'}
	  }$
      \\\\
	  $\inferrule* [left=\bf{SInpTau}]{
	      P \xrightarrow{\tau} P^{'}
	  }{
	    \underline{x(y)}.P \xrightarrow{x(y)} P^{'}
	  }$
	&
      \\\hline
    \end{tabular}
    \caption{Multi $\pi$ late semantic with structural congruence}
    \label{multipisoloinputlateywith}
  \end{table}
\end{definition}

\begin{example}Multi-party synchronization
  We show an example of a derivation of three processes that synchronize with the late semantic. The three processes are $\underline{x(a)}.x(b).P$, $\overline{x}y.Q$ and $\overline{x}z.R$. We assume that:
  \begin{center}
      $a\notin fn(x(b))\cup fn (\underline{x(a)}.x(b).P)$
  \end{center}
  and
  \begin{center}
      $b\notin fn(\underline{x(a)}.x(b).P|\overline{x}y.Q)$
  \end{center}

  \begin{center}
  $
      \inferrule* [left=\bf{LCom}]{
	\underline{x(a)}.x(b).P|\overline{x}y.Q
	  \xrightarrow{x(b)}
	    P\{y/a\}|Q
	\\
	  \inferrule* [left=\bf{Pref}]{
	  }{
	    \overline{x}z.R	
	      \xrightarrow{\overline{x}z} 
		R
	  }
      }{
	(\underline{x(a)}.x(b).P|\overline{x}y.Q)|\overline{x}z.R
	  \xrightarrow{\tau}
	    (P\{y/a\}|Q)\{z/b\}|R
      }
  $
  \end{center}
  
  \begin{center}
  $\inferrule* [left=\bf{LComSeq}]{
      \inferrule* [left=\bf{SInp}]{
	\inferrule* [left=\bf{Pref}]{
	}{
	  x(b).P \xrightarrow{x(b)} P
	}
      }{
	\underline{x(a)}.x(b).P
	  \xrightarrow{x(a) \cdot x(b)} 
	    P
      }
    \\
      \inferrule* [left=\bf{Out}]{
      }{
	\overline{x}y.Q \xrightarrow{\overline{x}y} Q
      }
  }{
	\underline{x(a)}.x(b).P|\overline{x}y.Q)
	  \xrightarrow{x(b)}
	    P\{y/a\}|Q
  }$
  \end{center}

\end{example}


\openrigthchapter{Multi $\pi$ calculus input e output}

\section{Syntax}

As we did whit multi $\pi$ calculus, we suppose that we have a countable set of names $\mathbb{N}$, ranged over by lower case letters $a,b, \cdots, z$. This names are used for communication channels and values. Furthermore we have a set of identifiers, ranged over by $A$. We represent the agents or processes by upper case letters $P,Q, \cdots $. A multi $\pi$ process, in addiction to the same actions of a $\pi$ process, can perform also a strong prefix:
\begin{center}
  $\pi$ ::= $\overline{x}y$ | $x(z)$ | $\underline{x(y)}$ | $\underline{\overline{x}y}$ |$\tau$ 
\end{center}
The process are defined, just as original $\pi$ calculus, by the following grammar:
\begin{center}
  \begin{tabular}{l}
    $P,Q$ ::= $0$ | $\pi.P$ | $P|Q$ | $P+Q$ | $(\nu x) P$ | $A(y_{1}, \cdots, y_{n})$
  \end{tabular}
\end{center}
and they have the same intuitive meaning as for the $\pi$ calculus. The strong prefix input allows a process to make an atomic sequence of actions, so that more than one process can synchronize on this sequence. 

We have to extend the following definition to deal with the strong prefix:
\begin{center}
  \begin{tabular}{ll}
	$B(\underline{x(y)}.Q, I)\; =\; \{y,\overline{y}\}\cup B(Q, I)$
      &
	$F(\underline{x(y)}.Q, I)\; =\; \{x,\overline{x}\}\cup (F(Q, I)-\{y,\overline{y}\})$
    \\
	$B(\underline{\overline{x}y}.Q, I)\; =\; B(Q,I)$
      &
	$F(\underline{\overline{x}y}.Q, I)\; =\; \{x,\overline{x},y,\overline{y}\}\cup F(Q, I)$
    \\
  \end{tabular}
\end{center}

\section{Operational semantic}
\subsection{Early operational semantic with structural congruence}

\subsection{Late operational semantic with structural congruence}

The semantic of a multi $\pi$ process is labeled transition system such that
\begin{itemize}
  \item 
    the nodes are multi $\pi$ calculus process. The set of node is $\mathbb{P}_{m}$
  \item
    The set of actions is $\mathbb{A}_{m}$ and can contain
    \begin{itemize}
      \item 
	bound output $\overline{x}(y)$
      \item
	unbound output $\overline{x}y$ 
      \item
	bound input $x(z)$
    \end{itemize}
    We use $\alpha, \alpha_{1}, \alpha_{2},\cdots $ to range over the set of actions, we use $\sigma, \sigma_{1}, \sigma_{2}, \cdots $ to range over the set $\mathbb{A}_{m}^{+} \cup \{\tau\}$. 
  \item
    the transition relations is $\rightarrow\subseteq \mathbb{P}_{m}\times (\mathbb{A}_{m}^{+} \cup \{\tau\})\times \mathbb{P}_{m}$
\end{itemize}

In this case, a label can be a sequence of prefixes, whether in the original $\pi$ calculus a label can be only a prefix. We use the symbol $\cdot$ to denote the concatenation operator.

\begin{definition}\index{transition relation! multipi! late! with structural congruence}
  The \emph{late transition relation with structural congruence} is the smallest relation induced by the following rules:
  \begin{center}
    \begin{tabular}{ll}
 	  \bf{Pref}
 	  \begin{tabular}{c}
 	      $\alpha\; not\; a\; strong\; prefix$
 	    \\\hline
 	      $\alpha.P \;\xrightarrow{\alpha} P$
 	  \end{tabular}
	&
	  \bf{Par}
	  \begin{tabular}{c}
	      $P \;\xrightarrow{\sigma} P^{'}\;\; bn(\sigma)\cap fn(Q)=\emptyset$
	    \\\hline
	      $P|Q \;\xrightarrow{\sigma} P^{'}|Q$
	  \end{tabular}
      \\\\
	  \bf{SOut}
	  \begin{tabular}{c}
	      $P \;\xrightarrow{\sigma} P^{'}\;\; \sigma\neq \tau$
	    \\\hline
	      $\underline{\overline{x}y}.P \;\xrightarrow{\overline{x}y \cdot \sigma} P^{'}$
	  \end{tabular}
	&
	  $\inferrule* [left=\bf{LComSeq1}]{
	      P \;\xrightarrow{x(y)}\; P^{'}
	    \\
	      Q\;\xrightarrow{\overline{x}z\cdot \sigma} Q^{'}
	    \\
	      z\notin fn(P)
	  }{
	    P|Q \;\xrightarrow{\sigma} P^{'}\{z/y\}|Q^{'}
	  }$
      \\\\
	  \bf{Sum}
	  \begin{tabular}{c}
	      $P \;\xrightarrow{\sigma} P^{'}$
	    \\\hline
	      $P+Q \;\xrightarrow{\sigma} P^{'}$
	  \end{tabular}
	&
	$\inferrule* [left=\bf{Str}]{
	    P\equiv P^{'}
	  \\
	    P^{'}\; \;\xrightarrow{\alpha}\; Q^{'}
	  \\
	    Q\equiv Q^{'}
	}{
	    P\; \;\xrightarrow{\alpha}\; Q
	}$
      \\\\
	  \bf{Res}
	  \begin{tabular}{c}
	      $P \;\xrightarrow{\sigma} P^{'}\;\; z\notin n(\alpha)$
	    \\\hline
	      $(\nu z) P \;\xrightarrow{\sigma} (\nu z) P^{'}$
	  \end{tabular}
	&
	  $\inferrule* [left=\bf{LComSng}]{
	      P \;\xrightarrow{x(y)} P^{'}
	    \\
	      Q\;\xrightarrow{\overline{x}z} Q^{'}
	    \\
	      z\notin fn(P)
	  }{
	    P|Q \;\xrightarrow{\tau} P^{'}\{z/y\}|Q^{'}
	  }$
      \\\\
	  $\inferrule* [left=\bf{SInp}]{
	      P \;\xrightarrow{\sigma}\; P^{'}
	    \\
	      \sigma\neq \tau
	  }{
	    \underline{x(y)}.P \;\xrightarrow{x(y) \cdot \sigma} P^{'}
	  }$
	&
	  $\inferrule* [left=\bf{LComSeq2}]{
	      P \;\xrightarrow{\overline{x}z}\; P^{'}
	    \\
	      Q\;\xrightarrow{x(y)\cdot \sigma}\; Q^{'}
	    \\
	      z\notin fn(P)
	  }{
	    P|Q \;\xrightarrow{\sigma\{z/y\}}\; P^{'}|Q^{'}\{z/y\}
	  }$
      \\
    \end{tabular}
  \end{center}
\end{definition}


\subsection{Another attemp to late operational semantic with structural congruence}

\begin{definition}\index{transition relation! multipi! late! with structural congruence}
  The \emph{late transition relation with structural congruence} is the smallest relation induced by the following rules:
  \begin{center}
    \begin{tabular}{ll}
 	  \bf{Pref}
 	  \begin{tabular}{c}
 	      $\alpha\; not\; a\; strong\; prefix$
 	    \\\hline
 	      $\alpha.P \;\xrightarrow{\alpha} P$
 	  \end{tabular}
	&
	  \bf{Par}
	  \begin{tabular}{c}
	      $P \;\xrightarrow{\sigma} P^{'}\;\; bn(\sigma)\cap fn(Q)=\emptyset$
	    \\\hline
	      $P|Q \;\xrightarrow{\sigma} P^{'}|Q$
	  \end{tabular}
      \\\\
	  \bf{SOut}
	  \begin{tabular}{c}
	      $P \;\xrightarrow{\sigma} P^{'}\;\; \sigma\neq \tau$
	    \\\hline
	      $\underline{\overline{x}y}.P \;\xrightarrow{\overline{x}y \cdot \sigma} P^{'}$
	  \end{tabular}
	&
	  $\inferrule* [left=\bf{LCom}]{
	      P \;\xrightarrow{\sigma_{1}}\; P^{'}
	    \\
	      Q\;\xrightarrow{\sigma_{2}} Q^{'}
	    \\
	      Sync(\sigma_{1}, \sigma_{2}, \sigma_{3}, \delta_{1}, \delta_{2})
	  }{
	    P|Q \;\xrightarrow{\sigma_{3}} P^{'}\delta_{1}|Q^{'}\delta_{2}
	  }$
      \\\\
	  \bf{Sum}
	  \begin{tabular}{c}
	      $P \;\xrightarrow{\sigma} P^{'}$
	    \\\hline
	      $P+Q \;\xrightarrow{\sigma} P^{'}$
	  \end{tabular}
	&
	$\inferrule* [left=\bf{Str}]{
	    P\equiv P^{'}
	  \\
	    P^{'}\; \;\xrightarrow{\alpha}\; Q^{'}
	  \\
	    Q\equiv Q^{'}
	}{
	    P\; \;\xrightarrow{\alpha}\; Q
	}$
      \\\\
	  \bf{Res}
	  \begin{tabular}{c}
	      $P \;\xrightarrow{\sigma} P^{'}\;\; z\notin n(\alpha)$
	    \\\hline
	      $(\nu z) P \;\xrightarrow{\sigma} (\nu z) P^{'}$
	  \end{tabular}
	&
	  \bf{SInp}
	  \begin{tabular}{c}
	      $P \;\xrightarrow{\sigma} P^{'}\;\; \sigma\neq \tau$
	    \\\hline
	      $\underline{x(y)}.P \;\xrightarrow{x(y) \cdot \sigma} P^{'}$
	  \end{tabular}
      \\
    \end{tabular}
  \end{center}
\end{definition}

In what follows, the names $\delta, \delta_{1}, \delta_{2}$ represents substitutions, they can also be empty; the names $\sigma, \sigma_{1}, \sigma_{2}, \sigma_{3}$ are non empty sequences of actions. The relation $Sync$ is defined in the following way:
\begin{center}
  \begin{tabular}{ll}
	$\inferrule* [left=S1L]{
	}{
	  Sync(x(y),\overline{x}z,\tau,\{z/y\},\{\})
	}$
      &
	$\inferrule* [left=S1R]{
	}{
	  Sync(\overline{x}z, x(y), \tau, \{\}, \{z/y\})
	}$
    \\\\
	$\inferrule* [left=S2L]{
	}{
	  Sync(x(y),\overline{x}z\cdot \sigma,\sigma,\{z/y\},\{\})
	}$
      &
	$\inferrule* [left=S2R]{
	}{
	  Sync(\overline{x}z\cdot \sigma, x(y), \sigma, \{\}, \{z/y\})
	}$
    \\\\  
	$\inferrule* [left=S3L]{
	}{
	  Sync(x(y)\cdot \sigma, \overline{x}z, \sigma\{z/y\}, \{z/y\}, \{\})
	}$	
      &
	$\inferrule* [left=S3R]{
	}{
	  Sync(\overline{x}z,x(y)\cdot \sigma,\sigma\{z/y\},\{\},\{z/y\})
	}$	
    \\\\
	$\inferrule* [left=S4L]{
	  Sync(\sigma_{1}, \sigma_{2}\{z/y\}, \sigma_{3}, \delta_{1}, \delta_{2})
	}{
	  Sync(x(y)\cdot \sigma_{1}, \overline{x}z\cdot\sigma_{2}, \sigma_{3}, \{z/y\}\delta_{1}, \delta_{2})
	}$		
      &
	$\inferrule* [left=S4R]{
	  Sync(\sigma_{1}, \sigma_{2}\{z/y\}, \sigma_{3}, \delta_{1}, \delta_{2})
	}{
	  Sync(\overline{x}z\cdot\sigma_{1},x(y)\cdot \sigma_{2}, \sigma_{3}, \delta_{1}, \{z/y\}\delta_{2})
	}$		
    \\\\
	$\inferrule* [left=I1L]{
	  Sync(\sigma_{1}, \sigma_{2}, \tau, \delta_{1}, \delta_{2})
	}{
	  Sync(\alpha \cdot \sigma_{1}, \sigma_{2}, \alpha, \delta_{1}, \delta_{2})
	}$		
      &
	$\inferrule* [left=I1R]{
	  Sync(\sigma_{1}, \sigma_{2}, \tau, \delta_{1}, \delta_{2})
	}{
	  Sync(\sigma_{1}, \alpha \cdot \sigma_{2}, \alpha, \delta_{1}, \delta_{2})
	}$		
    \\\\
	$\inferrule* [left=I2L]{
	  Sync(\sigma_{1}, \sigma_{2}, \sigma_{3}, \delta_{1}, \delta_{2})
	}{
	  Sync(\alpha \cdot \sigma_{1}, \sigma_{2}, \alpha \cdot \sigma_{3}, \delta_{1}, \delta_{2})
	}$			
      &
	$\inferrule* [left=I2R]{
	  Sync(\sigma_{1}, \sigma_{2}, \sigma_{3}, \delta_{1}, \delta_{2})
	}{
	  Sync(\sigma_{1}, \alpha \cdot \sigma_{2}, \alpha \cdot \sigma_{3}, \delta_{1}, \delta_{2})
	}$			
    \\\\
	$\inferrule* [left=I3L]{
	}{
	  Sync(\alpha, \sigma, \alpha \cdot \sigma, \delta_{1}, \delta_{2})
	}$			
      &
	$\inferrule* [left=I3R]{
	}{
	  Sync(\sigma, \alpha, \alpha \cdot \sigma, \delta_{1}, \delta_{2})
	}$
    \\\\
	$\inferrule* [left=I3L]{
	}{
	  Sync(\epsilon, \sigma, \sigma, \delta_{1}, \delta_{2})
	}$			
      &
	$\inferrule* [left=I3R]{
	}{
	  Sync(\sigma, \epsilon, \sigma, \delta_{1}, \delta_{2})
	}$
    \\
  \end{tabular}
\end{center}



ALTERNATIVA $\sigma, \sigma_{1}, \sigma_{2}, \sigma_{3}$ sono sequenze di azioni anche eventualmente vuote
\begin{center}
  \begin{tabular}{ll}
	$\inferrule* [left=S1L]{
	}{
	  Sync(x(y),\overline{x}z,\tau,\{z/y\},\{\})
	}$
      &
	$\inferrule* [left=S1R]{
	}{
	  Sync(\overline{x}z, x(y), \tau, \{\}, \{z/y\})
	}$
    \\\\
	$\inferrule* [left=S2L]{
	  Int(\sigma_{1}\{z/y\}, \sigma_{2}, \sigma_{3}, \delta_{1}, \delta_{2})
	}{
	  Sync(x(y)\cdot\sigma_{1}, \overline{x}z\cdot \sigma_{2}, \sigma_{3}, \{z/y\}\delta_{1}, \delta_{2})
	}$
      &
	$\inferrule* [left=S2R]{
	  Int(\sigma_{1}, \sigma_{2}\{z/y\}, \sigma_{3}, \delta_{1}, \delta_{2})
	}{
	  Sync(\overline{x}z\cdot\sigma_{1}, x(y)\cdot \sigma_{2}, \sigma_{3}, \delta_{1}, \{z/y\}\delta_{2})
	}$
    \\\\  
	$\inferrule* [left=S3In]{
	  Sync(\sigma_{1}, \sigma_{2}, \tau, , )
	}{
	  Sync(x(y)\cdot \sigma_{1}, \sigma_{2}, x(y), , )
	}$	
      &
	$\inferrule* [left=S3Out]{
	  Sync(\sigma_{1}, \sigma_{2}, \tau, , )
	}{
	  Sync(\overline{x}y\cdot \sigma_{1}, \sigma_{2}, \overline{x}y, , )
	}$	
    \\\\
	$\inferrule* [left=S4In]{
	  Sync(\sigma_{1}, \sigma_{2}, \tau, , )
	}{
	  Sync(\sigma_{1}, x(y)\cdot\sigma_{2}, x(y), , )
	}$		
      &
	$\inferrule* [left=S4Out]{
	  Sync(\sigma_{1}, \sigma_{2}, \tau, , )
	}{
	  Sync(\sigma_{1}, \overline{x}z\cdot\sigma_{2}, \overline{x}z, , )
	}$		
    \\\\
	$\inferrule* [left=S5In]{
	    Sync(\sigma_{1}, \sigma_{2}, \sigma_{3}, , )
	  \\
	    \sigma_{3}\neq\tau
	}{
	  Sync(x(y)\cdot\sigma_{1}, \sigma_{2}, x(y)\sigma_{3}, , )
	}$		
      &
	$\inferrule* [left=S5Out]{
	    Sync(\sigma_{1}, \sigma_{2}, \sigma_{3}, , )
	  \\
	    \sigma_{3}\neq\tau
	}{
	  Sync(\overline{x}z\cdot\sigma_{1}, \sigma_{2}, \overline{x}z\sigma_{3}, , )
	}$		
    \\\\
	$\inferrule* [left=S6In]{
	    Sync(\sigma_{1}, \sigma_{2}, \sigma_{3}, , )
	  \\
	    \sigma_{3}\neq\tau
	}{
	  Sync(\sigma_{1}, x(y)\cdot\sigma_{2}, x(y)\cdot\sigma_{3}, , )
	}$		
      &
	$\inferrule* [left=S6Out]{
	    Sync(\sigma_{1}, \sigma_{2}, \sigma_{3}, , )
	  \\
	    \sigma_{3}\neq\tau
	}{
	  Sync(\sigma_{1}, \overline{x}z\cdot\sigma_{2}, \overline{x}z\cdot\sigma_{3}, , )
	}$		
    \\\\
	$\inferrule* [left=I1L]{
	}{
	  Int(x(y), \overline{x}y, \tau, \delta_{1}, \delta_{2})
	}$		
      &
	$\inferrule* [left=I1R]{
	}{
	  Int(\overline{x}y, x(y), \tau, \delta_{1}, \delta_{2})
	}$		
    \\\\
	$\inferrule* [left=I2In]{
	}{
	  Int(x(y), \epsilon, x(y), , )
	}$			
      &
	$\inferrule* [left=I2Out]{
	}{
	  Int(\overline{x}y, \epsilon, \overline{x}y, , )
	}$			
    \\\\
	$\inferrule* [left=I3In]{
	}{
	  Int(\epsilon, x(y), x(y), , )
	}$			
      &
	$\inferrule* [left=I3Out]{
	}{
	  Int(\epsilon, \overline{x}z, \overline{x}z, , )
	}$
    \\\\
	$\inferrule* [left=I4L]{
	  Int(\sigma_{1}, \sigma_{2}, \sigma_{3}, , )
	}{
	  Int(x(y)\cdot\sigma_{1}, \overline{x}z\cdot \sigma_{2}, \sigma_{3}, , )
	}$			
      &
	$\inferrule* [left=I4R]{
	  Int(\sigma_{1}, \sigma_{2}, \sigma_{3}, , )
	}{
	  Int(\overline{x}z\cdot\sigma_{1}, x(y)\cdot \sigma_{2}, \sigma_{3}, , )
	}$
    \\\\
	$\inferrule* [left=I5In]{
	  Int(\sigma_{1}, \sigma_{2}, \tau, , )
	}{
	  Int(x(y)\cdot\sigma_{1}, \sigma_{2}, x(y), , )
	}$			
      &
	$\inferrule* [left=I5Out]{
	  Int(\sigma_{1}, \sigma_{2}, \tau, , )
	}{
	  Int(\overline{x}z\cdot\sigma_{1}, \sigma_{2}, \overline{x}z, , )
	}$
    \\\\
	$\inferrule* [left=I6In]{
	    Int(\sigma_{1}, \sigma_{2}, \sigma_{3}, , )
	  \\
	    \sigma_{3}\neq\tau
	}{
	  Int(x(y)\cdot\sigma_{1}, \sigma_{2}, x(y)\sigma_{3}, , )
	}$			
      &
	$\inferrule* [left=I6Out]{
	    Int(\sigma_{1}, \sigma_{2}, \sigma_{3}, , )
	  \\
	    \sigma_{3}\neq\tau
	}{
	  Int(\overline{x}z\cdot\sigma_{1}, \sigma_{2}, \overline{x}z\sigma_{3}, , )
	}$
    \\\\
	$\inferrule* [left=I7Out]{
	  Int(\sigma_{1}, \sigma_{2}, \tau, , )
	}{
	  Int(\sigma_{1}, \overline{x}z\cdot \sigma_{2}, \overline{x}z, , )
	}$			
      &
	$\inferrule* [left=I7In]{
	  Int(\sigma_{1}, \sigma_{2}, \tau, , )
	}{
	  Int(\sigma_{1}, x(y)\cdot \sigma_{2}, x(y), , )
	}$
    \\\\
	$\inferrule* [left=I8In]{
	    Int(\sigma_{1}, \sigma_{2}, \sigma_{3}, , )
	  \\
	    \sigma_{3}\neq\tau
	}{
	  Int(\sigma_{1}, x(y)\cdot \sigma_{2}, x(y)\cdot\sigma_{3}, , )
	}$
      &
	$\inferrule* [left=I8Out]{
	    Int(\sigma_{1}, \sigma_{2}, \sigma_{3}, , )
	  \\
	    \sigma_{3}\neq\tau
	}{
	  Int(\sigma_{1}, \overline{x}z\cdot \sigma_{2}, \overline{x}z\cdot\sigma_{3}, , )
	}$			
    \\
  \end{tabular}
\end{center}



\begin{example}
  We want to prove that:
  \[
    \underline{\overline{a}x}.\overline{a}y.P|\underline{a(w)}.a(z).Q\; \xrightarrow{\tau} P|Q\{x/w\}\{y/z\}
  \]
  We start first noticing that
  \[
    \inferrule* [left=S4R]{
      \inferrule* [left=S1R]{
      }{
	Sync(\overline{a} y, a(z)\{x/w\}, \tau, \{\}, \{y/z\})
      }
    }{
      Sync(\overline{a}x \cdot \overline{a} y, a(w) \cdot a(z), \tau, \{\}, \{x/w\}\{y/z\})
    }
  \]
  and that 
  \begin{center}
    \begin{tabular}{ll}
	  $
	    \inferrule* [left=SOut]{
	      \inferrule* [left=Pref]{
	      }{
		\overline{a}y.P\; \xrightarrow{\overline{a} y}\; P
	      }
	    }{
	      \underline{\overline{a}x}.\overline{a}y.P\; \xrightarrow{\overline{a}x \cdot \overline{a} y}\; P
	    }
	  $  
	&
	  $
	    \inferrule* [left=SInp]{
	      \inferrule* [left=Pref]{
	      }{
		a(z).Q\; \xrightarrow{a(z)} Q
	      }
	    }{
	      \underline{a(w)}.a(z).Q\; \xrightarrow{a(w)\cdot a(z)} Q
	    }
	  $
    \end{tabular}
  \end{center}
  and in the end we just need to apply the rule $\bf{LCom}$
\end{example}


\begin{example}
\begin{equation*}
  \begin{fitch}
    \fa (\underline{\overline{a}f}.\overline{b}g.P
	|\underline{a(w)}.a(z).Q)
	|\underline{b(y)}.\overline{a}h.R\; 
	  \xrightarrow{\tau} 
	    (P|Q\{f/w\})\{h/z\}|R\{g/y\}
      & 
	LCom        
    \\
    \fa \fa \underline{\overline{a}f}.\overline{b}g.P
	    |\underline{a(w)}.a(z).Q
	      \xrightarrow{\overline{b}g\cdot a(z)} 
		P |Q\{f/w\}
      &  
	LCom
    \\
    \fa \fa \fa \underline{\overline{a}f}.\overline{b}g.P
		  \xrightarrow{\overline{a}f\cdot \overline{b}g} 
		    P
      &  
	SOut
    \\    
    \fa \fa \fa \fa \overline{b}g.P
		      \xrightarrow{\overline{b}g} 
			P
      &  
      Pref
    \\    

    \fa \fa \fa \underline{a(w)}.a(z).Q
		  \xrightarrow{a(w)\cdot a(z)} 
		    Q
      &  
	SInp
    \\
    \fa \fa \fa \fa a(z).Q
		      \xrightarrow{a(z)} 
			Q
      &  
	Pref
    \\

    \fa \fa \fa Sync(\overline{a}f\cdot \overline{b}g, a(w)\cdot a(z),\overline{b}g \cdot a(z), \{\}, \{f/w\})
      &  
	S4R
    \\
    \fa \fa \fa \fa Sync(\overline{b}g, a(z)\{f/w\},\overline{b}g \cdot a(z), \{\}, \{\})
      &  
	I3L
    \\
    \fa \fa \fa \fa \fa Sync(\epsilon, a(z), a(z), \{\}, \{\})
      &  
	I4R
    \\
    \fa \fa \underline{b(y)}.\overline{a}h.R\; 
	      \xrightarrow{b(y)\cdot \overline{a}h}
		R
      &  
	SInp
    \\
    \fa \fa \fa \overline{a}h.R\; 
		  \xrightarrow{\overline{a}h}
		    R
      &
	Pref
    \\
    \fa \fa Sync(\overline{b}g\cdot a(z), b(y)\cdot \overline{a}h, \tau, \{h/z\}, \{g/y\})
      &  
	S4R
    \\
    \fa \fa \fa Sync(a(z), \overline{a}h, \tau, \{h/z\}, \{g/y\})
      &  
	S1L
    \\
  \end{fitch}
\end{equation*}
\end{example}


\begin{example}
\begin{equation*}
  \begin{fitch}
    \fa \underline{x(a)}.\overline{a}z.P
	|\overline{x}b.Q\; 
	  \xrightarrow{\overline{b}z} 
	    P\{b/a\}|Q
      &
	LCom   
    \\
    \fa \fa \underline{x(a)}.\overline{a}z.P
	      \xrightarrow{x(a)\cdot \overline{a}z} 
		P
      &  
	SInp
    \\
    \fa \fa \fa \overline{a}z.P
	      \xrightarrow{\overline{a}z} 
		P
      &  
	Inp
    \\
    \fa \fa \overline{x}b.Q\; 
	      \xrightarrow{\overline{x}b} 
		Q
      &  
	Pref
    \\    
    \fa \fa Sync(x(a)\cdot \overline{a}z,\overline{x}b,\overline{b}z,\{b/a\},\{\})
      &  
	S3L
    \\
  \end{fitch}
\end{equation*}
\end{example}


\subsection{Step semantic}
%verificare se è una congruenza per il parallelo (come mi auguro sia).





% \openrigthchapter{Multi $\pi$ calculus versione 3}
% \input{multipi3}


%\openrigthchapter{Multi $\pi$ calculus 2}
%\input{multipi2}

\printindex

\begin{thebibliography}{99}

\bibitem{djkstra}
  E. W. Dijkstra,
  \emph{Hierarchical ordering of sequential processes},
  Acta Informatica 1(2):115-138, 
  1971
  .


\bibitem{gorrieriMCCS}
  Roberto Gorrieri,
  \emph{A Fully-Abstract Semantics for Atomicity},
  Dipartimento di scienze dell'informazione, 
  Universita' di Bologna
  .


\bibitem{parrow}
  Joachin Parrow, 
  \emph{An Introduction to the $\pi$ Calculus},
  Department Teleinformatics,
  Rotal Institute of Technology,
  Stockholm
  .


\bibitem{sangiorgiwalker}
  Davide Sangiorgi, David Walker,
  \emph{The $\pi$-calculus},
  Cambridge University Press
  .


\bibitem{mousavireniers}
  MohammedReza Mousavi, Michel A Reniers,
  \emph{Congruence for Structural Congruences},
  Department of Computer Science,
  Eindhoven University of Technology
  .


\end{thebibliography}


\end{document}
