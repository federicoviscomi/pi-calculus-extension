	\begin{description}
	  \item[base case] 
	    if the length of the derivation is one, then the last and only rule used in the derivation have to be one of $Out,\; EInp,\; Tau$. This rules are also contained in $R_{1}$
	  \item[inductive case]
	    the last rule used in the derivation can be
	    \begin{itemize}
	      \item
		a rule $R$ in $R_{1}\cap R_{2}$, in such a case we can apply the inductive hypothesis on the premises of the rule $R$ and then apply the rule itself 
	      \item
		the rule $Alp$, in such a case the last part of the derivation of $P\xrightarrow{\alpha}_{2}P^{'}$ is
		\[
		  \inferrule* [left=Alp]{
		      P\equiv Q
		    \\
		      Q\xrightarrow{\alpha}_{2}Q^{'}
		    \\
		      P^{'}\equiv Q^{'}
		  }{
		    P\xrightarrow{\alpha}_{2}P^{'}
		  }
		\]
		now we can apply the inductive hypothesis on the premise $Q\xrightarrow{\alpha}_{2}Q^{'}$ and get $Q\xrightarrow{\alpha}_{1}Q^{'}$. Now we cannot just apply the rule $Alp$ because $Alp\notin R_{1}$, so we show by induction on the length of the derivation of $Q\xrightarrow{\alpha}_{1}Q^{'}$ that we can avoid using the rule $Alp$.
		\begin{description}
		  \item[base case] 
		    if the length is one then the last and only rule used can be one of $Out, Tau, EInp$. This means that $P$ has a prefix at the top level. To prove the transition $P\xrightarrow{\alpha}_{1}P^{'}$ we apply the appropriate rule for prefix to $P$.
		  \item[inductive case]
		    if the length of the derivation of $Q\xrightarrow{\alpha}_{1}Q^{'}$ is greater than one, we need to look at the last rule used $R$ 



    \begin{itemize}
	\item
	  \bf{ParL}
	  \begin{tabular}{c}
	      $P \xrightarrow{\alpha} P^{'}\;\; bn(\alpha)\cap fn(Q)=\emptyset$
	    \\\hline
	      $P|Q \xrightarrow{\alpha} P^{'}|Q$
	  \end{tabular}
	\item
	  \bf{ParR}
	  \begin{tabular}{c}
	      $Q \xrightarrow{\alpha} Q^{'}\;\; bn(\alpha)\cap fn(Q)=\emptyset$
	    \\\hline
	      $P|Q \xrightarrow{\alpha} P|Q^{'}$
	  \end{tabular}
	\item
	  \bf{SumL}
	  \begin{tabular}{c}
	      $P \xrightarrow{\alpha} P^{'}$
	    \\\hline
	      $P+Q \xrightarrow{\alpha} P^{'}$
	  \end{tabular}
	\item
	  \bf{SumR}
	  \begin{tabular}{c}
	      $Q \xrightarrow{\alpha} Q^{'}$
	    \\\hline
	      $P+Q \xrightarrow{\alpha} Q^{'}$
	    \end{tabular}
	\item
	  \bf{Res}
	  \begin{tabular}{c}
	      $P \xrightarrow{\alpha} P^{'}\;\; z\notin n(\alpha)$
	    \\\hline
	      $(\nu z) P \xrightarrow{\alpha} (\nu z) P^{'}$
	  \end{tabular}
	\item
	  \bf{ResAlp}
	  \begin{tabular}{c}
	      $(\nu w)P\{w/z\} \xrightarrow{xz} P^{'}\;\; w\notin n(P)$
	    \\\hline
	      $(\nu z) P \xrightarrow{xz} P^{'}$
	  \end{tabular}
	\item
	  \bf{EComR}
	  \begin{tabular}{c}
	      $P \xrightarrow{\overline{x}y} P^{'}\;\; Q\xrightarrow{xy} Q^{'}$
	    \\\hline
	      $P|Q \xrightarrow{\tau} P^{'}|Q^{'}$
	  \end{tabular}
	\item
	  \bf{EComL}
	  \begin{tabular}{c}
	      $P \xrightarrow{xy} P^{'}\;\; Q\xrightarrow{\overline{x}y} Q^{'}$
	    \\\hline
	      $P|Q \xrightarrow{\tau} P^{'}|Q^{'}$
	  \end{tabular}
	\item
	    \bf{ClsL}
	    \begin{tabular}{c}
		$P \xrightarrow{\overline{x}(z)} P^{'}$  
		$Q \xrightarrow{xz} Q^{'}$ 
		$z\notin fn(Q)$
	      \\\hline
		$P|Q \xrightarrow{\tau} (\nu z)(P^{'}|Q^{'})$
	    \end{tabular}
	\item
	    \bf{ClsR}
	    \begin{tabular}{c}
		$P \xrightarrow{xz} P^{'}$  
		$Q \xrightarrow{\overline{x}(z)} Q^{'}$ 
		$z\notin fn(P)$
	      \\\hline
		$P|Q \xrightarrow{\tau} (\nu z)(P^{'}|Q^{'})$
	    \end{tabular}
	\item
	  \bf{Cns}
	  \begin{tabular}{c}
	      $A(\tilde{x}) \stackrel{def}{=} P\; P\{\tilde{y}/\tilde{x}\} \xrightarrow{\alpha} P^{'}$
	    \\\hline
	      $A(\tilde{y}) \xrightarrow{\alpha} P^{'}$
	  \end{tabular}
	\item
	  \bf{Opn}
	  \begin{tabular}{c}
	      $P \;\xrightarrow{\overline{x}z} P^{'}\;\; z\neq x$
	    \\\hline
	      $(\nu z) P \;\xrightarrow{\overline{x}(z)} P^{'}$
	  \end{tabular}
    \end{itemize}




		\end{description}
	    \end{itemize}
	\end{description}










\begin{theorem}
  $P\xrightarrow{\alpha}_{1}P^{'}\; \Leftrightarrow\; P\xrightarrow{\alpha}_{2}P^{'}$
  \begin{proof}
    TODO: dimostrare che tutte le derivazioni di una qualsiasi transizione sono finite(se e' vero)!
    The proof is by induction on the length of the derivation of a transaction, and then both the base case and the inductive case proceed by cases on the last rule used in the derivation.
    We call $R_{1}$ the set of rules for $\rightarrow_{1}$ and $R_{2}$ the set of rules for $\rightarrow_{2}$.
    \begin{description}
      \item[$\Leftarrow$]:
	The cases in which the last rule is in $R_{1}\cap R_{2}$ are easy: for the base cases a derivation of $P\xrightarrow{\alpha}_{2}P^{'}$ is also a derivation of $P\xrightarrow{\alpha}_{1}P^{'}$ because it uses only the prefix rules; for the inductive cases, let $R$ be the last rule used in the derivation of $P\xrightarrow{\alpha}_{2}P^{'}$, we apply the inductive hypothesis to the premises of $R$ and then apply $R$. Things are more complicated if $R$ is $Alp$. We show that we can move toward the top level of the derivation tree any occurrence of the rule $Alp$ until we can replace it by an instance of $ResAlp$, a prefix rule of $OpnAlp$. So suppose we have at the end of the derivation tree of $P\xrightarrow{\alpha}_{2}P^{'}$ the following:
	\[
	  \inferrule* [left=Alp]{
	      P\equiv_{\alpha} S
	    \\
	      S\xrightarrow{\alpha}_{2}S^{'}
	  }{
	    P\xrightarrow{\alpha}_{2}S^{'}
	  }
	\]
	now we proceed by cases on the last rule of the derivation of $S\xrightarrow{\alpha}_{2}S^{'}$:
	\begin{description}
	  \item[Out] 
	    The last rule used in the derivation of $S\xrightarrow{\alpha}_{2}S^{'}$ is $Out$ so $S$ is $\overline{x}y.S_{1}$ for some names $x,y$ and process $S_{1}$. Since $S$ is $\alpha$ equivalent to $P$ for the inversion lemma we have that $P$ is $\overline{x}y.P_{1}$ for some process $P_{1}$. So the situation is the following:
	    \[
	      \inferrule* [left=Alp]{
		  \overline{x}y.P_{1}\equiv_{\alpha}\overline{x}y.S_{1}
		\\
		  \inferrule* [left=Out]{
		  }{
		    \overline{x}y.S_{1}\xrightarrow{\overline{x}y}_{2}S_{1}
		  }
		\\
		  S_{1}\equiv_{\alpha} P_{1}
	      }{
		\overline{x}y.P_{1}\; \xrightarrow{\overline{x}y}_{2}\; P_{1}
	      }
	    \]
	    became
	    \[
	      \inferrule* [left=Out]{
	      }{
		\overline{x}y.P_{1}\; \xrightarrow{\overline{x}y}_{2}\; P_{1}
	      }
	    \]
	  \item[Tau]
	    \[
	      \inferrule* [left=Alp]{
		  \tau.P_{1}\equiv_{\alpha}\tau.S_{1}
		\\
		  \inferrule* [left=Tau]{
		  }{
		    \tau.S_{1}\xrightarrow{\tau}_{2}S_{1}
		  }
		\\
		  S_{1}\equiv_{\alpha} P_{1}
	      }{
		\tau.P_{1}\; \xrightarrow{\tau}_{2}\; P_{1}
	      }
	    \]
	    became
	    \[
	      \inferrule* [left=Tau]{
	      }{
		\tau.P_{1}\; \xrightarrow{\tau}_{2}\; P_{1}
	      }
	    \]
	  \item[EInp]
	    from the hypothesis $x(y).P_{1}\equiv_{\alpha}x(z).S_{1}$ and the inversion lemma we get $P_{1}\{z/y\}\equiv_{\alpha}S_{1}$, it follows that $S_{1}\{w/z\}\equiv_{\alpha} (P_{1}\{z/y\})\{w/z\}$. We also have
	    \[
		  \inferrule* [left=EInp]{
		  }{
		    x(z).S_{1}\xrightarrow{x w}_{2}S_{1}\{w/z\}
		  }	     
	    \]
	    \[
	      \inferrule* [left=Alp]{
		  x(y).P_{1}\equiv_{\alpha}x(z).S_{1}
		\\
		  \inferrule* [left=EInp]{
		  }{
		    x(z).S_{1}\xrightarrow{x w}_{2}S_{1}\{w/z\}
		  }
		\\
		  S_{1}\{w/z\}\equiv_{\alpha} (P_{1}\{z/y\})\{w/z\}
	      }{
		x(y).P_{1}\; \xrightarrow{xw}_{2}\; (P_{1}\{z/y\})\{w/z\}
	      }
	    \]
	    became
	    \[
	      \inferrule* [left=Alp]{
	      }{
		x(y).P_{1}\; \xrightarrow{xw}_{2}\; P_{1}\{w/y\}
	      }
	    \]
	  \item[SumL] 
	    in such case $S$ is $S_{1}+S_{2}$ for some $S_{1}$ and $S_{2}$. From $P\equiv_{\alpha} S_{1}+S_{2}$ we get that $P$ is $P_{1}+P_{2}$ for some $P_{1}$ and $P_{2}$ such that $S_{1}\equiv_{\alpha}P_{1}$ and $S_{2}\equiv_{\alpha}P_{2}$. We also now that $S^{'}$ is an $S_{1}^{'}$ such that $S_{1}\xrightarrow{\alpha}_{2}S_{1}^{'}$ and $S_{1}^{'}\equiv_{\alpha} S_{1}$. To sum up the last part of the derivation tree of $P\xrightarrow{\alpha}_{2}P^{'}$ is:
	    \[
	      \inferrule* [left=Alp]{
		  P_{1}+P_{2}\equiv_{\alpha}S_{1}+S_{2}
		\\
		  \inferrule* [left=SumL]{
		    S_{1}\xrightarrow{\alpha}_{2}S_{1}^{'}
		  }{
		    S_{1}+S_{2}\xrightarrow{\alpha}_{2}S_{1}^{'}
		  }
		\\
		  S_{1}^{'}\equiv_{\alpha} P_{1}^{'}
	      }{
		P_{1}+P_{2}\; \xrightarrow{\alpha}_{2}\; P_{1}^{'}
	      }
	    \]
	    we move upward the rule $Alp$ like this:
	    \[
	      \inferrule* [left=SumL]{
		\inferrule* [left=Alp]{
		    P_{1}\equiv_{\alpha}S_{1}
		  \\
		    S_{1}\xrightarrow{\alpha}_{2}S_{1}^{'}
		  \\
		    S_{1}^{'}\equiv_{\alpha} P_{1}^{'}
		}{
		  P_{1}\; \xrightarrow{\alpha}_{2}\; P_{1}^{'}
		}
	      }{
		P_{1}+P_{2}\; \xrightarrow{\alpha}_{2}\; P_{1}^{'}
	      }
	    \]
	    Now we can apply the inductive hypothesis to $P_{1}\; \xrightarrow{\alpha}_{2}\; P_{1}^{'}$ and get $P_{1}\; \xrightarrow{\alpha}_{1}\; P_{1}^{'}$. After this we apply the rule $SumL$ and get $P_{1}+P_{2}\; \xrightarrow{\alpha}_{1}\; P_{1}^{'}$. The case for $SumR$ is simmetric and so omitted.
	  \item[ParL]
	    \[
	      \inferrule* [left=Alp]{
		  P_{1}|P_{2}\equiv_{\alpha}S_{1}|S_{2}
		\\
		  \inferrule* [left=ParL]{
		      S_{1}\xrightarrow{\alpha}_{2}S_{1}^{'}
		    \\
		      bn(\alpha)\cap fn(S_{2})=\emptyset
		  }{
		    S_{1}|S_{2}\xrightarrow{\alpha}_{2}S_{1}^{'}|S_{2}
		  }
		\\
		  S_{1}^{'}\equiv_{\alpha} P_{1}^{'}
	      }{
		P_{1}|P_{2}\; \xrightarrow{\alpha}_{2}\; P_{1}^{'}|P_{2}
	      }
	    \]
	    this became
	    \[
	      \inferrule* [left=ParL]{
		  \inferrule* [left=Alp]{
		      P_{1}\equiv_{\alpha}S_{1}
		    \\
		      S_{1}\xrightarrow{\alpha}_{2}S_{1}^{'}
		    \\
		      S_{1}^{'}\equiv_{\alpha} P_{1}^{'}
		  }{
		    P_{1}\; \xrightarrow{\alpha}_{2}\; P_{1}^{'}
		  }
		\\
		   bn(\alpha)\cap fn(P_{2})=\emptyset
	      }{
		P_{1}|P_{2}\; \xrightarrow{\alpha}_{2}\; P_{1}^{'}|P_{2}
	      }
	    \]
	  \item[Res]
	    \[
	      \inferrule* [left=Alp]{
		  (\nu z)R\equiv_{\alpha}(\nu y)S
		\\
		  \inferrule* [left=Res]{
		      S\xrightarrow{\alpha}_{2}S^{'}
		    \\
		      z\notin n(\alpha)
		  }{
		    (\nu y) S \xrightarrow{\alpha}_{2} (\nu y) S^{'}
		  }
		\\
		  (\nu z)R^{'}\equiv_{\alpha} (\nu y) S^{'}
	      }{
		(\nu z) R \xrightarrow{\alpha}_{2} (\nu z) R^{'}
	      }
	    \]
	    this became:
	    \[
	      \inferrule* [left=Res]{
		  \inferrule* [left=Alp]{
		      R\equiv_{\alpha}S
		    \\
		      S\xrightarrow{\alpha}_{2}S^{'}
		    \\
		      R^{'}\equiv_{\alpha} S^{'}
		  }{
		    R\xrightarrow{\alpha}_{2}R^{'}
		  }
		\\
		  z\notin n(\alpha)
	      }{
		(\nu z) R \xrightarrow{\alpha}_{2} (\nu z) R^{'}
	      }
	    \]
	  \item[Alp]
	    \[
	      \inferrule* [left=Alp]{
		  P\equiv_{\alpha}S
		\\
		  \inferrule* [left=Alp]{
		      S\equiv_{\alpha}T
		    \\
		      T\xrightarrow{\alpha}_{2}T^{'}
		    \\
		      S^{'}\equiv_{\alpha}T^{'}
		  }{
		    S \xrightarrow{\alpha}_{2} S^{'}
		  }
		\\
		  P^{'}\equiv_{\alpha} S^{'}
	      }{
		P \xrightarrow{\alpha}_{2} P^{'}
	      }
	    \]
	    because of the transitivity of $\alpha$ equivalence we can replace the previous part of the tree by the following:
	    \[
	      \inferrule* [left=Alp]{
		  P\equiv_{\alpha}T
		\\
		  T \xrightarrow{\alpha}_{2} T^{'}
		\\
		  P^{'}\equiv_{\alpha} T^{'}
	      }{
		P \xrightarrow{\alpha}_{2} P^{'}
	      }
	    \]
	    so we can assume the tree has no two adjacent application of the rule $Alp$
	  \item[EComL]
	    \[
	      \inferrule* [left=Alp]{
		  P_{1}|P_{2}\equiv_{\alpha}S_{1}|S_{2}
		\\
		  \inferrule* [left=EComL]{
		      S_{1}\xrightarrow{\overline{x}y}_{2}S_{1}^{'}
		    \\
		      S_{2}\xrightarrow{xy}_{2}S_{2}^{'}
		  }{
		    S_{1}|S_{2} \xrightarrow{\tau}_{2} S_{1}^{'}|S_{2}^{'}
		  }
		\\
		  P_{1}^{'}|P_{2}^{'}\equiv_{\alpha}S_{1}^{'}|S_{2}^{'}
	      }{
		P_{1}|P_{2} \xrightarrow{\tau}_{2} P_{1}^{'}|P_{2}^{'}
	      }
	    \]
	    this became

	    \[
	      \inferrule* [left=EComL]{
		  \inferrule* [left=Alp]{
		      P_{1}\equiv_{\alpha}S_{1}
		    \\
		      S_{1}\xrightarrow{\overline{x}y}_{2}S_{1}^{'}
		    \\
		      P_{1}^{'}\equiv_{\alpha}S_{1}^{'}
		  }{
		    P_{1}\xrightarrow{\overline{x}y}_{2}P_{1}^{'}
		  }
		\\
		  \inferrule* [left=Alp]{
		      P_{2}\equiv_{\alpha}S_{2}
		    \\
		      S_{2}\xrightarrow{xy}_{2}S_{2}^{'}
		    \\
		      P_{2}^{'}\equiv_{\alpha}S_{2}^{'}
		  }{
		    P_{2} \xrightarrow{xy}_{2} P_{2}^{'}
		  }
	      }{
		P_{1}|P_{2} \xrightarrow{\tau}_{2} P_{1}^{'}|P_{2}^{'}
	      }
	    \]
	    The case for $EComR$ is simmetric and so omitted.
	  \item[ClsL]
	    \[
	      \inferrule* [left=Alp]{
% 		  P_{1}|P_{2}\equiv_{\alpha}S_{1}|S_{2}
% 		\\
		  \inferrule* [left=ClsL]{
		      S_{1} \xrightarrow{\overline{x}(z)} S_{1}^{'}
		    \\
		      S_{2} \xrightarrow{xz} S_{2}^{'}
% 		    \\
% 		      z\notin fn(S_{2})
		  }{
		   P_{1}|P_{2}\equiv_{\alpha} S_{1}|S_{2} \xrightarrow{\tau}_{2} (\nu z)(S_{1}^{'}|S_{2}^{'})
		  }
		\\
		  (\nu z)(P_{1}^{'}|P_{2}^{'})\equiv_{\alpha}(\nu z)(S_{1}^{'}|S_{2}^{'})
	      }{
		P_{1}|P_{2} \xrightarrow{\tau}_{2} (\nu z)(P_{1}^{'}|P_{2}^{'})
	      }
	    \]
	    where $z\notin fn(S_{2})$, this became:
	    \[
	      \inferrule* [left=ClsL]{
		  \inferrule* [left=Alp]{
		      P_{1}\equiv_{\alpha}S_{1}
		    \\
		      S_{1} \xrightarrow{\overline{x}(z)} S_{1}^{'}
		    \\
		      P_{1}^{'}\equiv_{\alpha}S_{1}^{'}
		  }{
		    P_{1} \xrightarrow{\overline{x}(z)} P_{1}^{'}
		  }
		\\
		  \inferrule* [left=Alp]{
		      P_{2}\equiv_{\alpha}S_{2}
		    \\
		      S_{2} \xrightarrow{xz} S_{2}^{'}
		    \\
		      P_{2}^{'}\equiv_{\alpha}S_{2}^{'}
		  }{
		      P_{2} \xrightarrow{xz} P_{2}^{'}
		  }
	      }{
		P_{1}|P_{2} \xrightarrow{\tau}_{2} (\nu z)(P_{1}^{'}|P_{2}^{'})
	      }
	    \]
	    where $z\notin fn(P_{2})$ because $S_{2}\equiv_{\alpha}P_{2}$ and $z\notin fn(S_{2})$ and the $\alpha$ conversion does not change any free name.
	  \item[Cns]
	    \[
	      \inferrule* [left=Alp]{
		  P\equiv_{\alpha}A(\tilde{y})
		\\
		  \inferrule* [left=Cns]{
		      A(\tilde{x}) \stackrel{def}{=} S
		    \\
		      S\{\tilde{y}/\tilde{x}\} \xrightarrow{\alpha} S^{'}
		  }{
		    A(\tilde{y}) \xrightarrow{\alpha} S^{'}
		  }
		\\
		  S^{'}\equiv_{\alpha}P^{'}
	      }{
		P \xrightarrow{\alpha} P^{'}
	      }
	    \]
	    Given $P\equiv_{\alpha}A(\tilde{y})$ and $A(\tilde{x})=S$ we create another identifier $B(\tilde{y})=P$ such that $A(\tilde{y})\equiv_{\alpha}B(\tilde{y})$ and replace the previous derivation with:
	    \[
	      \inferrule* [left=Cns]{
		  B(\tilde{y})\stackrel{def}{=}P
		\\
		  \inferrule* [left=Alp]{
		      B(\tilde{y})\equiv_{\alpha}S\{\tilde{y}/\tilde{x}\}
		    \\
		      S\{\tilde{y}/\tilde{x}\} \xrightarrow{\alpha} S^{'}
		    \\
		      S^{'}\equiv_{\alpha}P^{'}
		  }{
		    B(\tilde{y})\xrightarrow{\alpha} P^{'}
		  }
	      }{
		P\xrightarrow{\alpha} P^{'}
	      }
	    \]
	  \item[Opn]
	    \[
	      \inferrule* [left=Alp]{
		  (\nu z) S \equiv_{\alpha} (\nu z) R 
		\\
		  \inferrule* [left=Opn]{
		      R \xrightarrow{\overline{x}z} R^{'}
		    \\
		      z\neq x
		  }{
		    (\nu z) R \xrightarrow{\overline{x}(z)} R^{'}
		  }
		\\
		  R^{'} \equiv_{\alpha} P^{'}
	      }{
		(\nu z) S \xrightarrow{\overline{x}(z)} P^{'}
	      }
	    \]
	    became
	    \[
	      \inferrule* [left=Opn]{
		  \inferrule* [left=Alp]{
		      S \equiv_{\alpha} R 
		    \\
		      R \xrightarrow{\overline{x}z} R^{'}	
		    \\
		      R^{'} \equiv_{\alpha} P^{'}
		  }{
		    S \xrightarrow{\overline{x}z} P^{'}
		  }
		\\
		  z\neq x
	      }{
		(\nu z) S \xrightarrow{\overline{x}(z)} P^{'}
	      }
	    \]
	    Now we wonder if we can have the following case:
	    \[
	      \inferrule* [left=Alp]{
		  (\nu y) S \equiv_{\alpha} (\nu z) R 
		\\
		  \inferrule* [left=Opn]{
		      R \xrightarrow{\overline{x}z} R^{'}
		    \\
		      z\neq x
		  }{
		    (\nu z) R \xrightarrow{\overline{x}(z)} R^{'}
		  }
		\\
		  R^{'} \equiv_{\alpha} P^{'}
	      }{
		(\nu y) S \xrightarrow{\overline{x}(z)} P^{'}
	      }
	    \]
	    The answer is yes, for example
	    \[
	      \inferrule* [left=Alp]{
		  (\nu y) \overline{x}y.0 \equiv_{\alpha} (\nu z) \overline{x}z.0
		\\
		  \inferrule* [left=Opn]{
		      \inferrule* [left=Out]{
		      }{
			\overline{x}z.0 \xrightarrow{\overline{x}z}_{2} 0
		      }
		    \\
		      z\neq x
		  }{
		    (\nu z) \overline{x}z.0 \xrightarrow{\overline{x}(z)}_{2} 0
		  }
	      }{
		(\nu y) \overline{x}y.0 \xrightarrow{\overline{x}(z)}_{2} 0
	      }
	    \]
	    There is no way to prove $(\nu y) \overline{x}y.0 \xrightarrow{\overline{x}(z)}_{1} 0$ because the head of this derivation has a restriction at the top level so the only rules applicable are:
	    \begin{description}
	      \item[Res]
		this rule does not work because the restriction in the body of the derivation is missing
	      \item[ResAlp]
		this rule dose not work because it requires an input action as label for the transition
	      \item[Opn]
		this rule implies $\overline{x}y.0 \xrightarrow{\overline{x}z}_{1} 0$ which cannot be proved
	    \end{description}
	    The solution is adding the following rule:
	    \[
	      \inferrule* [left=OpnAlp]{
		  y\notin n(R)
		\\
		  R\xrightarrow{\overline{x}z}_{1}R^{'}
		\\
		  z\notin bn(R)
	      }{
		(\nu y)R\{y/z\}\xrightarrow{\overline{x}(z)}_{1}R^{'}
	      }
	    \]
	    now the derivation became:
	    \[
	      \inferrule* [left=OpnAlp]{
		  y\notin n(S\{z/y\})
		\\
		  \inferrule* [left=Alp]{
		      S\{z/y\} \equiv_{\alpha} R
		    \\
		      R \xrightarrow{\overline{x}z} R^{'}
		    \\
		      P^{'} \equiv_{\alpha} R^{'}
		  }{
		    S\{z/y\} \xrightarrow{\overline{x}z} P^{'}
		  }
		\\
		  z\notin bn(S)
	      }{
		(\nu y) S \xrightarrow{\overline{x}(z)} P^{'}
	      }
	    \]
	\end{description}
      \item[$\Rightarrow$]:
	$R_{2}$ contains the same rules as $R_{1}$ but $Alp$ is in $R_{2}$ only, whether $ResAlp$ and $OpnAlp$ are in $R_{1}$ only. We can mimic the rule $ResAlp$ using the rule $Alp$, so if at some point in a derivation tree of $P\xrightarrow{\alpha}_{1}P^{'}$ we use the rule $ResAlp$, we can replace it with an appropriate instance of the following rule: 
	\[
	  \inferrule *{
	      (\nu z)P\equiv_{\alpha} (\nu w)P\{w/z\}
	    \\
	      w\notin n(P)
	    \\
	      (\nu w)P\{w/z\}\;
		\xrightarrow{xz}\;
		  P^{'}
	    \\
	      P^{'}\equiv_{\alpha} P^{'}
	  }{
	    (\nu z)P\; 
	      \xrightarrow{xz}\;
		P^{'}
	  }
	\]
	which is in turn a particular case of the rule $Alp$.
	We can mimic the rule $OpnAlp$ in the following way:
	    \[
	      \inferrule* [left=Alp]{
		  (\nu y)R\{y/z\}\equiv_{\alpha} (\nu z)R
		\\
		  \inferrule* [left=Opn]{
		      z\neq x
		    \\
		      R\xrightarrow{\overline{x}z}R^{'}
		  }{
		    (\nu z)R\xrightarrow{\overline{x}(z)}R^{'}
		  }
	      }{
		(\nu y)R\{y/z\}\xrightarrow{\overline{x}(z)}R^{'}
	      }
	    \]
    \end{description}
  \end{proof}
\end{theorem}
