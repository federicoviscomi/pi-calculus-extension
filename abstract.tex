\begin{abstract}

\section{Abstract} 


Il $\pi$ calcolo e' un formalismo che descrive e analizza le proprieta' del calcolo concorrente. Nasce come proseguio del lavoro gia' svolto sul CCS (Calculus of Communicating Systems). L'aspetto appetibile del $\pi$ calcolo rispetto ai formalismi precedenti e' l'essere in grado di descrivere la computazione concorrente in sistemi la cui configurazione puo' cambiare nel tempo. Nel CCS e nel $\pi$ calcolo manca la possibilta' di modellare sequenze atomiche di azioni e di modellare la sincronizzazione multiparte. Il Multi CCS \cite{gorrieriMCCS} estende il CCS con un'operatore di strong prefixing proprio per colmare tale vuoto. In questa tesi si cerca di trasportare per analogia le soluzioni introdotte dal Multi CCS verso il $\pi$ calcolo. Il risultato finale e' un linguaggio chiamato Multi $\pi$ calcolo. 

In particolare il Multi $\pi$ calcolo permette la sincronizzazione transazionale e la sincronizzazione multiparte. aggiungere una sintesi brevissima dei risultati ottenuti sul Multi $\pi$ calcolo.

\end{abstract}
