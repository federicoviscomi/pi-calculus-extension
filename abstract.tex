\begin{abstract} \section{Introduzione} 


%LA TESI PUO' ESSERE IN INGLESE MA IL SOMMARIO(ABSTRACT) DEVE ESSERE IN ITALIANO!


Qual e' lo stato dell'arte? 
Stato dell'arte: pi-calcolo, mancanza di multi-party synch mobile, per ccs e' stato proposto multi-ccs per multy-party synch statica.
\newline
Cosa ho fatto? multi-pi
\newline
Perche'?
\newline

i lavori di Nobuko Yoshida e Kohei Honda non sono riuscito a trovarli

% Il primo capitolo introduce il multi ccs: una estensione del ccs con un operatore di strong prefixing in grado di modelare sequenze atomiche di azioni e sincronizzazione multiparte. Segue il secondo capitolo che tratta il pi calcolo: . Infine nel terzo, il passaggio dal ccs al multi ccs viene trasportato per analogia al pi calcolo e otteniamo il multi pi calcolo.

\end{abstract}
