
%modificate le regole alpres1 e alpinp1: adesso la sostutizione avviene su P
%modificata la dimostrazione di P alfacong Q implica fn(P)=fn(Q), per adattarsi alle nuove regole
%sistemata la dimostrazione di transitivita' di alfacong


The $\pi$ calculus is a mathematical model of processes whose interconnections change as they interact. The basic computational step is the transfer of a communications link between two processes. The idea that the names of the links belong to the same category as the transferred objects is one of the cornerstone of the calculus. The $\pi$ calculus allows channel names to be communicated along the channels themselves, and in this way it is able to describe concurrent computations whose network configuration may change during the computation.

A coverage of $\pi$ calculus is on \cite{parrow}, \cite{sangiorgiwalker} and \cite{milner}

\section{Syntax}

We suppose that we have a countable set of names $\mathbb{N}$, ranged over by lower case letters $a,b, \cdots, z$. This names are used for communication channels and values. Furthermore we have a set of identifiers, ranged over by $A$. We represent the agents or processes by upper case letters $P,Q, \cdots $. A process can perform the following actions:
\begin{center}
  $\pi$ ::= $\overline{x}y$ | $x(z)$ | $\tau$ 
\end{center}
The process are defined by the following grammar:
\begin{center}
  \begin{tabular}{l}
    $P,Q$ ::= $0$ | $\pi.P$ | $P|Q$ | $P+Q$ | $(\nu x) P$ | $A$ 
  \end{tabular}
\end{center}
and they have the following intuitive meaning:
\begin{description}
  \item[$0$] 
    is the empty process which cannot perform any actions
  \item[$\pi.P$] 
    is an action prefixing, this process can perform action $\pi$ e then behave like $P$, the action can be:
    \begin{description}
      \item[$\overline{x}y$] 
	is an output action, this sends the name $y$ along the name $x$. We can think about $x$ as a channel or a port, and about $y$ as an output datum sent over the channel
      \item[$x(z)$] 
	is an input action, this receives a name along the name $x$. $z$ is a variable which stores the received data.
      \item[$\tau$] 
	is a silent or invisible action, this means that a process can evolve to $P$ without interaction with the environment 
    \end{description}
    for any action which is not a $\tau$, the first name that appears in the action is called subject of the action and the second name is called object of the action.
  \item[$P+Q$] 
    is the sum, this process can enact either $P$ or $Q$
  \item[$P|Q$] 
    is the parallel composition, $P$ and $Q$ can execute concurrently and also synchronize with each other
  \item[$(\nu z) P$] 
    is the scope restriction. This process behave as $P$ but the name $z$ is local. This process cannot use the name $z$ to interact with other processes.
  \item[$A$] 
    is an identifier. Every identifier has a definition
    \begin{center}
      $A(x_{1}, \cdots, x_{n})=P$
    \end{center}
    the $x_{i}$s must be pairwise different. The intuition is that we can substitute for some of the $x_{i}$s in $P$ to get a $\pi$ calculus process. 
\end{description}

To resolve ambiguity we can use parenthesis and observe the conventions that prefixing and restriction bind more tightly than composition and prefixing binds more tightly than sum. 

\begin{definition} \index{binder} \index{bind} \index{name occurrence! bound} \index{scope}
  We say that the input prefix $x(z).P$ \emph{binds} $z$ in $P$ or is a \emph{binder} for $z$ in $P$. We also say that $P$ is the \emph{scope} of the binder and that any occurrence of $z$ in $P$ are \emph{bound} by the binder. Also the restriction operator $(\nu z)P$ is a binder for $z$ in $P$. 
\end{definition}


\begin{definition} \index{$bn$}
  $bn(P)$ is the set of names that have a bound occurrence in $P$ and is defined as $B(P, \emptyset)$, where $B(P, I)$, with $I$ a set of identifiers, is defined in table \ref{table:B}
\end{definition}

  \begin{table}
    \begin{tabular}{ll}
      \hline\\
	$B(0, I)\; =\; \emptyset$&$B(Q+R,I)\; =\; B(Q,I)\cup B(R,I)$
      \\\\
	$B(\overline{x}y.Q, I)\; =\; B(Q, I)$&$B(Q|R,I)\; =\; B(Q,I)\cup B(R,I)$
      \\\\
	$B(x(y).Q, I)\; =\; \{y,\overline{y}\}\cup B(Q, I)$&$B((\nu x)Q, I)\; =\; \{x, \overline{x}\}\cup B(Q, I)$
      \\\\
	$B(\tau.Q, I)\; =\; B(Q, I)$&
      \\\\
	\multicolumn{2}{l}{
	$B(A(\tilde{x}), I)=\left\{
	  \begin{array}{ll}
		B(Q, I\cup \{A\})\; 
		where\; A(\tilde{x})\stackrel{def}{=}Q
	      &
		if\; A\notin I
	    \\
		\emptyset
	      &
		if\; A\in I
	  \end{array}\right.$}
      \\\hline
    \end{tabular}
    \caption{Bound occurrences}
    \label{table:B}
  \end{table}



\begin{definition} \index{name occurrence! free}
  We say that a name $x$ is \emph{free} in $P$ if $P$ contains a non bound occurrence of $x$. We write $fn(P)$ for the set of names with a free occurrence in $P$. $fn(P)$ is defined in table \ref{F}
\end{definition}

  \begin{table}
    \begin{tabular}{lll}
      \hline\\
	  $fn(\overline{x}y.Q)\; =\; \{x,\overline{x},y,\overline{y}\}\cup fn(Q)$
	&
	  $fn(Q+R)\; =\; fn(Q)\cup fn(R)$
	&
	  $fn(0)\; =\; \emptyset$
      \\\\
	  $fn(x(y).Q)\; =\; \{x,\overline{x}\}\cup (fn(Q)-\{y,\overline{y}\})$
	&
	  $fn(Q|R)\; =\; fn(Q)\cup fn(R)$
      \\\\
	  $fn((\nu x)Q)\; =\; fn(Q)-\{x,\overline{x}\}$	  
	&
	  $fn(\tau.Q)\; =\; fn(Q)$
	&
	  $fn(A(\tilde{x}))=\{\tilde{x}\}$
      \\\hline
    \end{tabular}
    \caption{Free occurrences}
    \label{F}
  \end{table}



\begin{definition} \index{n}
  $n(P)$ which is the set of all names in $P$ and is defined in the following way:
  \begin{center}
    $n(P)\; =\; fn(P)\cup bn(P)$
  \end{center}
\end{definition}


\begin{definition}
  We say that $\tau$ and actions which does not have any binder $xy, \overline{x}y$ are \emph{free} actions. Whether the other actions are \emph{bound} actions.
\end{definition}


In a definition
\begin{center}
   $A(x_{1}, \cdots, x_{n})=P$
\end{center}
the $x_{1}, \cdots, x_{n}$ are all the free names contained in $P$, specifically 
\[
  fn(P) \subseteq\{x_{1}, \cdots, x_{n}\}
\]
If we look at the definitions of $bn$ and of $fn$ we notice that if $P$ contains another identifier whose definition is:
\[
  B(z_{1},\cdots, z_{h})=Q
\] 
then we have 
\[
  fn(Q)\subseteq\{x_{1}, \cdots, x_{n}\}
\]




\begin{definition}\index{syntactic substitution}
  $P\{b/a\}$ is the syntactic substitution of name $b$ for a different name $a$ inside a $\pi$ calculus process, and it consists in replacing every free occurrences of $a$ with $b$. If $b$ is a bound name in $P$, in order to avoid name capture we perform an appropriate $\alpha$ conversion. $P\{b/a\}$ is defined in table \ref{syntacticsubstitution}. There is the following short notation
  \[
    \{\tilde{x}/\tilde{y}\}\mbox{ means } \{x_{1}/y_{1}, \cdots, x_{n}/y_{n}\}
  \]
\end{definition}

  \begin{table}
    \begin{tabular}{l}
      \hline\\
	$0\{b/a\}\; =\; 0$
      \\\\
	$(\overline{x}y.Q)\{b/a\}\; =\; \overline{x}\{b/a\}y\{b/a\}.Q\{b/a\}$
      \\\\
	$(x(y).Q)\{b/a\}\; =\; x\{b/a\}(y).Q\{b/a\}$ if $y\neq a$ and $y\neq b$
      \\\\
	$(x(a).Q)\{b/a\}\; =\; x\{b/a\}(a).Q$
      \\\\
	$(x(b).Q)\{b/a\}\; =\; x\{b/a\}(c).((Q\{c/b\})\{b/a\})$ where $c\notin n(Q)$
      \\\\
	$(\tau.Q)\{b/a\}\; =\; \tau.Q\{b/a\}$
      \\\\
	if $a\in \{x_{1}, \cdots, x_{n}\}$ then
      \\
	$(A(x_{1}, \cdots, x_{n}\; | \;y_{1}, \cdots, y_{m}))\{b/a\}=$
      \\
	$\;\;\;\;\;\left\{
	  \begin{array}{ll}
		A(x_{1}\{b/a\}, \cdots, x_{n}\{b/a\}\; | \;y_{1}, \cdots, y_{m})
	      &
		if\; b\notin \{y_{1}, \cdots, y_{m}\}
	    \\
		A(x_{1}\{b/a\}, \cdots, x_{n}\{b/a\}\; | \;y_{1}, \cdots, y_{i-1},c,y_{i+1},\cdots, y_{m})
	      &
		if\; b = y_{i}
	    \\
	        \;\;\;\;where\; c\; is\; fresh
	      &
	  \end{array}\right.$
      \\\\
	if $a\notin \{x_{1}, \cdots, x_{n}\}$ then
      \\
	$(A(x_{1}, \cdots, x_{n}\; | \;y_{1}, \cdots, y_{m}))\{b/a\}=A(x_{1}, \cdots, x_{n}\; | \;y_{1}, \cdots, y_{m})$
      \\\\
	$(Q+R)\{b/a\}\; =\; Q\{b/a\} + R\{b/a\}$
      \\\\
	$(Q|R)\{b/a\}\; =\; Q\{b/a\} | R\{b/a\}$
      \\\\
	$((\nu y)Q)\{b/a\}\; =\;(\nu y)Q\{b/a\}$ if $y\neq a$ and $y\neq b$
      \\\\
	$((\nu a)Q)\{b/a\}\; =\;(\nu a)Q$
      \\\\
	$((\nu b)Q)\{b/a\}\; =\;(\nu c)((Q\{c/b\})\{b/a\})$ where $c\notin n(Q)$ if $a\in fn(Q)$ 
      \\\\
	$((\nu b)Q)\{b/a\}\; =\;(\nu b)Q$ if $a\notin fn(Q)$
      \\\hline
    \end{tabular}
    \caption{Syntatic substitution}
    \label{syntacticsubstitution}
  \end{table}




\section{Operational Semantic(without structural congruence)}
\subsection{Early operational semantic(without structural congruence)}
The semantic of a $\pi$ calculus process is a labeled transition system such that:
\begin{itemize}
  \item 
    the nodes are $\pi$ calculus process. The set of node is $\mathbb{P}$
  \item
    the actions can be:
    \begin{itemize}
      \item unbound input $xy$
      \item unbound output $\overline{x}y$
      \item the silent action $\tau$
      \item bound output $\overline{x}(y)$
    \end{itemize}
    The set of actions is $\mathbb{A}$, we use $\alpha$ to range over the set of actions.
  \item
    the transition relations is $\rightarrow\subseteq \mathbb{P}\times \mathbb{A}\times \mathbb{P}$
\end{itemize}
In the following section we present the early semantic without structural congruence and without $alpha$ conversion. 

\begin{definition}\index{transition relation! early! without structural congruence}
  The \emph{early transition relation} $\rightarrow\subseteq \mathbb{P}\times \mathbb{A} \times \mathbb{P}$ is the smallest relation induced by the rules in table \ref{transitionrelationearlywithoutstructuralcongruence}. Where with $\tilde{x}$ we mean a sequence of names $x_{1}, \cdots, x_{n}$.

  \begin{table}
    \begin{tabular}{ll}  
      \hline\\
	  \bf{Out}
	  \begin{tabular}{c}
	      $\;\;$
	    \\\hline
	      $\overline{x}y.P \xrightarrow{\overline{x}y} P$
	  \end{tabular}
	&
	  \bf{EInp}
	  \begin{tabular}{c}
	    \\\hline
	      $x(y).P \xrightarrow{xz} P\{z/y\}$
	  \end{tabular}
      \\\\
	  \bf{SumR}
	  \begin{tabular}{c}
	      $Q \xrightarrow{\alpha} Q^{'}$
	    \\\hline
	      $P+Q \xrightarrow{\alpha} Q^{'}$
	    \end{tabular}
	&
	  \bf{ParR}
	  \begin{tabular}{c}
	      $Q \xrightarrow{\alpha} Q^{'}\;\; bn(\alpha)\cap fn(Q)=\emptyset$
	    \\\hline
	      $P|Q \xrightarrow{\alpha} P|Q^{'}$
	  \end{tabular}
      \\\\
	  \bf{SumL}
	  \begin{tabular}{c}
	      $P \xrightarrow{\alpha} P^{'}$
	    \\\hline
	      $P+Q \xrightarrow{\alpha} P^{'}$
	  \end{tabular}
	&
	  \bf{ParL}
	  \begin{tabular}{c}
	      $P \xrightarrow{\alpha} P^{'}\;\; bn(\alpha)\cap fn(Q)=\emptyset$
	    \\\hline
	      $P|Q \xrightarrow{\alpha} P^{'}|Q$
	  \end{tabular}
      \\\\
	  \bf{Res}
	  \begin{tabular}{c}
	      $P \xrightarrow{\alpha} P^{'}\;\; z\notin n(\alpha)$
	    \\\hline
	      $(\nu z) P \xrightarrow{\alpha} (\nu z) P^{'}$
	  \end{tabular}
	&
	  \bf{ResAlp}
	  \begin{tabular}{c}
	      $(\nu w)P\{w/z\} \xrightarrow{xz} P^{'}\;\; w\notin n(P)$
	    \\\hline
	      $(\nu z) P \xrightarrow{xz} P^{'}$
	  \end{tabular}
      \\\\
	  \bf{EComR}
	  \begin{tabular}{c}
	      $P \xrightarrow{\overline{x}y} P^{'}\;\; Q\xrightarrow{xy} Q^{'}$
	    \\\hline
	      $P|Q \xrightarrow{\tau} P^{'}|Q^{'}$
	  \end{tabular}
	&
	    \bf{ClsL}
	    \begin{tabular}{c}
		$P \xrightarrow{\overline{x}(z)} P^{'}$  
		$Q \xrightarrow{xz} Q^{'}$ 
		$z\notin fn(Q)$
	      \\\hline
		$P|Q \xrightarrow{\tau} (\nu z)(P^{'}|Q^{'})$
	    \end{tabular}
      \\\\
	  \bf{EComL}
	  \begin{tabular}{c}
	      $P \xrightarrow{xy} P^{'}\;\; Q\xrightarrow{\overline{x}y} Q^{'}$
	    \\\hline
	      $P|Q \xrightarrow{\tau} P^{'}|Q^{'}$
	  \end{tabular}
	&
	    \bf{ClsR}
	    \begin{tabular}{c}
		$P \xrightarrow{xz} P^{'}$  
		$Q \xrightarrow{\overline{x}(z)} Q^{'}$ 
		$z\notin fn(P)$
	      \\\hline
		$P|Q \xrightarrow{\tau} (\nu z)(P^{'}|Q^{'})$
	    \end{tabular}
      \\\\
	  \bf{Tau}
	  \begin{tabular}{c}
	      $\;\;$
	    \\\hline
	      $\tau.P \xrightarrow{\tau} P$
	  \end{tabular}
	  &
	  \bf{Cns}
	  \begin{tabular}{c}
	      $A(\tilde{x}) \stackrel{def}{=} P\; P\{\tilde{w}/\tilde{x}\} \xrightarrow{\alpha} P^{'}$
	    \\\hline
	      $A(\tilde{x})\{\tilde{w}/\tilde{x}\} \xrightarrow{\alpha} P^{'}$
	  \end{tabular}
      \\\\
	  \bf{Opn}
	  \begin{tabular}{c}
	      $P \;\xrightarrow{\overline{x}z} P^{'}\;\; z\neq x$
	    \\\hline
	      $(\nu z) P \;\xrightarrow{\overline{x}(z)} P^{'}$
	  \end{tabular}
	&
	  \bf{OpnAlp}
	  \begin{tabular}{c}
		  $(\nu w)P\{w/z\} \xrightarrow{\overline{x}(w)} P^{'}
		\;
 		  w\notin n(P)
		\;
		  x \neq w \neq z.$
	    \\\hline
 		$(\nu z)P \xrightarrow{\overline{x}(w)} P^{'}$
	  \end{tabular}
      \\\hline
    \end{tabular}
    \caption{Early transition relation without structural congruence}
    \label{transitionrelationearlywithoutstructuralcongruence}
  \end{table}
\end{definition}

\begin{example}
  We show now an example of the so called scope extrusion, in particular we prove that
  \begin{center}
    $a(x).P\; |\; (\nu b)\overline{a}b.Q\; \xrightarrow{\tau}\; (\nu b) (P\{b/x\}\; |\; Q)$
  \end{center}
  where we suppose that $b\notin fn(P)$. In this example the scope of $(\nu b)$ moves from the right hand component to the left hand.
  \[
    \inferrule* [left=CloseR] {
	\inferrule* [left=Einp] {
	}{
	  a(x).P\; \xrightarrow{ab} P\{b/x\}
	}
      \\
	\inferrule* [left=Opn] {
	    \inferrule* [left=Out]{
	    }{
	      \overline{a}b.Q\; \xrightarrow{\overline{a}b} Q
	    }
	  \\
	    a\neq b
	}{
	  (\nu b)\overline{a}b.Q\; \xrightarrow{\overline{a}(b)} Q
	}
      \\
	b\notin fn((\nu b)\overline{a}b.Q)
    }{
      a(x).P\; |\; (\nu b)\overline{a}b.Q\; \xrightarrow{\tau}\; (\nu b) (P\{b/x\}\; |\; Q)
    }
  \]

\end{example}


\begin{example}
    We want to prove now that:
    \begin{center}
      $((\nu b) a(x).P)\; |\; \overline{a}b.Q\; 
	\xrightarrow{\tau}\; 
	((\nu c) (P\{c/b\}\{b/x\})) | Q$
    \end{center}
    where $b\notin bn(P)$
    \[
	    \inferrule* [left=ResAlp] {
		\inferrule* [left=Res] {
		    \inferrule* [left=EInp]{
		    }{
		      (a(x).P)\{c/b\}\;
			\xrightarrow{ab}\;
			  P\{c/b\}\{b/x\}
		    }
		  \\
		    c\notin n(a(b))
		}{
		  (\nu c)((a(x).P)\{c/b\})\;
		    \xrightarrow{ab}\;
		      (\nu c)(P\{c/b\}\{b/x\})
		}
	      \\
		b\notin n((a(x).P)\{c/b\})
	    }{
	      (\nu b) a(x).P\; 
		\xrightarrow{ab}\; 
		  (\nu c) P\{c/b\}\{b/x\}
	    }
    \]

      \[
  	\inferrule* [left=EComL] {
  	      (\nu b) a(x).P\; 
		\xrightarrow{ab}\; 
		  (\nu c) P\{c/b\}\{b/x\}
  	  \\
  	    \inferrule* [left=EOut] {
  	    }{
  	      \overline{a}b.Q\; 
		\xrightarrow{\overline{a}b}\; 
		  Q
  	    }
  	}{
	  ((\nu b) a(x).P)\; |\; \overline{a}b.Q\; 
	    \xrightarrow{\tau}\; 
	      ((\nu c) (P\{c/b\}\{b/x\})) | Q
  	}
      \]
\end{example}

\begin{example}
  We have to spend some time to deal with the change of bound names in an identifier. Suppose we have
  \[
    A(x)\stackrel{def}{=} \underbrace{x(y).x(a).0}_{P}
  \]
  From the definition of substitution it follows that
  \[
    A(x)\{y/x\}=A(y)
  \]
  The identifier $A(y)$ is expected to behave consistently with 
  \[
    P\{y/x\}=y(z).y(a).0
  \]
  so we have to prove
  \[
    A(y)\xrightarrow{yw}y(a).0
  \]
  We can prove this in the following way:
  \[
    \inferrule* [left=Cns]{
	A(x)\stackrel{def}{=} P
      \\
	\inferrule* [left=EInp]{
	}{
	  P\{y/x\}\xrightarrow{yw}y(a).0
	}
    }{
      A(y)\xrightarrow{yw}y(a).0
    }
  \]
\end{example}





% \begin{example}
%   Now we prove that
%   \[
%     \inferrule* [left=Com]{
% 	\overline{a}x.c(x).0|b(x).0\;
% 	  \xrightarrow{\overline{a}x}\;
% 	      c(x).0|b(x).0
%       \\
% 	a(x).0|\overline{b}x.\overline{c}x.0\;
% 	  \xrightarrow{ax}\;
% 	    0|\overline{b}x.\overline{c}x.0
%     }{
%        (\overline{a}x.c(x).0|b(x).0)|(a(x).0|\overline{b}x.\overline{c}x.0)\; 
%  	\xrightarrow{\tau}\; 
%  	  (c(x).0|b(x).0)|(0|\overline{b}x.\overline{c}x.0)
% %        (\overline{a}x.c(x).0|b(x).0)|(a(x).0|\overline{b}x.\overline{c}x.0)\; 
% %  	\xrightarrow{\tau}\; 
% %  	  (0|0)|(0|0)
%     }
%   \]
%   
% \end{example}






\subsection{Late operational semantic(without structural congruence)}


In this case the set of actions $\mathbb{A}$ contains
\begin{itemize}
      \item bound input $x(y)$
      \item unbound output $\overline{x}y$
      \item the silent action $\tau$
      \item bound output $\overline{x}(y)$
\end{itemize}


\begin{definition}\index{transition relation! pi! late! without structural congruence}
  The \emph{late transition relation without structural congruence} $\rightarrow\subseteq \mathbb{P}\times \mathbb{A} \times \mathbb{P}$ is the smallest relation induced by the rules in table \ref{transitionrelationpilatewithoutstructuralcongruence}. TUTTE LE SEMANTICHE LATE DEL PI CALCOLO SONO DA AGGIORNARE!!!! !!! !! !
  \begin{table}
    \begin{tabular}{ll}
      \hline\\
	  \bf{LInp}
	  \begin{tabular}{c}
 	    $z\notin fn(P)$
	    \\\hline
 	    $x(y).P \xrightarrow{x(z)} P\{z/y\}$
	  \end{tabular}
	&
	  \bf{Res}
	  \begin{tabular}{c}
	    $P \xrightarrow{\alpha} P^{'}$ $z\notin n(\alpha)$
	      \\\hline
	    $(\nu z) P \xrightarrow{\alpha} (\nu z) P^{'}$
	  \end{tabular}    
      \\\\
	  \bf{SumL}
	  \begin{tabular}{c}
	      $P \xrightarrow{\alpha} P^{'}$
	    \\\hline
	      $P+Q \xrightarrow{\alpha} P^{'}$
	  \end{tabular}
	&
	  \bf{SumR}
	  \begin{tabular}{c}
	      $Q \xrightarrow{\alpha} Q^{'}$
	    \\\hline
	      $P+Q \xrightarrow{\alpha} Q^{'}$
	  \end{tabular}
      \\\\
	  \bf{ParL}
	  \begin{tabular}{c}
	      $P \xrightarrow{\alpha} P^{'}\;\; bn(\alpha)\cap fn(Q)=\emptyset$
	    \\\hline
	      $P|Q \xrightarrow{\alpha} P^{'}|Q$
	  \end{tabular}
	&
	  \bf{ParR}
	  \begin{tabular}{c}
	      $Q \xrightarrow{\alpha} Q^{'}\;\; bn(\alpha)\cap fn(Q)=\emptyset$
	    \\\hline
	      $P|Q \xrightarrow{\alpha} P|Q^{'}$
	  \end{tabular}
      \\\\
	  \bf{ComL}
	  \begin{tabular}{c}
	      $P \xrightarrow{x(y)} P^{'}\;\; Q\xrightarrow{\overline{x}(z)} Q^{'}$
	    \\\hline
	      $P|Q \xrightarrow{\tau} P^{'}\{z/y\}|Q^{'}$
	  \end{tabular}	
	&
	  \bf{ComR}
	  \begin{tabular}{c}
	      $P \xrightarrow{\overline{x}(z)} P^{'}\;\; Q\xrightarrow{x(y)} Q^{'}$
	    \\\hline
	      $P|Q \xrightarrow{\tau} P^{'}|Q^{'}\{z/y\}$
	  \end{tabular}	
      \\\\
	  \bf{Opn}
	  \begin{tabular}{c}
	      $P \xrightarrow{\overline{x}z} P^{'}\;\; z\neq x$
	    \\\hline
	      $(\nu z) P \xrightarrow{\overline{x}(z)} P^{'}$
	  \end{tabular}
	&
	  \bf{Out}
	  \begin{tabular}{c}
	    \hline
	    $\overline{x}y.P \xrightarrow{\overline{x}y} P$
	  \end{tabular}
      \\\\
	  \bf{ClsL}
	  \begin{tabular}{c}
	      $P\; \xrightarrow{\overline{x}(z)}\; P^{'}$  $Q \xrightarrow{xz} Q^{'}$ $z\notin fn(Q)$
	    \\\hline
	      $P|Q\; \xrightarrow{\tau}\; (\nu z)(P^{'}|Q^{'})$
	  \end{tabular}
	&
	  \bf{ClsR}
	  \begin{tabular}{c}
	      $P \xrightarrow{xz} P^{'}$  $Q \xrightarrow{\overline{x}(z)} Q^{'}$ $z\notin fn(P)$
	    \\\hline
	      $P|Q \xrightarrow{\tau} (\nu z)(P^{'}|Q^{'})$
	  \end{tabular}
      \\\\
	  \bf{Tau}
	  \begin{tabular}{c}
	    \hline
	      $\tau.P \xrightarrow{\tau} P$
	  \end{tabular}
	&
	  \bf{Cns}
	  \begin{tabular}{c}
	    $A(\tilde{x}) \stackrel{def}{=} P\; P\{\tilde{y}/\tilde{x}\} \xrightarrow{\alpha} P^{'}$
	      \\\hline
	    $A(\tilde{y}) \xrightarrow{\alpha} P^{'}$
	  \end{tabular}
      \\\hline
    \end{tabular}
    \caption{Late semantic without structural congruence}
    \label{transitionrelationpilatewithoutstructuralcongruence}
  \end{table}
\end{definition}



\subsection{Distinction between late and early semantics}

There are some differences between late and early semantics:
\begin{description}
  \item[Communication]
    da scrivere
  \item[Input]
    da scrivere
  \item[Parallel composition]
    the side condition in the rule $Par$ for the late sematic is important because:
    $(x(z).P|Q)|\overline{x}y.R \xrightarrow{\tau} (P\{w/z\}|Q)\{y/w\}|R$ 
    
    da scrivere
\end{description}




\section{Structural congruence}

Structural congruences are a set of equations defining equality and congruence relations on process. They can be used in combination with an SOS semantic for languages. In some cases structural congruences help simplifying the SOS rules: for example they can capture inherent properties of composition operators(e.g. commutativity, associativity and zero element). Also, in process calculi, structural congruences let processes interact even in case they are not adjacent in the syntax. There is a possible trade off between what to include in the structural congruence and what to include in the transition rules: for example in the case of the commutativity of the sum operator. It is worth noticing that in most process calculi every structurally congruent processes should never be distinguished and thus any semantic must assign them the same behaviour.


\begin{definition}
  A \emph{change of bound names} in a process $P$ is the replacement of a subterm $x(z).Q$ of $P$ by $x(w).Q\{w/z\}$ or the replacement of a subterm $(\nu z)Q$ of $P$ by $(\nu w)Q\{w/z\}$ where in each case $w$ does not occur in $Q$.
\end{definition}


\begin{definition}\index{context}
  A \emph{context} $C[\cdot]$ is a process with a placeholder. If $C[\cdot]$ is a context and we replace the placeholder with $P$, than we obtain $C[P]$. In doing so, we make no $\alpha$ conversions.
\end{definition}


\begin{definition}\index{congruence}
  A \emph{congruence} is a binary relation on processes such that:
  \begin{itemize}
    \item 
      $S$ is an equivalence relation
    \item 
      $S$ is preserved by substitution in contexts: for each pair of processes $(P, Q)$ and for each context $C[\cdot]$
      \begin{center}
	$(P,Q)\in S\; \Rightarrow\; (C[P], C[Q])\in S$
      \end{center}
  \end{itemize}
\end{definition}

\begin{definition}\index{$\alpha$ convertible}
  Processes $P$ and $Q$ are \emph{$\alpha$ convertible} or \emph{$\alpha$ equivalent} if $Q$ can be obtained from $P$ by a finite number of changes of bound names. If $P$ and $Q$ are $\alpha$ equivalent then we write $P\equiv_{\alpha}Q$. Specifically the $\alpha$ equivalence is the smallest binary relation on processes that satisfies the laws in table \ref{alphaequivalence}
  \begin{table}
    \begin{tabular}{ll}
      \hline\\
	  $\inferrule*[left=AlpSum]{
	      P_{1}\equiv_{\alpha}Q_{1}
	    \\
	      P_{2}\equiv_{\alpha}Q_{2}
	  }{
	    P_{1}+P_{2}\equiv_{\alpha}Q_{1}+Q_{2}
	  }$
	&
	  $\inferrule*[left=AlpTau]{
	      P\equiv_{\alpha}Q
	  }{
	    \tau.P\equiv_{\alpha}\tau.Q
	  }$
      \\\\
	  $\inferrule*[left=AlpRes1]{
	      P\{y/x\}\equiv_{\alpha}Q
	    \\
	      x\neq y
	    \\
	      y\notin fn(P)
	  }{
	    (\nu x)P\equiv_{\alpha}(\nu y)Q
	  }$
	&
	  $\inferrule*[left=AlpRes]{
	      P\equiv_{\alpha}Q
	  }{
	    (\nu x)P\equiv_{\alpha}(\nu x)Q
	  }$
      \\\\
	  $\inferrule*[left=AlpInp1]{
	      P\{y/x\}\equiv_{\alpha}Q
	    \\
	      x\neq y
	    \\
	      y\notin fn(P)
	  }{
	    z(x).P\equiv_{\alpha}z(y).Q
	  }$
	&
	  $\inferrule*[left=AlpInp]{
	      P\equiv_{\alpha}Q
	  }{
	    x(y).P\equiv_{\alpha}x(y).Q
	  }$
    \\\\
	  $\inferrule*[left=AlpPar]{
	      P_{1}\equiv_{\alpha}Q_{1}
	    \\
	      P_{2}\equiv_{\alpha}Q_{2}
	  }{
	    P_{1}|P_{2}\equiv_{\alpha}Q_{1}|Q_{2}
	  }$
      &
	  $\inferrule*[left=AlpOut]{
	      P\equiv_{\alpha}Q
	  }{
	    \overline{x}y.P\equiv_{\alpha}\overline{x}y.Q
	  }$
    \\\\
	  $\inferrule*[left=AlpIde]{
	  }{
	    A(\tilde{x}|\tilde{y})\equiv_{\alpha}A(\tilde{x}|\tilde{y})
	  }$
	&
	  $\inferrule*[left=AlpZero]{
	  }{
	    0\equiv_{\alpha}0
	  }$
      \\\\\hline
    \end{tabular}
    \caption{$\alpha$ equivalence laws}
    \label{alphaequivalence}
  \end{table}
\end{definition}


It remains the problem of proving that $\alpha$ equivalence is well defined, i.e. if we change only some bound names in a process $P$ then we get a process $\alpha$ equivalent to $P$. 
% We assume that there is a well order relation $\leq$ between the names. The condition $a\leq b$ in rule $AlpSubst$ ensure that we can apply this rule only a finite amount of times thus any proof tree of $\alpha$ equivalence is finite. 


\begin{lemma}
  Inversion lemma for $\alpha$ equivalence
  \begin{itemize}
    \item 	
      If $P\equiv_{\alpha}0$ then $P$ is also the null process $0$
    \item
      If $P\equiv_{\alpha} \tau.Q_{1}$ then $P=\tau.P_{1}$ for some $P_{1}$ such that $P_{1}\equiv_{\alpha}Q_{1}$
    \item
      If $P\equiv_{\alpha} \overline{x}y.Q_{1}$ then $P=\overline{x}y.P_{1}$ for some $P_{1}$ such that $P_{1}\equiv_{\alpha}Q_{1}$
    \item
      If $P\equiv_{\alpha} z(y).Q_{1}$ then one and only one of the following cases holds:
      \begin{itemize}
	\item 
	  $P=z(x).P_{1}$ for some $P_{1}$ such that $P_{1}\{y/x\}\equiv_{\alpha}Q_{1}$
	\item
	  $P=z(y).P_{1}$ for some $P_{1}$ such that $P_{1}\equiv_{\alpha}Q_{1}$
      \end{itemize}
    \item
      If $P\equiv_{\alpha} Q_{1}+Q_{2}$ then $P=P_{1}+P_{2}$ for some $P_{1}$ and $P_{2}$ such that $P_{1}\equiv_{\alpha}Q_{1}$ and $P_{2}\equiv_{\alpha}Q_{2}$.
    \item 
      If $P\equiv_{\alpha} Q_{1}|Q_{2}$ then $P=P_{1}|P_{2}$ for some $P_{1}$ and $P_{2}$ such that $P_{1}\equiv_{\alpha}Q_{1}$ and $P_{2}\equiv_{\alpha}Q_{2}$.
    \item 
      If $P\equiv_{\alpha} (\nu y)Q_{1}$ then one and only one of the following cases holds:
      \begin{itemize}
	\item
	  $P=(\nu x)P_{1}$ such that $P_{1}\{y/x\}\equiv_{\alpha}Q_{1}$
	\item
	  $P=(\nu y).P_{1}$ for some $P_{1}$ such that $P_{1}\equiv_{\alpha}Q_{1}$
      \end{itemize}
    \item 
      If $P\equiv_{\alpha} A(\tilde{x})$ then $P$ is $Q$.
  \end{itemize}
    \begin{proof}
      This lemma works because given $Q$ we know which rules must be at the end of any proof tree of $P\equiv_{\alpha}Q$.
    \end{proof}
\end{lemma}

\begin{lemma}
  Let $P$ be a process and $y,w,z$ names such that $w=z$ or $w\notin fn(P)$ then $P\{w/z\}\{y/w\}\equiv_{\alpha}P$
non ho una dimostrazione ma lo da per scontato in \cite{milnerparrowwalker} paragrafo 1.3.1
\end{lemma}



\begin{definition}\index{structural congruence}
  We define a \emph{structural congruence $\equiv$} as the smallest congruence on processes that satisfies the axioms in table \ref{structuralcongrunce}
  \begin{table}
    \begin{tabular}{lll}
      \hline\\
	SC-ALP&$\begin{array}{c}P \equiv_{\alpha} Q\\\overline{P\equiv Q}\end{array}$&$\alpha$ conversion
      \\\\
	\multicolumn{3}{l}{abelian monoid laws for sum:}
      \\
	SC-SUM-ASC& $M_{1}+(M_{2}+M_{3})\equiv (M_{1}+M_{2})+M_{3}$ &associativity
      \\
	SC-SUM-COM& $M_{1}+M_{2}\equiv M_{2}+M_{1}$ &commutativity
      \\
	SC-SUM-INC& $M+0\equiv M$&zero element
      \\\\
	\multicolumn{3}{l}{abelian monoid laws for parallel:}
      \\
	SC-COM-ASC& $P_{1}|(P_{2}|P_{3})\equiv (P_{1}|P_{2})|P_{3}$ &associativity
      \\
	SC-COM-COM& $P_{1}|P_{2}\equiv P_{2}|P_{1}$ &commutativity
      \\
	SC-COM-INC& $P|0\equiv P$&zero element
      \\\\
	\multicolumn{3}{l}{scope extension laws:}
      \\
	SC-RES& $(\nu z) (\nu w) P \equiv (\nu w) (\nu z) P$ &
      \\
	SC-RES-INC& $(\nu z) 0 \equiv 0$ &
      \\
	SC-RES-COM& $(\nu z) (P_{1}|P_{2}) \equiv P_{1}|(\nu z) P_{2}$ if $z\notin fn(P_{1})$&
      \\
	SC-RES-SUM& $(\nu z) (P_{1}+P_{2}) \equiv P_{1}+(\nu z) P_{2}$ if $z\notin fn(P_{1})$&
      \\\\
	\multicolumn{3}{l}{unfolding law:}
      \\
	SC-IDE&$A(\tilde{w})\equiv P\{\tilde{w}/\tilde{x}\}$&if $A(\tilde{x})\stackrel{def}{=}P$
      \\\hline
    \end{tabular}
    \caption{Structural congruence axioms}
    \label{structuralcongrunce}
  \end{table}
\end{definition}

We can make some clarification on the axioms of the structural congruence:
\begin{description}
  \item[$unfolding$] 
    this just helps replace an identifier by its definition, with the appropriate parameter instantiation. The alternative is to use the rule $Cns$ in table \ref{transitionrelationearlywithoutstructuralcongruence}.
  \item[$\alpha\; conversion$]
    is the $\alpha$ conversion, i.e., the choice of bound names, it identifies agents like $x(y).\overline{z}y$ and $x(w).\overline{z}w$. In the semantic of $\pi$ calculus we can use the structural congruence with the rule SC-ALP or we can embed the $\alpha$ conversion in the SOS rules. 
    In the early case, the rule for input and the rules $ResAlp, OpnAlp, Cns$ take care of $\alpha$ conversion, whether in the late case the rule for communication and the rules is $ResAlp, OpnAlp, Cns$ are in charge for $\alpha$ conversion.
  \item[$abelian\; monoidal\; properties\; of\; some\; operators$]
    We can deal with associativity and commutativity properties of sum and parallel composition by using SOS rules or by axiom of the structural congruence. For example the commutativity of the sum can be expressed by the following two rules:
    \begin{center}
      \begin{tabular}{cc}
	\bf{SumL}
	  \begin{tabular}{c}
	    $P \xrightarrow{\alpha} P^{'}$\\
	    \hline
	    $P+Q \xrightarrow{\alpha} P^{'}$
	  \end{tabular}
	 &
	 \bf{SumR}
	   \begin{tabular}{c}
	    $Q \xrightarrow{\alpha} Q^{'}$\\
	    \hline
	    $P+Q \xrightarrow{\alpha} Q^{'}$
	   \end{tabular}
      \end{tabular}
    \end{center}
  or by the following rule and axiom:
    \begin{center}
      \begin{tabular}{cc}
	\bf{Sum}
	  \begin{tabular}{c}
	    $P \xrightarrow{\alpha} P^{'}$\\
	    \hline
	    $P+Q \xrightarrow{\alpha} P^{'}$
	  \end{tabular}
	 &
	 \bf{SC-SUM}
	   \begin{tabular}{c}
	    $P+Q \equiv Q+P$
	   \end{tabular}
      \end{tabular}
    \end{center}
    and the rule $Str$
  \item[$scope\; extension$]
    We can use the scope extension laws in table \ref{structuralcongrunce} or the rules $Opn$ and $Cls$ in table \ref{transitionrelationearlywithoutstructuralcongruence} to deal with the scope extension.
\end{description}

\begin{lemma}\label{freenamesandsubstitution}
  \[
    a\in fn(Q)\Rightarrow fn(Q\{b/a\})=(fn(Q)-\{a\}) \cup \{b\}
  \]
  \begin{proof}
%     The proof is a structural induction on $Q$:
%     \begin{description}
%       \item[$0$] this case does not exist.
%       \item[$\overline{x}y.Q_{1}$]:
% 	\begin{center}
% 	  \begin{tabular}{ll}
% 	    $fn(Q_{1}\{b/a\})=(fn(Q_{1})-\{a\})\cup \{b\}$ &inductive hypothesis\\
% 	    $\Rightarrow fn(Q_{1}\{b/a\})\cup \{x,y\}=((fn(Q_{1})-\{a\})\cup \{b\})\cup \{x,y\}$ & $x, y, a, b$ pairwise different\\
% 	    $\Rightarrow fn(\overline{x}y.(Q_{1}\{b/a\}))=((fn(Q_{1})-\{a\})\cup \{b\})\cup \{x,y\}$ &\\
% 	  \end{tabular}
% 	\end{center}
% 	$(\overline{x}y.Q_{1})\{b/a\}\; =\; \overline{x}\{b/a\}y\{b/a\}.Q_{1}\{b/a\}$
%       \item[$x(y).Q_{1}$]
% 	$(x(y).Q_{1})\{b/a\}\; =\; x\{b/a\}(y).Q_{1}\{b/a\}$ if $y\neq a$ and $y\neq b$
%       \item[$x(a).Q_{1}$]
% 	$(x(a).Q_{1})\{b/a\}\; =\; x\{b/a\}(a).Q_{1}$
%       \item[$x(b).Q_{1}$]
% 	$(x(b).Q_{1})\{b/a\}\; =\; x\{b/a\}(c).((Q_{1}\{c/b\})\{b/a\})$ where $c\notin n(Q_{1})$
%       \item[$\tau.Q_{1}$]
% 	$(\tau.Q_{1})\{b/a\}\; =\; \tau.Q_{1}\{b/a\}$
%       \item[$A(x_{1}, \cdots, x_{n}\; | \;y_{1}, \cdots, y_{m})$]
% 	if $a\in \{x_{1}, \cdots, x_{n}\}$ then
%     
% 	$(A(x_{1}, \cdots, x_{n}\; | \;y_{1}, \cdots, y_{m}))\{b/a\}=$
%     
% 	$\;\;\;\;\;\left\{
% 	  \begin{array}{ll}
% 		A(x_{1}\{b/a\}, \cdots, x_{n}\{b/a\}\; | \;y_{1}, \cdots, y_{m})
% 	      &
% 		if\; b\notin \{y_{1}, \cdots, y_{m}\}
% 	    \\
% 		A(x_{1}\{b/a\}, \cdots, x_{n}\{b/a\}\; | \;y_{1}, \cdots, y_{i-1},c,y_{i+1},\cdots, y_{m})
% 	      &
% 		if\; b = y_{i}
% 	    \\
% 	        \;\;\;\;where\; c\; is\; fresh
% 	      &
% 	  \end{array}\right.$
%       \item[$A(x_{1}, \cdots, x_{n}\; | \;y_{1}, \cdots, y_{m})$]
% 	if $a\notin \{x_{1}, \cdots, x_{n}\}$ then
%       
% 	$(A(x_{1}, \cdots, x_{n}\; | \;y_{1}, \cdots, y_{m}))\{b/a\}=A(x_{1}, \cdots, x_{n}\; | \;y_{1}, \cdots, y_{m})$
%       \item[$Q_{1}+Q_{2}$]:
% 	\begin{center}
% 	  \begin{tabular}{ll}
% 	    inductive hypothesis:&\\
% 	    $fn(Q_{1}\{b/a\})=(fn(Q_{1})-\{a\})\cup \{b\}$ and $fn(Q_{2}\{b/a\})=(fn(Q_{2})-\{a\})\cup \{b\}$ &\\
% 	    $\Rightarrow fn(Q_{1}\{b/a\})\cup fn(Q_{2}\{b/a\})=((fn(Q_{1})-\{a\})\cup \{b\}) \cup ((fn(Q_{2})-\{a\})\cup \{b\})$  &\\
% 	    $\Rightarrow fn(Q_{1}\{b/a\})\cup fn(Q_{2}\{b/a\})=((fn(Q_{1}) \cup fn(Q_{2}))-\{a\})\cup \{b\}$  &\\
% 	    $\Rightarrow fn(Q_{1}\{b/a\}+Q_{2}\{b/a\})=((fn(Q_{1}+Q_{2}))-\{a\})\cup \{b\}$  &\\
% 	    $\Rightarrow fn((Q_{1}+Q_{2})\{b/a\})=((fn(Q_{1}+Q_{2}))-\{a\})\cup \{b\}$  &\\
% 	  \end{tabular}
% 	\end{center}
%       \item[$Q_{1}|Q_{2}$] this case is very similar to the previous one.
%       \item[$(\nu y)Q_{1}$]
% 	$((\nu y)Q_{1})\{b/a\}\; =\;(\nu y)Q_{1}\{b/a\}$ if $y\neq a$ and $y\neq b$
%       \item[$(\nu a)Q_{1}$]
% 	$((\nu a)Q_{1})\{b/a\}\; =\;(\nu a)Q_{1}$
%       \item[$(\nu b)Q_{1}$]
% 	$((\nu b)Q_{1})\{b/a\}\; =\;(\nu c)((Q_{1}\{c/b\})\{b/a\})$ where $c\notin n(Q_{1})$ if $a\in fn(Q_{1})$ 
%       \item[$(\nu b)Q_{1}$]
% 	$((\nu b)Q_{1})\{b/a\}\; =\;(\nu b)Q_{1}$ if $a\notin fn(Q_{1})$
%     \end{description}
  \end{proof}
\end{lemma}



\begin{lemma}\label{alphaequivalentsamefreenames}
  $P\equiv_{\alpha}Q\Rightarrow fn(P)=fn(Q)$
  \begin{proof}
    The proof goes by induction on rules 
    \begin{description}
      \item[$AlpZero$]
	the lemma holds because $P$ and $Q$ are the same process.
      \item[$AlpTau$]:
	\begin{center}
	  \begin{tabular}{ll}
	    &rule premise\\
	    $P\equiv_{\alpha}Q$&inductive hypothesis\\
	    $\Rightarrow fn(P)=fn(Q)$&definition of $fn$\\
	    $\Rightarrow fn(\tau.P)=fn(\tau.Q)$&\\
	  \end{tabular}
	\end{center}
      \item[$AlpOut$]:
	\begin{center}
	  \begin{tabular}{ll}
	    &rule premise\\
	    $P\equiv_{\alpha}Q$&inductive hypothesis\\
	    $\Rightarrow fn(P)=fn(Q)$&\\
	    $\Rightarrow fn(P)\cup \{x,y\}=fn(Q)\cup \{x,y\}$&definition of $fn$\\
	    $\Rightarrow fn(\overline{x}y.P)=fn(\overline{x}y.Q)$&\\
	  \end{tabular}
	\end{center}
      \item[$AlpRes1$]:
	we consider two cases:
	\begin{description}
	  \item[$x\notin fn(P)$]:
	    \begin{center}
	      \begin{tabular}{ll}
		&rule premises\\
		$P\{y/x\}\equiv_{\alpha}Q$ and $y\notin fn(P)$&inductive hypothesis\\
		$\Rightarrow fn(P\{y/x\})=fn(Q)$&$x\notin fn(P)$ and def of substitution\\
		$\Rightarrow fn(P)=fn(Q)$&$y\notin fn(P)$\\
		$\Rightarrow fn(P)=fn(Q)$ and $y\notin fn(Q)$&\\
	      \end{tabular}
	    \end{center}
	    Since $x\notin fn(P)$ then $fn(P)=fn(P)-\{x\}$. Since $y\notin fn(Q)$ then $fn(Q)=fn(Q)-\{y\}$. From $fn(P)=fn(P)-\{x\}$, $fn(Q)=fn(Q)-\{y\}$, $fn(P)=fn(Q)$ and the definition of substitution it follows that $fn((\nu x)P)=fn((\nu y)Q)$
	  \item[$x\in fn(P)$]:
	    \begin{center}
	      \begin{tabular}{ll}
		&rule premise\\
		$P\{y/x\}\equiv_{\alpha}Q$&inductive hypothesis\\
		$\Rightarrow fn(P\{y/x\})=fn(Q)$&\\
		$\Rightarrow fn(P\{y/x\})-\{y\}=fn(Q)-\{y\}$&\\
		$\Rightarrow (fn(P)-\{x\}\cup \{y\})-\{y\}=fn(Q)-\{y\}$&\\
		$\Rightarrow fn(P)-\{x\}=fn(Q)-\{y\}$&definition of $fn$\\
		$\Rightarrow fn((\nu x)P)=fn((\nu y)Q)$&\\
	      \end{tabular}
	    \end{center}
	\end{description}
      \item[$AlpInp1$]:
	we consider two cases:
	\begin{description}
	  \item[$x\notin fn(P)$]:
	    \begin{center}
	      \begin{tabular}{ll}
		&rule premises\\
		$P\{y/x\}\equiv_{\alpha}Q$ and $y\notin fn(P)$&inductive hypothesis\\
		$\Rightarrow fn(P\{y/x\})=fn(Q)$&$x\notin fn(P)$ and def of substitution\\
		$\Rightarrow fn(P)=fn(Q)$&$y\notin fn(P)$\\
		$\Rightarrow fn(P)=fn(Q)$ and $y\notin fn(Q)$&\\
	      \end{tabular}
	    \end{center}
	    Since $x\notin fn(P)$ then $fn(P)=fn(P)-\{x\}$. Since $y\notin fn(Q)$ then $fn(Q)=fn(Q)-\{y\}$. From $fn(P)=fn(P)-\{x\}$, $fn(Q)=fn(Q)-\{y\}$ and $fn(P)=fn(Q)$ it follows that $fn(P)-\{x\}=fn(Q)-\{y\}$ and so $(fn(P)-\{x\})\cup \{z\}=(fn(Q)-\{y\})\cup \{z\}$ which gives $fn(z(x).P)=fn(z(y).Q)$.
	  \item[$x\in fn(P)$]:
	    \begin{center}
	      \begin{tabular}{ll}
		&rule premise\\
		$P\{y/x\}\equiv_{\alpha}Q$&inductive hypothesis\\
		$\Rightarrow fn(P\{y/x\})=fn(Q)$&\\
		$\Rightarrow fn(P)-\{y\}=fn(Q)-\{y\}$&lemma \ref{freenamesandsubstitution}\\
		$\Rightarrow (fn(P)-\{x\}\cup \{y\})-\{y\}=fn(Q)-\{y\}$&\\
		$\Rightarrow fn(P)-\{x\}=fn(Q)-\{y\}$&\\
		$\Rightarrow (fn(P)-\{x\})\cup \{z\}=(fn(Q)-\{y\})\cup \{z\}$&definition of $fn$\\
		$\Rightarrow fn(z(x).P)=fn(z(y).Q)$&\\
	      \end{tabular}
	    \end{center}
	\end{description}
      \item[$AlpSum$]:
	\begin{center}
	  \begin{tabular}{ll}
	    &rule premises\\
	    $P_{1}\equiv_{\alpha}Q_{1}$ and $P_{2}\equiv_{\alpha}Q_{2}$&inductive hypothesis\\
	    $\Rightarrow fn(P_{1})=fn(Q_{1})$ and $fn(P_{2})=fn(Q_{2})$&\\
	    $\Rightarrow fn(P_{1})\cup fn(P_{2})=fn(Q_{1})\cap fn(Q_{2})$&definition of $fn$\\
	    $\Rightarrow fn(P_{1}+P_{2})=fn(Q_{1}+Q_{2})$&\\
	  \end{tabular}
	\end{center}
      \item[$AlpPar$]:
	\begin{center}
	  \begin{tabular}{ll}
	    &rule premises\\
	    $P_{1}\equiv_{\alpha}Q_{1}$ and $P_{2}\equiv_{\alpha}Q_{2}$&inductive hypothesis\\
	    $\Rightarrow fn(P_{1})=fn(Q_{1})$ and $fn(P_{2})=fn(Q_{2})$&\\
	    $\Rightarrow fn(P_{1})\cup fn(P_{2})=fn(Q_{1})\cap fn(Q_{2})$&definition of $fn$\\
	    $\Rightarrow fn(P_{1}|P_{2})=fn(Q_{1}|Q_{2})$&\\
	  \end{tabular}
	\end{center}
      \item[$AlpRes$]:
	\begin{center}
	  \begin{tabular}{ll}
	    &rule premise\\
	    $P\equiv_{\alpha}Q$&inductive hypothesis\\
	    $\Rightarrow fn(P)=fn(Q)$&\\
	    $\Rightarrow fn(P)- \{x\}=fn(Q)- \{x\}$&definition of $fn$\\
	    $\Rightarrow fn((\nu x)P)=fn((\nu x)Q)$&\\
	  \end{tabular}
	\end{center}
      \item[$AlpInp$]:
	\begin{center}
	  \begin{tabular}{ll}
	    &rule premise\\
	    $P\equiv_{\alpha}Q\{x/y\}$&inductive hypothesis\\
	    $\Rightarrow fn(P)=fn(Q)$&\\
	    $\Rightarrow (fn(P)-\{y\})\cup \{x\}=(fn(Q)-\{y\})\cup \{x\}$&definition of $fn$\\
	    $\Rightarrow fn(x(y).P)=fn(x(y).Q)$&\\
	  \end{tabular}
	\end{center}
      \item[$AlpIde$]
	the lemma holds because $P$ and $Q$ are the same process.
    \end{description}
  \end{proof}
\end{lemma}

\begin{lemma}\label{alphaequivalencecommutativity}
  $x\notin fn(P)\Rightarrow P\{x/y\}\{b/a\}\equiv_{\alpha}P\{b/a\}\{x/y\}$ 
\end{lemma}



\begin{lemma}\label{alphaequivalencesubstitution}
  $\alpha$ equivalence is invariant with respect to substitution. In other words 
  \begin{center}
    \begin{tabular}{lll}
	$P\equiv_{\alpha}Q$
      &
	
      &
	
    \\
	$b\notin fn(P)$
      &
	$\Rightarrow$
      &
	$P\{b/a\}\equiv_{\alpha}Q\{b/a\}$
    \\
	$b\notin fn(Q)$
      &
	
      &
	
    \\
    \end{tabular}
  \end{center}
  \begin{proof}:
    If $a$ and $b$ are the same name then the substitution has no effect and the lemma holds. Otherwise:
    \begin{center}
      \begin{tabular}{ll}
	&lemma hypothesis\\
	$P \equiv_{\alpha} Q$&lemma \ref{alphaequivalentsamefreenames}\\
	$\Rightarrow fn(P) = fn(Q)$&\\
	$\Rightarrow a\notin fn(P) \wedge a\notin fn(Q)$ or $a\in fn(P) \wedge a\in fn(Q)$&\\
      \end{tabular}
    \end{center}
    In the former case $a$ is not a free name in $P$ and $Q$ so the substitutions have no effects and the lemma holds. In the latter case $a$ is a free names in both processes: the proof goes by induction on the length of the proof tree of $P\equiv_{\alpha}Q$ and then by cases on the last rule of the proof tree. Let $x,y,a$ and $b$ be pairwise different.
    \begin{description}
      \item[$base\; case$] 
	The length of the proof is one and the rule used can be only: $AlpZero$ or $AlpIde$: the lemma holds because $P$ and $Q$ are syntacticly the same process.
      \item[$inductive\; case$]:
	\begin{description}
	  \item[$AlpTau$]:
	    \begin{center}
	      \begin{tabular}{ll}
		&rule premise\\
		$P_{1}\equiv_{\alpha}Q_{1}$&inductive hypothesis\\
		$\Rightarrow P_{1}\{b/a\}\equiv_{\alpha}Q_{1}\{b/a\}$&rule $AlpTau$\\
		$\Rightarrow \tau.(P_{1}\{b/a\})\equiv_{\alpha}\tau.(Q_{1}\{b/a\}) $&definition of substitution\\
		$\Rightarrow (\tau.P_{1})\{b/a\}\equiv_{\alpha}(\tau.Q_{1})\{b/a\} $&\\
	      \end{tabular}
	    \end{center}
	  \item[$AlpSum$]:
	    \begin{center}
	      \begin{tabular}{ll}
		&rule premises\\
		$P_{1}\equiv Q_{1}$ and $P_{2}\equiv Q_{2}$&inductive hypothesis\\
		$\Rightarrow P_{1}\{b/a\}\equiv Q_{1}\{b/a\}$ and $P_{2}\{b/a\}\equiv Q_{2}\{b/a\}$&rule $AlpSum$\\
		$\Rightarrow P_{1}\{b/a\}+P_{2}\{b/a\}\equiv Q_{1}\{b/a\}+Q_{2}\{b/a\} $&definition of substitution\\
		$\Rightarrow (P_{1}+P_{2})\{b/a\}\equiv_{\alpha}(Q_{1}+Q_{2})\{b/a\} $&\\
	      \end{tabular}
	    \end{center}
	  \item[$AlpPar$]:
	    this case is very similar to the previous one.
	  \item[$AlpOut$]:
	    \begin{center}
	      \begin{tabular}{ll}
		&rule premise\\
		$P_{1}\equiv_{\alpha}Q_{1}$&inductive hypothesis\\
		$\Rightarrow P_{1}\{b/a\}\equiv_{\alpha}Q_{1}\{b/a\}$&rule $AlpOut$\\
		$\Rightarrow \overline{x}\{b/a\}y\{b/a\}.P_{1}\{b/a\}\equiv_{\alpha}\overline{x}\{b/a\}y\{b/a\}.Q_{1}\{b/a\} $&definition of substitution\\
		$\Rightarrow (\overline{x}y.P_{1})\{b/a\}\equiv_{\alpha}(\overline{x}y.Q_{1})\{b/a\} $&\\
	      \end{tabular}
	    \end{center}
	  \item[$AlpInp$]:
	    \begin{center}
	      \begin{tabular}{ll}
		  &
		    rule premise
		\\
		    $P_{1}\equiv_{\alpha}Q_{1}$
		  &
		    inductive hypothesis
		\\
		    $\Rightarrow P_{1}\{b/a\}\equiv_{\alpha}Q_{1}\{b/a\}$
		  &
		    rule $AlpInp$
		\\
		    $\Rightarrow x\{b/a\}(y).P_{1}\{b/a\}\equiv_{\alpha}x\{b/a\}(y).Q_{1}\{b/a\} $
		  &
		    definition of substitution
		\\
		    $\Rightarrow (x(y).P_{1})\{b/a\}\equiv_{\alpha}(x(y).Q_{1})\{b/a\} $
		  &
		\\
	      \end{tabular}
	    \end{center}	    	    
 	    \begin{center}
 	      \begin{tabular}{ll}
 		&rule premise\\
 		$P_{1}\equiv_{\alpha}Q_{1}$&rule $AlpIn$\\
 		$\Rightarrow b(a).P_{1}\equiv_{\alpha}b(a).Q_{1} $&definition of substitution\\
 		$\Rightarrow a\{b/a\}(a).P_{1}\equiv_{\alpha}a\{b/a\}(a).Q_{1} $&definition of substitution\\
 		$\Rightarrow (a(a).P_{1})\{b/a\}\equiv_{\alpha}(a(a).Q_{1})\{b/a\} $&\\
 	      \end{tabular}
 	    \end{center}	    	    
	    \begin{center}
	      \begin{tabular}{ll}
		&rule premise\\
		$P_{1}\equiv_{\alpha}Q_{1}$&inductive hypothesis\\
		$\Rightarrow P_{1}\{b/a\}\equiv_{\alpha}Q_{1}\{b/a\} $&rule $AlpIn$\\
		$\Rightarrow b\{b/a\}(x).(P_{1}\{b/a\})\equiv_{\alpha}b\{b/a\}(x).(Q_{1}\{b/a\}) $&definition of substitution\\
		$\Rightarrow (b(x).P_{1})\{b/a\}\equiv_{\alpha}(b(x).Q_{1})\{b/a\} $&\\
	      \end{tabular}
	    \end{center}	    	    
	  \item[$AlpInp1$]:
	    we have various cases:
	    \begin{itemize}
	      \item 
		the last part of the proof tree of $P\equiv_{\alpha}Q$ is
		\[\inferrule*[left=AlpInp1]{
		    P_{1}\equiv_{\alpha}Q_{1}\{x/y\}
		  \\
		    x\neq y	      
		  \\
		    x\notin fn(Q_{1})
		}{
		  \underbrace{z(x).P_{1}}_{P}
		    \equiv_{\alpha}
		      \underbrace{z(y).Q_{1}}_{Q}
		}\]
		\begin{center}
		  \begin{tabular}{ll}
		      &
		    rule premise
		  \\
		    $P_{1}\equiv_{\alpha}Q_{1}\{x/y\}$ and $x\notin fn(Q_{1})$ 
		      &
		    inductive hypothesis
		  \\
		    $\Rightarrow P_{1}\{b/a\}\equiv_{\alpha}Q_{1}\{x/y\}\{b/a\}$
		      &
		    transitivity and lemma \ref{alphaequivalencecommutativity}
		  \\
		    $\Rightarrow P_{1}\{b/a\}\equiv_{\alpha}Q_{1}\{b/a\}\{x/y\}$
		      &
		    rule $AlpInp1$
		  \\
		    $\Rightarrow z(x).(P_{1}\{b/a\})\equiv_{\alpha}z(y).(Q_{1}\{b/a\})$
		      &
		    definition of substitution
		  \\
		    $\Rightarrow (z(x).P_{1})\{b/a\}\equiv_{\alpha}(z(y).Q_{1})\{b/a\}$
		      &
		    
		  \\
		  \end{tabular}
		\end{center}	    	
	      \item 
		the last part of the proof tree of $P\equiv_{\alpha}Q$ is
		\[\inferrule*[left=AlpInp1]{
		    P_{1}\equiv_{\alpha}Q_{1}\{x/y\}
		  \\
		    x\neq y	      
		  \\
		    x\notin fn(Q_{1})
		}{
		  \underbrace{b(x).P_{1}}_{P}
		    \equiv_{\alpha}
		      \underbrace{b(y).Q_{1}}_{Q}
		}\]
		\begin{center}
		  \begin{tabular}{ll}
		      &
		    rule premise
		  \\
		    $P_{1}\equiv_{\alpha}Q_{1}\{x/y\}$ and $x\notin fn(Q_{1})$ 
		      &
		    inductive hypothesis
		  \\
		    $\Rightarrow P_{1}\{b/a\}\equiv_{\alpha}Q_{1}\{x/y\}\{b/a\}$
		      &
		    transitivity and lemma \ref{alphaequivalencecommutativity}
		  \\
		    $\Rightarrow P_{1}\{b/a\}\equiv_{\alpha}Q_{1}\{b/a\}\{x/y\}$
		      &
		    rule $AlpInp1$
		  \\
		    $\Rightarrow b(x).(P_{1}\{b/a\})\equiv_{\alpha}b(y).(Q_{1}\{b/a\})$
		      &
		    definition of substitution
		  \\
		    $\Rightarrow (b(x).P_{1})\{b/a\}\equiv_{\alpha}(b(y).Q_{1})\{b/a\}$
		      &
		    
		  \\
		  \end{tabular}
		\end{center}	    	
	      \item 
		the last part of the proof tree of $P\equiv_{\alpha}Q$ is
		\[\inferrule*[left=AlpInp1]{
		    P_{1}\equiv_{\alpha}Q_{1}\{x/y\}
		  \\
		    x\neq y	      
		  \\
		    x\notin fn(Q_{1})
		}{
		  \underbrace{a(x).P_{1}}_{P}
		    \equiv_{\alpha}
		      \underbrace{a(y).Q_{1}}_{Q}
		}\]
		\begin{center}
		  \begin{tabular}{ll}
		      &
		    rule premise
		  \\
		    $P_{1}\equiv_{\alpha}Q_{1}\{x/y\}$ and $x\notin fn(Q_{1})$ 
		      &
		    inductive hypothesis
		  \\
		    $\Rightarrow P_{1}\{b/a\}\equiv_{\alpha}Q_{1}\{x/y\}\{b/a\}$
		      &
		    transitivity and lemma \ref{alphaequivalencecommutativity}
		  \\
		    $\Rightarrow P_{1}\{b/a\}\equiv_{\alpha}Q_{1}\{b/a\}\{x/y\}$
		      &
		    rule $AlpInp1$
		  \\
		    $\Rightarrow a(x).(P_{1}\{b/a\})\equiv_{\alpha}a(y).(Q_{1}\{b/a\})$
		      &
		    definition of substitution
		  \\
		    $\Rightarrow (a(x).P_{1})\{b/a\}\equiv_{\alpha}(a(y).Q_{1})\{b/a\}$
		      &
		    
		  \\
		  \end{tabular}
		\end{center}	    	
	      \item 
		the last part of the proof tree of $P\equiv_{\alpha}Q$ is
		\[\inferrule*[left=AlpInp1]{
		    P_{1}\equiv_{\alpha}Q_{1}\{a/y\}
		  \\
		    a\neq y	      
		  \\
		    a\notin fn(Q_{1})
		}{
		  \underbrace{a(a).P_{1}}_{P}
		    \equiv_{\alpha}
		      \underbrace{a(y).Q_{1}}_{Q}
		}\]
		\begin{center}
		  \begin{tabular}{ll}
		      &
		    rule premise
		  \\
		    $P_{1}\equiv_{\alpha}Q_{1}\{a/y\}$ and $x\notin fn(Q_{1})$ 
		      &
		    inductive hypothesis
		  \\
		    $\Rightarrow P_{1}\{b/a\}\equiv_{\alpha}Q_{1}\{a/y\}\{b/a\}$
		      &
		    transitivity and lemma \ref{alphaequivalencecommutativity}
		  \\
		    $\Rightarrow P_{1}\{b/a\}\equiv_{\alpha}Q_{1}\{b/a\}\{a/y\}$
		      &
		    rule $AlpInp1$
		  \\
		    $\Rightarrow a(a).(P_{1}\{b/a\})\equiv_{\alpha}a(y).(Q_{1}\{b/a\})$
		      &
		    definition of substitution
		  \\
		    $\Rightarrow (a(a).P_{1})\{b/a\}\equiv_{\alpha}(a(y).Q_{1})\{b/a\}$
		      &
		    
		  \\
		  \end{tabular}
		\end{center}	    	
	      \item 
		the last part of the proof tree of $P\equiv_{\alpha}Q$ is
		\[\inferrule*[left=AlpInp1]{
		    P_{1}\equiv_{\alpha}Q_{1}\{x/a\}
		  \\
		    x\neq a
		  \\
		    x\notin fn(Q_{1})
		}{
		  \underbrace{a(x).P_{1}}_{P}
		    \equiv_{\alpha}
		      \underbrace{a(a).Q_{1}}_{Q}
		}\]
		\begin{center}
		  \begin{tabular}{ll}
		      &
		    rule premise
		  \\
		    $P_{1}\equiv_{\alpha}Q_{1}\{x/a\}$ and $x\notin fn(Q_{1})$ 
		      &
		    inductive hypothesis
		  \\
		    $\Rightarrow P_{1}\{b/a\}\equiv_{\alpha}Q_{1}\{x/a\}\{b/a\}$
		      &
		    transitivity and lemma \ref{alphaequivalencecommutativity}
		  \\
		    $\Rightarrow P_{1}\{b/a\}\equiv_{\alpha}Q_{1}\{b/a\}\{x/a\}$
		      &
		    rule $AlpInp1$
		  \\
		    $\Rightarrow a(x).(P_{1}\{b/a\})\equiv_{\alpha}a(a).(Q_{1}\{b/a\})$
		      &
		    definition of substitution
		  \\
		    $\Rightarrow (a(x).P_{1})\{b/a\}\equiv_{\alpha}(a(a).Q_{1})\{b/a\}$
		      &		    
		  \\
		  \end{tabular}
		\end{center}	    	
	      \item mancano x x y e x y x
	    \end{itemize}
	  \item[$AlpRes$]:
	    \begin{center}
	      \begin{tabular}{ll}
		&rule premise\\
		$P_{1}\equiv_{\alpha}Q_{1}$&inductive hypothesis\\
		$\Rightarrow P_{1}\{b/a\}\equiv_{\alpha}Q_{1}\{b/a\}$&rule $AlpRes$\\
		$\Rightarrow (\nu x)(P_{1}\{b/a\})\equiv_{\alpha}(\nu x)(Q_{1}\{b/a\}) $&definition of substitution\\
		$\Rightarrow ((\nu x)P_{1})\{b/a\}\equiv_{\alpha}((\nu x)Q_{1})\{b/a\} $&\\
	      \end{tabular}
	    \end{center}	    	    
	  \item[$AlpRes1$]:
	    \[\inferrule*[left=AlpRes1]{
		P_{1}\equiv_{\alpha}Q_{1}\{x/y\}
	      \\
		x\neq y
	      \\
		x\notin fn(Q_{1})
	    }{
	      \underbrace{(\nu x)P_{1}}_{P}\equiv_{\alpha}\underbrace{(\nu y)Q_{1}}_{Q}
	    }\]
	    \begin{center}
	      \begin{tabular}{ll}
		  &
		    rule premises
		\\
		    $P_{1}\equiv_{\alpha}Q_{1}\{x/y\}$ and $x\neq y$ and $x\notin fn(Q_{1})$
		  &
		    inductive hypothesis
		\\
		    $P_{1}\{b/a\}\equiv_{\alpha} Q_{1}\{x/y\}\{b/a\}$
		  &
		    lemma \ref{alphaequivalencecommutativity} and transitivity
		\\
		    $P_{1}\{b/a\}\equiv_{\alpha} Q_{1}\{b/a\}\{x/y\}$
		  &
		    rule $AlpRes1$
		\\
		    $(\nu x)(P_{1}\{b/a\})\equiv_{\alpha} (\nu y)(Q_{1}\{b/a\})$
		  &
		    definition of substitution
		\\
		    $((\nu x)P_{1})\{b/a\}\equiv_{\alpha} ((\nu y)Q_{1})\{b/a\}$
		  &
		\\
	      \end{tabular}
	    \end{center}
	\end{description}
    \end{description}
  \end{proof}
\end{lemma}


\begin{lemma}
  \[
    P\equiv_{\alpha} P \{x/y\}\{y/x\}
  \]
  esistono delle precondizioni per le quali il lemma e' vero? esistono delle precondizioni per le quali si puo' addirittura avere l'uguaglianza sintattica?
\end{lemma}


In the proof of equivalence of the semantics in the next section we need the following lemmas


\begin{lemma}
  $P\{x/y\}\equiv_{\alpha}Q$ if and only if $P\equiv_{\alpha}Q\{y/x\}$. 

  NON FUNZIONA LA DIMOSTRAZIONE! staro' forse esagerando?
  \begin{proof}
    The proof is an induction on the length of the proof tree of $P\{x/y\}\equiv_{\alpha}Q$ and then by cases on the last rule:
    \begin{description}
      \item[base case]
	the last rule can be
	\begin{description}
	  \item[$AlpZero$]
	    in this case both $P$ and $Q$ are the null process $0$ so the thesis holds.
	  \item[$AlpIde$]
 	    for this rule to apply $P\{x/y\}$ and $Q$ must be some identifier $A$ with the same variable. Suppose that 
	    $P=A(\tilde{a}|\tilde{b})$
	    There can be some different cases:
	    \begin{description}
	      \item[$y\in \tilde{a}$]
		we can suppose that $\tilde{a}=y, \tilde{c}$ then
		\begin{description}
		  \item[$x\in \tilde{b}$]
		    we can suppose that $\tilde{b}=x, \tilde{d}$, then
		    \[
		      Q=P\{x/y\}=A(x, \tilde{c}|z, \tilde{d})
		    \]
		    where $z$ is a fresh name. We need now the identifier equal to $Q\{y/x\}=A(x, \tilde{c}|z, \tilde{d})\{y/x\}$ so we have to distinguish two cases:
		    \begin{description}
		      \item[$x\in tilde{d}$]
			
		      \item[$x\notin tilde{d}$]
		      \[
			Q\{y/x\}=A(x, \tilde{c}|z, \tilde{d})\{y/x\}=A(y, \tilde{c}|z, \tilde{d})
		      \]			
		    \end{description}
		  \item[$y\notin \tilde{y}$]
		    in this case there is no need to change bound names so 
		    \[
		      Q\{y/x\}=A(y, \tilde{z}|\tilde{y})
		    \]
		\end{description}
	      \item[$x\notin \tilde{x}$]
		then 
		\[
		  Q\{y/x\}=Q=A(\tilde{x}|\tilde{y})
		\]
		  
	    \end{description}

% 	    \begin{description}
% 	      \item[$y\notin \tilde{x}$]:
% 		\begin{description}
% 		  \item[$y\notin \tilde{y}$]:
% 		    \begin{description}
% 		      \item[$x\notin \tilde{x}$]:
% 			\begin{description}
% 			  \item[$x\notin \tilde{y}$]:
% 
% 			  \item[$x\in \tilde{y}$]:
% 			    
% 			\end{description}
% 		      \item[$x\in \tilde{x}$]:
% 			\begin{description}
%  			  \item[$x\notin \tilde{y}$]:
%  		
%  			  \item[$x\in \tilde{y}$]:
%  		
%  			\end{description}		
%  		    \end{description}
%  		  \item[$y\in \tilde{y}$]:
%  		    \begin{description}
%  		      \item[$x\notin \tilde{x}$]:
%  			\begin{description}
%  			  \item[$x\notin \tilde{y}$]:
%  		
%  			  \item[$x\in \tilde{y}$]:
%  		
%  			\end{description}
%  		      \item[$x\in \tilde{x}$]:
%  			\begin{description}
%  			  \item[$x\notin \tilde{y}$]:
%  		
%  			  \item[$x\in \tilde{y}$]:
%  		
%  			\end{description}		
%  		    \end{description}
%  		\end{description}
%  	      \item[$y\in \tilde{x}$]:
%  		\begin{description}
%  		  \item[$y\notin \tilde{y}$]:
%  		    \begin{description}
%  		      \item[$x\notin \tilde{x}$]:
%  			\begin{description}
%  			  \item[$x\notin \tilde{y}$]:
%  		
%  			  \item[$x\in \tilde{y}$]:
%  		
%  			\end{description}
%  		      \item[$x\in \tilde{x}$]:
%  			\begin{description}
%  			  \item[$x\notin \tilde{y}$]:
%  		
%  			  \item[$x\in \tilde{y}$]:
%  		
%  			\end{description}		
%  		    \end{description}
%  		  \item[$y\in \tilde{y}$]:
%  		    \begin{description}
%  		      \item[$x\notin \tilde{x}$]:
%  			\begin{description}
%  			  \item[$x\notin \tilde{y}$]:
% 		
% 			  \item[$x\in \tilde{y}$]:
% 		
% 			\end{description}
% 		      \item[$x\in \tilde{x}$]:
% 			\begin{description}
% 			  \item[$x\notin \tilde{y}$]:
% 		
% 			  \item[$x\in \tilde{y}$]:
% 		
% 			\end{description}		
% 		    \end{description}		
% 		\end{description}		
% 	    \end{description}

 	\end{description}
      \item[inductive case]
	
    \end{description}
  \end{proof}
\end{lemma}

% \begin{definition}
%   We call depth of a $\pi$ calculus process the depth of its syntax tree and we use the letter $\delta$ to represent it. Formally:
% \end{definition}


\begin{lemma}
  The $\alpha$ equivalence is an equivalence relation.
  \begin{proof}:
    \begin{description}
      \item[reflexivity]
	We prove $P\equiv_{\alpha}P$ by structural induction on $P$:
	\begin{description}
	  \item[$0$]:
	    \[\inferrule* [left=AlpZero]{
	    }{
	      0\equiv_{\alpha}0
	    }\]
	  \item[$\tau.P_{1}$]:
	    for induction $P_{1}\equiv_{\alpha}P_{1}$ so
	    \[\inferrule* [left=AlpTau]{
	      P_{1}\equiv_{\alpha}P_{1}
	    }{
	      \tau.P_{1}\equiv_{\alpha}\tau.P_{1}
	    }\]
	  \item[$x(y).P_{1}$]:
	    for induction $P_{1}\equiv_{\alpha}P_{1}$ so
	    \[\inferrule* [left=AlpInp]{
	      P_{1}\equiv_{\alpha}P_{1}
	    }{
	      x(y).P_{1}\equiv_{\alpha}x(y).P_{1}
	    }\]
	  \item[$\overline{x}y.P_{1}$]:
	    for induction $P_{1}\equiv_{\alpha}P_{1}$ so
	    \[\inferrule* [left=AlpOut]{
	      P_{1}\equiv_{\alpha}P_{1}
	    }{
	      \overline{x}y.P_{1}\equiv_{\alpha}\overline{x}y.P_{1}
	    }\]
	  \item[$P_{1}+P_{2}$]:
	    for induction $P_{1}\equiv_{\alpha}P_{1}$ and $P_{2}\equiv_{\alpha}P_{2}$ so
	    \[\inferrule* [left=AlpSum]{
		  P_{1}\equiv_{\alpha}P_{1}
		\\
		  P_{2}\equiv_{\alpha}P_{2}
	    }{
	      P_{1}+P_{2}\equiv_{\alpha}P_{1}+P_{2}
	    }\]
	  \item[$P_{1}|P_{2}$]:
	    for induction $P_{1}\equiv_{\alpha}P_{1}$ and $P_{2}\equiv_{\alpha}P_{2}$ so
	    \[\inferrule* [left=AlpPar]{
		  P_{1}\equiv_{\alpha}P_{1}
		\\
		  P_{2}\equiv_{\alpha}P_{2}
	    }{
	      P_{1}|P_{2}\equiv_{\alpha}P_{1}|P_{2}
	    }\]
	  \item[$(\nu x)P_{1}$]:
	    for induction $P_{1}\equiv_{\alpha}P_{1}$ so
	    \[\inferrule* [left=AlpRes]{
	      P_{1}\equiv_{\alpha}P_{1}
	    }{
	      (\nu x)P_{1}\equiv_{\alpha}(\nu x)P_{1}
	    }\]
	  \item[$A(\tilde{x}|\tilde{y})$]:
	    \[\inferrule* [left=AlpIde]{
	    }{
	      A(\tilde{x}|\tilde{y})\equiv_{\alpha}A(\tilde{x}|\tilde{y})
	    }\]
	\end{description}
      \item[symmetry] % LA SIMMETRIA VIENE USATA DALLA TRANSITIVITA' IN SEGUITO
	A proof of 
	\[
	  P\equiv_{\alpha}Q\Rightarrow Q\equiv_{\alpha}P
	\]
	can go by induction on the length of the proof tree of $P\equiv_{\alpha}Q$ and then by cases on the last rule used. Nevertheless we notice that the base case rules $AlpZero$ and $AlpIde$ are symmetric and the inductive case rules are symmetric except for $AlpRes1$ and $AlpInp1$. So we provide with the cases for those last two rules:
	\begin{description}
	  \item[$AlpRes1$]
	    the last part of the proof tree is
	    \[\inferrule*[left=AlpRes1]{
		P\{y/x\}\equiv_{\alpha}Q
	      \\
		x\neq y
	      \\
		y\notin fn(P)
	    }{
	      (\nu x)P\equiv_{\alpha}(\nu y)Q
	    }\]
	    we apply the inductive hypothesis on $P\{y/x\}\equiv_{\alpha}Q$ and get $Q\equiv_{\alpha}P\{y/x\}$ which implies $Q\{x/y\}\equiv_{\alpha}P$
	    \begin{center}
	      DA DIMOSTRARE
	      $Q\equiv_{\alpha}P\{y/x\}$ and $y\notin fn(P)$ implies $Q\{x/y\}\equiv_{\alpha}P$ and $x\notin fn(Q)$
	    \end{center}
	    so an application of the same rule yields:
	    \[\inferrule*[left=AlpRes1]{
		Q\{x/y\}\equiv_{\alpha}P
	      \\
		x\neq y
	      \\
		x\notin fn(Q)
	    }{
	      (\nu y)QP\equiv_{\alpha}(\nu x)
	    }\]
	  \item[$AlpInp1$]
	    this is very similar to the previous.
	\end{description}
      \item[transitivity]
	suppose 
	\begin{center}
	  $P\equiv_{\alpha}Q$ and $Q\equiv_{\alpha}R$
	\end{center}
	we prove the thesis $P\equiv_{\alpha}R$ by induction on the length of the proof tree of $P\equiv_{\alpha}Q$. If the tree has only one node then the rule used must be $AlpZero$ or $AlpIde$. In the former case both $P$ and $Q$ are $0$ and so $0\equiv_{\alpha}R$. For symmetry and the inversion lemma then $R$ is also $0$. In the latter case a similar argument applies. If the proof tree has more than one node then we proceed by cases on the last rule
	\begin{description}
	  \item[$AlpInp$]:
	    In this case $P=x(y).P_{1}$, $Q=x(y).Q_{1}$ and $P_{1}\equiv_{\alpha}Q_{1}$ and $x(y).Q_{1}\equiv_{\alpha}R$ which implies for symmetry and the inversion lemma that one of the following cases holds:
	    \begin{itemize}
	      \item 
		$R=x(y).R_{1}$ and $Q_{1}\equiv_{\alpha}R_{1}$:
		\begin{center}
		  \begin{tabular}{ll}
			$P_{1}\equiv_{\alpha}Q_{1}$ and $Q_{1}\equiv_{\alpha}R_{1}$
		      &
			inductive hypothesis
		    \\
			$\Rightarrow P_{1}\equiv_{\alpha}R_{1}$
		      &
			rule $AlpInp$
		    \\
			$\Rightarrow x(y).P_{1}\equiv_{\alpha}x(y).R_{1}$
		      &
		    \\
		  \end{tabular}
		\end{center}
	      \item 
		$R=x(z).R_{1}$ and $Q_{1}\{y/z\}\equiv_{\alpha}R_{1}$:
		\begin{center}
		  \begin{tabular}{ll}
			$P_{1}\equiv_{\alpha}Q_{1}$
		      &
			lemma \ref{alphaequivalencesubstitution}
		    \\
			$\Rightarrow P_{1}\{y/z\}\equiv_{\alpha}Q_{1}\{y/z\}$
		      &
			inductive hypothesis
		    \\
			$\Rightarrow P_{1}\{y/z\}\equiv_{\alpha}R_{1}$
		      &
			rule $AlpInp1$
		    \\
			$\Rightarrow x(y).P_{1}\equiv_{\alpha}x(z).R_{1}$
		      &
		    \\
		  \end{tabular}
		\end{center}		
		 
	    \end{itemize}
	  \item[$AlpRes$]:
	    
	  \item[$AlpInp1$]:
	    
	  \item[$AlpRes1$]:
	    
	  \item[$AlpSum$]:
	    
	  \item[$AlpPar$]:
	    
	  \item[$AlpSum$]:
	    
	  \item[$AlpTau$]:
	    
	  \item[$AlpOut$]:
	    
	\end{description}
    \end{description}
  \end{proof}
\end{lemma}




\begin{lemma}
  E' FALSO!!!!! !!!! !!! !! !:\begin{itemize}
    \item
      If $P\equiv \tau.Q$ then $P=\tau.P_{1}$ for some $P_{1}$ such that $P_{1}\equiv Q$
    \item
      If $P\equiv \overline{x}y.Q$ then $P=\overline{x}y.P_{1}$ for some $P_{1}$ such that $P_{1}\equiv Q$
    \item
      If $P\equiv x(y).Q$ then one and only one of the following cases holds:
      \begin{itemize}
	\item 
	  $P=x(z).P_{1}$ for some $P_{1}$ such that $P_{1}\{z/y\}\equiv Q$
	\item
	  $P=x(y).P_{1}$ for some $P_{1}$ such that $P_{1}\equiv Q$
      \end{itemize}
    \item
      If $P\equiv Q_{1}+Q_{2}$ then $P=P_{1}+P_{2}$ for some $P_{1}$ and $P_{2}$ such that $P_{1}\equiv Q_{1}$ and $P_{2}\equiv Q_{2}$.
    \item 
      If $P\equiv Q_{1}|Q_{2}$ then $P=P_{1}|P_{2}$ for some $P_{1}$ and $P_{2}$ such that $P_{1}\equiv Q_{1}$ and $P_{2}\equiv Q_{2}$.
    \item 
      If $P\equiv (\nu y)Q$ then one and only one of the following cases holds:
      \begin{itemize}
        \item 
	  $P=(\nu z)P_{1}$ such that $P_{1}\{z/y\}\equiv Q$
	\item
	  $P=(\nu y).P_{1}$ for some $P_{1}$ such that $P_{1}\equiv Q$
      \end{itemize}
    \item 
      If $P\equiv A(\tilde{x}|\tilde{y})$ then ??? ?? ?
  \end{itemize}
  \begin{proof}
    
  \end{proof}
\end{lemma}


\section{Operational semantic with structural congruence}

\subsection{Early semantic with $\alpha$ conversion only}
In this subsection we introduce the early operational semantic for $\pi$ calculus with the use of a minimal structural congruence, specifically we exploit only the easy of $\alpha$ conversion.

\begin{definition}\index{transition relation! early! with $\alpha$ conversion}
  The \emph{early transition relation with $\alpha$ conversion} $\rightarrow\subseteq \mathbb{P}\times \mathbb{A} \times \mathbb{P}$ is the smallest relation induced by the rules in table \ref{transitionrelationearlywithalphaconversion}.

  \begin{table}
    \begin{tabular}{ll}  
      \hline\\
	  \bf{Out}
	  \begin{tabular}{c}
	      $\;\;$
	    \\\hline
	      $\overline{x}y.P \xrightarrow{\overline{x}y} P$
	  \end{tabular}
	&
	  \bf{EInp}
	  \begin{tabular}{c}
	    \\\hline
	      $x(y).P \xrightarrow{xz} P\{z/y\}$
	  \end{tabular}
      \\\\
	  \bf{ParL}
	  \begin{tabular}{c}
	      $P \xrightarrow{\alpha} P^{'}\;\; bn(\alpha)\cap fn(Q)=\emptyset$
	    \\\hline
	      $P|Q \xrightarrow{\alpha} P^{'}|Q$
	  \end{tabular}
	&
	  \bf{ParR}
	  \begin{tabular}{c}
	      $Q \xrightarrow{\alpha} Q^{'}\;\; bn(\alpha)\cap fn(Q)=\emptyset$
	    \\\hline
	      $P|Q \xrightarrow{\alpha} P|Q^{'}$
	  \end{tabular}
      \\\\
	  \bf{SumL}
	  \begin{tabular}{c}
	      $P \xrightarrow{\alpha} P^{'}$
	    \\\hline
	      $P+Q \xrightarrow{\alpha} P^{'}$
	  \end{tabular}
	&
	  \bf{SumR}
	  \begin{tabular}{c}
	      $Q \xrightarrow{\alpha} Q^{'}$
	    \\\hline
	      $P+Q \xrightarrow{\alpha} Q^{'}$
	    \end{tabular}
      \\\\
	  \bf{Res}
	  \begin{tabular}{c}
	      $P \xrightarrow{\alpha} P^{'}\;\; z\notin n(\alpha)$
	    \\\hline
	      $(\nu z) P \xrightarrow{\alpha} (\nu z) P^{'}$
	  \end{tabular}
	&
	  \bf{Alp}
	  \begin{tabular}{c}
	      $P\equiv_{\alpha}Q\; P\xrightarrow{\alpha}P^{'}$
	    \\\hline
	      $Q\xrightarrow{\alpha}P^{'}$
	  \end{tabular}
      \\\\
	  \bf{EComL}
	  \begin{tabular}{c}
	      $P \xrightarrow{xy} P^{'}\;\; Q\xrightarrow{\overline{x}y} Q^{'}$
	    \\\hline
	      $P|Q \xrightarrow{\tau} P^{'}|Q^{'}$
	  \end{tabular}
	&
	  \bf{EComR}
	  \begin{tabular}{c}
	      $P \xrightarrow{\overline{x}y} P^{'}\;\; Q\xrightarrow{xy} Q^{'}$
	    \\\hline
	      $P|Q \xrightarrow{\tau} P^{'}|Q^{'}$
	  \end{tabular}
      \\\\
	    \bf{ClsL}
	    \begin{tabular}{c}
		$P \xrightarrow{\overline{x}(z)} P^{'}$  
		$Q \xrightarrow{xz} Q^{'}$ 
		$z\notin fn(Q)$
	      \\\hline
		$P|Q \xrightarrow{\tau} (\nu z)(P^{'}|Q^{'})$
	    \end{tabular}
	&
	    \bf{ClsR}
	    \begin{tabular}{c}
		$P \xrightarrow{xz} P^{'}$  
		$Q \xrightarrow{\overline{x}(z)} Q^{'}$ 
		$z\notin fn(P)$
	      \\\hline
		$P|Q \xrightarrow{\tau} (\nu z)(P^{'}|Q^{'})$
	    \end{tabular}
      \\\\
	  \bf{Cns}
	  \begin{tabular}{c}
	      $A(\tilde{x}|\tilde{y}) \stackrel{def}{=} P\; P\{\tilde{w}/\tilde{x}\} \xrightarrow{\alpha} P^{'}$
	    \\\hline
	      $A(\tilde{x}|\tilde{y})\{\tilde{w}/\tilde{x}\} \xrightarrow{\alpha} P^{'}$
	  \end{tabular}
	&
	  \bf{Opn}
	  \begin{tabular}{c}
	      $P \;\xrightarrow{\overline{x}z} P^{'}\;\; z\neq x$
	    \\\hline
	      $(\nu z) P \;\xrightarrow{\overline{x}(z)} P^{'}$
	  \end{tabular}
      \\\\
	  \bf{Tau}
	  \begin{tabular}{c}
	      $\;\;$
	    \\\hline
	      $\tau.P \xrightarrow{\tau} P$
	  \end{tabular}
	&
      \\\hline
    \end{tabular}
    \caption{Early transition relation with $\alpha$ conversion}
    \label{transitionrelationearlywithalphaconversion}
  \end{table}
\end{definition}



\subsection{Early semantic with structural congruence}

\begin{definition}\index{transition relation! pi! early! with structural congruence}
  The \emph{early transition relation with structural congruence} $\rightarrow\subseteq \mathbb{P}\times \mathbb{A} \times \mathbb{P}$ is the smallest relation induced by the rules in table \ref{earlysemanticwithstructuralcongruence}.

  \begin{table}
    \begin{tabular}{lll}
      \hline\\
	  \bf{Out}
	  \begin{tabular}{c}
	    \hline
	    $\overline{x}y.P \xrightarrow{\overline{x}y} P$
	  \end{tabular}
	  &
	  \bf{EInp}
	  \begin{tabular}{c}
	    \hline
	    $x(z).P \xrightarrow{xy} P\{y/z\}$
	  \end{tabular}
	  &
	  \bf{Par}
	  \begin{tabular}{c}
	    $P \xrightarrow{\alpha} P^{'}\;\; bn(\alpha)\cap fn(Q)=\emptyset$\\
	    \hline
	    $P|Q \xrightarrow{\alpha} P^{'}|Q$
	  \end{tabular}
      \\\\
	  \bf{Sum}
	  \begin{tabular}{c}
	    $P \xrightarrow{\alpha} P^{'}$\\
	    \hline
	    $P+Q \xrightarrow{\alpha} P^{'}$
	  \end{tabular}
	  &
	    \bf{ECom}
	    \begin{tabular}{c}
	      $P \xrightarrow{xy} P^{'}\;\; Q\xrightarrow{\overline{x}y} Q^{'}$\\
	      \hline
	      $P|Q \xrightarrow{\tau} P^{'}|Q^{'}$
	    \end{tabular}
	  &
	  \bf{Res}
	  \begin{tabular}{c}
	    $P \xrightarrow{\alpha} P^{'}\;\; z\notin n(\alpha)$\\
	    \hline
	    $(\nu z) P \xrightarrow{\alpha} (\nu z) P^{'}$
	  \end{tabular}
      \\\\
	  \bf{Tau}
	  \begin{tabular}{c}
	    \hline
	    $\tau.P \xrightarrow{\tau} P$
	  \end{tabular}
	  &
	  \bf{Opn}
	  \begin{tabular}{c}
	    $P \xrightarrow{\overline{x}z} P^{'}\;\; z\neq x$\\
	    \hline
	    $(\nu z) P \xrightarrow{\overline{x}(z)} P^{'}$
	  \end{tabular}
	  &
	  \bf{Str}
	  \begin{tabular}{c}
	    $P\equiv P^{'}\;\; P\xrightarrow{\alpha} Q\;\; Q\equiv Q^{'}$\\
	    \hline
	    $P^{'} \xrightarrow{\alpha} Q^{'}$
	  \end{tabular}
      \\\hline
    \end{tabular}
    \caption{Early semantic with structural congruence}
    \label{earlysemanticwithstructuralcongruence}
  \end{table}
\end{definition}


\begin{example}
  We prove now that
  \begin{center}
    $a(x).P\; |\; (\nu b)\overline{a}b.Q\; \xrightarrow{\tau}\; (\nu b)(P\{b/x\}\; |\; Q)$
  \end{center}
  where $b\notin fn(P)$.
  This follows from
  \[
    a(x).P\; |\; (\nu b)\overline{a}b.Q\; \equiv\; (\nu b)(a(x).P\; |\; \overline{a}b.Q)
  \]
  and
  \[
    (\nu b)(a(x).P\; |\; \overline{a}b.Q) \xrightarrow{\tau} (\nu b)(P\{b/x\}\; |\; Q)
  \]
  with the rule $Str$. We can prove the last transition in the following way:
  \[
    \inferrule* [left=Res] {
      \inferrule* [left=Com] {
	  \inferrule* [left=EInp] {
	  }{
	    a(x).P\; \xrightarrow{ab}\; P\{b/x\}
	  }
	\\
	  \inferrule* [left=Out] {
	  }{
	    \overline{a}b.Q\; \xrightarrow{\overline{a}b}\; Q
	  }
      }{
	a(x).P\; |\; \overline{a}b.Q\; \xrightarrow{\tau}\; P\{b/x\}\; |\; Q
      }
    }{
      (\nu b)(a(x).P\; |\; \overline{a}b.Q)\; \xrightarrow{\tau}\; (\nu b)(P\{b/x\}\; |\; Q)
    }
  \]

\end{example}

\begin{example}
    We want to prove now that:
    \begin{center}
      $((\nu b) a(x).P)\; |\; \overline{a}b.Q\; \xrightarrow{\tau}\; (\nu c) (P\{c/b\}\{b/x\}\; |\; Q)$
    \end{center}
    where the name $c$ is not in the free names of $Q$. We can exploit the structural congruence and get that
    \[
      ((\nu b) a(x).P) | \overline{a}b.Q\; \equiv\; (\nu c) (a(x).(P\{c/b\}) | \overline{a}b.Q)     
    \]
    then we have
    \[
	\inferrule* [left=Res] {
	  \inferrule* [left=Com]{
	      \inferrule* [left=EInp]{
	      }{
		a(x).P\{c/b\}\; \xrightarrow{ab}\; P\{c/b\}\{b/x\}
	      }
	    \\
	      \inferrule* [left=Out]{
	      }{
		\overline{a}b.Q\; \xrightarrow{\overline{a}b}\; Q
	      }
	  }{
	      (a(x).(P\{c/b\}) | \overline{a}b.Q)\; \xrightarrow{\tau}\; (P\{c/b\}\{b/x\} | Q)
	  }
	}{
	  (\nu c) (a(x).(P\{c/b\}) | \overline{a}b.Q)\; \xrightarrow{\tau}\; (\nu c) (P\{c/b\}\{b/x\} | Q)
	}
    \]
    Now we just apply the rule $Str$ to prove the thesis.
\end{example}



\subsection{Late semantic with structural congruence}

\begin{definition}\index{transition relation! pi! late! with structural congruence}
  The \emph{late transition relation with structural congruence} $\rightarrow\subseteq \mathbb{P}\times \mathbb{A} \times \mathbb{P}$ is the smallest relation induced by the rules in table \ref{latewith}.
  \begin{table}
    \begin{tabular}{ll}
	\hline\\
	\bf{Prf}
	\begin{tabular}{c}
	  \hline
	  $\alpha.P \xrightarrow{\alpha} P$
	\end{tabular}
      &
	\bf{Sum}
	\begin{tabular}{c}
	    $P \xrightarrow{\alpha} P^{'}$
	  \\\hline
	    $P+Q \xrightarrow{\alpha} P^{'}$
	\end{tabular}
    \\\\
	\bf{Par}
	\begin{tabular}{c}
	    $P \xrightarrow{\alpha} P^{'}\;\; bn(\alpha)\cap fn(Q)=\emptyset$
	  \\\hline
	    $P|Q \xrightarrow{\alpha} P^{'}|Q$
	\end{tabular}
      &
	\bf{Res}
	\begin{tabular}{c}
	    $P \xrightarrow{\alpha} P^{'}$ $z\notin n(\alpha)$
	  \\\hline
	    $(\nu z) P \xrightarrow{\alpha} (\nu z) P^{'}$
	\end{tabular}
    \\\\    
	\bf{LCom}
	\begin{tabular}{c}
	    $P \xrightarrow{x(y)} P^{'}\;\; Q\xrightarrow{\overline{x}z} Q^{'}$
	  \\\hline
	  $P|Q \xrightarrow{\tau} P^{'}\{z/y\}|Q^{'}$
	\end{tabular}
      &
	\bf{Str}
	\begin{tabular}{c}
	    $P\equiv P^{'}\;\; P\xrightarrow{\alpha} Q\;\; Q\equiv Q^{'}$
	  \\\hline
	    $P^{'} \xrightarrow{\alpha} Q^{'}$
	\end{tabular}
    \\\\
	\bf{Opn}
	\begin{tabular}{c}
	    $P \xrightarrow{\overline{x}z} P^{'}\;\; z\neq x$
	  \\\hline
	    $(\nu z) P \xrightarrow{\overline{x}(z)} P^{'}$
	\end{tabular}
      &
    \\\hline
    \end{tabular}
    \caption{Late semantic with structural congruence}
    \label{latewith}
  \end{table}
\end{definition}


\begin{example}
  We prove now that
  \begin{center}
    $a(x).P\; |\; (\nu b)\overline{a}b.Q\; \xrightarrow{\tau} P\{b/x\}\; |\; Q$
  \end{center}
  where $b\notin fn(P)$. This follows from
  \[
    a(x).P\; |\; (\nu b)\overline{a}b.Q\; \equiv\; (\nu b)(a(x).P\; |\; \overline{a}b.Q)
  \]
  and
  \[
    (\nu b)(a(x).P\; |\; \overline{a}b.Q) \xrightarrow{\tau} (\nu b)(P\{b/x\}\; |\; Q)
  \]
  with the rule $Str$. We can prove the last transition in the following way:
  \[
    \inferrule* [left=Res] {
	\inferrule* [left=LCom] {
	    \inferrule* [left=LInp] {
	      b\notin fn(P)
	    }{
	      a(x).P\; \xrightarrow{ab}\; P\{b/x\}
	    }
	  \\
	    \inferrule* [left=Out] {
	    }{
	      \overline{a}b.Q\; \xrightarrow{\overline{a}b}\; Q
	    }
	}{
	  a(x).P\; |\; \overline{a}b.Q\; \xrightarrow{\tau}\; P\{b/x\}\; |\; Q
	}
      \\
	b\notin n(\tau)
    }{
      (\nu b)(a(x).P\; |\; \overline{a}b.Q)\; \xrightarrow{\tau}\; (\nu b)(P\{b/x\}\; |\; Q)
    }
  \]

\end{example}

\begin{example}
    We want to prove now that:
    \begin{center}
      $((\nu b) a(x).P)\; |\; \overline{a}b.Q\; \xrightarrow{\tau}\; (\nu c) (P\{c/b\}\{b/x\}\; |\; Q)$
    \end{center}
    where the name $c$ is not in the free names of $Q$ and is not in the names of $P$. We can exploit the structural congruence and get that
    \[
      ((\nu b) a(x).P) | \overline{a}b.Q\; \equiv\; (\nu c) (a(x).(P\{c/b\}) | \overline{a}b.Q)     
    \]
    then we have
    \[
	\inferrule* [left=Res] {
	    \inferrule* [left=LCom]{
		\inferrule* [left=LInp]{
		  b\notin fn(P\{c/b\})
		}{
		  a(x).P\{c/b\}\; \xrightarrow{ab}\; P\{c/b\}\{b/x\}
		}
	      \\
		\inferrule* [left=Out]{
		}{
		  \overline{a}b.Q\; \xrightarrow{\overline{a}b}\; Q
		}
	    }{
	      (a(x).(P\{c/b\}) | \overline{a}b.Q)\; \xrightarrow{\tau}\; (P\{c/b\}\{b/x\} | Q)
	    }
	  \\
	    c\notin n(\tau)
	}{
	  (\nu c) (a(x).(P\{c/b\}) | \overline{a}b.Q)\; \xrightarrow{\tau}\; (\nu c) (P\{c/b\}\{b/x\} | Q)
	}
    \]
    Now we just apply the rule $Str$ to prove the thesis.
\end{example}



\section{Equivalence of the semantics}
\subsection{Equivalence of the early semantics}
In this subsection we write $\rightarrow_{1}$ for the early semantic without structural congruence, $\rightarrow_{2}$ for the early semantic with just $\alpha$ conversion and $\rightarrow_{3}$ for the early semantic with the full structural congruence. We call $R_{1}$ the set of rules for $\rightarrow_{1}$, $R_{2}$ the set of rules for $\rightarrow_{2}$ and $R_{3}$ the set of rules for $\rightarrow_{3}$. 
In the following section we will need:

\begin{lemma} %Dove lo uso? Serve davvero?
\[
  P\equiv Q\; \Rightarrow\; fn(Q)= fn(P)
\]
  \begin{proof}
    A proof can go by induction on the proof tree of $P\equiv Q$ and then by cases on the last rule used in the proof tree. 
    \begin{description}
      \item[base case]
	The last and only rule of the proof tree can be one of the following axioms:
	\begin{description}
	  \item[SC-ALP]$\begin{array}{c}P \equiv_{\alpha} Q\\\overline{P\equiv Q}\end{array}$
	  \item[SC-SUM-ASC] $M_{1}+(M_{2}+M_{3})\equiv (M_{1}+M_{2})+M_{3}$ 
	  \item[SC-SUM-COM] $M_{1}+M_{2}\equiv M_{2}+M_{1}$ 
	  \item[SC-SUM-INC] $M+0\equiv M$
	  \item[SC-COM-ASC] $P_{1}|(P_{2}|P_{3})\equiv (P_{1}|P_{2})|P_{3}$ 
	  \item[SC-COM-COM] $P_{1}|P_{2}\equiv P_{2}|P_{1}$ 	
	  \item[SC-COM-INC] $P|0\equiv P$
	  \item[SC-RES] $(\nu z) (\nu w) P \equiv (\nu w) (\nu z) P$ 
	  \item[SC-RES-INC] $(\nu z) 0 \equiv 0$ 
	  \item[SC-RES-COM] $(\nu z) (P_{1}|P_{2}) \equiv P_{1}|(\nu z) P_{2}$ if $z\notin fn(P_{1})$
	  \item[SC-RES-SUM] $(\nu z) (P_{1}+P_{2}) \equiv P_{1}+(\nu z) P_{2}$ if $z\notin fn(P_{1})$
	  \item[SC-IDE]$A(\tilde{w}|\tilde{y})\equiv P\{\tilde{w}/\tilde{x}\}$
	\end{description}
      \item[inductive case]
	\item[SC-REFL] $\inferrule*{}{P\equiv P}$
	\item[SC-SIMM] $\inferrule*{Q\equiv P}{P\equiv Q}$
	\item[SC-TRAN] $\inferrule*{P\equiv Q \\ Q\equiv R}{P\equiv R}$
	\item[SC-CONG] $\inferrule*{P\equiv Q}{C[P]\equiv C[Q]}$
    \end{description}
  \end{proof}
\end{lemma}

We would like to prove that $P\xrightarrow{\alpha}_{2}P^{'}\; \Rightarrow\; P\xrightarrow{\alpha}_{1}P^{'}$ but this is false because
\[
  \inferrule* [left=Alp]{
      \overline{x}y.x(y).0 \equiv_{\alpha} \overline{x}y.x(w).0
    \\
      \inferrule* [left=Out]{
      }{
	\overline{x}y.x(w).0\xrightarrow{\overline{x}y}_{2}x(w).0
      }
  }{
    \overline{x}y.x(y).0\xrightarrow{\overline{x}y}_{2}x(w).0
  }
\]
so we want to prove 
\[
  \overline{x}y.x(y).0\xrightarrow{\overline{x}y}_{1}x(w).0
\] 
The head of the transition has an output prefixing at the top level so the only rule we could use is $Out$, but the application of $Out$ yields 
\[
  \overline{x}y.x(y).0\xrightarrow{\overline{x}y}_{1}x(y).0
\] 
which is not want we want. So we prove a weaker version
\begin{theorem}
\[
  P\xrightarrow{\alpha}_{2}P^{'}\; \Rightarrow\; \exists P^{''}: P^{''}\equiv_{\alpha}P^{'}\; and\; P\xrightarrow{\alpha}_{1}P^{''}
\]
  \begin{proof}
    The proof goes by induction on the depth of the derivation tree of $P\xrightarrow{\alpha}_{2}P^{'}$ and then by cases on the last rule used:
    \begin{description}
      \item[base case]
	If the depth of the derivation tree is one, the rule used has to be a prefix rule 
	\[
	  \{Out, EInp, Tau\}\subseteq R_{1}\cap R_{2}
	\]
	so a derivation tree of $P\xrightarrow{\alpha}_{2}P^{'}$ is also a derivation tree of $P\xrightarrow{\alpha}_{1}P^{'}$
      \item[inductive case]
	If the depth of the derivation tree is more than one, then we proceed by cases on the last rule $R$. If the rule $R$ is not a prefix rule and it is in common between the two semantics:
	\[
	  R\in \{ParL, ParR, SumL, SumR, Res, EComL, EComR, ClsL, ClsR, Cns, Opn\}
	\]
	then we just apply the inductive hypothesis on the premises of $R$ and then reapply $R$ to get the desired derivation tree. We show just the case for $SumL$ when the end of the derivation tree is 
	\[
	  \inferrule* [left=SumL]{
	    P_{1}\xrightarrow{\alpha}_{2}P_{1}^{'}
	  }{
	    \underbrace{P_{1}+P_{2}}_{P}\xrightarrow{\alpha}_{2}\underbrace{P_{1}^{'}}_{P^{'}}
	  }
	\]
	\begin{center}
	  \begin{tabular}{ll}
	    &rule premise\\
	    $P_{1}\xrightarrow{\alpha}_{2}P_{1}^{'}$&inductive hypothesis\\
	    $\Rightarrow  P_{1}\xrightarrow{\alpha}_{1}P_{1}^{''}$ and $P_{1}^{'}\equiv_{\alpha}P_{1}^{''}$&rule $SumL$\\
	    $\Rightarrow  P_{1}+P_{2}\xrightarrow{\alpha}_{1}P_{1}^{''}$&\\
	  \end{tabular}
	\end{center}
	
	If the rule $R$ is in 
	\[
	  R_{2}-R_{1}=\{Alp\}
	\]
	then the last part of the derivation tree of $P\xrightarrow{\alpha}_{2}P^{'}$ is
	\[
	  \inferrule* [left=Alp]{
	      P\equiv_{\alpha}Q
	    \\
	      \inferrule* [left=S]{
		\cdots
	      }{
		Q\xrightarrow{\alpha}_{2}P^{'}
	      }
	  }{
	    P\xrightarrow{\alpha}_{2}P^{'}
	  }
	\]
	and the proof goes by cases on $S$ the last rule in the proof tree of $Q\xrightarrow{\alpha}_{2}P^{'}$:
	\begin{description}
	  \item[Out]:
	    If $S=Out$ then there exists some names $x,y$ and a process $Q_{1}$ such that 
	    \[
	      Q=\overline{x}y.Q_{1}
	    \]
	    and $\alpha=\overline{x}y$. 
 	    \begin{center}
 	      \begin{tabular}{ll}
 		$P\equiv_{\alpha}\overline{x}y.Q_{1}$&inversion lemma\\
 		$\Rightarrow P=\overline{x}y.P_{1}$ and $P_{1}\equiv_{\alpha}Q_{1}$&rule $Out$\\
 		$\Rightarrow \overline{x}y.P_{1}\xrightarrow{\overline{x}y}_{1}P_{1}$&\\
 	      \end{tabular}
 	    \end{center}
	  \item[EInp]
	    If $S=EInp$ then there exists some names $x,y,z$ and a process $Q_{1}$ such that $Q=x(y).Q_{1}$, $\alpha=xz$ and $P^{'}=Q_{1}\{z/y\}$. Since 
	    \[
	      P\equiv_{\alpha}x(y).Q_{1}
	    \]
	    then for the inversion lemma we have two cases:
	    \begin{itemize}
	      \item 
		:
		\begin{center}
		  \begin{tabular}{ll}
		    $P=x(y).P_{1}$ and $P_{1}\equiv_{\alpha}Q_{1}$& rule $EInp$\\
		    $\Rightarrow x(y).P_{1}\xrightarrow{xz}_{1}P_{1}\{z/y\}$& \\
		  \end{tabular}
		\end{center}
		This is what we want because for lemma \ref{alphaequivalencesubstitution}
		\[
		  P_{1}\equiv_{\alpha}Q_{1}\Rightarrow P_{1}\{z/y\}\equiv_{\alpha}Q_{1}\{z/y\}
		\]
	      \item
		:
		\begin{center}
		  \begin{tabular}{ll}
		    $P=x(w).P_{1}$ and $P_{1}\{y/w\}\equiv_{\alpha}Q_{1}$& rule $EInp$\\
		    $\Rightarrow x(w).P_{1}\xrightarrow{x z}_{1}P_{1}\{z/w\}$& \\
		  \end{tabular}
		\end{center}
		This is what we want because 
		\begin{center}
		  \begin{tabular}{ll}
		      $P_{1}\{y/w\}\equiv_{\alpha}Q_{1}$
		    &
		      lemma \ref{alphaequivalencesubstitution}
		  \\
		      $\Rightarrow P_{1}\{y/w\}\{z/y\}\equiv_{\alpha}Q_{1}\{z/y\}$
		    &
		      
		  \\
		      $\Rightarrow P_{1}\{z/w\}\equiv_{\alpha}Q_{1}\{z/y\}$
		    &
		      
		  \\
		  \end{tabular}
		\end{center}		
	    \end{itemize}
	  \item[Tau]
	    If $S=Tau$ then there exists a process $Q_{1}$ such that $Q=\tau.Q_{1}$ and $\alpha=\tau$ and $P^{'}=Q_{1}$. 
	    \begin{center}
	      \begin{tabular}{ll}
		$P\equiv_{\alpha}\tau.Q_{1}$&inversion lemma\\
		$\Rightarrow P=\tau.P_{1}$ and $P_{1}\equiv_{\alpha}Q_{1}$& rule $Tau$\\
		$\Rightarrow \tau.P_{1}\xrightarrow{\tau}_{1}P_{1}$& \\
	      \end{tabular}
	    \end{center}
	  \item[ParL]
	    If $S=ParL$ then there exists some processes $Q_{1},Q_{2}$ such that 
	    \[
	      Q=Q_{1}|Q_{2}
	    \]
	    Since 
	    \[
	      P\equiv_{\alpha}Q_{1}|Q_{2}
	    \]
	    then for the inversion lemma there exists $P_{1},P_{2}$ such that 
	    \[
	      P=P_{1}|P_{2}\; and\; P_{1}\equiv_{\alpha}Q_{1}\; and P_{2}\equiv_{\alpha}Q_{2}
	    \]
	    and so the last part of the derivation tree of $P\xrightarrow{\alpha}_{2}P^{'}$ looks like this:
	    \[
	      \inferrule* [left=Alp]{
		  P_{1}|P_{2}\equiv_{\alpha}Q_{1}|Q_{2}
		\\
		  \inferrule* [left=ParL]{
		      Q_{1}\xrightarrow{\alpha}_{2}Q_{1}^{'}
		    \\
		      bn(\alpha)\cap fn(Q_{2})=\emptyset
		  }{
		    Q_{1}|Q_{2}\xrightarrow{\alpha}_{2}Q_{1}^{'}|Q_{2}
		  }
	      }{
		\underbrace{P_{1}|P_{2}}_{P}\xrightarrow{\alpha}_{2}\underbrace{Q_{1}^{'}|Q_{2}}_{P^{'}}
	      }
	    \]
	    from this hypothesis we can create the following proof tree of $P_{1}\xrightarrow{\alpha}_{2}Q_{1}^{'}$:
	    \[
	      \inferrule* [left=Alp]{
		  P_{1}\equiv_{\alpha}Q_{1}
		\\
		  Q_{1}\xrightarrow{\alpha}_{2}Q_{1}^{'}
	      }{
		P_{1}\xrightarrow{\alpha}_{2}Q_{1}^{'}
	      }
	    \]
	    this proof tree is smaller than the proof tree of $P_{1}|P_{2}\xrightarrow{\alpha}_{2}Q_{1}^{'}|Q_{2}$ so we can apply the inductive hypothesis and get that there exists a process $Q_{1}^{''}$ such that
	    \[
	      Q_{1}^{'}\equiv Q_{1}^{''}\; and\; P_{1}\xrightarrow{\alpha}_{1}Q_{1}^{''} 
	    \]
	    then we apply again the rule $ParL$ and get 
	    \[
	      \inferrule* [left=ParL]{
		  P_{1}\xrightarrow{\alpha}_{1}Q_{1}^{''} 
		\\
		  bn(\alpha) \cap fn(P_{2})=\emptyset
	      }{
		\underbrace{P_{1}|P_{2}}_{P}\xrightarrow{\alpha}_{1}\underbrace{Q_{1}^{''}|P_{2}}_{P^{''}}
	      }
	    \]
	    The second premise of the previous instance holds because:
	    \[
	      bn(\alpha)\cap fn(Q_{2})=\emptyset\; and\; P_{2}\equiv_{\alpha}Q_{2}\Rightarrow bn(\alpha)\cap fn(P_{2})=\emptyset
	    \]
	  \item[ParR, SumL, SumR, EComL, EComR, ClsL, ClsR] 
	    This cases are similar to the previous.
	  \item[Res]
	    If $S=Res$ then there exists some name $z$ and a process $Q_{1}$ such that 
	    \[
	      Q=(\nu z)Q_{1}
	    \]
	    and $P^{'}=(\nu z)Q_{1}^{'}$. Since 
	    \[
	      P\equiv_{\alpha}(\nu z)Q_{1}
	    \]
	    then for the inversion lemma we have two cases:
	    \begin{itemize}
	      \item 
		there exists some $P_{1}$ such that 
		\[
		  P=(\nu z)P_{1}\; and\; P_{1}\equiv_{\alpha}Q_{1}
		\]
		and so the last part of the derivation tree of $P\xrightarrow{\alpha}_{2}P^{'}$ looks like this:
		\[
		  \inferrule* [left=Alp]{
		      (\nu z)P_{1}\equiv_{\alpha}(\nu z)Q_{1}
		    \\
		      \inferrule* [left=Res]{
			  Q_{1}\xrightarrow{\alpha}_{2}Q_{1}^{'}
			\\
			  z\notin n(\alpha)
		      }{
			(\nu z)Q_{1}\xrightarrow{\alpha}_{2}(\nu z)Q_{1}^{'}
		      }
		  }{
		    (\nu z)P_{1}\xrightarrow{\alpha}_{2}(\nu z)Q_{1}^{'}
		  }
		\]
		from this we create the following proof tree of $P_{1}\xrightarrow{\alpha}_{2}Q_{1}^{'}$:
		\[
		  \inferrule* [left=Alp]{
		      P_{1}\equiv_{\alpha}Q_{1}
		    \\
		      Q_{1}\xrightarrow{\alpha}_{2}Q_{1}^{'}
		  }{
		    P_{1}\xrightarrow{\alpha}_{2}Q_{1}^{'}
		  }		
		\]
		to which we can apply the inductive hypothesis and get that there exists a process $Q_{1}^{''}$ such that
		\[
		  P_{1}\xrightarrow{\alpha}_{1}Q_{1}^{''}\;and\; Q_{1}^{''}\equiv_{\alpha}Q_{1}^{'}
		\]
		then we apply the rule $Res$ to get
		\[
		      \inferrule* [left=Res]{
			  P_{1}\xrightarrow{\alpha}_{1}Q_{1}^{''}
			\\
			  z\notin n(\alpha)
		      }{
			(\nu z)P_{1}\xrightarrow{\alpha}_{1}(\nu z)Q_{1}^{''}
		      }		 
		\]
		this satisfies the thesis of the theorem because  
		\[
		  (\nu z)Q_{1}^{''}\equiv(\nu z)Q_{1}^{'}
		\]
	      \item
		there exists some $P_{1}$ such that 
		\[
		  P=(\nu y)P_{1}\; and\; P_{1}\{z/y\}\equiv_{\alpha}Q_{1}
		\]
		and so the last part of the derivation tree of $P\xrightarrow{\alpha}_{2}P^{'}$ looks like this:
		\[
		  \inferrule* [left=Alp]{
		      (\nu y)P_{1}\equiv_{\alpha}(\nu z)Q_{1}
		    \\
		      \inferrule* [left=Res]{
			  Q_{1}\xrightarrow{\alpha}_{2}Q_{1}^{'}
			\\
			  z\notin n(\alpha)
		      }{
			(\nu z)Q_{1}\xrightarrow{\alpha}_{2}(\nu z)Q_{1}^{'}
		      }
		  }{
		    (\nu y)P_{1}\xrightarrow{\alpha}_{2}(\nu z)Q_{1}^{'}
		  }
		\]
		from this we create the following proof tree of $P_{1}\{z/y\}\xrightarrow{\alpha}_{2}Q_{1}^{'}$:
		\[
		  \inferrule* [left=Alp]{
		      P_{1}\{z/y\}\equiv_{\alpha}Q_{1}
		    \\
		      Q_{1}\xrightarrow{\alpha}_{2}Q_{1}^{'}
		  }{
		    P_{1}\{z/y\}\xrightarrow{\alpha}_{2}Q_{1}^{'}
		  }		
		\]
		to which we can apply the inductive hypothesis and get that there exists a process $Q_{1}^{''}$ such that
		\[
		  P_{1}\{z/y\}\xrightarrow{\alpha}_{1}Q_{1}^{''}\;and\; Q_{1}^{''}\equiv_{\alpha}Q_{1}^{'}
		\]
		then we apply the rule $Res$ and $ResAlp$ to get
		\[
		  \inferrule* [left=ResAlp]{
		      \inferrule* [left=Res]{
			  P_{1}\{z/y\} \xrightarrow{\alpha}_{1}Q_{1}^{''}
			\\
			  z\notin n(\alpha)
		      }{
			(\nu z)P_{1}\{z/y\} \xrightarrow{\alpha}_{1}(\nu z)Q_{1}^{''}
		      }		 
		  }{
		    (\nu y)P_{1} \xrightarrow{\alpha}_{1}(\nu z)Q_{1}^{''}
		  }
		\]
		this satisfies the thesis of the theorem because  
		\[
		  (\nu z)Q_{1}^{''}\equiv(\nu z)Q_{1}^{'}
		\]
	    \end{itemize}
	  \item[Alp]
	    we can assume that there are no two consecutive application of the rule $Alp$ because we can merge them thanks to the transitivity of the alpha equivalence.
	  \item[Opn]
	    If $S=Opn$ then there exists some names $x,z$ and a process $Q_{1}$ such that 
	    \[
	      Q=(\nu z)Q_{1}
	    \]
	    and $P^{'}=Q_{1}^{'}$ and $\alpha=\overline{x}(z)$. Since 
	    \[
	      P\equiv_{\alpha}(\nu z)Q_{1}
	    \]
	    then for the inversion lemma we have two cases:
	    \begin{itemize}
	      \item 
		there exists some $P_{1}$ such that 
		\[
		  P=(\nu z)P_{1}\; and\; P_{1}\equiv_{\alpha}Q_{1}
		\]
		and so the last part of the derivation tree of $P\xrightarrow{\alpha}_{2}P^{'}$ looks like this:
		\[
		  \inferrule* [left=Alp]{
		      (\nu z)P_{1}\equiv_{\alpha}(\nu z)Q_{1}
		    \\
		      \inferrule* [left=Opn]{
			  Q_{1}\xrightarrow{\overline{x}z}_{2}Q_{1}^{'}
			\\
			  z\neq x
		      }{
			(\nu z)Q_{1}\xrightarrow{\overline{x}(z)}_{2}Q_{1}^{'}
		      }
		  }{
		    (\nu z)P_{1}\xrightarrow{\overline{x}(z)}_{2}Q_{1}^{'}
		  }
		\]
		from this we create the following proof tree of $P_{1}\xrightarrow{\overline{x}z}_{2}Q_{1}^{'}$:
		\[
		  \inferrule* [left=Alp]{
		      P_{1}\equiv_{\alpha}Q_{1}
		    \\
		      Q_{1}\xrightarrow{\overline{x}z}_{2}Q_{1}^{'}
		  }{
		    P_{1}\xrightarrow{\overline{x}z}_{2}Q_{1}^{'}
		  }		
		\]
		to which we can apply the inductive hypothesis and get that there exists a process $Q_{1}^{''}$ such that
		\[
		  P_{1}\xrightarrow{\overline{x}z}_{1}Q_{1}^{''}\;and\; Q_{1}^{''}\equiv_{\alpha}Q_{1}^{'}
		\]
		then we apply the rule $Opn$ to get
		\[
		      \inferrule* [left=Opn]{
			  P_{1}\xrightarrow{\overline{x}z}_{1}Q_{1}^{''}
			\\
			  z\neq x
		      }{
			(\nu z)P_{1}\xrightarrow{\alpha}_{1}Q_{1}^{''}
		      }		 
		\]
	      \item
		there exists some $P_{1}$ such that 
		\[
		  P=(\nu y)P_{1}\; and\; P_{1}\{z/y\}\equiv_{\alpha}Q_{1}
		\]
		and so the last part of the derivation tree of $P\xrightarrow{\alpha}_{2}P^{'}$ looks like this:
		\[
		  \inferrule* [left=Alp]{
		      (\nu y)P_{1}\equiv_{\alpha}(\nu z)Q_{1}
		    \\
		      \inferrule* [left=Opn]{
			  Q_{1}\xrightarrow{\overline{x}z}_{2}Q_{1}^{'}
			\\
			  z\neq x
		      }{
			(\nu z)Q_{1}\xrightarrow{\overline{x}z}_{2}Q_{1}^{'}
		      }
		  }{
		    (\nu y)P_{1}\xrightarrow{\alpha}_{2}Q_{1}^{'}
		  }
		\]
		from this we create the following proof tree of $P_{1}\{z/y\}\xrightarrow{\alpha}_{2}Q_{1}^{'}$:
		\[
		  \inferrule* [left=Alp]{
		      P_{1}\{z/y\}\equiv_{\alpha}Q_{1}
		    \\
		      Q_{1}\xrightarrow{\alpha}_{2}Q_{1}^{'}
		  }{
		    P_{1}\{z/y\}\xrightarrow{\alpha}_{2}Q_{1}^{'}
		  }		
		\]
		to which we can apply the inductive hypothesis and get that there exists a process $Q_{1}^{''}$ such that
		\[
		  P_{1}\{z/y\}\xrightarrow{\alpha}_{1}Q_{1}^{''}\;and\; Q_{1}^{''}\equiv_{\alpha}Q_{1}^{'}
		\]
		then we apply the rule $Opn$ and $OpnAlp$ to get
		\[
		  \inferrule* [left=OpnAlp]{
		      \inferrule* [left=Opn]{
			  P_{1}\{z/y\} \xrightarrow{\overline{x}z}_{1}Q_{1}^{''}
			\\
			  z\neq x
		      }{
			(\nu z)P_{1}\{z/y\} \xrightarrow{\overline{x}(z)}_{1}Q_{1}^{''}
		      }
		    \\
		      z\notin n(P)
		    \\
		      x\neq y\neq z
		  }{
		    (\nu y)P_{1} \xrightarrow{\overline{x}(z)}_{1}Q_{1}^{''}
		  }
		\]
	    \end{itemize}
	  \item[Cns]
	    Since there is no process $\alpha$ equivalent to an identifier except for the identifier itself, the last part of the derivation tree of $P\xrightarrow{\alpha}_{2}P^{'}$ looks like this:
 		\[
 		  \inferrule* [left=Alp]{
 		      A(\tilde{x}|\tilde{y})\{\tilde{w}/\tilde{x}\}\equiv_{\alpha}A(\tilde{x}|\tilde{y})\{\tilde{w}/\tilde{x}\}
 		    \\
 		      \inferrule* [left=Cns]{
 			  A(\tilde{x}|\tilde{y})\stackrel{def}{=}R
 			\\
 			  R\{\tilde{w}/\tilde{x}\}\xrightarrow{\alpha}_{2}P^{'}
 		      }{
 			A(\tilde{x}|\tilde{y})\{\tilde{w}/\tilde{x}\}\xrightarrow{\alpha}_{2}P^{'}
 		      }
 		  }{
 		    A(\tilde{x}|\tilde{y})\{\tilde{w}/\tilde{x}\}\xrightarrow{\alpha}_{2}P^{'}
 		  }
 		\]
	    here we can apply the inductive hypothesis on the conclusion of $S$ and get that there exists a process $P^{''}$ such that $A(\tilde{x}|\tilde{y})\{\tilde{w}/\tilde{x}\}\xrightarrow{\alpha}_{1}P^{''}$ and $P^{'}\equiv_{\alpha}P^{''}$
% 	    We have
% 	    \begin{center}
% 	      \begin{tabular}{ll}
% 		  $P\equiv_{\alpha}A(\tilde{w}|\tilde{y})$ &
% 		\\
% 		  $\Rightarrow P\{\tilde{x}/\tilde{w}\}\equiv_{\alpha}A(\tilde{x}|\tilde{y})$  & inversion lemma
% 		\\
% 		  $\Rightarrow P\{\tilde{x}/\tilde{w}\}\equiv_{\alpha}R$ & 
% 		\\
% 		  $\Rightarrow P\equiv_{\alpha}R\{\tilde{w}/\tilde{x}\}$ & 
% 	      \end{tabular}
% 	    \end{center}
% 	    now we create the following proof which is shorter then the inductive case and so suitable for inductive hypothesis application
% 	    \[
% 		  \inferrule* [left=Alp]{
% 		      P\equiv_{\alpha}R\{\tilde{w}/\tilde{x}\}
% 		    \\
% 		      R\{\tilde{w}/\tilde{x}\}\xrightarrow{\alpha}_{2}P^{'}
% 		  }{
% 		    P\xrightarrow{\alpha}_{2}P^{'}
% 		  }	      
% 	    \]
% 	    the inductive hypothesis tells us that there exists a process $P^{''}$ such that $P^{''}\equiv P^{'}$ and $P\xrightarrow{\alpha}_{1}P^{''}$
	\end{description}
    \end{description}
  \end{proof}
\end{theorem}



\begin{theorem}
  $P\xrightarrow{\alpha}_{1}P^{'}\; \Rightarrow\; P\xrightarrow{\alpha}_{2}P^{'}$
  \begin{proof}
    The proof can go by induction on the length of the derivation of a transaction, and then both the base case and the inductive case proceed by cases on the last rule used in the derivation. However it is not necessary to show all the details of the proof because the rules in $R_{2}$ are almost the same as the rules in $R_{1}$, the only difference is that in $R_{2}$ we have the rule $Alp$ instead of $ResAlp$ and $OpnAlp$. The rule $Alp$ can mimic the rule $ResAlp$ in the following way:
	\[
	  \inferrule *{
	      (\nu z)P\equiv_{\alpha} (\nu w)P\{w/z\}
	    \\
	      w\notin n(P)
	    \\
	      (\nu w)P\{w/z\}\;
		\xrightarrow{xz}\;
		  P^{'}
	  }{
	    (\nu z)P\; 
	      \xrightarrow{xz}\;
		P^{'}
	  }
	\]
	And the rule $Alp$ can mimic the rule $OpnAlp$ in the following way:
	\[
	  \inferrule *{
	      (\nu z)P\equiv_{\alpha} (\nu w)P\{w/z\}
	    \\
	      w\notin n(P)
	    \\
	      (\nu w)P\{w/z\}\;
		\xrightarrow{\overline{x}(w)}\;
		  P^{'}
	    \\
	      x\neq w\neq z
	  }{
	    (\nu z)P\; 
	      \xrightarrow{\overline{x}(w)}\;
		P^{'}
	  }
	\]
  \end{proof}
\end{theorem}


\begin{theorem}
  $P\xrightarrow{\alpha}_{2}P^{'}\; \Leftrightarrow\; \exists P^{''}: P^{'}\equiv P^{''}\; and\; P\xrightarrow{\alpha}_{3}P^{''}$
  \begin{proof}
    \begin{description}
      \item[$\Rightarrow$] 
	First we prove $P\xrightarrow{\alpha}_{2}P^{'}\; \Rightarrow\; \exists P^{''}: P^{'}\equiv P^{''}\; and\; P\xrightarrow{\alpha}_{3}P^{''}$. The proof is by induction on the length of the derivation of $P\xrightarrow{\alpha}_{2}P^{'}$, and then both the base case and the inductive case proceed by cases on the last rule used.
	\begin{description}
	  \item[base case]
	    in this case the rule used can be one of the following $Out, EInp, Tau$ which are also in $R_{3}$ so a derivation of $P\xrightarrow{\alpha}_{2}P^{'}$ is also a derivation of $P\xrightarrow{\alpha}_{3}P^{'}$
	  \item[inductive case]:
	    \begin{itemize}
	      \item 
		the last rule used can be one in $R_{2}\cap R_{3}=\{Res, Opn\}$ and so for example we have 
		\[
		  \inferrule* [left=Res]{
		      P\xrightarrow{\alpha}_{2} P^{'}
		    \\
		      z\notin n(\alpha)
		  }{
		    (\nu z) P\xrightarrow{\alpha}_{2}(\nu z) P^{'}
		  }
		\]
		we apply the inductive hypothesis on $P\xrightarrow{\alpha}_{2} P^{'}$ and get $\exists P^{''}$ such that $P^{'}\equiv P^{''}$ and $P\xrightarrow{\alpha}_{3} P^{''}$. The proof we want is:
		\[
		  \inferrule* [left=Res]{
		      P\xrightarrow{\alpha}_{3} P^{''}
		    \\
		      z\notin n(\alpha)
		  }{
		    (\nu z) P\xrightarrow{\alpha}_{3}(\nu z) P^{''}
		  }
		\]
		and $(\nu z) P^{''}\equiv (\nu z) P^{'}$
	      \item
		the last rule used can be one in $\{ParL, ParR, SumL, SumR, EComL, EComR\}$, in this case we can proceed as in the previous case and if necessary add an application of $Str$ thus exploiting the commutativity of sum or parallel composition. For example
		\[
		  \inferrule* [left=ParR]{
		      Q\xrightarrow{\alpha}_{2}Q^{'}
		    \\
		      bn(\alpha)\cap fn(Q)=\emptyset
		  }{
		      P|Q\xrightarrow{\alpha}_{2}P|Q^{'}
		  }
		\]	
		now we apply the inductive hypothesis to $Q\xrightarrow{\alpha}_{2}Q^{'}$ and get $Q\xrightarrow{\alpha}_{3}Q^{''}$ for a $Q^{''}$ such that $Q^{'}\equiv Q^{''}$. The proof we want is
		\[
		  \inferrule* [left=Str]{
		      P|Q\equiv Q|P
		    \\
		      \inferrule* [left=Par]{
			  Q\xrightarrow{\alpha}_{3}Q^{''}
			\\
			  bn(\alpha)\cap fn(Q)=\emptyset
		      }{
			  Q|P\xrightarrow{\alpha}_{3}Q^{''}|P
		      }
		    }{
		      P|Q\xrightarrow{\alpha}_{3}Q^{''}|P
		    }
		\]
		and $Q^{''}|P\equiv P|Q^{'}$
	      \item
		if the last rule used is $Cns$:
		\[
		    \inferrule* [left=Cns]{
			A(\tilde{x}|\tilde{z})\stackrel{def}{=}P
		      \\
			P\{\tilde{y}/\tilde{x}\}\xrightarrow{\alpha}_{2}P^{'}
		    }{
		      A(\tilde{y}|\tilde{z})\xrightarrow{\alpha}_{2}P^{'}
		    }
		\]
		we apply the inductive hypothesis on the premise and get $P\{\tilde{y}/\tilde{x}\}\xrightarrow{\alpha}_{3}P^{''}$ such that $P^{''}\equiv P^{'}$. Now the proof we want is
		\[
		    \inferrule* [left=Str]{
			A(\tilde{y}|\tilde{z})\equiv P\{\tilde{y}/\tilde{x}\}
		      \\
			P\{\tilde{y}/\tilde{x}\}\xrightarrow{\alpha}_{3}P^{''}
		    }{
		      A(\tilde{y}|\tilde{z})\xrightarrow{\alpha}_{3}P^{''}
		    }
		\]		
	      \item
		if the last rule is $Alp$, then we just notice that this rule is a particular case of $Str$
	      \item
		if the last rule is $ClsL$(the case for $ClsR$ is simmetric) then we have
		\[
		    \inferrule* [left=ClsL]{
			P\xrightarrow{\overline{x}(z)}_{2}P^{'}
		      \\
			Q\xrightarrow{xz}_{2}Q^{'}
		      \\
			z\notin fn(Q)
		    }{
		      P|Q\xrightarrow{\tau}_{2}(\nu z)(P^{'}|Q^{'})
		    }
		\]
		there is no easy way to mimic this rule with the rules in $R_{3}$. But if in the derivation tree we have an introduction of the bound output $\overline{x}(z)$ followed directly by an elimination of the same bound output such as:
		\[
		    \inferrule* [left=ClsL]{
			\inferrule* [left=Opn]{
			    P\xrightarrow{\overline{x}z}_{2}P^{'}
			  \\
			    z\neq x
			}{
			  (\nu z)P\xrightarrow{\overline{x}(z)}_{2}P^{'}
			}
		      \\
			Q\xrightarrow{xz}_{2}Q^{'}
		      \\
			z\notin fn(Q)
		    }{
		      ((\nu z)P)|Q\xrightarrow{\tau}_{2}(\nu z)(P^{'}|Q^{'})
		    }
		\]
		we can apply the inductive hypothesis and get that 
		\[
		  P\xrightarrow{\overline{x}z}_{3}P^{''}\; and\; Q\xrightarrow{xz}_{3}Q^{''}
		\]
		where $P^{'}\equiv P^{''}$ and $Q^{'}\equiv Q^{''}$, so we create the needed proof in the following way
		\[
		    \inferrule* [left=Str]{
			(\nu z)(P|Q)\equiv ((\nu z)P)|Q
		      \\
			\inferrule* [left=Res]{
			  \inferrule* [left=Com]{
			      P\xrightarrow{\overline{x}z}_{3}P^{''}
			    \\
			      Q\xrightarrow{xz}_{3}Q^{''}
			  }{
			    P|Q\xrightarrow{\tau}_{3}P^{''}|Q^{''}
			  }
			}{
			  (\nu z)(P|Q)\xrightarrow{\tau}_{3}(\nu z)(P^{''}|Q^{''})
			}
		    }{
		      ((\nu z)P)|Q\xrightarrow{\tau}_{3}(\nu z)(P^{''}|Q^{''})
		    }
		\]
		We can always take a derivation tree in $R_{2}$ and move downward each occurrence of $Opn$ until we find the appropriate occurrence of $ClsL$. In this process we might need to use the structural congruence, in particular the scope extension axioms. We can attempt to prove that in the following way:
		\[
		  P\xrightarrow{\overline{x}(z)}_{2}P^{'}\; \Rightarrow\; \exists R: (\nu z)R\equiv P
		\]
		and if $(\nu z)R\xrightarrow{\overline{x}(z)}_{2}P^{'}$ then there exists a derivation tree for this transition such that the last rule used is $Opn$
	    \end{itemize}	    
	\end{description}
      \item[$\Leftarrow$] 
	
	PRIMA DEVO DIMOSTRARE IL LEMMA DI INVERSIONE PER LA CONGRUENZA STRUTTURALE(SE E' VERO)

	Secondly we prove $P\xrightarrow{\alpha}_{3}P^{'}\; \Rightarrow\; \exists P^{''}: P^{'}\equiv P^{''}\; and\; P\xrightarrow{\alpha}_{2}P^{''}$. The proof is by induction on the length of the derivation of $P\xrightarrow{\alpha}_{3}P^{'}$, and then both the base case and the inductive case proceed by cases on the last rule used.
	\begin{description}
	  \item[base case]
	    in this case the rule used can be one of the following $Out, EInp, Tau$ which are also in $R_{2}$ so a derivation of $P\xrightarrow{\alpha}_{3}P^{'}$ is also a derivation of $P\xrightarrow{\alpha}_{2}P^{'}$
	  \item[inductive case]:
	    \begin{itemize}
	      \item 
		the last rule used can be one in $R_{2}\cap R_{3}=\{Res, Opn\}$, this goes like in the previous proof for the opposite direction with the transition numbers swapped.
	      \item
		the last rule used can be one of $Par$, $Sum$ or $ECom$, in this case we apply the inductive hypothesis to the premises and the apply the appropriate rule: $ParL$, $SumL$ or $EComL$. For example
		\[
		  \inferrule* [left=Par]{
		      P\xrightarrow{\alpha}_{3}P^{'}
		    \\
		      bn(\alpha)\cap fn(Q)=\emptyset
		  }{
		      P|Q\xrightarrow{\alpha}_{3}P^{'}|Q
		  }
		\]	
		now we apply the inductive hypothesis to $P\xrightarrow{\alpha}_{3}P^{'}$ and get $P\xrightarrow{\alpha}_{2}P^{''}$ for a $P^{''}$ such that $P^{'}\equiv P^{''}$. The proof we want is
		\[
		      \inferrule* [left=ParL]{
			  P\xrightarrow{\alpha}_{2}P^{''}
			\\
			  bn(\alpha)\cap fn(Q)=\emptyset
		      }{
			  P|Q\xrightarrow{\alpha}_{2}P|Q^{''}
		      }
		\]
		and $Q^{''}|P\equiv P|Q^{'}$
	      \item
		if the last rule is $Str$, then we have
		\[
		  \inferrule* [left=Str]{
		      P\equiv Q
		    \\
		      Q\xrightarrow{\alpha}_{3}P^{'}
		  }{
		    P\xrightarrow{\alpha}_{3}P^{'}
		  }
		\]
		we proceed by cases on the premise $Q\xrightarrow{\alpha}_{3}P^{'}$. In the cases of prefix we can just use the appropriate prefix rule of $R_{2}$ and get rid of the $Str$. In the other cases we can move upward the occurrence of $Str$, after that we have one or two smaller derivation trees that are suitable to application of the inductive hypothesis and finally we apply some appropriate rules in $R_{2}$.
		\begin{description}
		  \item[Out]
		    Since we are using the rule $Out$, $Q=\overline{x}y.Q_{1}$ for some $Q_{1}$. $Q\equiv P$ means for the inversion lemma for structural congruence that $P=\overline{x}y.P_{1}$ for some $P_{1}\equiv Q_{1}$. The last part of the derivation tree is 
		    \[
		      \inferrule* [left=Str]{
			  \overline{x}y.P_{1}\equiv \overline{x}y.Q_{1}
			\\
			  \inferrule* [left=Out]{
			  }{
			    \overline{x}y.Q_{1}\xrightarrow{\overline{x}y}_{3}Q_{1}
			  }
		      }{
			\overline{x}y.P_{1}\xrightarrow{\overline{x}y}_{3}Q_{1}
		      }
		    \]
		    So we get 
		    \[
		      \inferrule* [left=Out]{
		      }{
			\overline{x}y.P_{1}\xrightarrow{\overline{x}y}_{2}P_{1}
		      }
		    \]
		    where $P_{1}\equiv Q_{1}$
		  \item[Tau] this is very similar to the previous case
		  \item[EInp]
		    Since we are using the rule $EInp$, $Q=x(y).Q_{1}$ for some $Q_{1}$. From $Q\equiv P$ using the inversion lemma for structural congruence we can have two cases:
		    \begin{itemize}
		      \item 
			$P=x(y).P_{1}$ for some $P_{1}\equiv Q_{1}$. The last part of the derivation tree is 
			\[
			  \inferrule* [left=Str]{
			      x(y).P_{1}\equiv x(y).Q_{1}
			    \\
			      \inferrule* [left=EInp]{
			      }{
				x(y).Q_{1}\xrightarrow{xw}_{3}Q_{1}\{w/y\}
			      }
			  }{
			    x(y).P_{1}\xrightarrow{xw}_{3}Q_{1}\{w/y\}
			  }
			\]
			So we get 
			\[
			  \inferrule* [left=EInp]{
			  }{
			    x(y).P_{1}\xrightarrow{xw}_{2}P_{1}\{w/y\}
			  }
			\]
			where $P_{1}\equiv Q_{1}$ implies $P_{1}\{w/y\}\equiv Q_{1}\{w/y\}$
		      \item
			$P=x(z).P_{1}$ for some $P_{1}\equiv Q_{1}\{z/y\}$. The last part of the derivation tree is 
			\[
			  \inferrule* [left=Str]{
			      x(z).P_{1}\equiv x(y).Q_{1}
			    \\
			      \inferrule* [left=EInp]{
			      }{
				x(y).Q_{1}\xrightarrow{xw}_{3}Q_{1}\{w/y\}
			      }
			  }{
			    x(z).P_{1}\xrightarrow{xw}_{3}Q_{1}\{w/y\}
			  }
			\]
			So we get 
			\[
			  \inferrule* [left=EInp]{
			  }{
			    x(z).P_{1}\xrightarrow{xw}_{2}P_{1}\{w/z\}
			  }
			\]
			where $P_{1}\equiv Q_{1}\{z/y\}$ implies $P_{1}\{w/z\}\equiv Q_{1}\{z/y\}\{w/z\}\equiv Q_{1}\{w/y\}$
		    \end{itemize}
		  \item[Par]
		    Since we are using the rule $Par$, $Q=Q_{1}|Q_{2}$ for some $Q_{1},Q_{2}$. $Q\equiv P$ means for the inversion lemma for structural congruence that $P=P_{1}|P_{2}$ for some $P_{1},P_{2}$ such that $P_{1}\equiv Q_{1}$ and $P_{2}\equiv Q_{2}$. The last part of the derivation tree is 
		    \[
		      \inferrule* [left=Str]{
			  P_{1}|P_{2}\equiv Q_{1}|Q_{2}
			\\
			  \inferrule* [left=Par]{
			      Q_{1}\xrightarrow{\alpha}_{3}Q_{1}^{'}
			    \\
			      bn(\alpha)\cap fn(Q_{2})=\emptyset
			  }{
			    Q_{1}|Q_{2}\xrightarrow{\alpha}_{3}Q_{1}^{'}|Q_{2}
			  }
		      }{
			P_{1}|P_{2}\xrightarrow{\alpha}_{3}Q_{1}^{'}|Q_{2}
		      }
		    \]
		    the first step is the creation of this proof tree:
		    \[
			  \inferrule* [left=Str]{
			      P_{1}\equiv Q_{1}
			    \\
			      Q_{1}\xrightarrow{\alpha}_{3}Q_{1}^{'}
			  }{
			    P_{1}\xrightarrow{\alpha}_{3}Q_{1}^{'}
			  }
		    \]
		    which is smaller then the inductive case, so we apply the inductive hypothesis and get $P_{1}\xrightarrow{\alpha}_{2}Q_{1}^{''}$ where $Q_{1}^{'}\equiv Q_{1}^{''}$. The last step is 
		    \[
		      \inferrule* [left=ParL]{
			  P_{1}\xrightarrow{\alpha}_{2}Q_{1}^{''}
			\\
			  bn(\alpha)\cap fn(P_{2})=\emptyset
		      }{
			P_{1}|P_{2}\xrightarrow{\alpha}_{2}Q_{1}^{''}|P_{2}
		      }
		    \]
		  \item[Sum] this case is very similar to the previous.
		  \item[ECom] this case is also similar to the $Par$ case.
		  \item[Res]
		    Since we are using the rule $Res$, $Q=(\nu z)Q_{1}$ for some $Q_{1}$ and some $z$. $(\nu z)Q_{1}\equiv P$ means thanks to the inversion lemma for structural congruence that one of the following cases holds:
                    \begin{itemize}
		      \item
			$P=(\nu z)P_{1}$ for some $P_{1}$ such that $P_{1}\equiv Q_{1}$. The last part of the derivation tree is 
			\[
			  \inferrule* [left=Str]{
			      (\nu z)P_{1}\equiv (\nu z)Q_{1}
			    \\
			      \inferrule* [left=Res]{
				  Q_{1}\xrightarrow{\alpha}_{3}Q_{1}^{'}
				\\
				  z\notin n(\alpha)
			      }{
				(\nu z)Q_{1}\xrightarrow{\alpha}_{3}(\nu z)Q_{1}^{'}
			      }
			  }{
			    (\nu z)P_{1}\xrightarrow{\alpha}_{3}(\nu z)Q_{1}^{'}
			  }
			\]
			first we create the following proof:
			\[
			      \inferrule* [left=Str]{
				  P_{1}\equiv Q_{1}
				\\
				  Q_{1}
				    \xrightarrow{\alpha}_{3}
				      Q_{1}^{'}
			      }{
				P_{1}\xrightarrow{\alpha}_{3}Q_{1}^{'}
			      }
			\]
			now we can apply the inductive hypothesis and get $P_{1}\xrightarrow{\alpha}_{2}Q_{1}^{''}$ where $Q_{1}^{'}\equiv Q_{1}^{''}$. The last step is 
			\[
			  \inferrule* [left=Res]{
			      P_{1}\xrightarrow{\alpha}_{2}Q_{1}^{''}
			    \\
			      z\notin n(\alpha)
			  }{
			    (\nu z)P_{1}\xrightarrow{\alpha}_{2}(\nu z)Q_{1}^{''}
			  }
			\]
		      \item
			$P=(\nu y)P_{1}$ for some $P_{1}$ such that $P_{1}\{z/y\}\equiv Q_{1}$. The last part of the derivation tree is 
			\[
			  \inferrule* [left=Str]{
			      (\nu y)P_{1}\equiv (\nu z)Q_{1}
			    \\
			      \inferrule* [left=Res]{
				  Q_{1}\xrightarrow{\alpha}_{3}Q_{1}^{'}
				\\
				  z\notin n(\alpha)
			      }{
				(\nu z)Q_{1}\xrightarrow{\alpha}_{3}(\nu z)Q_{1}^{'}
			      }
			  }{
			    (\nu y)P_{1}\xrightarrow{\alpha}_{3}(\nu z)Q_{1}^{'}
			  }
			\]
			we create the following proof of $P_{1}\{z/y\}\xrightarrow{\alpha}_{3}Q_{1}^{'}$:
			\[
			  \inferrule* [left=Str]{
			      P_{1}\{z/y\}\equiv Q_{1}
			    \\
			      Q_{1}\xrightarrow{\alpha}_{3}Q_{1}^{'}
			  }{
			    P_{1}\{z/y\}\xrightarrow{\alpha}_{3}Q_{1}^{'}
			  }
			\]
			this proof tree is shorter then the one of $(\nu y)P_{1}\xrightarrow{\alpha}_{3}(\nu z)Q_{1}^{'}$ so we can apply the inductive hypothesis and get that there exists a process $Q_{1}^{''}$ such that 
			\[
			  P_{1}\{z/y\}\xrightarrow{\alpha}_{2}Q_{1}^{''}\; and\; Q_{1}^{''}\equiv Q_{1}^{'}
			\]
			now we can apply the rules $Res$ and $Alp$ to get the desired proof tree:
			\[
			  \inferrule* [left=Alp]{
				(\nu z)P_{1}\{z/y\}\equiv_{\alpha} (\nu y)P_{1}
			      \\
				\inferrule* [left=Res]{
				    P_{1}\{z/y\}
				      \xrightarrow{\alpha}_{2}
					Q_{1}^{''}
				  \\
				    z\notin(\alpha)
				  }{
				    (\nu z)P_{1}\{z/y\}
				      \xrightarrow{\alpha}_{2}
					(\nu z)Q_{1}^{''}
				  }
			  }{
			    (\nu y)P_{1}\xrightarrow{\alpha}_{2}(\nu z)Q_{1}^{''}
			  }
			\]
                      \end{itemize}
		  \item[Opn]
		    Since we are using the rule $Opn$, $Q=(\nu z) Q_{1}$ for some $Q_{1}$. $(\nu z) Q_{1}\equiv P$ means for the inversion lemma for structural congruence that
		    \begin{itemize}
		      \item
			$P=(\nu z) P_{1}$ for some $P_{1}$ such that $P_{1}\equiv Q_{1}$. The last part of the derivation tree is 
			\[
			  \inferrule* [left=Str]{
			      (\nu z) P_{1}\equiv (\nu z) Q_{1}
			    \\
			      \inferrule* [left=Opn]{
				  Q_{1} \xrightarrow{\overline{x}z} Q_{1}^{'}
				\\
				  z\neq x
			      }{
				(\nu z) Q_{1} \xrightarrow{\overline{x}(z)} Q_{1}^{'}
			      }
			  }{
			    (\nu z) P_{1} \xrightarrow{\overline{x}(z)} Q_{1}^{'}
			  }
			\]
			first:
			\[
			      \inferrule* [left=Str]{
				  P_{1}\equiv Q_{1}
				\\
				  Q_{1}\xrightarrow{\overline{x}z}_{3}Q_{1}^{'}
			      }{
				P_{1}\xrightarrow{\overline{x}z}_{3}Q_{1}^{'}
			      }
			\]
			then we apply the inductive hypothesis and get $P_{1}\xrightarrow{\overline{x}z}_{2}Q_{1}^{''}$ where $Q_{1}^{'}\equiv Q_{1}^{''}$. The last step is 
			\[
			  \inferrule* [left=Res]{
			      P_{1}\xrightarrow{\overline{x}z}_{2}Q_{1}^{''}
			    \\
			      z\neq x
			  }{
			    (\nu z)P_{1}\xrightarrow{\overline{x}z}_{2}Q_{1}^{''}
			  }
			\]
		      \item
			$P=(\nu z) P_{1}$ for some $P_{1}$ such that $P_{1}\equiv Q_{1}$. The last part of the derivation tree is 
			\[
			  \inferrule* [left=Str]{
			      (\nu z) P_{1}\equiv (\nu z) Q_{1}
			    \\
			      \inferrule* [left=Opn]{
				  Q_{1} \xrightarrow{\overline{x}z} Q_{1}^{'}
				\\
				  z\neq x
			      }{
				(\nu z) Q_{1} \xrightarrow{\overline{x}(z)} Q_{1}^{'}
			      }
			  }{
			    (\nu z) P_{1} \xrightarrow{\overline{x}(z)} Q_{1}^{'}
			  }
			\]
			the first step is:
			\[
			      \inferrule* [left=Str]{
				  P_{1}\equiv Q_{1}
				\\
				  Q_{1}\xrightarrow{\overline{x}z}_{3}Q_{1}^{'}
			      }{
				P_{1}\xrightarrow{\overline{x}z}_{3}Q_{1}^{'}
			      }
			\]
			then we apply the inductive hypothesis and get $P_{1}\xrightarrow{\overline{x}z}_{2}Q_{1}^{''}$ where $Q_{1}^{'}\equiv Q_{1}^{''}$. The last step is 
			\[
			  \inferrule* [left=Res]{
			      P_{1}\xrightarrow{\overline{x}z}_{2}Q_{1}^{''}
			    \\
			      z\neq x
			  }{
			    (\nu z)P_{1}\xrightarrow{\overline{x}z}_{2}Q_{1}^{''}
			  }
			\]
		      \item
			$P=(\nu y) P_{1}$ for some $P_{1}$ such that $P_{1}\{z/y\}\equiv Q_{1}$. The last part of the derivation tree is 
			\[
			  \inferrule* [left=Str]{
			      (\nu y) P_{1}\equiv (\nu z) Q_{1}
			    \\
			      \inferrule* [left=Opn]{
				  Q_{1} \xrightarrow{\overline{x}z}_{3} Q_{1}^{'}
				\\
				  z\neq x
			      }{
				(\nu z) Q_{1} \xrightarrow{\overline{x}(z)}_{3} Q_{1}^{'}
			      }
			  }{
			    (\nu y) P_{1} \xrightarrow{\overline{x}(z)}_{3} Q_{1}^{'}
			  }
			\]
			we can create the following proof of $P_{1}\{z/y\} \xrightarrow{\overline{x}z}_{3} Q_{1}^{'}$:
			\[
			  \inferrule* [left=Str]{
			      P_{1}\{z/y\}\equiv Q_{1}
			    \\
			      Q_{1} 
				\xrightarrow{\overline{x}z}_{3} 
				  Q_{1}^{'}
			  }{
			    P_{1}\{z/y\} 
			      \xrightarrow{\overline{x}z}_{3} 
				Q_{1}^{'}
			  }
			\]
			this proof tree is shorter then the one of $(\nu y) P_{1} \xrightarrow{\overline{x}(z)}_{3} Q_{1}^{'}$ so we can apply the inductive hypothesis and get that there exists a process $Q_{1}^{''}$ such that
			\[
			  Q_{1}^{''}\equiv Q_{1}^{'}\; and\;
			    P_{1}\{z/y\} 
			      \xrightarrow{\overline{x}z}_{2} 
				Q_{1}^{''} 
			\]
			so now we only need to apply the rules $Opn$ and $Alp$:
			\[
			  \inferrule* [left=Alp]{
			      (\nu y) P_{1}\equiv_{\alpha} (\nu z) P_{1}\{z/y\}
			    \\
			      \inferrule* [left=Opn]{
				  P_{1}\{z/y\} \xrightarrow{\overline{x}z}_{2} Q_{1}^{''}
				\\
				  z\neq x
			      }{
				(\nu z) P_{1}\{z/y\} \xrightarrow{\overline{x}(z)}_{3} Q_{1}^{''}
			      }
			  }{
			    (\nu y) P_{1} \xrightarrow{\overline{x}(z)}_{2} Q_{1}^{''}
			  }
			\]
		  \end{itemize}
		\end{description}
	    \end{itemize}	    
	\end{description}
    \end{description}
  \end{proof}
\end{theorem}



\subsection{Equivalence of the late semantics}



\section{Bisimilarity, congruence and equivalence}

We present here some behavioural equivalences and some of their properties. In the following we will use the phrase $bn(\alpha)$ is fresh in a definition to mean that the name in $bn(\alpha)$, if any, is different from any free name occurring in any of the agents in the definition. We write $\rightarrow_{E}$ for the early semantic and $\rightarrow_{L}$ for the late semantic. It's not a concern which late semantic we are talking about because we have proved them equivalent.


\subsection{Late bisimilarity}

\begin{definition}
  A \emph{strong late bisimulation}(according to \cite{parrow}) is a binary simmetric relation $\mathbf{S}$ on processes such that for each process $P$ and $Q$, $P\mathbf{S}Q$ implies:
  \begin{itemize}
    \item
      if $P \xrightarrow{a(x)}_{L} P^{'}$ and $x\notin fn(P)\cup fn(Q)$ then there exists a process $Q^{'}$ such that $Q \xrightarrow{a(x)}_{L} Q^{'}$ and for all $u$ $P^{'}\{u/x\}\mathbf{S}Q^{'}\{u/x\}$
    \item 
      if $P \xrightarrow{\alpha}_{L} P^{'}$, $\alpha$ is not an input and $bn(\alpha) \cap (fn(P)\cup fn(Q)) = \emptyset$ then there exists a process $Q^{'}$ such that $Q \xrightarrow{\alpha}_{L} Q^{'}$ and $P^{'}\mathbf{S}Q^{'}$
  \end{itemize}
  $P$ and $Q$ are \emph{late bisimilar} written $P\dot{\sim}_{L}Q$ if there exists a strong late bisimulation $\mathbf{S}$ such that $P\mathbf{S}Q$.
\end{definition}

\begin{example}
  Strong late bisimulation is not closed under substitution in general:
  \[
    a(u).0|\overline{b}v.0\; \dot{\sim}_{L}\; a(u).\overline{b}v.0 + \overline{b}v.a(u).0
  \]
  and the bisimulation(without the simmetric part) is  the following:
  \[
    \{(a(u).0|\overline{b}v.0, a(u).\overline{b}v.0 + \overline{b}v.a(u).0),\;\; (a(u).0|0,a(u).0),\;\; (0|0,0),\;\; (0|\overline{b}v.0,\overline{b}v.0)\} 
  \]
  If we apply the substitution $\{a/b\}$ to each process then they are not strongly bisimilar anymore because $(a(u).0|\overline{b}v.0)\{a/b\}$ is $a(u).0|\overline{a}v.0$ and this process can perform an invisible action whether $(a(u).\overline{b}v.0 + \overline{b}v.a(u).0)\{a/b\}$ cannot.
\end{example}

We refer to strong late bisimulation as strong \emph{ground} late bisimulation, because it is not preserved by substitution.

\begin{proposition}
  If $P \dot{\sim} Q$ and $\sigma$ is injective then $P\sigma \dot{\sim} Q\sigma$
%   \begin{proof}
%     We have to establish that transitions are preserved by injective substitution, i.e., that for injective $\sigma$ 
%     \[
%       P\xrightarrow{\alpha}P^{'}\; \Rightarrow\; P\sigma\xrightarrow{\alpha\sigma}P^{'}\sigma
%     \]
%     by induction on the inference of $P\xrightarrow{\alpha}P^{'}$. For this purpose we consider the late semantic with structural congruence:
%     %The last rule of a derivation of $P\xrightarrow{\alpha}P^{'}$ can be:
%     \begin{description}
%       \item[$Pref$] 		
% 	An application of the rule pref let us prove that $\alpha\sigma.P\sigma\xrightarrow{\alpha\sigma}P\sigma$
%       \item[$Str$]
% 	The premises of the rule tell us that: $P\xrightarrow{\alpha}Q$, $P^{'}\equiv P$ and $Q\equiv Q^{'}$. We apply the inductive hypothesis to the first premise and get $P\sigma\xrightarrow{\alpha\sigma}Q\sigma$. From the other two premises we get that $P^{'}\sigma\equiv P\sigma$ and $Q\sigma\equiv Q^{'}\sigma$. We should prove this separately. So applying the rule $Str$ we get the result $P^{'}\sigma\xrightarrow{\alpha\sigma}Q^{'}\sigma$.
%       \item[$Sum$]
% 	
%       \item[$Par$]
%       \item[$Com$]
%       \item[$Res$]
%       \item[$Opn$]
%     \end{description}
%   \end{proof}
\end{proposition}

\begin{proposition}
  $\dot{\sim}_{L}$ is an equivalence
\end{proposition}

\begin{proposition}
  $\dot{\sim}_{L}$ is preserved by all operators except input prefix
\end{proposition}

\begin{definition}
  Two processes $P$ and $Q$ are \emph{strong late equivalent} written $P\sim_{L}Q$ is for each substitution $\sigma$ $P\sigma \dot{\sim}_{L}Q\sigma$
\end{definition}

\begin{example}
  If $z\notin fn(R)\cup \{x\}$ then $x(y).R \dot{\sim}_{L} (z)x(y).R$
\end{example}



\subsection{Early bisimilarity}

\begin{definition}\index{bisimulation! strong! early! with early semantic}
  A \emph{strong early bisimulation}(according to \cite{parrow})is a symmetric binary relation $\mathbf{S}$ on processes such that for each process $P$ and $Q$: $P\mathbf{S} Q$, $P \xrightarrow{\alpha}_{E} P^{'}$ and $bn(\alpha) \cap (fn(P) \cup fn(Q))=\emptyset$ implies that there exists $Q^{'}$ such that $Q \xrightarrow{\alpha}_{E} Q^{'}$ and $P^{'}\mathbf{S}Q^{'}$. $P$ and $Q$ are \emph{early bisimilar} written $P\dot{\sim}_{E}Q$ if there exists a strong early bisimulation $\mathbf{S}$ such that $P\mathbf{S}Q$
\end{definition}

\begin{definition}
  Two processes $P$ and $Q$ are \emph{strong early equivalent} written $P\sim_{E}Q$ if for each substitution $\sigma$ $P\sigma \dot{\sim}_{E}Q\sigma$
\end{definition}


% \begin{proposition}
%   Early bisimilarity is preserved by all operators except input prefix. 
% \end{proposition}
% 
% \begin{proposition}
%   The early congruence is the largest congruence in $\dot{\sim}_{E}$.  
% \end{proposition}

% In the following definition we consider a subcalculus without restriction. 
% \begin{definition}\index{bisimulation! strong! open! early}
%   A \emph{strong open bisimulation} is a symmetric binary relation $\mathbb{R}$ on agents satisfying the following for all substitutions $\sigma$: $P\mathbb{R} Q$ and $P\sigma\; \xrightarrow{\alpha}_{E}\; P^{'}$ where $bn(\alpha)$ is fresh implies that
%   \begin{center}
%     $\exists Q^{'}:\;\; Q\sigma\xrightarrow{\alpha}_{E}Q^{'}\; \wedge\; P^{'}\mathbb{R}Q^{'}$
%   \end{center}
%   $P$ and $Q$ are strongly open bisimilar, written $P\; \dot{\sim}_{O}\; Q$ if they are related by an open bisimulation.
% \end{definition}
% 
% \begin{proposition}
%   strong open bisimulation is also a late bisimulation, is closed under substitution, is an equivalence and a congruence
% \end{proposition}


\subsection{Congruence}

\begin{definition}\index{congruence! strong}
  We say that two agents $P$ and $Q$ are \emph{strongly congruent}, written $P\sim Q$ if
  \begin{center}
    $P\sigma \dot{\sim} Q\sigma$ for all substitution $\sigma$    
  \end{center}
\end{definition}

\begin{proposition}
  Strong congruence is the largest congruence in bisimilarity.
\end{proposition}


\subsection{Open bisimilarity}

\begin{definition}
  A \emph{distinction} is a finite symmetric and irreflexive binary relation on names. A substitution $\sigma$ \emph{respects} a distinction $D$ if for each name $a,b$ $aDb$ implies $\sigma(a)\neq \sigma(b)$. We write $D\sigma$ for the composition of the two relation.
\end{definition}


\begin{definition}
  An \emph{strong open simulation}(according to \cite{parrow}) is $\{S_{D}\}_{D\in \mathbb{D}}$ a family of binary relations on processes such that for each process $P, Q$, for each distinction $D\in \mathbb{D}$, for each name substitution $\sigma$ which respects $D$ if $P S_{D} Q$, $P\sigma \xrightarrow{\alpha} P^{'}$ and $bn(\alpha)\cap (fn(P\sigma)\cup fn(Q\sigma))=\emptyset$ then:
   \begin{itemize}
    \item 
      if $\alpha=\overline{a}(x)$ then there exists $Q^{'}$ such that $Q\sigma \xrightarrow{\overline{a}(x)} Q^{'}$ and $P^{'} S_{D^{'}} Q^{'}$ where $D^{'}=D\sigma \cup \{x\}\times (fn(P\sigma)\cup fn(Q\sigma)) \cup  (fn(P\sigma)\cup fn(Q\sigma))\times\{x\}$
    \item
      if $\alpha$ is not a bound output then there exists $Q^{'}$ such that $Q\sigma \xrightarrow{\alpha} Q^{'}$ and $P^{'} S_{D\sigma} Q^{'}$
  \end{itemize}
  $P$ and $Q$ are \emph{open D bisimilar}, written $P \dot{\sim}_{O}^{D} Q$ if there exists a member $S_{D}$ of an open bisimulation such that $P S_{D} Q$; they are \emph{open bisimilar} if they are open $\emptyset$ bisimilar, written $P \dot{\sim}_{O} D$.
 

\end{definition}
 
 















