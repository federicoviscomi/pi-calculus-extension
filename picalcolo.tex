


The $\pi$ calculus is a mathematical model of processes whose interconnections change as they interact. The basic computational step is the transfer of a communication link between two processes. The idea that the names of the links belong to the same category as the transferred objects is one of the cornerstone of the calculus. The $\pi$ calculus allows channel names to be communicated along the channels themselves, and in this way it is able to describe concurrent computations whose network configuration may change during the computation.

A coverage of $\pi$ calculus is on \cite{parrow}, \cite{sangiorgiwalker} and \cite{milner}

\section{Syntax}

We suppose that we have a countable set of names $\mathbb{N}$, ranged over by lower case letters $a,b, \cdots, z$. This names are used for communication channels and values. Furthermore we have a set of identifiers, ranged over by $A$. We represent the agents or processes by upper case letters $P,Q, \cdots $. A process can perform the following actions:
\begin{center}
  $\pi$ ::= $\overline{x}y$ | $x(z)$ | $\tau$ 
\end{center}
The process are defined by the following grammar:
\begin{center}
  \begin{tabular}{l}
    $P,Q$ ::= $0$ | $\pi.P$ | $P|Q$ | $P+Q$ | $(\nu x) P$ | $A(\tilde{x})$ 
  \end{tabular}
\end{center}
and they have the following intuitive meaning:
\begin{description}
  \item[$0$] 
    is the empty process which cannot perform any actions
  \item[$\pi.P$] 
    is an action prefixing, this process can perform action $\pi$ e then behave like $P$, the action can be:
    \begin{description}
      \item[$\overline{x}y$] 
	is an output action, this sends the name $y$ along the name $x$. We can think about $x$ as a channel or a port, and about $y$ as an output datum sent over the channel
      \item[$x(z)$] 
	is an input action, this receives a name along the name $x$. $z$ is a variable which stores the received data.
      \item[$\tau$] 
	is a silent or invisible action, this means that a process can evolve to $P$ without interaction with the environment 
    \end{description}
    for any action which is not a $\tau$, the first name that appears in the action is called subject of the action and the second name is called object of the action.
  \item[$P+Q$] 
    is the sum, this process can enact either $P$ or $Q$
  \item[$P|Q$] 
    is the parallel composition, $P$ and $Q$ can execute concurrently and also synchronize with each other
  \item[$(\nu z) P$] 
    is the scope restriction. This process behave as $P$ but the name $z$ is local. This process cannot use the name $z$ to interact with other processes.
  \item[$A(\tilde{x})$] 
    is an identifier. Every identifier has a definition
    \begin{center}
      $A(x_{1}, \cdots, x_{n})=P$
    \end{center}
    the $x_{i}$s must be pairwise different. The intuition is that we can substitute for some of the $x_{i}$s in $P$ to get a $\pi$ calculus process. We can write $\tilde{x}$ for $x_{1}, \cdots, x_{n}$.
\end{description}

To resolve ambiguity we can use parenthesis and observe the conventions that prefixing and restriction bind more tightly than composition and prefixing binds more tightly than sum. 

\begin{definition}
  We say that the input prefix $x(z).P$ \emph{binds} $z$ in $P$ or is a \emph{binder} for $z$ in $P$. We also say that $P$ is the \emph{scope} of the binder and that any occurrence of $z$ in $P$ are \emph{bound} by the binder. Also the restriction operator $(\nu z)P$ is a binder for $z$ in $P$. 
\end{definition}


\begin{definition}
  $bn(P)$ is the set of names that have a bound occurrence in $P$ and is defined as $B(P, \emptyset)$, where $B(P, I)$, with $I$ a set of identifiers, is defined in table \ref{table:B}
\end{definition}

  \begin{table}
    \begin{tabular}{ll}
      \multicolumn{2}{l}{\line(1,0){415}}\\
	$B(0, I)\; =\; \emptyset$&$B(Q+R,I)\; =\; B(Q,I)\cup B(R,I)$
      \\\\
	$B(\overline{x}y.Q, I)\; =\; B(Q, I)$&$B(Q|R,I)\; =\; B(Q,I)\cup B(R,I)$
      \\\\
	$B(x(y).Q, I)\; =\; \{y,\overline{y}\}\cup B(Q, I)$&$B((\nu x)Q, I)\; =\; \{x, \overline{x}\}\cup B(Q, I)$
      \\\\
	$B(\tau.Q, I)\; =\; B(Q, I)$&
      \\\\
	\multicolumn{2}{l}{
	$B(A, I)=\left\{
	  \begin{array}{ll}
		B(Q, I\cup \{A\})\; 
		where\; A(\tilde{x})\stackrel{def}{=}Q
	      &
		if\; A\notin I
	    \\
		\emptyset
	      &
		if\; A\in I
	  \end{array}\right.$}
      \\\multicolumn{2}{l}{\line(1,0){415}}
    \end{tabular}
    \caption{Bound occurrences}
    \label{table:B}
  \end{table}



\begin{definition}
  We say that a name $x$ is \emph{free} in $P$ if $P$ contains a non bound occurrence of $x$. We write $fn(P)$ for the set of names with a free occurrence in $P$. $fn(P)$ is defined in table \ref{F}
\end{definition}

  \begin{table}
    \begin{tabular}{lll}
      \multicolumn{3}{l}{\line(1,0){415}}\\
	  $fn(\overline{x}y.Q)\; =\; \{x,\overline{x},y,\overline{y}\}\cup fn(Q)$
	&
	  $fn(Q+R)\; =\; fn(Q)\cup fn(R)$
	&
	  $fn(0)\; =\; \emptyset$
      \\\\
	  $fn(x(y).Q)\; =\; \{x,\overline{x}\}\cup (fn(Q)-\{y,\overline{y}\})$
	&
	  $fn(Q|R)\; =\; fn(Q)\cup fn(R)$
      \\\\
	  $fn((\nu x)Q)\; =\; fn(Q)-\{x,\overline{x}\}$	  
	&
	  $fn(\tau.Q)\; =\; fn(Q)$
	&
	  $\inferrule{A(\tilde{x})\stackrel{def}{=}P}{fn(A)=\{\tilde{x}\}}$
      \\\multicolumn{3}{l}{\line(1,0){415}}
    \end{tabular}
    \caption{Free occurrences}
    \label{F}
  \end{table}



\begin{definition}
  $n(P)$ which is the set of all names in $P$ and is defined in the following way:
  \begin{center}
    $n(P)\; =\; fn(P)\cup bn(P)$
  \end{center}
\end{definition}


\begin{definition}
  We say that $\tau$ and actions which does not have any binder, such as $xy, \overline{x}y$, are \emph{free} actions. Whether the other actions are \emph{bound} actions.
\end{definition}


In a definition $A(\tilde{x})=P$ the $\tilde{x}$ are exactly the free names contained in $P$, specifically $fn(P) = \{\tilde{x}\}$. If we look at the definitions of $bn$ and of $fn$ we notice that if $P$ contains another identifier whose definition is: $B(\tilde{z})=Q$ then we have $fn(Q)\subseteq\{\tilde{x}\}$.


\begin{definition}
  $P\{b/a\}$ is the syntactic substitution of name $b$ for a different name $a$ inside a $\pi$ calculus process and it consists in replacing every free occurrences of $a$ with $b$. If $b$ is a bound name in $P$, in order to avoid name capture we perform an appropriate $\alpha$ conversion. $P\{b/a\}$ is defined in table \ref{syntacticsubstitution}. We use the notation $\{\tilde{x}/\tilde{y}\}$ as a short for $\{x_{1}/y_{1}, \cdots, x_{n}/y_{n}\}$ which is not the composition of the substitutions $\{x_{1}/y_{1}\} \circ \ldots \circ \{x_{n}/y_{n}\}$ 
  \begin{table}
    \begin{tabular}{lll}
      \multicolumn{3}{l}{\line(1,0){415}}\\
	$0\{b/a\}\; =\; 0$
      &
	$(\overline{x}y.Q)\{b/a\}\; =\; \overline{x}\{b/a\}y\{b/a\}.Q\{b/a\}$
      &
	$(\tau.Q)\{b/a\}\; =\; \tau.Q\{b/a\}$
      \\
    \end{tabular}
      \\
    \begin{tabular}{ll}
      \\
	$\inferrule{y\neq a \\ y\neq b}{(x(y).Q)\{b/a\}\; =\; x\{b/a\}(y).Q\{b/a\}}$
      &
	$\inferrule{
	  c\notin n(x(b).Q)
	}{
	  (x(b).Q)\{b/a\}\; =\; x\{b/a\}(c).((Q\{c/b\})\{b/a\})
	}$
      \\
    \end{tabular}
      \\
    \begin{tabular}{l}
      \\
	$(x(a).Q)\{b/a\}\; =\; x\{b/a\}(a).Q$
      \\\\
	$\inferrule{
	    a \in \tilde{x}
	  \\
	    A(\tilde{x}) \stackrel{def}{=} P
	}{
	    A(\tilde{x})\{b/a\} = A(\tilde{x}\{b/a\})
	}$
      \\
    \end{tabular}
      \\
    \begin{tabular}{ll}
      \\
	$(Q+R)\{b/a\}\; =\; Q\{b/a\} + R\{b/a\}$
      &
	$(Q|R)\{b/a\}\; =\; Q\{b/a\} | R\{b/a\}$
      \\
    \end{tabular}
      \\
    \begin{tabular}{ll}
      \\
	$\inferrule{y\neq a \\ y\neq b}{((\nu y)Q)\{b/a\}\; =\;(\nu y)Q\{b/a\}}$ 
      &
	$((\nu a)Q)\{b/a\}\; =\;(\nu a)Q$
      \\\\
	$\inferrule{c\notin n((\nu b)Q) \\ a\in fn(Q)}{((\nu b)Q)\{b/a\}\; =\;(\nu c)((Q\{c/b\})\{b/a\})}$ 
      &
      \\\multicolumn{2}{l}{\line(1,0){415}}
    \end{tabular}
    \caption{Syntatic substitution}
    \label{syntacticsubstitution}
  \end{table}

\end{definition}

\section{Operational Semantic(without structural congruence)}
\subsection{Early operational semantic(without structural congruence)}
The semantic of a $\pi$ calculus process is a labeled transition system such that:
\begin{itemize}
  \item 
    the nodes are $\pi$ calculus process. The set of node is $\mathbb{P}$
  \item
    the actions can be:
    \begin{itemize}
      \item unbound input $xy$
      \item unbound output $\overline{x}y$
      \item the silent action $\tau$
      \item bound output $\overline{x}(y)$
    \end{itemize}
    The set of actions is $\mathbb{A}$, we use $\alpha$ to range over the set of actions.
  \item
    the transition relations is $\rightarrow\subseteq \mathbb{P}\times \mathbb{A}\times \mathbb{P}$
\end{itemize}

\begin{definition}
  The \emph{early transition relation} $\rightarrow\subseteq \mathbb{P}\times \mathbb{A} \times \mathbb{P}$ is the smallest relation induced by the rules in table \ref{transitionrelationearlywithoutstructuralcongruence}. 
  \begin{table}
    \begin{tabular}{lll}  
      \multicolumn{3}{l}{\line(1,0){415}}\\\\
	  $\inferrule* [left=\bf{Out}]{
	  }{
	    \overline{x}y.P \xrightarrow{\overline{x}y} P
	  }$
	&
	  $\inferrule* [left=\bf{EInp}]{
	  }{
	    x(y).P \xrightarrow{xz} P\{z/y\}
	  }$
	&
	  $\inferrule* [left=\bf{Tau}]{
	  }{
	    \tau.P \xrightarrow{\tau} P
	  }$
      \\
    \end{tabular}
    \\
    \begin{tabular}{ll}  
      \\
	  $\inferrule* [left=\bf{SumL}]{
	    P \xrightarrow{\alpha} P^{'}
	  }{
	    P+Q \xrightarrow{\alpha} P^{'}
	  }$
	&
	  $\inferrule* [left=\bf{SumR}]{
	    Q \xrightarrow{\alpha} Q^{'}
	  }{
	    P+Q \xrightarrow{\alpha} P^{'}
	  }$
      \\
    \end{tabular}
    \\
    \begin{tabular}{ll}  
      \\
	  $\inferrule* [left=\bf{ParL}]{
	      P \xrightarrow{\alpha} P^{'}
	    \\
	      bn(\alpha)\cap fn(Q)=\emptyset
	  }{
	    P|Q \xrightarrow{\alpha} P^{'}|Q
	  }$
	&
	  $\inferrule* [left=\bf{ParR}]{
	      Q \xrightarrow{\alpha} Q^{'}
	    \\
	      bn(\alpha)\cap fn(P)=\emptyset
	  }{
	    P|Q \xrightarrow{\alpha} P|Q^{'}
	  }$
      \\
    \end{tabular}
    \\
    \begin{tabular}{ll}  
      \\
	  $\inferrule* [left=\bf{Res}]{
	      P \xrightarrow{\alpha} P^{'}
	    \\ 
	      z\notin n(\alpha)
	  }{
	    (\nu z) P \xrightarrow{\alpha} (\nu z) P^{'}
	  }$
	&
      \\\\
	  $\inferrule* [left=\bf{ResAlp1}]{
	      (\nu w)P\{w/z\} \xrightarrow{\alpha} P^{'}
	    \\ 
	      w\notin n(P)
	  }{
	    (\nu z) P \xrightarrow{\alpha} P^{'}
	  }$
	&
	  $\inferrule* [left=\bf{ResAlp2}]{
	      P \xrightarrow{\alpha} P^{'}
	    \\ 
	      w\notin n(P)
	  }{
	    (\nu w)P\{w/z\} \xrightarrow{\alpha} (\nu w)P^{'}
	  }$
      \\
    \end{tabular}
    \\
    \begin{tabular}{ll}  
      \\
	  $\inferrule* [left=\bf{EComL}]{
	      P \xrightarrow{xy} P^{'}
	    \\
	      Q\xrightarrow{\overline{x}y} Q^{'}
	  }{
	    P|Q \xrightarrow{\tau} P^{'}|Q^{'}
	  }$
	&
	  $\inferrule* [left=\bf{EComR}]{
	      P \xrightarrow{\overline{x}y} P^{'}
	    \\
	      Q\xrightarrow{xy} Q^{'}
	  }{
	    P|Q \xrightarrow{\tau} P^{'}|Q^{'}
	  }$
      \\
    \end{tabular}
    \\
    \begin{tabular}{ll}  
      \\
	  $\inferrule* [left=\bf{ClsL}]{
	      P \xrightarrow{\overline{x}(z)} P^{'}
	    \\
	      Q \xrightarrow{xz} Q^{'}
	    \\
	      z\notin fn(Q)
	  }{
	    P|Q \xrightarrow{\tau} (\nu z)(P^{'}|Q^{'})
	  }$
	&
	  $\inferrule* [left=\bf{ClsR}]{
	      P \xrightarrow{xz} P^{'}
	    \\
	      Q \xrightarrow{\overline{x}(z)} Q^{'}
	    \\
	      z\notin fn(P)
	  }{
	    P|Q \xrightarrow{\tau} (\nu z)(P^{'}|Q^{'})
	  }$
      \\
    \end{tabular}
    \\
    \begin{tabular}{ll}  
      \\
	  $\inferrule* [left=\bf{Opn}]{
	      P \xrightarrow{\overline{x}z} P^{'}
	    \\ 
	      z\neq x
	  }{
	    (\nu z) P \;\xrightarrow{\overline{x}(z)} P^{'}
	  }$
	&
	  $\inferrule* [left=\bf{OpnAlp}]{
	      (\nu w)P\{w/z\} \xrightarrow{\overline{x}(w)} P^{'}
	    \\ 
	      w\notin n(P)
	    \\
	      x \neq w \neq z
	  }{
	    (\nu z)P \xrightarrow{\overline{x}(w)} P^{'}
	  }$
      \\
    \end{tabular}
    \\
    \begin{tabular}{l}  
      \\
	$\inferrule* [left=\bf{Ide}]{
	    A(\tilde{x}) \stackrel{def}{=} P
	  \\
	    P\{\tilde{w}/\tilde{x}\} \xrightarrow{\alpha} P^{'}
	}{
	  A(\tilde{x})\{\tilde{w}/\tilde{x}\} \xrightarrow{\alpha} P^{'}
	}$
      \\\multicolumn{1}{l}{\line(1,0){415}}
    \end{tabular}
    \caption{Early semantic without structural congruence and without explicit $\alpha$ conversion}
    \label{transitionrelationearlywithoutstructuralcongruence}
  \end{table}
\end{definition}

\begin{example}
  We show now an example of the so called scope extrusion, in particular we prove that
  \begin{center}
    $a(x).P\; |\; (\nu b)\overline{a}b.Q\; \xrightarrow{\tau}\; (\nu b) (P\{b/x\}\; |\; Q)$
  \end{center}
  where we suppose that $b\notin fn(P)$. In this example the scope of $(\nu b)$ moves from the right hand component to the left hand.
  \[
    \inferrule* [left=CloseR] {
	\inferrule* [left=Einp] {
	}{
	  a(x).P\; \xrightarrow{ab} P\{b/x\}
	}
      \\
	\inferrule* [left=Opn] {
	    \inferrule* [left=Out]{
	    }{
	      \overline{a}b.Q\; \xrightarrow{\overline{a}b} Q
	    }
	  \\
	    a\neq b
	}{
	  (\nu b)\overline{a}b.Q\; \xrightarrow{\overline{a}(b)} Q
	}
      \\
	b\notin fn((\nu b)\overline{a}b.Q)
    }{
      a(x).P\; |\; (\nu b)\overline{a}b.Q\; \xrightarrow{\tau}\; (\nu b) (P\{b/x\}\; |\; Q)
    }
  \]

\end{example}


\begin{example}
    We want to prove now that:
    \begin{center}
      $((\nu b) a(x).P)\; |\; \overline{a}b.Q\; 
	\xrightarrow{\tau}\; 
	((\nu c) (P\{c/b\}\{b/x\})) | Q$
    \end{center}
    where $b\notin bn(P)$
    \[
	    \inferrule* [left=ResAlp1] {
		\inferrule* [left=Res] {
		    \inferrule* [left=EInp]{
		    }{
		      (a(x).P)\{c/b\}\;
			\xrightarrow{ab}\;
			  P\{c/b\}\{b/x\}
		    }
		  \\
		    c\notin n(a(b))
		}{
		  (\nu c)((a(x).P)\{c/b\})\;
		    \xrightarrow{ab}\;
		      (\nu c)(P\{c/b\}\{b/x\})
		}
	      \\
		b\notin n((a(x).P)\{c/b\})
	    }{
	      (\nu b) a(x).P\; 
		\xrightarrow{ab}\; 
		  (\nu c) P\{c/b\}\{b/x\}
	    }
    \]

      \[
  	\inferrule* [left=EComL] {
  	      (\nu b) a(x).P\; 
		\xrightarrow{ab}\; 
		  (\nu c) P\{c/b\}\{b/x\}
  	  \\
  	    \inferrule* [left=EOut] {
  	    }{
  	      \overline{a}b.Q\; 
		\xrightarrow{\overline{a}b}\; 
		  Q
  	    }
  	}{
	  ((\nu b) a(x).P)\; |\; \overline{a}b.Q\; 
	    \xrightarrow{\tau}\; 
	      ((\nu c) (P\{c/b\}\{b/x\})) | Q
  	}
      \]
\end{example}

\begin{example}
  We have to spend some time to deal with the change of bound names in an identifier. Suppose we have
  \[
    A(x)\stackrel{def}{=} \underbrace{x(y).x(a).0}_{P}
  \]
  From the definition of substitution it follows that $A(x)\{y/x\}=A(y)$. The identifier $A(y)$ is expected to behave consistently with $P\{y/x\}=y(z).y(a).0$. So for example we have to prove that $A(y)\xrightarrow{yw}y(a).0$. We can prove this in the following way:
  \[
    \inferrule* [left=\bf{Ide}]{
	A(x)\stackrel{def}{=} P
      \\
	\inferrule* [left=\bf{EInp}]{
	}{
	  P\{y/x\}\xrightarrow{yw}y(a).0
	}
    }{
      A(y)\xrightarrow{yw}y(a).0
    }
  \]
\end{example}


\subsection{Late operational semantic(without structural congruence)}


In this case the set of actions $\mathbb{A}$ contains
\begin{itemize}
      \item bound input $x(y)$
      \item unbound output $\overline{x}y$
      \item the silent action $\tau$
      \item bound output $\overline{x}(y)$
\end{itemize}


\begin{definition}
  The \emph{late transition relation without structural congruence} $\rightarrow\subseteq \mathbb{P}\times \mathbb{A} \times \mathbb{P}$ is the smallest relation induced by the rules in table \ref{transitionrelationpilatewithoutstructuralcongruence}.
  \begin{table}
    \begin{tabular}{ll}
     \multicolumn{2}{l}{\line(1,0){415}}\\
	  \bf{LInp}
	  \begin{tabular}{c}
 	    $z\notin fn(P)$
	    \\\hline
 	    $x(y).P \xrightarrow{x(z)} P\{z/y\}$
	  \end{tabular}
	&
	  \bf{Res}
	  \begin{tabular}{c}
	    $P \xrightarrow{\alpha} P^{'}$ $z\notin n(\alpha)$
	      \\\hline
	    $(\nu z) P \xrightarrow{\alpha} (\nu z) P^{'}$
	  \end{tabular}    
      \\\\
	  \bf{SumL}
	  \begin{tabular}{c}
	      $P \xrightarrow{\alpha} P^{'}$
	    \\\hline
	      $P+Q \xrightarrow{\alpha} P^{'}$
	  \end{tabular}
	&
	  \bf{SumR}
	  \begin{tabular}{c}
	      $Q \xrightarrow{\alpha} Q^{'}$
	    \\\hline
	      $P+Q \xrightarrow{\alpha} Q^{'}$
	  \end{tabular}
      \\\\
	  \bf{ParL}
	  \begin{tabular}{c}
	      $P \xrightarrow{\alpha} P^{'}\;\; bn(\alpha)\cap fn(Q)=\emptyset$
	    \\\hline
	      $P|Q \xrightarrow{\alpha} P^{'}|Q$
	  \end{tabular}
	&
	  \bf{ParR}
	  \begin{tabular}{c}
	      $Q \xrightarrow{\alpha} Q^{'}\;\; bn(\alpha)\cap fn(Q)=\emptyset$
	    \\\hline
	      $P|Q \xrightarrow{\alpha} P|Q^{'}$
	  \end{tabular}
      \\\\
	  \bf{ComL}
	  \begin{tabular}{c}
	      $P \xrightarrow{x(y)} P^{'}\;\; Q\xrightarrow{\overline{x}(z)} Q^{'}$
	    \\\hline
	      $P|Q \xrightarrow{\tau} P^{'}\{z/y\}|Q^{'}$
	  \end{tabular}	
	&
	  \bf{ComR}
	  \begin{tabular}{c}
	      $P \xrightarrow{\overline{x}(z)} P^{'}\;\; Q\xrightarrow{x(y)} Q^{'}$
	    \\\hline
	      $P|Q \xrightarrow{\tau} P^{'}|Q^{'}\{z/y\}$
	  \end{tabular}	
      \\\\
	  \bf{Opn}
	  \begin{tabular}{c}
	      $P \xrightarrow{\overline{x}z} P^{'}\;\; z\neq x$
	    \\\hline
	      $(\nu z) P \xrightarrow{\overline{x}(z)} P^{'}$
	  \end{tabular}
	&
	  \bf{Out}
	  \begin{tabular}{c}
	    \hline
	    $\overline{x}y.P \xrightarrow{\overline{x}y} P$
	  \end{tabular}
      \\\\
	  \bf{ClsL}
	  \begin{tabular}{c}
	      $P\; \xrightarrow{\overline{x}(z)}\; P^{'}$  $Q \xrightarrow{xz} Q^{'}$ $z\notin fn(Q)$
	    \\\hline
	      $P|Q\; \xrightarrow{\tau}\; (\nu z)(P^{'}|Q^{'})$
	  \end{tabular}
	&
	  \bf{ClsR}
	  \begin{tabular}{c}
	      $P \xrightarrow{xz} P^{'}$  $Q \xrightarrow{\overline{x}(z)} Q^{'}$ $z\notin fn(P)$
	    \\\hline
	      $P|Q \xrightarrow{\tau} (\nu z)(P^{'}|Q^{'})$
	  \end{tabular}
      \\\\
	  \bf{Tau}
	  \begin{tabular}{c}
	    \hline
	      $\tau.P \xrightarrow{\tau} P$
	  \end{tabular}
	&
	  \bf{Ide}
	  \begin{tabular}{c}
	    $A(\tilde{x}) \stackrel{def}{=} P\; P\{\tilde{y}/\tilde{x}\} \xrightarrow{\alpha} P^{'}$
	      \\\hline
	    $A(\tilde{y}) \xrightarrow{\alpha} P^{'}$
	  \end{tabular}
      \\\multicolumn{2}{l}{\line(1,0){415}}
    \end{tabular}
    \caption{Late semantic without structural congruence and without explicit $\alpha$ conversion}
    \label{transitionrelationpilatewithoutstructuralcongruence}
  \end{table}
\end{definition}





\section{Structural congruence}

Structural congruences are a set of equations defining equality and congruence relations on process. They can be used in combination with an SOS semantic for languages. In some cases structural congruences help simplifying the SOS rules: for example they can capture inherent properties of composition operators(e.g. commutativity, associativity and zero element). Also, in process calculi, structural congruences let processes interact even in case they are not adjacent in the syntax. There is a possible trade off between what to include in the structural congruence and what to include in the transition rules: for example in the case of the commutativity of the sum operator. It is worth noticing that in most process calculi every structurally congruent processes should never be distinguished and thus any semantic must assign them the same behaviour.


\begin{definition}
  A \emph{change of bound names} in a process $P$ is the replacement of a subterm $x(z).Q$ of $P$ by $x(w).Q\{w/z\}$ or the replacement of a subterm $(\nu z)Q$ of $P$ by $(\nu w)Q\{w/z\}$ where in each case $w$ does not occur in $Q$.
\end{definition}


\begin{definition}
  A \emph{context} $C[\cdot]$ is a process with a placeholder. If $C[\cdot]$ is a context and we replace the placeholder with $P$, than we obtain $C[P]$. In doing so, we make no $\alpha$ conversions.
\end{definition}


\begin{definition}
  A \emph{congruence} is a binary relation on processes such that:
  \begin{itemize}
    \item 
      $S$ is an equivalence relation
    \item 
      $S$ is preserved by substitution in contexts: for each pair of processes $(P, Q)$ and for each context $C[\cdot]$
      \begin{center}
	$(P,Q)\in S\; \Rightarrow\; (C[P], C[Q])\in S$
      \end{center}
  \end{itemize}
\end{definition}

\begin{definition}
  Processes $P$ and $Q$ are \emph{$\alpha$ convertible} or \emph{$\alpha$ equivalent} if $Q$ can be obtained from $P$ by a finite number of changes of bound names. If $P$ and $Q$ are $\alpha$ equivalent then we write $P\equiv_{\alpha}Q$. Specifically the $\alpha$ equivalence is the smallest binary relation on processes that satisfies the laws in table \ref{alphaequivalence}
  \begin{table}
    \begin{tabular}{lll}
      \multicolumn{3}{l}{\line(1,0){415}}\\
	  $\inferrule*[left=AlpOut]{
	      P\equiv_{\alpha}Q
	  }{
	    \overline{x}y.P\equiv_{\alpha}\overline{x}y.Q
	  }$
	&
	  $\inferrule*[left=AlpTau]{
	      P\equiv_{\alpha}Q
	  }{
	    \tau.P\equiv_{\alpha}\tau.Q
	  }$
	&
	  $\inferrule*[left=AlpInp]{
	      P\equiv_{\alpha}Q
	  }{
	    x(y).P\equiv_{\alpha}x(y).Q
	  }$
      \\
    \end{tabular}
    \\
    \begin{tabular}{lll}
      \\
	  $\inferrule*[left=AlpIde]{
	  }{
	    A(\tilde{x})\equiv_{\alpha}A(\tilde{x})
	  }$
	&
	  $\inferrule*[left=AlpZero]{
	  }{
	    0\equiv_{\alpha}0
	  }$
	&
	  $\inferrule*[left=AlpRes]{
	      P\equiv_{\alpha}Q
	  }{
	    (\nu x)P\equiv_{\alpha}(\nu x)Q
	  }$
      \\
    \end{tabular}
    \\
    \begin{tabular}{ll}
      \\
	  $\inferrule*[left=AlpPar]{
	      P_{1}\equiv_{\alpha}Q_{1}
	    \\
	      P_{2}\equiv_{\alpha}Q_{2}
	  }{
	    P_{1}|P_{2}\equiv_{\alpha}Q_{1}|Q_{2}
	  }$
      &
	  $\inferrule*[left=AlpSum]{
	      P_{1}\equiv_{\alpha}Q_{1}
	    \\
	      P_{2}\equiv_{\alpha}Q_{2}
	  }{
	    P_{1}+P_{2}\equiv_{\alpha}Q_{1}+Q_{2}
	  }$
      \\
    \end{tabular}
    \\
    \begin{tabular}{l}
      \\
	  $\inferrule*[left=AlpRes1]{
	      P\equiv_{\alpha}Q
	    \\
	      x\neq y
	    \\
	      y\notin n(Q)
	    \\
	      x\in fn(Q)
	  }{
	    (\nu x)P\equiv_{\alpha}(\nu y)Q\{y/x\}
	  }$
      \\\\
	  $\inferrule*[left=AlpInp1]{
	      P\equiv_{\alpha}Q
	    \\
	      x\neq y
	    \\
	      y\notin n(Q)
	    \\
	      x\in fn(Q)
	  }{
	    z(x).P\equiv_{\alpha}z(y).Q\{y/x\}
	  }$
      \\
    \end{tabular}
    \\
    \begin{tabular}{l}
      \\
	  $\inferrule*[left=AlpRes2]{
	      P\equiv_{\alpha}Q
	    \\
	      x\neq y
	    \\
	      x\notin n(P)
	    \\
	      y\in fn(P)
	  }{
	    (\nu x)P\{x/y\}\equiv_{\alpha}(\nu y)Q
	  }$
      \\\\
	  $\inferrule*[left=AlpInp2]{
	      P\equiv_{\alpha}Q\{x/y\}
	    \\
	      x\neq y
	    \\
	      x\notin n(P)
	    \\
	      y\in fn(P)
	  }{
	    z(x).P\{x/y\}\equiv_{\alpha}z(y).Q
	  }$
    \\\\\multicolumn{1}{l}{\line(1,0){415}}
    \end{tabular}
    \caption{$\alpha$ equivalence laws}
    \label{alphaequivalence}
  \end{table}
\end{definition}


% \begin{definition}
%   Processes $P$ and $Q$ are \emph{$\alpha$ convertible} or \emph{$\alpha$ equivalent} if $Q$ can be obtained from $P$ by a finite number of changes of bound names. If $P$ and $Q$ are $\alpha$ equivalent then we write $P\equiv_{\alpha}Q$. Specifically the $\alpha$ equivalence is the smallest binary relation on processes that satisfies the laws in table \ref{alphaequivalence}
%   \begin{table}
%     \begin{tabular}{lll}
%       \multicolumn{3}{l}{\line(1,0){415}}\\
% 	  $\inferrule*[left=AlpOut]{
% 	      P\equiv_{\alpha}Q
% 	  }{
% 	    \overline{x}y.P\equiv_{\alpha}\overline{x}y.Q
% 	  }$
% 	&
% 	  $\inferrule*[left=AlpTau]{
% 	      P\equiv_{\alpha}Q
% 	  }{
% 	    \tau.P\equiv_{\alpha}\tau.Q
% 	  }$
% 	&
% 	  $\inferrule*[left=AlpInp]{
% 	      P\equiv_{\alpha}Q
% 	  }{
% 	    x(y).P\equiv_{\alpha}x(y).Q
% 	  }$
%       \\
%     \end{tabular}
%     \\
%     \begin{tabular}{lll}
%       \\
% 	  $\inferrule*[left=AlpIde]{
% 	  }{
% 	    A(\tilde{x})\equiv_{\alpha}A(\tilde{x})
% 	  }$
% 	&
% 	  $\inferrule*[left=AlpZero]{
% 	  }{
% 	    0\equiv_{\alpha}0
% 	  }$
% 	&
% 	  $\inferrule*[left=AlpRes]{
% 	      P\equiv_{\alpha}Q
% 	  }{
% 	    (\nu x)P\equiv_{\alpha}(\nu x)Q
% 	  }$
%       \\
%     \end{tabular}
%     \\
%     \begin{tabular}{ll}
%       \\
% 	  $\inferrule*[left=AlpPar]{
% 	      P_{1}\equiv_{\alpha}Q_{1}
% 	    \\
% 	      P_{2}\equiv_{\alpha}Q_{2}
% 	  }{
% 	    P_{1}|P_{2}\equiv_{\alpha}Q_{1}|Q_{2}
% 	  }$
%       &
% 	  $\inferrule*[left=AlpSum]{
% 	      P_{1}\equiv_{\alpha}Q_{1}
% 	    \\
% 	      P_{2}\equiv_{\alpha}Q_{2}
% 	  }{
% 	    P_{1}+P_{2}\equiv_{\alpha}Q_{1}+Q_{2}
% 	  }$
%       \\
%     \end{tabular}
%     \\
%     \begin{tabular}{l}
%       \\
% 	  $\inferrule*[left=AlpRes1]{
% 	      P\{y/x\}\equiv_{\alpha}Q
% 	    \\
% 	      x\neq y
% 	    \\
% 	      y\notin n(P)
% 	    \\
% 	      x\in fn(P)
% 	  }{
% 	    (\nu x)P\equiv_{\alpha}(\nu y)Q
% 	  }$
%       \\\\
% 	  $\inferrule*[left=AlpInp1]{
% 	      P\{y/x\}\equiv_{\alpha}Q
% 	    \\
% 	      x\neq y
% 	    \\
% 	      y\notin n(P)
% 	    \\
% 	      x\in fn(P)
% 	  }{
% 	    z(x).P\equiv_{\alpha}z(y).Q
% 	  }$
%       \\
%     \end{tabular}
%     \\
%     \begin{tabular}{l}
%       \\
% 	  $\inferrule*[left=AlpRes2]{
% 	      P\equiv_{\alpha}Q\{x/y\}
% 	    \\
% 	      x\neq y
% 	    \\
% 	      x\notin n(Q)
% 	    \\
% 	      y\in fn(Q)
% 	  }{
% 	    (\nu x)P\equiv_{\alpha}(\nu y)Q
% 	  }$
%       \\\\
% 	  $\inferrule*[left=AlpInp2]{
% 	      P\equiv_{\alpha}Q\{x/y\}
% 	    \\
% 	      x\neq y
% 	    \\
% 	      x\notin n(Q)
% 	    \\
% 	      y\in fn(Q)
% 	  }{
% 	    z(x).P\equiv_{\alpha}z(y).Q
% 	  }$
%     \\\\\multicolumn{1}{l}{\line(1,0){415}}
%     \end{tabular}
%     \caption{$\alpha$ equivalence laws}
%     \label{alphaequivalence}
%   \end{table}
% \end{definition}


It remains the problem of proving that $\alpha$ equivalence is well defined, i.e. if we change only some bound names in a process $P$ then we get a process $\alpha$ equivalent to $P$. 

% \begin{lemma}
%   Inversion lemma for $\alpha$ equivalence
%   \begin{itemize}
%     \item 	
%       If $P\equiv_{\alpha}0$ then $P$ is also the null process $0$
%     \item
%       If $P\equiv_{\alpha} \tau.Q_{1}$ then $P=\tau.P_{1}$ for some $P_{1}$ such that $P_{1}\equiv_{\alpha}Q_{1}$
%     \item
%       If $P\equiv_{\alpha} \overline{x}y.Q_{1}$ then $P=\overline{x}y.P_{1}$ for some $P_{1}$ such that $P_{1}\equiv_{\alpha}Q_{1}$
%     \item
%       If $P\equiv_{\alpha} z(y).Q_{1}$ then one of the following cases holds:
%       \begin{itemize}
% 	\item
% 	  $P=z(y).P_{1}$ for some $P_{1}$ such that $P_{1}\equiv_{\alpha}Q_{1}$
% 	\item 
% 	  $P=z(x).(P_{1}\{x/y\})$ for some $P_{1}$ such that $P_{1}\equiv_{\alpha}Q_{1}$
% 	\item 
% 	  $P=z(x).P_{1}$ for some $P_{1}$ such that $P_{1}\equiv_{\alpha}Q_{1}$
%       \end{itemize}
%     \item
%       If $P\equiv_{\alpha} Q_{1}+Q_{2}$ then $P=P_{1}+P_{2}$ for some $P_{1}$ and $P_{2}$ such that $P_{1}\equiv_{\alpha}Q_{1}$ and $P_{2}\equiv_{\alpha}Q_{2}$.
%     \item 
%       If $P\equiv_{\alpha} Q_{1}|Q_{2}$ then $P=P_{1}|P_{2}$ for some $P_{1}$ and $P_{2}$ such that $P_{1}\equiv_{\alpha}Q_{1}$ and $P_{2}\equiv_{\alpha}Q_{2}$.
%     \item 
%       If $P\equiv_{\alpha} (\nu y)Q_{1}$ then one of the following cases holds:
%       \begin{itemize}
% 	\item
% 	  $P=(\nu y).P_{1}$ for some $P_{1}$ such that $P_{1}\equiv_{\alpha}Q_{1}$
% 	\item
% 	  $P=(\nu x)P_{1}$ such that $P_{1}\{y/x\}\equiv_{\alpha}Q_{1}$ and $y\notin fn(P_{1})$
% 	\item
% 	  $P=(\nu x)P_{1}$ such that $P_{1}\equiv_{\alpha}Q_{1}\{x/y\}$ and $x\notin fn(Q_{1})$
%       \end{itemize}
%     \item 
%       If $P\equiv_{\alpha} A(\tilde{x})$ then $P$ is $Q$.
%   \end{itemize}
%     \begin{proof}
%       This lemma works because given $Q$ we know which rules must be at the end of any proof tree of $P\equiv_{\alpha}Q$.
%     \end{proof}
% \end{lemma}


According to \cite{milnerparrowwalker}
% (paragraph 1.3.1) 
the following lemma holds:
\begin{lemma}
  Let $P$ be a process and $y,w,z$ names such that $w=z$ or $w\notin fn(P)$ then $P\{w/z\}\{y/w\}\equiv_{\alpha}P$.
\end{lemma}






\begin{definition}
  \emph{structural congruence $\equiv$} is the smallest relation on processes that satisfies the axioms in table \ref{structuralCongrunce}
  \begin{table}
    \begin{tabular}{ll}
      \multicolumn{2}{l}{\line(1,0){415}}\\
	$\inferrule* [left=\bf{SumAsc1}]{}{M_{1}+(M_{2}+M_{3})\equiv (M_{1}+M_{2})+M_{3}}$ &$\inferrule* [left=\bf{ParAsc1}]{}{P_{1}|(P_{2}|P_{3})\equiv (P_{1}|P_{2})|P_{3}}$
      \\
	$\inferrule* [left=\bf{SumAsc2}]{}{(M_{1}+M_{2})+M_{3}\equiv M_{1}+(M_{2}+M_{3})}$ &$\inferrule* [left=\bf{ParAsc2}]{}{(P_{1}|P_{2})|P_{3}\equiv P_{1}|(P_{2}|P_{3})}$ 
      \\
	 & 
      \\
      \end{tabular}
      \\
      \begin{tabular}{lll}
      \\
	$\inferrule* [left=\bf{ParCom}]{}{P_{1}|P_{2}\equiv P_{2}|P_{1}}$ &
	$\inferrule* [left=\bf{ResCom}]{}{(\nu x) (\nu y) P \equiv (\nu y) (\nu x) P}$ &
	$\inferrule* [left=\bf{SumCom}]{}{M_{1}+M_{2}\equiv M_{2}+M_{1}}$
      \\
      \end{tabular}
      \\
      \begin{tabular}{ll}
      \\
	$\inferrule* [left=\bf{ScpExtPar1}]{z\notin fn(P_{1})}{(\nu z) (P_{1}|P_{2}) \equiv P_{1}|(\nu z) P_{2}}$ & $\inferrule* [left=\bf{ScpExtPar2}]{z\notin fn(P_{1})}{P_{1}|(\nu z) P_{2} \equiv (\nu z) (P_{1}|P_{2})}$ 
      \\
	$\inferrule* [left=\bf{ScpExtSum1}]{z\notin fn(P_{1})}{(\nu z) (P_{1}+P_{2}) \equiv P_{1}+(\nu z) P_{2}}$ & $\inferrule* [left=\bf{ScpExtSum2}]{z\notin fn(P_{1})}{P_{1}+(\nu z) P_{2} \equiv (\nu z) (P_{1}+P_{2})}$ 
      \\
      \end{tabular}
      \\
      \begin{tabular}{lll}
      \\
	  $\inferrule* [left=\bf{Ide}]{A(\tilde{x})\stackrel{def}{=}P}{A(\tilde{w})\equiv P\{\tilde{w}/\tilde{x}\}}$
	&
	  $\inferrule* [left=\bf{Trans}]{P \equiv Q \\ Q \equiv R}{P \equiv R}$
	&
	  $\inferrule* [left=\bf{Alp}] {
	    P \equiv_{\alpha} Q
	  }{
	    P\equiv Q
	  }$
      \\
      \end{tabular}
      \\
      \begin{tabular}{ll}
      \\
	  $\inferrule* [left=\bf{Cong1}]{
	    P \equiv Q
	  }{
	    C[P] \equiv C[Q]
	  }$
	&
	  $\inferrule* [left=\bf{Cong2}]{
	    P_{1} \equiv Q_{1} 
	  \\ 
	    P_{2} \equiv Q_{2}
	  \\
	    C[\_,\_]\in\{\_+\_, \_|\_ \}
	  }{
	    C[P_{1}, P_{2}] \equiv C[Q_{1}, Q_{2}]
	  }$
      \\
      \\\multicolumn{2}{l}{\line(1,0){415}}
    \end{tabular}
    \caption{Structural congruence rules}
    \label{structuralCongrunce}
  \end{table}
\end{definition}

\begin{proposition}
  $\equiv$ as defined in table \ref{structuralCongrunce} is a congruence and an equivalence relation.
  \begin{proof}
    $\equiv$ is a congruence thanks to rules $Cong1$ and $Cong2$. Reflexivity holds for rule $Alp$. Symmetry holds because all the rules are symmetric or have a symmetric counterpart. Transitivity holds because of rule $Trans$.
  \end{proof}
\end{proposition}

We can make some clarification on the axioms of the structural congruence:
\begin{description}
  \item[$unfolding$] 
    this just helps replace an identifier by its definition, with the appropriate parameter instantiation. The alternative is to use the rule $Cns$ in table \ref{transitionrelationearlywithoutstructuralcongruence}.
  \item[$\alpha\; conversion$]
    is the $\alpha$ conversion, i.e., the choice of bound names, it identifies agents like $x(y).\overline{z}y$ and $x(w).\overline{z}w$. In the semantic of $\pi$ calculus we can use the structural congruence with the rule SC-ALP or we can embed the $\alpha$ conversion in the SOS rules. 
    In the early case, the rule for input and the rules $ResAlp1, OpnAlp, Cns$ take care of $\alpha$ conversion, whether in the late case the rule for communication and the rules is $ResAlp1, OpnAlp, Cns$ are in charge for $\alpha$ conversion.
  \item[$abelian\; monoidal\; properties\; of\; some\; operators$]
    We can deal with associativity and commutativity properties of sum and parallel composition by using SOS rules or by axiom of the structural congruence. For example the commutativity of the sum can be expressed by the following two rules:
    \begin{center}
      \begin{tabular}{cc}
	\bf{SumL}
	  \begin{tabular}{c}
	    $P \xrightarrow{\alpha} P^{'}$\\
	    \hline
	    $P+Q \xrightarrow{\alpha} P^{'}$
	  \end{tabular}
	 &
	 \bf{SumR}
	   \begin{tabular}{c}
	    $Q \xrightarrow{\alpha} Q^{'}$\\
	    \hline
	    $P+Q \xrightarrow{\alpha} Q^{'}$
	   \end{tabular}
      \end{tabular}
    \end{center}
  or by the following rule and axiom:
    \begin{center}
      \begin{tabular}{cc}
	\bf{Sum}
	  \begin{tabular}{c}
	    $P \xrightarrow{\alpha} P^{'}$\\
	    \hline
	    $P+Q \xrightarrow{\alpha} P^{'}$
	  \end{tabular}
	 &
	 \bf{SC-SUM}
	   \begin{tabular}{c}
	    $P+Q \equiv Q+P$
	   \end{tabular}
      \end{tabular}
    \end{center}
    and the rule $Cong$
  \item[$scope\; extension$]
    We can use the scope extension laws in table \ref{structuralCongrunce} or the rules $Opn$ and $Cls$ in table \ref{transitionrelationearlywithoutstructuralcongruence} to deal with the scope extension.
\end{description}

\begin{lemma}\label{freenamesandsubstitution}
  \[
    a\in fn(Q)\Rightarrow fn(Q\{b/a\})=(fn(Q)-\{a\}) \cup \{b\}
  \]
\end{lemma}

\begin{lemma}\label{alphaequivalentsamefreenames}
  $P\equiv_{\alpha}Q\Rightarrow fn(P)=fn(Q)$
  \begin{proof}
    The proof goes by induction on rules 
    \begin{description}
      \item[$AlpZero$]
	the lemma holds because $P$ and $Q$ are the same process.
      \item[$AlpTau$]:
	\begin{center}
	  \begin{tabular}{ll}
	    &rule premise\\
	    $P\equiv_{\alpha}Q$&inductive hypothesis\\
	    $\Rightarrow fn(P)=fn(Q)$&definition of $fn$\\
	    $\Rightarrow fn(\tau.P)=fn(\tau.Q)$&\\
	  \end{tabular}
	\end{center}
      \item[$AlpOut$]:
	\begin{center}
	  \begin{tabular}{ll}
	    &rule premise\\
	    $P\equiv_{\alpha}Q$&inductive hypothesis\\
	    $\Rightarrow fn(P)=fn(Q)$&\\
	    $\Rightarrow fn(P)\cup \{x,y\}=fn(Q)\cup \{x,y\}$&definition of $fn$\\
	    $\Rightarrow fn(\overline{x}y.P)=fn(\overline{x}y.Q)$&\\
	  \end{tabular}
	\end{center}
      \item[$AlpRes1$]:
	\begin{center}
	  \begin{tabular}{ll}
	    &rule premise\\
	    $P\equiv_{\alpha}Q$&inductive hypothesis\\
	    $\Rightarrow fn(P)=fn(Q)$&\\
	    $\Rightarrow fn(P)-\{y\}=fn(Q)- \{y\}$&definition of $fn$\\
	    $\Rightarrow fn(P)-\{y\}=fn((\nu y)Q)$&\\
	    $\Rightarrow ((fn(P)-\{y\})\cup \{x\})-\{x\}=fn((\nu y)Q)$&\\
	    $\Rightarrow fn(P\{x/y\})-\{x\}=fn((\nu y)Q)$&\\
	    $\Rightarrow fn((\nu x)(P\{x/y\}))=fn((\nu y)Q)$&\\
	  \end{tabular}
	\end{center}
      \item[$AlpRes2$]: similar.
      \item[$AlpInp1$]:
	\begin{center}
	  $fn(a(x).(P\{x/y\})) = (fn(P\{x/y\})-\{x\}) \cup \{a\} = (((fn(P)-\{y\})\cup \{x\})-\{x\}) \cup \{a\} = (fn(P)-\{y\}) \cup \{a\} = (fn(Q)-\{y\})\cup \{a\} = fn(a(x).Q)$
	\end{center}
      \item[$AlpInp2$]: similar.
      \item[$AlpSum$]:
	\begin{center}
	  \begin{tabular}{ll}
	    &rule premises\\
	    $P_{1}\equiv_{\alpha}Q_{1}$ and $P_{2}\equiv_{\alpha}Q_{2}$&inductive hypothesis\\
	    $\Rightarrow fn(P_{1})=fn(Q_{1})$ and $fn(P_{2})=fn(Q_{2})$&\\
	    $\Rightarrow fn(P_{1})\cup fn(P_{2})=fn(Q_{1})\cap fn(Q_{2})$&definition of $fn$\\
	    $\Rightarrow fn(P_{1}+P_{2})=fn(Q_{1}+Q_{2})$&\\
	  \end{tabular}
	\end{center}
      \item[$AlpPar$]:
	\begin{center}
	  \begin{tabular}{ll}
	    &rule premises\\
	    $P_{1}\equiv_{\alpha}Q_{1}$ and $P_{2}\equiv_{\alpha}Q_{2}$&inductive hypothesis\\
	    $\Rightarrow fn(P_{1})=fn(Q_{1})$ and $fn(P_{2})=fn(Q_{2})$&\\
	    $\Rightarrow fn(P_{1})\cup fn(P_{2})=fn(Q_{1})\cap fn(Q_{2})$&definition of $fn$\\
	    $\Rightarrow fn(P_{1}|P_{2})=fn(Q_{1}|Q_{2})$&\\
	  \end{tabular}
	\end{center}
      \item[$AlpRes$]:
	\begin{center}
	  \begin{tabular}{ll}
	    &rule premise\\
	    $P\equiv_{\alpha}Q$&inductive hypothesis\\
	    $\Rightarrow fn(P)=fn(Q)$&\\
	    $\Rightarrow fn(P)- \{x\}=fn(Q)- \{x\}$&definition of $fn$\\
	    $\Rightarrow fn((\nu x)P)=fn((\nu x)Q)$&\\
	  \end{tabular}
	\end{center}
      \item[$AlpInp$]:
	\begin{center}
	  \begin{tabular}{ll}
	    &rule premise\\
	    $P\equiv_{\alpha}Q\{x/y\}$&inductive hypothesis\\
	    $\Rightarrow fn(P)=fn(Q)$&\\
	    $\Rightarrow (fn(P)-\{y\})\cup \{x\}=(fn(Q)-\{y\})\cup \{x\}$&definition of $fn$\\
	    $\Rightarrow fn(x(y).P)=fn(x(y).Q)$&\\
	  \end{tabular}
	\end{center}
      \item[$AlpIde$]
	the lemma holds because $P$ and $Q$ are the same process.
    \end{description}
  \end{proof}
\end{lemma}

\begin{lemma}\label{alphaequivalencecommutativity}
  $x\notin n(P)$, $y\in fn(P)$, $a\in fn(P)$ imply $P\{x/y\}\{b/a\}\equiv_{\alpha}P\{b/a\}\{x/y\}$ 
\end{lemma}



\begin{lemma}\label{alphaequivalencesubstitution}
  $\alpha$ equivalence is invariant with respect to substitution. In other words 
  \begin{center}
    \begin{tabular}{lll}
	$P\equiv_{\alpha}Q$
      &
	
      &
	
    \\
	$b\notin fn(P)$
      &
	$\Rightarrow$
      &
	$P\{b/a\}\equiv_{\alpha}Q\{b/a\}$
    \\
	$b\notin fn(Q)$
      &
	
      &
	
    \\
    \end{tabular}
  \end{center}
  \begin{proof}:
    If $a$ and $b$ are the same name then the substitution has no effect and the lemma holds. Otherwise:
    \begin{center}
      \begin{tabular}{ll}
	&lemma hypothesis\\
	$P \equiv_{\alpha} Q$&lemma \ref{alphaequivalentsamefreenames}\\
	$\Rightarrow fn(P) = fn(Q)$&\\
	$\Rightarrow a\notin fn(P) \wedge a\notin fn(Q)$ or $a\in fn(P) \wedge a\in fn(Q)$&\\
      \end{tabular}
    \end{center}
    In the former case $a$ is not a free name in $P$ and $Q$ so the substitutions have no effects and the lemma holds. In the latter case $a$ is a free names in both processes: the proof goes by induction on the length of the proof tree of $P\equiv_{\alpha}Q$ and then by cases on the last rule of the proof tree. Let $x,y,a$ and $b$ be pairwise different.
    \begin{description}
      \item[$AlpZero,\; AlpIde$] 
	The length of the proof is one and the rule used can be only: $AlpZero$ or $AlpIde$: the lemma holds because $P$ and $Q$ are syntacticly the same process.
      \item[$AlpTau$]:
	    \begin{center}
	      \begin{tabular}{ll}
		&rule premise\\
		$P_{1}\equiv_{\alpha}Q_{1}$&inductive hypothesis\\
		$\Rightarrow P_{1}\{b/a\}\equiv_{\alpha}Q_{1}\{b/a\}$&rule $AlpTau$\\
		$\Rightarrow \tau.(P_{1}\{b/a\})\equiv_{\alpha}\tau.(Q_{1}\{b/a\}) $&definition of substitution\\
		$\Rightarrow (\tau.P_{1})\{b/a\}\equiv_{\alpha}(\tau.Q_{1})\{b/a\} $&\\
	      \end{tabular}
	    \end{center}
      \item[$AlpSum$]:
	    \begin{center}
	      \begin{tabular}{ll}
		&rule premises\\
		$P_{1}\equiv Q_{1}$ and $P_{2}\equiv Q_{2}$&inductive hypothesis\\
		$\Rightarrow P_{1}\{b/a\}\equiv Q_{1}\{b/a\}$ and $P_{2}\{b/a\}\equiv Q_{2}\{b/a\}$&rule $AlpSum$\\
		$\Rightarrow P_{1}\{b/a\}+P_{2}\{b/a\}\equiv Q_{1}\{b/a\}+Q_{2}\{b/a\} $&definition of substitution\\
		$\Rightarrow (P_{1}+P_{2})\{b/a\}\equiv_{\alpha}(Q_{1}+Q_{2})\{b/a\} $&\\
	      \end{tabular}
	    \end{center}
      \item[$AlpPar$]:
	    this case is very similar to the previous one.
      \item[$AlpOut$]:
	    \begin{center}
	      \begin{tabular}{ll}
		&rule premise\\
		$P_{1}\equiv_{\alpha}Q_{1}$&inductive hypothesis\\
		$\Rightarrow P_{1}\{b/a\}\equiv_{\alpha}Q_{1}\{b/a\}$&rule $AlpOut$\\
		$\Rightarrow \overline{x}\{b/a\}y\{b/a\}.P_{1}\{b/a\}\equiv_{\alpha}\overline{x}\{b/a\}y\{b/a\}.Q_{1}\{b/a\} $&definition of substitution\\
		$\Rightarrow (\overline{x}y.P_{1})\{b/a\}\equiv_{\alpha}(\overline{x}y.Q_{1})\{b/a\} $&\\
	      \end{tabular}
	    \end{center}
      \item[$AlpInp$]:
	    \begin{center}
	      \begin{tabular}{ll}
		  &
		    rule premise
		\\
		    $P_{1}\equiv_{\alpha}Q_{1}$
		  &
		    inductive hypothesis
		\\
		    $\Rightarrow P_{1}\{b/a\}\equiv_{\alpha}Q_{1}\{b/a\}$
		  &
		    rule $AlpInp$
		\\
		    $\Rightarrow x\{b/a\}(y).P_{1}\{b/a\}\equiv_{\alpha}x\{b/a\}(y).Q_{1}\{b/a\} $
		  &
		    definition of substitution
		\\
		    $\Rightarrow (x(y).P_{1})\{b/a\}\equiv_{\alpha}(x(y).Q_{1})\{b/a\} $
		  &
		\\
	      \end{tabular}
	    \end{center}	    	    
 	    \begin{center}
 	      \begin{tabular}{ll}
 		&rule premise\\
 		$P_{1}\equiv_{\alpha}Q_{1}$&rule $AlpInp$\\
 		$\Rightarrow b(a).P_{1}\equiv_{\alpha}b(a).Q_{1} $&definition of substitution\\
 		$\Rightarrow a\{b/a\}(a).P_{1}\equiv_{\alpha}a\{b/a\}(a).Q_{1} $&definition of substitution\\
 		$\Rightarrow (a(a).P_{1})\{b/a\}\equiv_{\alpha}(a(a).Q_{1})\{b/a\} $&\\
 	      \end{tabular}
 	    \end{center}	    	    
	    \begin{center}
	      \begin{tabular}{ll}
		&rule premise\\
		$P_{1}\equiv_{\alpha}Q_{1}$&inductive hypothesis\\
		$\Rightarrow P_{1}\{b/a\}\equiv_{\alpha}Q_{1}\{b/a\} $&rule $AlpInp$\\
		$\Rightarrow b\{b/a\}(x).(P_{1}\{b/a\})\equiv_{\alpha}b\{b/a\}(x).(Q_{1}\{b/a\}) $&definition of substitution\\
		$\Rightarrow (b(x).P_{1})\{b/a\}\equiv_{\alpha}(b(x).Q_{1})\{b/a\} $&\\
	      \end{tabular}
	    \end{center}	    	    
      \item[$AlpInp1$]:
% 	    we have various cases:
% 	    \begin{itemize}
% 	      \item 
% 		the last part of the proof tree of $P\equiv_{\alpha}Q$ is
% 		\[\inferrule*[left=AlpInp1]{
% 		    P_{1}\equiv_{\alpha}Q_{1}\{x/y\}
% 		  \\
% 		    x\neq y	      
% 		  \\
% 		    x\notin n(Q_{1})
% 		  \\
% 		    y\in fn(Q_{1})
% 		}{
% 		  \underbrace{z(x).P_{1}}_{P}
% 		    \equiv_{\alpha}
% 		      \underbrace{z(y).Q_{1}}_{Q}
% 		}\]
% 		\begin{center}
% 		  \begin{tabular}{ll}
% 		      &
% 		    rule premise
% 		  \\
% 		    $P_{1}\equiv_{\alpha}Q_{1}\{x/y\}$ and $x\notin fn(Q_{1})$ 
% 		      &
% 		    inductive hypothesis
% 		  \\
% 		    $\Rightarrow P_{1}\{b/a\}\equiv_{\alpha}Q_{1}\{x/y\}\{b/a\}$
% 		      &
% 		    transitivity and lemma \ref{alphaequivalencecommutativity}
% 		  \\
% 		    $\Rightarrow P_{1}\{b/a\}\equiv_{\alpha}Q_{1}\{b/a\}\{x/y\}$
% 		      &
% 		    rule $AlpInp1$
% 		  \\
% 		    $\Rightarrow z(x).(P_{1}\{b/a\})\equiv_{\alpha}z(y).(Q_{1}\{b/a\})$
% 		      &
% 		    definition of substitution
% 		  \\
% 		    $\Rightarrow (z(x).P_{1})\{b/a\}\equiv_{\alpha}(z(y).Q_{1})\{b/a\}$
% 		      &
% 		    
% 		  \\
% 		  \end{tabular}
% 		\end{center}	    	
% 	      \item 
% 		the last part of the proof tree of $P\equiv_{\alpha}Q$ is
% 		\[\inferrule*[left=AlpInp1]{
% 		    P_{1}\equiv_{\alpha}Q_{1}\{x/y\}
% 		  \\
% 		    x\neq y	      
% 		  \\
% 		    x\notin fn(Q_{1})
% 		}{
% 		  \underbrace{b(x).P_{1}}_{P}
% 		    \equiv_{\alpha}
% 		      \underbrace{b(y).Q_{1}}_{Q}
% 		}\]
% 		\begin{center}
% 		  \begin{tabular}{ll}
% 		      &
% 		    rule premise
% 		  \\
% 		    $P_{1}\equiv_{\alpha}Q_{1}\{x/y\}$ and $x\notin fn(Q_{1})$ 
% 		      &
% 		    inductive hypothesis
% 		  \\
% 		    $\Rightarrow P_{1}\{b/a\}\equiv_{\alpha}Q_{1}\{x/y\}\{b/a\}$
% 		      &
% 		    transitivity and lemma \ref{alphaequivalencecommutativity}
% 		  \\
% 		    $\Rightarrow P_{1}\{b/a\}\equiv_{\alpha}Q_{1}\{b/a\}\{x/y\}$
% 		      &
% 		    rule $AlpInp1$
% 		  \\
% 		    $\Rightarrow b(x).(P_{1}\{b/a\})\equiv_{\alpha}b(y).(Q_{1}\{b/a\})$
% 		      &
% 		    definition of substitution
% 		  \\
% 		    $\Rightarrow (b(x).P_{1})\{b/a\}\equiv_{\alpha}(b(y).Q_{1})\{b/a\}$
% 		      &
% 		    
% 		  \\
% 		  \end{tabular}
% 		\end{center}	    	
% 	      \item 
% 		the last part of the proof tree of $P\equiv_{\alpha}Q$ is
% 		\[\inferrule*[left=AlpInp1]{
% 		    P_{1}\equiv_{\alpha}Q_{1}\{x/y\}
% 		  \\
% 		    x\neq y	      
% 		  \\
% 		    x\notin fn(Q_{1})
% 		}{
% 		  \underbrace{a(x).P_{1}}_{P}
% 		    \equiv_{\alpha}
% 		      \underbrace{a(y).Q_{1}}_{Q}
% 		}\]
% 		\begin{center}
% 		  \begin{tabular}{ll}
% 		      &
% 		    rule premise
% 		  \\
% 		    $P_{1}\equiv_{\alpha}Q_{1}\{x/y\}$ and $x\notin fn(Q_{1})$ 
% 		      &
% 		    inductive hypothesis
% 		  \\
% 		    $\Rightarrow P_{1}\{b/a\}\equiv_{\alpha}Q_{1}\{x/y\}\{b/a\}$
% 		      &
% 		    transitivity and lemma \ref{alphaequivalencecommutativity}
% 		  \\
% 		    $\Rightarrow P_{1}\{b/a\}\equiv_{\alpha}Q_{1}\{b/a\}\{x/y\}$
% 		      &
% 		    rule $AlpInp1$
% 		  \\
% 		    $\Rightarrow a(x).(P_{1}\{b/a\})\equiv_{\alpha}a(y).(Q_{1}\{b/a\})$
% 		      &
% 		    definition of substitution
% 		  \\
% 		    $\Rightarrow (a(x).P_{1})\{b/a\}\equiv_{\alpha}(a(y).Q_{1})\{b/a\}$
% 		      &
% 		    
% 		  \\
% 		  \end{tabular}
% 		\end{center}	    	
% 	      \item 
% 		the last part of the proof tree of $P\equiv_{\alpha}Q$ is
% 		\[\inferrule*[left=AlpInp1]{
% 		    P_{1}\equiv_{\alpha}Q_{1}\{a/y\}
% 		  \\
% 		    a\neq y	      
% 		  \\
% 		    a\notin fn(Q_{1})
% 		}{
% 		  \underbrace{a(a).P_{1}}_{P}
% 		    \equiv_{\alpha}
% 		      \underbrace{a(y).Q_{1}}_{Q}
% 		}\]
% 		\begin{center}
% 		  \begin{tabular}{ll}
% 		      &
% 		    rule premise
% 		  \\
% 		    $P_{1}\equiv_{\alpha}Q_{1}\{a/y\}$ and $x\notin fn(Q_{1})$ 
% 		      &
% 		    inductive hypothesis
% 		  \\
% 		    $\Rightarrow P_{1}\{b/a\}\equiv_{\alpha}Q_{1}\{a/y\}\{b/a\}$
% 		      &
% 		    transitivity and lemma \ref{alphaequivalencecommutativity}
% 		  \\
% 		    $\Rightarrow P_{1}\{b/a\}\equiv_{\alpha}Q_{1}\{b/a\}\{a/y\}$
% 		      &
% 		    rule $AlpInp1$
% 		  \\
% 		    $\Rightarrow a(a).(P_{1}\{b/a\})\equiv_{\alpha}a(y).(Q_{1}\{b/a\})$
% 		      &
% 		    definition of substitution
% 		  \\
% 		    $\Rightarrow (a(a).P_{1})\{b/a\}\equiv_{\alpha}(a(y).Q_{1})\{b/a\}$
% 		      &
% 		    
% 		  \\
% 		  \end{tabular}
% 		\end{center}	    	
% 	      \item 
% 		the last part of the proof tree of $P\equiv_{\alpha}Q$ is
% 		\[\inferrule*[left=AlpInp1]{
% 		    P_{1}\equiv_{\alpha}Q_{1}\{x/a\}
% 		  \\
% 		    x\neq a
% 		  \\
% 		    x\notin fn(Q_{1})
% 		}{
% 		  \underbrace{a(x).P_{1}}_{P}
% 		    \equiv_{\alpha}
% 		      \underbrace{a(a).Q_{1}}_{Q}
% 		}\]
% 		\begin{center}
% 		  \begin{tabular}{ll}
% 		      &
% 		    rule premise
% 		  \\
% 		    $P_{1}\equiv_{\alpha}Q_{1}\{x/a\}$ and $x\notin fn(Q_{1})$ 
% 		      &
% 		    inductive hypothesis
% 		  \\
% 		    $\Rightarrow P_{1}\{b/a\}\equiv_{\alpha}Q_{1}\{x/a\}\{b/a\}$
% 		      &
% 		    transitivity and lemma \ref{alphaequivalencecommutativity}
% 		  \\
% 		    $\Rightarrow P_{1}\{b/a\}\equiv_{\alpha}Q_{1}\{b/a\}\{x/a\}$
% 		      &
% 		    rule $AlpInp1$
% 		  \\
% 		    $\Rightarrow a(x).(P_{1}\{b/a\})\equiv_{\alpha}a(a).(Q_{1}\{b/a\})$
% 		      &
% 		    definition of substitution
% 		  \\
% 		    $\Rightarrow (a(x).P_{1})\{b/a\}\equiv_{\alpha}(a(a).Q_{1})\{b/a\}$
% 		      &		    
% 		  \\
% 		  \end{tabular}
% 		\end{center}	    	
% 	      \item mancano x x y e x y x
% 	    \end{itemize}
	  \item[$AlpRes$]:
	    \begin{center}
	      \begin{tabular}{ll}
		&rule premise\\
		$P_{1}\equiv_{\alpha}Q_{1}$&inductive hypothesis\\
		$\Rightarrow P_{1}\{b/a\}\equiv_{\alpha}Q_{1}\{b/a\}$&rule $AlpRes$\\
		$\Rightarrow (\nu x)(P_{1}\{b/a\})\equiv_{\alpha}(\nu x)(Q_{1}\{b/a\}) $&definition of substitution\\
		$\Rightarrow ((\nu x)P_{1})\{b/a\}\equiv_{\alpha}((\nu x)Q_{1})\{b/a\} $&\\
	      \end{tabular}
	    \end{center}	    	    
      \item[$AlpRes1$]:
% 	    \[\inferrule*[left=AlpRes1]{
% 		P_{1}\equiv_{\alpha}Q_{1}\{x/y\}
% 	      \\
% 		x\neq y
% 	      \\
% 		x\notin fn(Q_{1})
% 	    }{
% 	      \underbrace{(\nu x)P_{1}}_{P}\equiv_{\alpha}\underbrace{(\nu y)Q_{1}}_{Q}
% 	    }\]
% 	    \begin{center}
% 	      \begin{tabular}{ll}
% 		  &
% 		    rule premises
% 		\\
% 		    $P_{1}\equiv_{\alpha}Q_{1}\{x/y\}$ and $x\neq y$ and $x\notin fn(Q_{1})$
% 		  &
% 		    inductive hypothesis
% 		\\
% 		    $P_{1}\{b/a\}\equiv_{\alpha} Q_{1}\{x/y\}\{b/a\}$
% 		  &
% 		    lemma \ref{alphaequivalencecommutativity} and transitivity
% 		\\
% 		    $P_{1}\{b/a\}\equiv_{\alpha} Q_{1}\{b/a\}\{x/y\}$
% 		  &
% 		    rule $AlpRes1$
% 		\\
% 		    $(\nu x)(P_{1}\{b/a\})\equiv_{\alpha} (\nu y)(Q_{1}\{b/a\})$
% 		  &
% 		    definition of substitution
% 		\\
% 		    $((\nu x)P_{1})\{b/a\}\equiv_{\alpha} ((\nu y)Q_{1})\{b/a\}$
% 		  &
% 		\\
% 	      \end{tabular}
% 	    \end{center}
    \end{description}
  \end{proof}
\end{lemma}


\begin{lemma}
  \[
    P\equiv_{\alpha} P \{x/y\}\{y/x\}
  \]
\end{lemma}


In the proof of equivalence of the semantics in the next section we need the following lemmas


\begin{lemma}\label{alphaEquivalenceReflexivity}
  The $\alpha$ equivalence is reflexive.
  \begin{proof}:
	We prove $P\equiv_{\alpha}P$ by structural induction on $P$:
	\begin{description}
	  \item[$0$]:
	    \[\inferrule* [left=AlpZero]{
	    }{
	      0\equiv_{\alpha}0
	    }\]
	  \item[$\tau.P_{1}$]:
	    for induction $P_{1}\equiv_{\alpha}P_{1}$ so
	    \[\inferrule* [left=AlpTau]{
	      P_{1}\equiv_{\alpha}P_{1}
	    }{
	      \tau.P_{1}\equiv_{\alpha}\tau.P_{1}
	    }\]
	  \item[$x(y).P_{1}$]:
	    for induction $P_{1}\equiv_{\alpha}P_{1}$ so
	    \[\inferrule* [left=AlpInp]{
	      P_{1}\equiv_{\alpha}P_{1}
	    }{
	      x(y).P_{1}\equiv_{\alpha}x(y).P_{1}
	    }\]
	  \item[$\overline{x}y.P_{1}$]:
	    for induction $P_{1}\equiv_{\alpha}P_{1}$ so
	    \[\inferrule* [left=AlpOut]{
	      P_{1}\equiv_{\alpha}P_{1}
	    }{
	      \overline{x}y.P_{1}\equiv_{\alpha}\overline{x}y.P_{1}
	    }\]
	  \item[$P_{1}+P_{2}$]:
	    for induction $P_{1}\equiv_{\alpha}P_{1}$ and $P_{2}\equiv_{\alpha}P_{2}$ so
	    \[\inferrule* [left=AlpSum]{
		  P_{1}\equiv_{\alpha}P_{1}
		\\
		  P_{2}\equiv_{\alpha}P_{2}
	    }{
	      P_{1}+P_{2}\equiv_{\alpha}P_{1}+P_{2}
	    }\]
	  \item[$P_{1}|P_{2}$]:
	    for induction $P_{1}\equiv_{\alpha}P_{1}$ and $P_{2}\equiv_{\alpha}P_{2}$ so
	    \[\inferrule* [left=AlpPar]{
		  P_{1}\equiv_{\alpha}P_{1}
		\\
		  P_{2}\equiv_{\alpha}P_{2}
	    }{
	      P_{1}|P_{2}\equiv_{\alpha}P_{1}|P_{2}
	    }\]
	  \item[$(\nu x)P_{1}$]:
	    for induction $P_{1}\equiv_{\alpha}P_{1}$ so
	    \[\inferrule* [left=AlpRes]{
	      P_{1}\equiv_{\alpha}P_{1}
	    }{
	      (\nu x)P_{1}\equiv_{\alpha}(\nu x)P_{1}
	    }\]
	  \item[$A(\tilde{x}|\tilde{y})$]:
	    \[\inferrule* [left=AlpIde]{
	    }{
	      A(\tilde{x})\equiv_{\alpha}A(\tilde{x})
	    }\]
	\end{description}
  \end{proof}
\end{lemma}

\begin{lemma}\label{alphaEquivalenceSymmetry}
  $\alpha$ equivalence is symmetric.
  \begin{proof}
    Every rule is symmetric or it has a symmetric counterpart.
  \end{proof}
\end{lemma}

\begin{lemma}\label{alphaEquivalenceTransitive}
  $\alpha$ equivalence is transitive.
%   \begin{proof}
%     Suppose $P\equiv_{\alpha}Q$ and $Q\equiv_{\alpha}R$. We prove the thesis $P\equiv_{\alpha}R$ by induction on the sum of the size of the derivation of $P\equiv_{\alpha}Q$ and $Q\equiv_{\alpha}R$. The last rules pair used in the derivations can be:
%     \begin{description}
%       \item[$(AlpOut, AlpOut)$]:
% 	\begin{center}
% 	  \begin{tabular}{ll}
% 	      $\inferrule* [left=\bf{AlpOut}]{
% 		P \equiv_{\alpha} Q
% 	      }{
% 		\overline{a}b.P \equiv_{\alpha} \overline{a}b.Q
% 	      }$
% 	    &
% 	      $\inferrule* [left=\bf{AlpOut}]{
% 		Q \equiv_{\alpha} R
% 	      }{
% 		\overline{a}b.Q \equiv_{\alpha} \overline{a}b.R
% 	      }$
% 	  \end{tabular}
% 	\end{center}
% 	for inductive hypothesis $P\equiv R$ and for rule $AlpOut$: $\overline{a}b.P \equiv_{\alpha} \overline{a}b.R$
%       \item[$(AlpTau, AlpTau)$] similar.
%       \item[$(AlpInp, AlpInp)$] similar.
%       \item[$(AlpIde, AlpIde)$]
% 	$P,Q$ and $R$ are the same.
%       \item[$(AlpZero, AlpZero)$]
% 	$P,Q$ and $R$ are the same.
%       \item[$(AlpSum, AlpSum)$]:
% 	\begin{center}
% 	  \begin{tabular}{ll}
% 	      $\inferrule* [left=\bf{AlpSum}]{
% 		  P_{1} \equiv_{\alpha} Q_{1}
% 		\\
% 		  P_{2} \equiv_{\alpha} Q_{2}
% 	      }{
% 		P_{1}+P_{2} \equiv_{\alpha} Q_{1}+Q_{2}
% 	      }$
% 	    &
% 	      $\inferrule* [left=\bf{AlpSum}]{
% 		  Q_{1} \equiv_{\alpha} R_{1}
% 		\\
% 		  Q_{2} \equiv_{\alpha} R_{2}
% 	      }{
% 		Q_{1}+Q_{2} \equiv_{\alpha} R_{1}+R_{2}
% 	      }$
% 	  \end{tabular}
% 	\end{center}
% 	for inductive hypothesis $P_{1}\equiv R_{1}$, $P_{2}\equiv R_{2}$ and for rule $AlpSum$: $P_{1}+P_{2} \equiv_{\alpha} R_{1}+R_{2}$
%       \item[$(AlpPar, AlpPar)$]
% 	similar.
%       \item[$(AlpRes, AlpRes)$]
% 	similar.
%       \item[$(AlpInp, AlpInp1)$]:
% 	\begin{center}
% 	  \begin{tabular}{ll}
% 	      $\inferrule* [left=\bf{AlpInp}]{
% 		P \equiv_{\alpha} Q
% 	      }{
% 		a(b).P \equiv_{\alpha} a(b).Q
% 	      }$
% 	    &
% 	      $\inferrule* [left=\bf{AlpInp1}]{
% 		Q \equiv_{\alpha} R
% 	      }{
% 		a(b).Q \equiv_{\alpha} a(c).(R\{c/b\})
% 	      }$
% 	  \end{tabular}
% 	\end{center}
% 	for inductive hypothesis $P\equiv R$ and for rule $AlpInp1$: $a(b).P \equiv_{\alpha} a(c).(R\{c/b\})$
% %       \item[$(AlpInp, AlpInp2)$] this case does not exist.
%       \item[$(AlpRes, AlpRes1)$]:
% 	\begin{center}
% 	  \begin{tabular}{ll}
% 	      $\inferrule* [left=\bf{AlpRes}]{
% 		P \equiv_{\alpha} Q
% 	      }{
% 		(\nu x)P \equiv_{\alpha} (\nu x)Q
% 	      }$
% 	    &
% 	      $\inferrule* [left=\bf{AlpRes1}]{
% 		Q \equiv_{\alpha} R
% 	      }{
% 		(\nu x)Q \equiv_{\alpha} (\nu y)(R\{y/x\})
% 	      }$
% 	  \end{tabular}
% 	\end{center}
% 	for inductive hypothesis $P\equiv R$ and for rule $AlpRes1$: $(\nu x)P \equiv_{\alpha} (\nu y)(R\{y/x\})$
% %       \item[$(AlpRes, AlpRes2)$] this case does not exist.
%       \item[$(AlpInp1, AlpInp)$]:
% 	\begin{center}
% 	  \begin{tabular}{ll}
% 	      $\inferrule* [left=\bf{AlpInp1}]{
% 		P \equiv_{\alpha} Q
% 	      }{
% 		a(b).P \equiv_{\alpha} a(c).(Q\{c/b\})
% 	      }$
% 	    &
% 	      $\inferrule* [left=\bf{AlpInp}]{
% 		Q\{c/b\} \equiv_{\alpha} R
% 	      }{
% 		a(c).(Q\{c/b\}) \equiv_{\alpha} a(c).R
% 	      }$
% 	  \end{tabular}
% 	\end{center}
% 	for inductive hypothesis and lemma \ref{alphaequivalencesubstitution}: $P\{c/b\}\equiv R$ and for rule $AlpInp2$: $a(b).(P\{c/b\}) \equiv_{\alpha} a(c).R$
%       \item[$(AlpInp1, AlpInp1)$]:
% 	\begin{center}
% 	  \begin{tabular}{ll}
% 	      $\inferrule* [left=\bf{AlpInp1}]{
% 		P \equiv_{\alpha} Q
% 	      }{
% 		a(b).P \equiv_{\alpha} a(c).(Q\{c/b\})
% 	      }$
% 	    &
% 	      $\inferrule* [left=\bf{AlpInp1}]{
% 		Q\{c/b\} \equiv_{\alpha} R
% 	      }{
% 		a(c).(Q\{c/b\}) \equiv_{\alpha} a(d).(R\{d/c\})
% 	      }$
% 	  \end{tabular}
% 	\end{center}
%       \item[$(AlpInp1, AlpInp2)$]
%       \item[$(AlpInp2, AlpInp)$]
%       \item[$(AlpInp2, AlpInp1)$]
%       \item[$(AlpInp2, AlpInp2)$]
% 
%       \item[$(AlpRes1, AlpRes)$]
%       \item[$(AlpRes1, AlpRes1)$]
%       \item[$(AlpRes1, AlpRes2)$]
%       \item[$(AlpRes2, AlpRes)$]
%       \item[$(AlpRes2, AlpRes1)$]
%       \item[$(AlpRes2, AlpRes2)$]
%     \end{description}
%   \end{proof}
\end{lemma}

\begin{theorem}
  $\alpha$ equivalence is an equivalence relation.
  \begin{proof}
    Follows from lemmas \ref{alphaEquivalenceReflexivity}, \ref{alphaEquivalenceSymmetry} and \ref{alphaEquivalenceTransitive}.
  \end{proof}
\end{theorem}



\section{Operational semantic with structural congruence}

\subsection{Early semantic with $\alpha$ conversion only}
In this subsection we introduce the early operational semantic for $\pi$ calculus with the use of a minimal structural congruence, specifically we exploit only the easy of $\alpha$ conversion.

\begin{definition}
  The \emph{early transition relation with $\alpha$ conversion} $\rightarrow\subseteq \mathbb{P}\times \mathbb{A} \times \mathbb{P}$ is the smallest relation induced by the rules in table \ref{transitionrelationearlywithalphaconversion}.

  \begin{table}
    \begin{tabular}{lll}  
      	\multicolumn{3}{l}{\line(1,0){415}}\\\\
	  $\inferrule* [left=\bf{Out}]{
	  }{
	    \overline{x}y.P \xrightarrow{\overline{x}y} P
	  }$
	&
	  $\inferrule* [left=\bf{EInp}]{
	  }{
	    x(y).P \xrightarrow{xz} P\{z/y\}
	  }$
	&
	  $\inferrule* [left=\bf{Tau}]{
	  }{
	    \tau.P \xrightarrow{\tau} P
	  }$
      \\
    \end{tabular}
    \\
    \begin{tabular}{ll}  
      \\
	  $\inferrule* [left=\bf{ParL}]{
	      P \xrightarrow{\alpha} P^{'}
	    \\
	      bn(\alpha)\cap fn(Q)=\emptyset
	  }{
	    P|Q \xrightarrow{\alpha} P^{'}|Q
	  }$
	&
	  $\inferrule* [left=\bf{ParR}]{
	      Q \xrightarrow{\alpha} Q^{'}
	    \\
	      bn(\alpha)\cap fn(P)=\emptyset
	  }{
	    P|Q \xrightarrow{\alpha} P|Q^{'}
	  }$
      \\\\
	  $\inferrule* [left=\bf{SumL}]{
	      P \xrightarrow{\alpha} P^{'}
	    \\
	      bn(\alpha)\cap fn(Q)=\emptyset
	  }{
	    P+Q \xrightarrow{\alpha} P^{'}
	  }$
	&
	  $\inferrule* [left=\bf{SumR}]{
	      Q \xrightarrow{\alpha} Q^{'}
	    \\
	      bn(\alpha)\cap fn(P)=\emptyset
	  }{
	    P+Q \xrightarrow{\alpha} Q^{'}
	  }$
      \\\\
	  $\inferrule* [left=\bf{Res}]{
	      P \xrightarrow{\alpha} P^{'}
	    \\
	      z\notin n(\alpha)
	  }{
	    (\nu z) P \xrightarrow{\alpha} (\nu z) P^{'}
	  }$
	&
	  $\inferrule* [left=\bf{Alp}]{
	      P\equiv_{\alpha}Q
	    \\
	      P\xrightarrow{\alpha}P^{'}
	  }{
	    Q\xrightarrow{\alpha}P^{'}
	  }$
      \\\\
	  $\inferrule* [left=\bf{EComL}]{
	      P \xrightarrow{xy} P^{'}
	    \\
	      Q\xrightarrow{\overline{x}y} Q^{'}
	  }{
	    P|Q \xrightarrow{\tau} P^{'}|Q^{'}
	  }$
	&
	  $\inferrule* [left=\bf{EComR}]{
	      P \xrightarrow{\overline{x}y} P^{'}
	    \\
	      Q\xrightarrow{xy} Q^{'}
	  }{
	    P|Q \xrightarrow{\tau} P^{'}|Q^{'}
	  }$
      \\\\
	  $\inferrule* [left=\bf{ClsL}]{
	      P \xrightarrow{\overline{x}(z)} P^{'}
	    \\
	      Q \xrightarrow{xz} Q^{'}
	    \\
	      z\notin fn(Q)
	  }{
	    P|Q \xrightarrow{\tau} (\nu z)(P^{'}|Q^{'})
	  }$
	&
	  $\inferrule* [left=\bf{ClsR}]{
	      P \xrightarrow{xz} P^{'}
	    \\
	      Q \xrightarrow{\overline{x}(z)} Q^{'}
	    \\
	      z\notin fn(P)
	  }{
	    P|Q \xrightarrow{\tau} (\nu z)(P^{'}|Q^{'})
	  }$
      \\\\
	  $\inferrule* [left=\bf{Ide}]{
	      A(\tilde{x}) \stackrel{def}{=} P
	    \\
	      P\{\tilde{w}/\tilde{x}\} \xrightarrow{\alpha} P^{'}
	  }{
	    A(\tilde{x}) \xrightarrow{\alpha} P^{'}
	  }$
	&
	  $\inferrule* [left=\bf{Opn}]{
	      P \xrightarrow{\overline{x}z} P^{'}
	    \\
	      z\neq x
	  }{
	    (\nu z) P \xrightarrow{\overline{x}(z)} P^{'}
	  }$
      \\	\multicolumn{2}{l}{\line(1,0){415}}
    \end{tabular}
    \caption{Early transition relation with $\alpha$ conversion but without structural congruence}
    \label{transitionrelationearlywithalphaconversion}
  \end{table}
\end{definition}

The following example shows why the condition $bn(\alpha)\cap fn(Q)=\emptyset$ in the rule $Sum$ is desirable:
\begin{example}
  without the side condition we are able to prove:
  \begin{center}
    $\inferrule* [left=\bf{ClsL}]{
	\inferrule* [left=\bf{Sum}]{
	  \inferrule* [left=\bf{Opn}]{
	    (\nu y)\overline{x}y.0
	      \xrightarrow{\overline{x}y}
		(\nu y)0
	  }{
	    (\nu y)\overline{x}y.0
	      \xrightarrow{\overline{x}(y)}
		(\nu y)0
	  }
	}{
	  ((\nu y)\overline{x}y.0) + \overline{y}x.0
	    \xrightarrow{\overline{x}(y)}
	      (\nu y)0
	}
      \\
	\inferrule* [left=\bf{EInp}]{
	}{
	  x(z).0
	    \xrightarrow{xy}
	      0
	}
    }{
      (((\nu y)\overline{x}y.0) + \overline{y}x.0)|x(z).0
	\xrightarrow{\tau}
	  (\nu y)0
    }$
  \end{center}
  but $(((\nu y)\overline{x}y.0) + \overline{y}x.0)|x(z).0 \not \equiv (\nu y)(\overline{x}y.0 + \overline{y}x.0|x(z).0)$
\end{example}



\subsection{Early semantic with structural congruence}

\begin{definition}
  The \emph{early transition relation with structural congruence} $\rightarrow\subseteq \mathbb{P}\times \mathbb{A} \times \mathbb{P}$ is the smallest relation induced by the rules in table \ref{earlysemanticwithstructuralcongruence}.

  \begin{table}
    \begin{tabular}{lll}  
      	\multicolumn{3}{l}{\line(1,0){415}}\\\\
	  $\inferrule* [left=\bf{Out}]{
	  }{
	    \overline{x}y.P \xrightarrow{\overline{x}y} P
	  }$
	&
	  $\inferrule* [left=\bf{EInp}]{
	  }{
	    x(y).P \xrightarrow{xz} P\{z/y\}
	  }$
	&
	  $\inferrule* [left=\bf{Tau}]{
	  }{
	    \tau.P \xrightarrow{\tau} P
	  }$
      \\
    \end{tabular}
    \\
    \begin{tabular}{ll}  
      \\
	  $\inferrule* [left=\bf{Par}]{
	      P \xrightarrow{\alpha} P^{'}
	    \\
	      bn(\alpha)\cap fn(Q)=\emptyset
	  }{
	    P|Q \xrightarrow{\alpha} P^{'}|Q
	  }$
	  &
	  $\inferrule* [left=\bf{Sum}]{
	      P \xrightarrow{\alpha} P^{'}
	    \\
	      bn(\alpha)\cap fn(Q)=\emptyset
	  }{
	    P+Q \xrightarrow{\alpha} P^{'}
	  }$
      \\\\
	  $\inferrule* [left=\bf{ECom}]{
	      P \xrightarrow{xy} P^{'}
	    \\
	      Q\xrightarrow{\overline{x}y} Q^{'}
	  }{
	    P|Q \xrightarrow{\tau} P^{'}|Q^{'}
	  }$
	  &
	  $\inferrule* [left=\bf{Cong}]{
	      P\equiv P^{'}
	    \\
	      P^{'}\xrightarrow{\alpha} Q
	  }{
	    P \xrightarrow{\alpha} Q
	  }$
      \\\\
	  $\inferrule* [left=\bf{Opn}]{
	      P \xrightarrow{\overline{x}z} P^{'}
	    \\
	      z\neq x
	  }{
	    (\nu z) P \xrightarrow{\overline{x}(z)} P^{'}
	  }$
	  &
	  $\inferrule* [left=\bf{Res}]{
	      P \xrightarrow{\alpha} P^{'}
	    \\
	      z\notin n(\alpha)
	  }{
	    (\nu z) P \xrightarrow{\alpha} (\nu z) P^{'}
	  }$
      \\	\multicolumn{2}{l}{\line(1,0){415}}
    \end{tabular}
    \caption{Early semantic with structural congruence}
    \label{earlysemanticwithstructuralcongruence}
  \end{table}
\end{definition}


\begin{example}
  We prove now that
  \begin{center}
    $a(x).P\; |\; (\nu b)\overline{a}b.Q\; \xrightarrow{\tau}\; (\nu b)(P\{b/x\}\; |\; Q)$
  \end{center}
  where $b\notin fn(P)$.
  This follows from
  \[
    a(x).P\; |\; (\nu b)\overline{a}b.Q\; \equiv\; (\nu b)(a(x).P\; |\; \overline{a}b.Q)
  \]
  and
  \[
    (\nu b)(a(x).P\; |\; \overline{a}b.Q) \xrightarrow{\tau} (\nu b)(P\{b/x\}\; |\; Q)
  \]
  with the rule $Cong$. We can prove the last transition in the following way:
  \[
    \inferrule* [left=Res] {
      \inferrule* [left=Com] {
	  \inferrule* [left=EInp] {
	  }{
	    a(x).P\; \xrightarrow{ab}\; P\{b/x\}
	  }
	\\
	  \inferrule* [left=Out] {
	  }{
	    \overline{a}b.Q\; \xrightarrow{\overline{a}b}\; Q
	  }
      }{
	a(x).P\; |\; \overline{a}b.Q\; \xrightarrow{\tau}\; P\{b/x\}\; |\; Q
      }
    }{
      (\nu b)(a(x).P\; |\; \overline{a}b.Q)\; \xrightarrow{\tau}\; (\nu b)(P\{b/x\}\; |\; Q)
    }
  \]

\end{example}

\begin{example}
    We want to prove now that:
    \begin{center}
      $((\nu b) a(x).P)\; |\; \overline{a}b.Q\; \xrightarrow{\tau}\; (\nu c) (P\{c/b\}\{b/x\}\; |\; Q)$
    \end{center}
    where the name $c$ is not in the free names of $Q$. We can exploit the structural congruence and get that
    \[
      ((\nu b) a(x).P) | \overline{a}b.Q\; \equiv\; (\nu c) (a(x).(P\{c/b\}) | \overline{a}b.Q)     
    \]
    then we have
    \[
	\inferrule* [left=Res] {
	  \inferrule* [left=Com]{
	      \inferrule* [left=EInp]{
	      }{
		a(x).P\{c/b\}\; \xrightarrow{ab}\; P\{c/b\}\{b/x\}
	      }
	    \\
	      \inferrule* [left=Out]{
	      }{
		\overline{a}b.Q\; \xrightarrow{\overline{a}b}\; Q
	      }
	  }{
	      (a(x).(P\{c/b\}) | \overline{a}b.Q)\; \xrightarrow{\tau}\; (P\{c/b\}\{b/x\} | Q)
	  }
	}{
	  (\nu c) (a(x).(P\{c/b\}) | \overline{a}b.Q)\; \xrightarrow{\tau}\; (\nu c) (P\{c/b\}\{b/x\} | Q)
	}
    \]
    Now we just apply the rule $Cong$ to prove the thesis.
\end{example}



\subsection{Late semantic with structural congruence}

\begin{definition}
  The \emph{late transition relation with structural congruence} $\rightarrow\subseteq \mathbb{P}\times \mathbb{A} \times \mathbb{P}$ is the smallest relation induced by the rules in table \ref{latewith}.
  \begin{table}
    \begin{tabular}{lll}
      \multicolumn{3}{l}{\line(1,0){415}}\\\\
	$\inferrule* [left=\bf{Tau}]{
	}{
	  \tau.P \xrightarrow{\tau} P
	}$
      &
	$\inferrule* [left=\bf{LInp}]{
	}{
	  x(y).P \xrightarrow{x(y)} P
	}$
      &
	$\inferrule* [left=\bf{Out}]{
	}{
	  \overline{x}y.P \xrightarrow{\overline{x}y} P
	}$
      \\
    \end{tabular}
    \\
    \begin{tabular}{ll}
      \\
	$\inferrule* [left=\bf{Opn}]{
	    P \xrightarrow{\overline{x}z} P^{'}
	  \\
	    z\neq x
	}{
	  (\nu z) P \xrightarrow{\overline{x}(z)} P^{'}
	}$
      &
	$\inferrule* [left=\bf{Res}]{
	    P \xrightarrow{\alpha} P^{'}
	  \\
	    z\notin n(\alpha)
	}{
	  (\nu z) P \xrightarrow{\alpha} (\nu z) P^{'}
	}$
    \\\\
	$\inferrule* [left=\bf{LCom}]{
	    P \xrightarrow{x(y)} P^{'}
	  \\
	    Q\xrightarrow{\overline{x}z} Q^{'}
	}{
	  P|Q \xrightarrow{\tau} P^{'}\{z/y\}|Q^{'}
	}$
      &
	$\inferrule* [left=\bf{Par}]{
	    P \xrightarrow{\alpha} P^{'}
	  \\
	    bn(\alpha)\cap fn(Q)=\emptyset
	}{
	  P|Q \xrightarrow{\alpha} P^{'}|Q
	}$
    \\\\
	$\inferrule* [left=\bf{Cong}]{
	    P\equiv P^{'}
	  \\
	    P\xrightarrow{\alpha} Q
	}{
	  P^{'} \xrightarrow{\alpha} Q^{'}
	}$
      &
	$\inferrule* [left=\bf{Sum}]{
	  P \xrightarrow{\alpha} P^{'}
	}{
	  P+Q \xrightarrow{\alpha} P^{'}
	}$
    \\\multicolumn{2}{l}{\line(1,0){415}}
    \end{tabular}
    \caption{Late semantic with structural congruence}
    \label{latewith}
  \end{table}
\end{definition}


\begin{example}
  We prove now that
  \begin{center}
    $a(x).P\; |\; (\nu b)\overline{a}b.Q\; \xrightarrow{\tau} P\{b/x\}\; |\; Q$
  \end{center}
  where $b\notin fn(P)$. This follows from
  \[
    a(x).P\; |\; (\nu b)\overline{a}b.Q\; \equiv\; (\nu b)(a(x).P\; |\; \overline{a}b.Q)
  \]
  and
  \[
    (\nu b)(a(x).P\; |\; \overline{a}b.Q) \xrightarrow{\tau} (\nu b)(P\{b/x\}\; |\; Q)
  \]
  with the rule $Cong$. We can prove the last transition in the following way:
  \[
    \inferrule* [left=Res] {
	\inferrule* [left=LCom] {
	    \inferrule* [left=LInp] {
	      b\notin fn(P)
	    }{
	      a(x).P\; \xrightarrow{ab}\; P\{b/x\}
	    }
	  \\
	    \inferrule* [left=Out] {
	    }{
	      \overline{a}b.Q\; \xrightarrow{\overline{a}b}\; Q
	    }
	}{
	  a(x).P\; |\; \overline{a}b.Q\; \xrightarrow{\tau}\; P\{b/x\}\; |\; Q
	}
      \\
	b\notin n(\tau)
    }{
      (\nu b)(a(x).P\; |\; \overline{a}b.Q)\; \xrightarrow{\tau}\; (\nu b)(P\{b/x\}\; |\; Q)
    }
  \]

\end{example}

\begin{example}
    We want to prove now that:
    \begin{center}
      $((\nu b) a(x).P)\; |\; \overline{a}b.Q\; \xrightarrow{\tau}\; (\nu c) (P\{c/b\}\{b/x\}\; |\; Q)$
    \end{center}
    where the name $c$ is not in the free names of $Q$ and is not in the names of $P$. We can exploit the structural congruence and get that
    \[
      ((\nu b) a(x).P) | \overline{a}b.Q\; \equiv\; (\nu c) (a(x).(P\{c/b\}) | \overline{a}b.Q)     
    \]
    then we have
    \[
	\inferrule* [left=Res] {
	    \inferrule* [left=LCom]{
		\inferrule* [left=LInp]{
		  b\notin fn(P\{c/b\})
		}{
		  a(x).P\{c/b\}\; \xrightarrow{ab}\; P\{c/b\}\{b/x\}
		}
	      \\
		\inferrule* [left=Out]{
		}{
		  \overline{a}b.Q\; \xrightarrow{\overline{a}b}\; Q
		}
	    }{
	      (a(x).(P\{c/b\}) | \overline{a}b.Q)\; \xrightarrow{\tau}\; (P\{c/b\}\{b/x\} | Q)
	    }
	  \\
	    c\notin n(\tau)
	}{
	  (\nu c) (a(x).(P\{c/b\}) | \overline{a}b.Q)\; \xrightarrow{\tau}\; (\nu c) (P\{c/b\}\{b/x\} | Q)
	}
    \]
    Now we just apply the rule $Cong$ to prove the thesis.
\end{example}



\section{Equivalence of the semantics}
\subsection{Equivalence of the early semantics}
In this subsection we write $\rightarrow_{1}$ for the early semantic without structural congruence, $\rightarrow_{2}$ for the early semantic with just $\alpha$ conversion and $\rightarrow_{3}$ for the early semantic with the full structural congruence. We call $R_{1}$ the set of rules for $\rightarrow_{1}$, $R_{2}$ the set of rules for $\rightarrow_{2}$ and $R_{3}$ the set of rules for $\rightarrow_{3}$. 


\begin{lemma}
  Structurally equivalent process have the same free names:
  \begin{center}
    $P\equiv Q\; \Rightarrow\; fn(Q)= fn(P)$
  \end{center}
  \begin{proof}
    The proof is easy and is an induction on the rules of structural congruence.
  \end{proof}
\end{lemma}

We would like to prove that $P\xrightarrow{\alpha}_{2}P^{'}\; \Rightarrow\; P\xrightarrow{\alpha}_{1}P^{'}$ but this is false because
\[
  \inferrule* [left=Alp]{
      \overline{x}y.x(y).0 \equiv_{\alpha} \overline{x}y.x(w).0
    \\
      \inferrule* [left=Out]{
      }{
	\overline{x}y.x(w).0\xrightarrow{\overline{x}y}_{2}x(w).0
      }
  }{
    \overline{x}y.x(y).0\xrightarrow{\overline{x}y}_{2}x(w).0
  }
\]
so we want to prove 
\[
  \overline{x}y.x(y).0\xrightarrow{\overline{x}y}_{1}x(w).0
\] 
The head of the transition has an output prefixing at the top level so the only rule we could use is $Out$, but the application of $Out$ yields 
\[
  \overline{x}y.x(y).0\xrightarrow{\overline{x}y}_{1}x(y).0
\] 
which is not want we want. But we prove some weaker results.

\begin{lemma}\label{moveAlpDown}
  Let $\twoheadrightarrow_{2}$ be the semantic of table \ref{transitionrelationearlywithalphaconversion} but without rule $Alp$. If $P\xrightarrow{\alpha}_{2}P^{'}$ then there exist a process $Q$ such that $P \equiv_{\alpha} Q \stackrel{\alpha}{\twoheadrightarrow}_{2} P^{'}$.
  \begin{proof}
    We prove by cases that in a derivation of $P\xrightarrow{\alpha}_{2}P^{'}$ we can move downward to the end of the derivation any occurrence of the rule $Alp$:
    \begin{description}
     \item[$ParL$]:
	\begin{center}
	  $\inferrule* [left=\bf{ParL}]{
	    \inferrule* [left=\bf{Alp}]{
		P \equiv_{\alpha} R
	      \\
		R \xrightarrow{\alpha} P^{'}
	    }{
	      P \xrightarrow{\alpha} P^{'}
	    }
	    \\
	    bn(\alpha) \cap fn(Q) = \emptyset
	  }{
	    P|Q \xrightarrow{\alpha} P^{'}
	  }$
	\end{center}
	became
	\begin{center}
	  $\inferrule* [left=\bf{Alp}]{
	      \inferrule* [left=\bf{AlpPar}]{
		P \equiv_{\alpha} R
	      }{
		P|Q \equiv_{\alpha} R|Q
	      }
	    \\
	      \inferrule* [left=\bf{ParL}]{
		  R \xrightarrow{\alpha} P^{'}
		\\
		  bn(\alpha) \cap fn(Q) = \emptyset
	      }{
		R|Q \xrightarrow{\alpha} P^{'}
	      }
	  }{
	    P|Q \xrightarrow{\alpha} P^{'}
	  }$
	\end{center}
      \item[$ParR, SumL, SumR$] similar.
      \item[$Alp$] 
	since $\alpha$ equivalence is transitive, we can merge any pair of consecutive instance of the rule $Alp$
     \item[$Res$]:
	\begin{center}
	  $\inferrule* [left=\bf{Res}]{
	      \inferrule* [left=\bf{Alp}]{
		  P \equiv_{\alpha} R
		\\
		  R \xrightarrow{\alpha} P^{'}
	      }{
		P \xrightarrow{\alpha} P^{'}
	      }
	    \\
	      z\notin n(\alpha)
	  }{
	    (\nu z)P \xrightarrow{\alpha} (\nu z)P^{'}
	  }$
	\end{center}
	became
	\begin{center}
	  $\inferrule* [left=\bf{Alp}]{
	      \inferrule* [left=\bf{AlpRes}]{
		P \equiv_{\alpha} R
	      }{
		(\nu z)P \equiv_{\alpha} (\nu z)R
	      }
	    \\
	      \inferrule* [left=\bf{Res}]{
		  R \xrightarrow{\alpha} P^{'}
		\\
		  z\notin n(\alpha)
	      }{
		(\nu z)R \xrightarrow{\alpha} (\nu z)P^{'}
	      }
	  }{
	    (\nu z)P \xrightarrow{\alpha} (\nu z)P^{'}
	  }$
	\end{center}
      \item[$Opn$]
	similar.
      \item[$EComL$]:
	\begin{center}
	  $\inferrule* [left=\bf{EComL}]{
	      \inferrule* [left=\bf{Alp}]{
		  P \equiv_{\alpha} R
		\\
		  R \xrightarrow{xy} P^{'}
	      }{
		P \xrightarrow{xy} P^{'}
	      }
	    \\
	      \inferrule* [left=\bf{Alp}]{
		  Q \equiv_{\alpha} S
		\\
		  S \xrightarrow{\overline{x}y} S^{'}
	      }{
		Q \xrightarrow{\overline{x}y} S^{'}
	      }
	  }{
	    P|Q \xrightarrow{\tau} P^{'}|S^{'}
	  }$
	\end{center}
	became
	\begin{center}
	  $\inferrule* [left=\bf{Alp}]{
	      \inferrule* [left=\bf{AlpPar}]{
		  P \equiv_{\alpha} R
		\\
		  Q \equiv_{\alpha} S
	      }{
		P|Q \equiv_{\alpha} R|S
	      }
	    \\
	      \inferrule* [left=\bf{EComL}]{
		  R \xrightarrow{xy} P^{'}
		\\
		  S \xrightarrow{\overline{x}y} S^{'}
	      }{
		R|S \xrightarrow{\tau} S^{'}
	      }
	  }{
	    P|Q \xrightarrow{\tau} P^{'}|S^{'}
	  }$
	\end{center}
      \item[$EComR$] 
	similar.
      \item[$ClsR$]:
	\begin{center}
	  $\inferrule* [left=\bf{ClsR}]{
	      \inferrule* [left=\bf{Alp}]{
		  P \equiv_{\alpha} R
		\\
		  R \xrightarrow{xy} P^{'}
	      }{
		P \xrightarrow{xy} P^{'}
	      }
	    \\
	      \inferrule* [left=\bf{Alp}]{
		  Q \equiv_{\alpha} S
		\\
		  S \xrightarrow{\overline{x}(y)} S^{'}
	      }{
		Q \xrightarrow{\overline{x}(y)} S^{'}
	      }
	    \\
	      z\notin fn(P)
	  }{
	    P|Q \xrightarrow{\tau} (\nu z)(P^{'}|S^{'})
	  }$
	\end{center}
	became
	\begin{center}
	  $\inferrule* [left=\bf{Alp}]{
	      \inferrule* [left=\bf{AlpPar}]{
		  P \equiv_{\alpha} R
		\\
		  Q \equiv_{\alpha} S
	      }{
		P|Q \equiv_{\alpha} R|S
	      }
	    \\
	      \inferrule* [left=\bf{ClsR}]{
		  R \xrightarrow{xy} P^{'}
		\\
		  S \xrightarrow{\overline{x}y} S^{'}
		\\
		  z\notin fn(R)
	      }{
		R|S \xrightarrow{\tau} (\nu y)(P^{'}|S^{'})
	      }
	  }{
	    P|Q \xrightarrow{\tau} (\nu y)(P^{'}|S^{'})
	  }$
	\end{center}
      \item[$ClsL$]
	similar.
      \item[$Ide$]:
	\begin{center}
	  $\inferrule* [left=\bf{Ide}]{
	      A(\tilde{x}) \stackrel{def}{=} P
	    \\
	      \inferrule* [left=\bf{Alp}]{
		  P\{\tilde{w}/\tilde{x}\} \equiv_{\alpha} R
		\\
		  R \xrightarrow{\alpha} P^{'}
	      }{
		P\{\tilde{w}/\tilde{x}\} \xrightarrow{\alpha} P^{'}
	      }
	  }{
	    A(\tilde{w}) \xrightarrow{\alpha} P^{'}
	  }$
	\end{center}
	we can add a new definition $A(\tilde{w})\stackrel{def}{=} R$ and this derivation became:
	\begin{center}
	  $\inferrule* [left=\bf{Ide}]{	
	      A(\tilde{w})\stackrel{def}{=} R
	    \\
	      R \xrightarrow{\alpha} P^{'}
	  }{
	    A(\tilde{w}) \xrightarrow{\alpha} P^{'}
	  }$
	\end{center}
    \end{description}
  \end{proof}
\end{lemma}


\begin{theorem}
  If $P\xrightarrow{\alpha}_{2}P^{'}$ then there exists a process $Q$ such that $P \equiv_{\alpha} Q \xrightarrow{\alpha}_{1} P^{'}$.
  \begin{proof}
    This result follows from lemma \ref{moveAlpDown} observing that $\twoheadrightarrow_{2}\subseteq \rightarrow_{1}$.
  \end{proof}
\end{theorem}




\begin{theorem}
  If $P\xrightarrow{\alpha}_{2}P^{'}$ then there exists a process $P^{''}$ such that $P\xrightarrow{\alpha}_{1}P^{''}$ and $P^{''}\equiv_{\alpha}P^{'}$
  \begin{proof}
    For lemma \ref{moveAlpDown} there exists a process $Q$ such that $P \equiv_{\alpha} Q \stackrel{\alpha}{\twoheadrightarrow}_{2} P^{'}$. The proof proceed by cases on the last rule used in the derivation of $P \equiv_{\alpha} Q$:
    \begin{description}
      \item[$AlpIde$]
 	in this case there is an identifier such that $P=A(\tilde{x})=Q$ so the conclusion holds.
      \item[$AlpInp$]
 	in this case $P=x(y).P_{1}$, $Q=x(y).Q_{1}$, $P_{1}\equiv_{\alpha} Q_{1}$ and $\alpha=xz$. For rule $EInp$: $x(y).Q_{1}\xrightarrow{xz} Q_{1}$.
      \item[$AlpInp1$]:
	\begin{center}
 	  $\inferrule* [left=\bf{Alp}]{
	      \inferrule* [left=\bf{AlpInp1}]{
		P_{1} \equiv_{\alpha} Q_{1}
	      }{
 		a(x).P_{1} \equiv_{\alpha} a(y).(Q_{1}\{y/x\})
	      }
	    \\
	      \inferrule* [left=\bf{EInp}]{
	      }{
		a(y).Q_{1}\{y/x\} \xrightarrow{az}_{2} Q_{1}\{y/x\}\{z/y\}
	      }
 	  }{
	    a(x).P_{1} \xrightarrow{az}_{2} Q_{1}\{y/x\}\{z/y\}
 	  }$
 	\end{center}
 	For rule $EInp$: $a(x).P_{1} \xrightarrow{az}_{1} P_{1}$. $P_{1} \equiv_{\alpha} Q_{1}$ imply $P_{1} \equiv_{\alpha} Q_{1}\{y/x\}\{z/y\}$
       \item[$AlpInp2$]:
	\begin{center}
 	  $\inferrule* [left=\bf{Alp}]{
	      \inferrule* [left=\bf{AlpInp2}]{
		P_{1} \equiv_{\alpha} Q_{1}
	      }{
 		a(x).(P_{1}\{x/y\}) \equiv_{\alpha} a(y).Q_{1}
	      }
	    \\
	      \inferrule* [left=\bf{EInp}]{
	      }{
		a(y).Q_{1} \xrightarrow{az}_{2} Q_{1}\{z/y\}
	      }
 	  }{
	    a(x).(P_{1}\{x/y\}) \xrightarrow{az}_{2} Q_{1}\{z/y\}
 	  }$
 	\end{center}
 	For rule $EInp$: $a(x).(P_{1}\{x/y\}) \xrightarrow{az}_{1} P_{1}\{x/y\}\{z/x\}$. $P_{1} \equiv_{\alpha} Q_{1}$ imply $P_{1}\{x/y\}\{z/x\} \equiv_{\alpha} Q_{1}\{z/y\}$
       \item[$AlpOut$]
 	in this case $P=\overline{a}x.P_{1}$, $Q=\overline{a}x.Q_{1}$, $P_{1} \equiv_{\alpha} Q_{1}$ and $\alpha=\overline{a}x$. For rule $Out$: $\overline{a}x.P_{1}\xrightarrow{\overline{a}x} P_{1}$ and $\overline{a}x.Q_{1}\xrightarrow{\overline{a}x} Q_{1}$.
       \item[$AlpTau$]
 	similar.
       \item[$AlpPar$]
 	in this case $P=P_{1}|P_{2}$, $Q=Q_{1}|Q_{2}$, $P_{1} \equiv_{\alpha} Q_{1}$ and $P_{2} \equiv_{\alpha} Q_{2}$. The last rule used in the derivation of $Q_{1}|Q_{2} \stackrel{\alpha}{\twoheadrightarrow}_{2} P^{'}|Q_{2}$ can be:
 	\begin{description}
 	  \item[$ParL$]
 	    in this case $P_{1}\equiv Q_{1} \stackrel{\alpha}{\twoheadrightarrow}_{2} P^{'}$ and for inductive hypothesis $P_{1} \xrightarrow{\alpha}_{1} P^{'}$. For rule $ParL$: $P_{1}|P_{2} \xrightarrow{\alpha}_{1} P^{'}|P_{2} \equiv_{\alpha} P^{'}|Q_{2}$
 	  \item[$ParR$]
 	    similar.
 	  \item[$EComL$]
 	    in this case $P_{1}\equiv Q_{1} \stackrel{xy}{\twoheadrightarrow}_{2} Q_{1}^{'}$ and $P_{2}\equiv Q_{2} \stackrel{\overline{x}y}{\twoheadrightarrow}_{2} Q_{2}^{'}$, for inductive hypothesis $P_{1} \xrightarrow{xy}_{1} Q_{1}^{'}$ and $P_{2} \xrightarrow{\overline{x}y}_{1} Q_{2}^{'}$. For rule $EComL$: $P_{1}|P_{2} \xrightarrow{\tau}_{1} Q_{1}^{'}|Q_{2}^{'}$
 	  \item[$EComR$]
 	    similar.
 	\end{description}
       \item[$AlpSum$]
 	similar.
       \item[$AlpRes$]
 	in this case $P=(\nu x)P_{1}$, $Q=(\nu x)Q_{1}$ and $P_{1} \equiv Q_{1}$. The last rule used in the derivation of $Q \stackrel{\alpha}{\twoheadrightarrow}_{2} P^{'}$ can be:
 	\begin{description}
 	  \item[$Res$]
 	    in this case $P_{1}\equiv Q_{1} \stackrel{\alpha}{\twoheadrightarrow}_{2} Q_{1}^{'}$ and $x\notin n(\alpha)$. For inductive hypothesis $P_{1} \xrightarrow{\alpha}_{1} Q_{1}^{'}$. For rule $Res$: $(\nu x)P_{1} \xrightarrow{\alpha}_{1} (\nu x)Q_{1}^{'}$
 	  \item[$Opn$]
 	    similar.
 	\end{description}
      \item[$AlpRes1$]
 	The last rule used in the derivation of $Q \stackrel{\alpha}{\twoheadrightarrow}_{2} P^{'}$ can be:
 	\begin{description}
 	  \item[$Res$]:
 	    \begin{center}
 	      $\inferrule* [left=\bf{Alp}]{
 		  \inferrule* [left=\bf{AlpRes1}]{
 		    P_{1} \equiv_{\alpha} Q_{1}
 		  }{
 		    (\nu x)P_{1} \equiv_{\alpha} (\nu y)(Q_{1}\{y/x\})
 		  }
 		\\
 		  \inferrule* [left=\bf{Res}]{
 		      Q_{1}\{y/x\} \xrightarrow{\alpha}_{2} Q^{'}
 		    \\
 		      y\notin n(\alpha)
 		  }{
 		    (\nu y)(Q_{1}\{y/x\}) \xrightarrow{\alpha}_{2} (\nu y)Q^{'}
 		  }
 	      }{
 		(\nu x)P_{1} \xrightarrow{\alpha}_{2} (\nu y)Q^{'}
 	      }$
 	    \end{center}
 	    $P_{1} \equiv_{\alpha} Q_{1}$ imply $P_{1}\{y/x\} \equiv_{\alpha} Q_{1}\{y/x\}$, for inductive hypothesis $P_{1}\{y/x\} \xrightarrow{\alpha}_{1} Q^{'}$, now we can derive
 	    \begin{center}
 	      $\inferrule* [left=\bf{ResAlp1}]{
		  \inferrule* [left=\bf{Res}]{
		      P_{1}\{y/x\} \xrightarrow{\alpha}_{1} Q^{'}
 		    \\
 		      y\notin n(\alpha)
		  }{
		    (\nu y)(P_{1}\{y/x\}) \xrightarrow{\alpha}_{1} (\nu y)Q^{'}
		  }
 	      }{
 		(\nu x)P_{1} \xrightarrow{\alpha}_{1} (\nu y)Q_{1}^{'}
 	      }$
 	    \end{center}
 	  \item[$Opn$]
 	    similar.
 	\end{description}
       \item[$AlpRes2$]
 	The last rule used in the derivation of $Q \stackrel{\alpha}{\twoheadrightarrow}_{2} P^{'}$ can be:
 	\begin{description}
 	  \item[$Res$]:
 	    \begin{center}
 	      $\inferrule* [left=\bf{Alp}]{
 		  \inferrule* [left=\bf{AlpRes2}]{
 		    P_{1} \equiv_{\alpha} Q_{1}
 		  }{
 		    (\nu x)(P_{1}\{x/y\}) \equiv_{\alpha} (\nu y)Q_{1}
 		  }
 		\\
 		  \inferrule* [left=\bf{Res}]{
 		      Q_{1} \xrightarrow{\alpha}_{2} Q^{'}
 		    \\
 		      y\notin n(\alpha)
 		  }{
 		    (\nu y)Q_{1} \xrightarrow{\alpha}_{2} (\nu y)Q^{'}
 		  }
 	      }{
 		(\nu x)(P_{1}\{x/y\}) \xrightarrow{\alpha}_{2} (\nu y)Q^{'}
 	      }$
 	    \end{center}
 	    $P_{1} \equiv_{\alpha} Q_{1} \xrightarrow{\alpha}_{2} Q^{'}$ imply for inductive hypothesis that $P_{1} \xrightarrow{\alpha}_{1} Q^{'}$, now we can derive
 	    \begin{center}
 	      $\inferrule* [left=\bf{ResAlp2}]{
		  P_{1} \xrightarrow{\alpha}_{1} Q^{'}
		\\
		  y\notin n(\alpha)
 	      }{
 		(\nu x)P_{1}\{x/y\} \xrightarrow{\alpha}_{1} (\nu y)Q^{'}
 	      }$
 	    \end{center}
 	  \item[$Opn$]
 	    similar.
 	\end{description}
       \item[$AlpZero$]
 	this case does not exist.
     \end{description}
  \end{proof}
\end{theorem}



\begin{theorem}
  $P\xrightarrow{\alpha}_{1}P^{'}\; \Rightarrow\; P\xrightarrow{\alpha}_{2}P^{'}$
  \begin{proof}
    The proof can go by induction on the length of the derivation of a transaction, and then both the base case and the inductive case proceed by cases on the last rule used in the derivation. However it is not necessary to show all the details of the proof because the rules in $R_{2}$ are almost the same as the rules in $R_{1}$, the only difference is that in $R_{2}$ we have the rule $Alp$ instead of $ResAlp1$ and $OpnAlp$. The rule $Alp$ can mimic the rule $ResAlp1$ in the following way:
	\[
	  \inferrule *{
	      (\nu z)P\equiv_{\alpha} (\nu w)P\{w/z\}
	    \\
	      w\notin n(P)
	    \\
	      (\nu w)P\{w/z\}\;
		\xrightarrow{xz}\;
		  P^{'}
	  }{
	    (\nu z)P\; 
	      \xrightarrow{xz}\;
		P^{'}
	  }
	\]
	And the rule $Alp$ can mimic the rule $OpnAlp$ in the following way:
	\[
	  \inferrule *{
	      (\nu z)P\equiv_{\alpha} (\nu w)P\{w/z\}
	    \\
	      w\notin n(P)
	    \\
	      (\nu w)P\{w/z\}\;
		\xrightarrow{\overline{x}(w)}\;
		  P^{'}
	    \\
	      x\neq w\neq z
	  }{
	    (\nu z)P\; 
	      \xrightarrow{\overline{x}(w)}\;
		P^{'}
	  }
	\]
  \end{proof}
\end{theorem}


\begin{lemma}\label{OpenClose}
  If $P \xrightarrow{\overline{x}(y)}_{2} P^{'}$ then there is a process $R$ such that $P \equiv R \xrightarrow{\overline{x}(y)}_{2} P^{'}$ and the last rule in this derivation is the instance of rule $Opn$ used to open the scope of $y$.
  \begin{proof}
    The derivation of $P \xrightarrow{\overline{x}(y)}_{2} P^{'}$ must contain an instance of $Opn$. The proof consists in showing that we can move this instance of $Opn$ downward in the inference tree of $P \xrightarrow{\overline{x}(y)}_{2} P^{'}$. The proof goes by induction on the depth of the derivation of $P \xrightarrow{\overline{x}(y)}_{2} P^{'}$ and then by cases on the last rule applied:
    \begin{description}
      \item[$Opn$] if the derivation ends with $Opn$ then the conclusion holds.
% 	\begin{center}
% 	  $\inferrule* [left=\bf{ClsL}]{
% 	      \inferrule* [left=\bf{Opn}]{
% 		  P_{1} \xrightarrow{\overline{x}y}_{2} P_{1}^{'}
% 	      \\
% 		x\neq y
% 	    }{
% 	      (\nu y)P_{1} \xrightarrow{\overline{x}(y)}_{2} P_{1}^{'}
% 	    }
% 	    \\
% 	      Q \xrightarrow{xy}_{2} Q^{'}
% 	    \\
% 	      y\notin fn(Q)
% 	  }{
% 	    ((\nu y)P_{1})|Q \xrightarrow{\tau}_{2} (\nu y)(P_{1}^{'}|Q^{'})
% 	  }$
% 	\end{center}
% 	became:
% 	\begin{center}
% 	  $\inferrule* [left=\bf{Res}]{
% 	      \inferrule* [left=\bf{EComL}]{
% 		P_{1} \xrightarrow{\overline{x}y}_{2} P_{1}^{'}
% 	      \\
% 		Q \xrightarrow{xy}_{2} Q^{'}
% 	    }{
% 	      P_{1}|Q \xrightarrow{\tau}_{2} P_{1}|Q
% 	    }
% 	  }{
% 	    (\nu y)(P_{1}|Q) \xrightarrow{\tau}_{2} (\nu y)(P_{1}^{'}|Q^{'})
% 	  }$
% 	\end{center}
% 	Since $y\notin fn(Q)$ it holds that $(\nu y)(\overline{x}y.P_{1})|Q\equiv (\nu y)(\overline{x}y.P_{1}|Q)$.
      \item[$SumL$]:
	\begin{center}
	  $\inferrule* [left=\bf{SumL}]{
	      \inferrule* [left=\bf{Opn}]{
		P_{1} \xrightarrow{\overline{x}y}_{2} P^{'}
	      \\
		x\neq y
	      }{
		(\nu y)P_{1} \xrightarrow{\overline{x}(y)}_{2} P^{'}
	      }
	    \\
	      bn(\overline{x}(y)) \cap fn(R) = \emptyset
	  }{
	    P=(\nu y)P_{1}+R \xrightarrow{\overline{x}(y)}_{2} P^{'}
	  }$
	\end{center}
	became:
	\begin{center}
	  $\inferrule* [left=\bf{Opn}]{
	      \inferrule* [left=\bf{SumL}]{
		  P_{1} \xrightarrow{\overline{x}y}_{2} P^{'}
	      }{
		P_{1}+R \xrightarrow{\overline{x}y}_{2} P^{'}
	      }
	    \\
	      x\neq y
	  }{
	    (\nu y)(P_{1}+R) \xrightarrow{\overline{x}(y)}_{2} P^{'}
	  }$
	\end{center}
	$bn(\overline{x}(y)) \cap fn(R) = \emptyset$ imply $y\notin fn(R)$ and so $(\nu y)(P_{1}+R) \equiv (\nu y)P_{1}+R$.
      \item[$SumR$] 
	symmetric to the previous case.
      \item[$ParL$]:
	\begin{center}
	  $\inferrule* [left=\bf{ParL}]{
	      \inferrule* [left=\bf{Opn}]{
		P_{1} \xrightarrow{\overline{x}y}_{2} P^{'}
	      \\
		x\neq y
	      }{
		(\nu y)P_{1} \xrightarrow{\overline{x}(y)}_{2} P^{'}
	      }
	    \\
	      bn(\overline{x}(y)) \cap fn(R) = \emptyset
	  }{
	    P=(\nu y)P_{1}|R \xrightarrow{\overline{x}(y)}_{2} P^{'}|R
	  }$
	\end{center}
	became:
	\begin{center}
	  $\inferrule* [left=\bf{Opn}]{
	      \inferrule* [left=\bf{ParL}]{
		  P_{1} \xrightarrow{\overline{x}y}_{2} P^{'}
	      }{
		P_{1}|R \xrightarrow{\overline{x}y}_{2} P^{'}|R
	      }
	    \\
	      x\neq y
	  }{
	    (\nu y)(P_{1}|R) \xrightarrow{\overline{x}(y)}_{2} P^{'}|R
	  }$
	\end{center}
	$bn(\overline{x}(y)) \cap fn(R) = \emptyset$ imply $y\notin fn(R)$ and so $(\nu y)(P_{1}|R) \equiv (\nu y)P_{1}|R$.
      \item[$ParR$]
	symmetric to the previous case.
      \item[$Res$]:
	\begin{center}
	  $\inferrule* [left=\bf{Res}]{
	      \inferrule* [left=\bf{Opn}]{
		P_{1} \xrightarrow{\overline{x}y}_{2} P^{'}
	      \\
		x\neq y
	      }{
		(\nu y)P_{1} \xrightarrow{\overline{x}(y)}_{2} P^{'}
	      }
	    \\
	      w \notin n(\overline{x}(y))
	  }{
	    P=(\nu w)(\nu y)P_{1} \xrightarrow{\overline{x}(y)}_{2} (\nu w)P^{'}
	  }$
	\end{center}
	became:
	\begin{center}
	  $\inferrule* [left=\bf{Opn}]{
	      \inferrule* [left=\bf{Res}]{
		  P_{1} \xrightarrow{\overline{x}y}_{2} P^{'}
		\\
		  w \notin n(\overline{x}y)
	      }{
		(\nu w)P_{1} \xrightarrow{\overline{x}y}_{2} (\nu w)P^{'}
	      }
	    \\
	      x\neq y
	  }{
	    (\nu y)(\nu w)P_{1} \xrightarrow{\overline{x}(y)}_{2} (\nu w)P^{'}
	  }$
	\end{center}
	$(\nu y)(\nu w)P_{1} \equiv (\nu w)(\nu y)P_{1}$.
      \item[$Alp(1)$]:
	\begin{center}
	  $\inferrule* [left=\bf{Alp}]{
	      \inferrule* {
		P_{1} \equiv_{\alpha} R_{1}
	      }{
		(\nu y)P_{1} \equiv_{\alpha} (\nu y)R_{1}
	      }
	    \\
	      \inferrule* [left=\bf{Opn}]{
		R_{1} \xrightarrow{\overline{x}y}_{2} R_{1}^{'}
	      \\
		x\neq y
	      }{
		(\nu y)R_{1} \xrightarrow{\overline{x}(y)}_{2} R_{1}^{'}
	      }
	  }{
	    P=(\nu y)P_{1} \xrightarrow{\overline{x}(y)}_{2} R_{1}^{'}=P^{'}
	  }$
	\end{center}
	became:
	\begin{center}
	  $\inferrule* [left=\bf{Opn}]{
	      \inferrule* [left=\bf{Alp}]{
		  P_{1} \equiv_{\alpha} R_{1}
		\\
		  R_{1} \xrightarrow{\overline{x}y}_{2} R_{1}^{'}
	      }{
		P_{1} \xrightarrow{\overline{x}y}_{2} R_{1}^{'}
	      }
	    \\
	      x\neq y
	  }{
	    (\nu y)P_{1} \xrightarrow{\overline{x}(y)}_{2} R_{1}^{'}
	  }$
	\end{center}
      \item[$Alp(2)$]:
	\begin{center}
	  $\inferrule* [left=\bf{Alp}]{
	      \inferrule* {
		  P_{1}\{w/y\} \equiv_{\alpha} R_{1}
		\\
		  w \notin n(P_{1})
	      }{
		(\nu w)P_{1} \equiv_{\alpha} (\nu y)R_{1}
	      }
	    \\
	      \inferrule* [left=\bf{Opn}]{
		R_{1} \xrightarrow{\overline{x}y}_{2} R_{1}^{'}
	      \\
		x\neq y
	      }{
		(\nu y)R_{1} \xrightarrow{\overline{x}(y)}_{2} R_{1}^{'}
	      }
	  }{
	    P=(\nu w)P_{1} \xrightarrow{\overline{x}(y)}_{2} R_{1}^{'}=P^{'}
	  }$
	\end{center}
	became:
	\begin{center}
	  $\inferrule* [left=\bf{Opn}]{
	      \inferrule* [left=\bf{Alp}]{
		  P_{1}\{y/w\} \equiv_{\alpha} R_{1}
		\\
		  R_{1} \xrightarrow{\overline{x}y}_{2} R_{1}^{'}
	      }{
		P_{1}\{y/w\} \xrightarrow{\overline{x}y}_{2} R_{1}^{'}
	      }
	    \\
	      x\neq y
	  }{
	    (\nu y)P_{1}\{y/w\} \xrightarrow{\overline{x}(y)}_{2} R_{1}^{'}
	  }$
	\end{center}
	and $(\nu y)P_{1}\{y/w\} \equiv (\nu w)P_{1}$
    \end{description}
  \end{proof}
\end{lemma}


\begin{lemma}
  $P\xrightarrow{\alpha}_{2} P^{'}$ imply that there exist processes $Q, Q^{'}$ such that  $P \equiv Q \xrightarrow{\alpha}_{3} Q^{'} \equiv P^{'}$
  \begin{proof}
	First we prove $P\xrightarrow{\alpha}_{2}P^{'}\; \Rightarrow\; \exists P^{''}: P^{'}\equiv P^{''}\; and\; P\xrightarrow{\alpha}_{3}P^{''}$. The proof is by induction on the length of the derivation of $P\xrightarrow{\alpha}_{2}P^{'}$, and then both the base case and the inductive case proceed by cases on the last rule used.
	\begin{description}
	  \item[base case]
	    in this case the rule used can be one of the following $Out, EInp, Tau$ which are also in $R_{3}$ so a derivation of $P\xrightarrow{\alpha}_{2}P^{'}$ is also a derivation of $P\xrightarrow{\alpha}_{3}P^{'}$
	  \item[inductive case]:
	    \begin{itemize}
	      \item 
		the last rule used can be one in $R_{2}\cap R_{3}=\{Res, Opn\}$ and so for example we have 
		\[
		  \inferrule* [left=Res]{
		      P\xrightarrow{\alpha}_{2} P^{'}
		    \\
		      z\notin n(\alpha)
		  }{
		    (\nu z) P\xrightarrow{\alpha}_{2}(\nu z) P^{'}
		  }
		\]
		we apply the inductive hypothesis on $P\xrightarrow{\alpha}_{2} P^{'}$ and get $\exists P^{''}$ such that $P^{'}\equiv P^{''}$ and $P\xrightarrow{\alpha}_{3} P^{''}$. The proof we want is:
		\[
		  \inferrule* [left=Res]{
		      P\xrightarrow{\alpha}_{3} P^{''}
		    \\
		      z\notin n(\alpha)
		  }{
		    (\nu z) P\xrightarrow{\alpha}_{3}(\nu z) P^{''}
		  }
		\]
		and $(\nu z) P^{''}\equiv (\nu z) P^{'}$
	      \item
		the last rule used can be one in $\{ParL, ParR, SumL, SumR, EComL, EComR\}$, in this case we can proceed as in the previous case and if necessary add an application of $Cong$ thus exploiting the commutativity of sum or parallel composition. For example
		\[
		  \inferrule* [left=ParR]{
		      Q\xrightarrow{\alpha}_{2}Q^{'}
		    \\
		      bn(\alpha)\cap fn(Q)=\emptyset
		  }{
		      P|Q\xrightarrow{\alpha}_{2}P|Q^{'}
		  }
		\]	
		now we apply the inductive hypothesis to $Q\xrightarrow{\alpha}_{2}Q^{'}$ and get $Q\xrightarrow{\alpha}_{3}Q^{''}$ for a $Q^{''}$ such that $Q^{'}\equiv Q^{''}$. The proof we want is
		\[
		  \inferrule* [left=Str]{
		      P|Q\equiv Q|P
		    \\
		      \inferrule* [left=Par]{
			  Q\xrightarrow{\alpha}_{3}Q^{''}
			\\
			  bn(\alpha)\cap fn(Q)=\emptyset
		      }{
			  Q|P\xrightarrow{\alpha}_{3}Q^{''}|P
		      }
		    }{
		      P|Q\xrightarrow{\alpha}_{3}Q^{''}|P
		    }
		\]
		and $Q^{''}|P\equiv P|Q^{'}$
	      \item
		if the last rule used is $Cns$:
		\[
		    \inferrule* [left=Cns]{
			A(\tilde{x}|\tilde{z})\stackrel{def}{=}P
		      \\
			P\{\tilde{y}/\tilde{x}\}\xrightarrow{\alpha}_{2}P^{'}
		    }{
		      A(\tilde{y}|\tilde{z})\xrightarrow{\alpha}_{2}P^{'}
		    }
		\]
		we apply the inductive hypothesis on the premise and get $P\{\tilde{y}/\tilde{x}\}\xrightarrow{\alpha}_{3}P^{''}$ such that $P^{''}\equiv P^{'}$. Now the proof we want is
		\[
		    \inferrule* [left=Str]{
			A(\tilde{y}|\tilde{z})\equiv P\{\tilde{y}/\tilde{x}\}
		      \\
			P\{\tilde{y}/\tilde{x}\}\xrightarrow{\alpha}_{3}P^{''}
		    }{
		      A(\tilde{y}|\tilde{z})\xrightarrow{\alpha}_{3}P^{''}
		    }
		\]		
	      \item
		if the last rule is $Alp$, then we just notice that this rule is a particular case of $Cong$
% 	      \item
% 		if the last rule is $ClsL$(the case for $ClsR$ is simmetric) then we have
% 		\[
% 		    \inferrule* [left=ClsL]{
% 			P\xrightarrow{\overline{x}(z)}_{2}P^{'}
% 		      \\
% 			Q\xrightarrow{xz}_{2}Q^{'}
% 		      \\
% 			z\notin fn(Q)
% 		    }{
% 		      P|Q\xrightarrow{\tau}_{2}(\nu z)(P^{'}|Q^{'})
% 		    }
% 		\]
% 		there is no easy way to mimic this rule with the rules in $R_{3}$. But if in the derivation tree we have an introduction of the bound output $\overline{x}(z)$ followed directly by an elimination of the same bound output such as:
% 		\[
% 		    \inferrule* [left=ClsL]{
% 			\inferrule* [left=Opn]{
% 			    P\xrightarrow{\overline{x}z}_{2}P^{'}
% 			  \\
% 			    z\neq x
% 			}{
% 			  (\nu z)P\xrightarrow{\overline{x}(z)}_{2}P^{'}
% 			}
% 		      \\
% 			Q\xrightarrow{xz}_{2}Q^{'}
% 		      \\
% 			z\notin fn(Q)
% 		    }{
% 		      ((\nu z)P)|Q\xrightarrow{\tau}_{2}(\nu z)(P^{'}|Q^{'})
% 		    }
% 		\]
% 		we can apply the inductive hypothesis and get that 
% 		\[
% 		  P\xrightarrow{\overline{x}z}_{3}P^{''}\; and\; Q\xrightarrow{xz}_{3}Q^{''}
% 		\]
% 		where $P^{'}\equiv P^{''}$ and $Q^{'}\equiv Q^{''}$, so we create the needed proof in the following way
% 		\[
% 		    \inferrule* [left=Str]{
% 			(\nu z)(P|Q)\equiv ((\nu z)P)|Q
% 		      \\
% 			\inferrule* [left=Res]{
% 			  \inferrule* [left=Com]{
% 			      P\xrightarrow{\overline{x}z}_{3}P^{''}
% 			    \\
% 			      Q\xrightarrow{xz}_{3}Q^{''}
% 			  }{
% 			    P|Q\xrightarrow{\tau}_{3}P^{''}|Q^{''}
% 			  }
% 			}{
% 			  (\nu z)(P|Q)\xrightarrow{\tau}_{3}(\nu z)(P^{''}|Q^{''})
% 			}
% 		    }{
% 		      ((\nu z)P)|Q\xrightarrow{\tau}_{3}(\nu z)(P^{''}|Q^{''})
% 		    }
% 		\]
% 		We can always take a derivation tree in $R_{2}$ and move downward each occurrence of $Opn$ until we find the appropriate occurrence of $ClsL$. In this process we might need to use the structural congruence, in particular the scope extension axioms. We can attempt to prove that in the following way:
% 		\[
% 		  P\xrightarrow{\overline{x}(z)}_{2}P^{'}\; \Rightarrow\; \exists R: (\nu z)R\equiv P
% 		\]
% 		and if $(\nu z)R\xrightarrow{\overline{x}(z)}_{2}P^{'}$ then there exists a derivation tree for this transition such that the last rule used is $Opn$
	      \item
		if the last rule is $ClsL$(the case for $ClsR$ is simmetric) then we have
		\[
		    \inferrule* [left=ClsL]{
			P\xrightarrow{\overline{x}(z)}_{2}P^{'}
		      \\
			Q\xrightarrow{xz}_{2}Q^{'}
		      \\
			z\notin fn(Q)
		    }{
		      P|Q\xrightarrow{\tau}_{2}(\nu z)(P^{'}|Q^{'})
		    }
		\]
		$P\xrightarrow{\overline{x}(z)}_{2}P^{'}$ for lemma \ref{OpenClose} imply that there exist processes $(\nu z)R$ such that $P \equiv (\nu z)R \xrightarrow{\overline{x}(z)}_{2} P^{'}$ and the derivation of $(\nu z)R \xrightarrow{\tau}_{2} R^{'}$ ends with the instance of $Opn$ that opens the scope of $z$. So 
		\[
		  \inferrule* [left=Res]{
		    \inferrule* [left=EComL]{
			R \xrightarrow{\overline{x}z}_{2} P^{'}
		      \\
			Q \xrightarrow{xz}_{2} Q^{'}
		    }{
		      R|Q \xrightarrow{\tau}_{2} P^{'}|Q^{'}
		    }
		  }{
		    (\nu z)(R|Q) \xrightarrow{\tau}_{2} (\nu z)(P^{'}|Q^{'})
		  }
		\]
		$P \equiv (\nu z)R$ and $z \notin fn(Q)$ imply $(\nu z)(R|Q) \equiv P|Q$. The conclusion follows after applying the inductive hypothesis on $(\nu z)(R|Q) \xrightarrow{\tau}_{3} (\nu z)(P^{'}|Q^{'})$ and the transitivity of structural congruence.
	    \end{itemize}	    
	\end{description}
  \end{proof}
  \label{2inCong3}
\end{lemma}

\begin{theorem}
  $P\xrightarrow{\alpha}_{2} P^{'}$ imply that there exist processes $Q^{'}$ such that  $P \xrightarrow{\alpha}_{3} Q^{'} \equiv P^{'}$.
  \begin{proof}
    For lemma \ref{2inCong3} there exist processes $Q, Q^{'}$ such that $P \equiv Q \xrightarrow{\alpha}_{3} Q^{'} \equiv P^{'}$. So for rule $Cong$: $P \xrightarrow{\alpha}_{3} Q^{'} \equiv P^{'}$.
  \end{proof}
\end{theorem}

% VALE IL TEOREMA SEGUENTE?
% \begin{theorem}
%   $P\xrightarrow{\alpha}_{2} P^{'}$ imply $P \xrightarrow{\alpha}_{3} Q^{'}$
% \end{theorem}



\begin{lemma}\label{moveCongDowPi}
  Let $\twoheadrightarrow_{3}$ be the semantic in table \ref{earlysemanticwithstructuralcongruence} but without rule $Cong$. $P\xrightarrow{\alpha}_{3}P^{'}$ imply that there exist a process $Q$ such that $P\equiv Q \stackrel{\alpha}{\twoheadrightarrow}_{3} P^{'}$.
  \begin{proof}
    The proof needs to show that in any proof tree we can move downward any instance of a rule $Cong$ until the proof tree has only on instance of the rule $Cong$ and this is at the end. There are some cases to consider:
    \begin{description}
      \item[$Sum$]:
	\begin{center}
	  $\inferrule* [left=\bf{Sum}]{
	      \inferrule* [left=\bf{Cong}]{
		  P\equiv R
		\\
		  R \xrightarrow{\alpha} P^{'}
	      }{
		P \xrightarrow{\alpha} P^{'}
	      }
	    \\
	      bn(\alpha) \cap fn(Q) = \emptyset
	  }{
	    P+Q \xrightarrow{\alpha} P^{'}
	  }$
	\end{center}
	became:
	\begin{center}
	  $\inferrule* [left=\bf{Cong}]{
	      \inferrule* {
		P \equiv R
	      }{
		P+Q \equiv R+Q
	      }
	    \\
	      \inferrule* [left=\bf{Sum}]{
		  R \xrightarrow{\alpha} P^{'}
		\\
		  bn(\alpha) \cap fn(Q) = \emptyset
	      }{
		R+Q \xrightarrow{\alpha} P^{'}
	      }
	  }{
	    P+Q \xrightarrow{\alpha} P^{'}
	  }$
	\end{center}
      \item[$Par$]:
	\begin{center}
	  $\inferrule* [left=\bf{Par}]{
	      \inferrule* [left=\bf{Cong}]{
		  P\equiv R
		\\
		  R \xrightarrow{\alpha} P^{'}
	      }{
		P \xrightarrow{\alpha} P^{'}
	      }
	    \\
	      bn(\alpha) \cap fn(Q) = \emptyset
	  }{
	    P|Q \xrightarrow{\alpha} P^{'}|Q
	  }$
	\end{center}
	became:
	\begin{center}
	  $\inferrule* [left=\bf{Cong}]{
	      \inferrule* {
		P \equiv R
	      }{
		P|Q \equiv R|Q
	      }
	    \\
	      \inferrule* [left=\bf{Par}]{
		  R \xrightarrow{\alpha} P^{'}
		\\
		  bn(\alpha) \cap fn(Q) = \emptyset
	      }{
		R|Q \xrightarrow{\alpha} P^{'}|Q
	      }
	  }{
	    P+Q \xrightarrow{\alpha} P^{'}|Q
	  }$
	\end{center}
      \item[$ECom$]:
	\begin{center}
	  $\inferrule* [left=\bf{ECom}]{
	      \inferrule* [left=\bf{Cong}]{
		  P\equiv R
		\\
		  R \xrightarrow{xy} P^{'}
	      }{
		P \xrightarrow{xy} P^{'}
	      }
	    \\
	      Q \xrightarrow{\overline{x}y} Q^{'}
	  }{
	    P|Q \xrightarrow{\tau} P^{'}|Q^{'}
	  }$
	\end{center}
	became:
	\begin{center}
	  $\inferrule* [left=\bf{Cong}]{
	      \inferrule* {
		P \equiv R
	      }{
		P|Q \equiv R|Q
	      }
	    \\
	      \inferrule* [left=\bf{ECom}]{
		  R \xrightarrow{xy} P^{'}
		\\
		  Q \xrightarrow{\overline{x}y} Q^{'}
	      }{
		R|Q \xrightarrow{\tau} P^{'}|Q^{'}
	      }
	  }{
	    P|Q \xrightarrow{\alpha} P^{'}|Q^{'}
	  }$
	\end{center}
      \item[$Res$]:
	\begin{center}
	  $\inferrule* [left=\bf{Res}]{
	      \inferrule* [left=\bf{Cong}]{
		  P\equiv R
		\\
		  R \xrightarrow{\alpha} P^{'}
	      }{
		P \xrightarrow{\alpha} P^{'}
	      }
	    \\
	      x\notin n(\alpha)
	  }{
	    (\nu x)P \xrightarrow{\alpha} (\nu x)P^{'}
	  }$
	\end{center}
	became:
	\begin{center}
	  $\inferrule* [left=\bf{Cong}]{
	      \inferrule* {
		P \equiv R
	      }{
		(\nu x)P \equiv (\nu x)R
	      }
	    \\
	      \inferrule* [left=\bf{Res}]{
		  R \xrightarrow{\alpha} P^{'}
		\\
		  x\notin n(\alpha)
	      }{
		(\nu x)R \xrightarrow{\alpha} (\nu x)P^{'}
	      }
	  }{
	    (\nu x)P \xrightarrow{\alpha} (\nu x)P^{'}
	  }$
	\end{center}
      \item[$Opn$]:
	\begin{center}
	  $\inferrule* [left=\bf{Opn}]{
	      \inferrule* [left=\bf{Cong}]{
		  P\equiv R
		\\
		  R \xrightarrow{\overline{y}x} P^{'}
	      }{
		P \xrightarrow{\overline{y}x} P^{'}
	      }
	  }{
	    (\nu x)P \xrightarrow{\overline{y}(x)} P^{'}
	  }$
	\end{center}
	became:
	\begin{center}
	  $\inferrule* [left=\bf{Cong}]{
	      \inferrule* {
		P \equiv R
	      }{
		(\nu x)P \equiv (\nu x)R
	      }
	    \\
	      \inferrule* [left=\bf{Opn}]{
		  R \xrightarrow{\overline{y}x} P^{'}
	      }{
		(\nu x)R \xrightarrow{\overline{y}(x)} P^{'}
	      }
	  }{
	    (\nu x)P \xrightarrow{\overline{y}(x)} P^{'}
	  }$
	\end{center}
      \item[$Cong$]:
	\begin{center}
	  $\inferrule* [left=\bf{Cong}]{
	      P\equiv R
	    \\
	      \inferrule* [left=\bf{Cong}]{
		  R\equiv S
		\\
		  S \xrightarrow{\alpha} P^{'}
	      }{
		R \xrightarrow{\alpha} P^{'}
	      }
	  }{
	    P \xrightarrow{\alpha} P^{'}
	  }$
	\end{center}
	became:
	\begin{center}
	  $\inferrule* [left=\bf{Cong}]{
	      \inferrule* {
		  P \equiv R
		\\
		  R \equiv S
	      }{
		P \equiv S
	      }
	    \\
	      S \xrightarrow{\alpha} P^{'}
	  }{
	    P \xrightarrow{\alpha} P^{'}
	  }$
	\end{center}
% 	since structural congruence is transitive we can assume that in every derivation of a transition there are no two consecutive instance of the rule $Cong$.
    \end{description}
  \end{proof}
\end{lemma}


\begin{theorem}
  $P\xrightarrow{\alpha}_{3}P^{'}$ imply that there exist a proces $Q$ such that $P\equiv Q \xrightarrow{\alpha}_{2} P^{'}$.	
  \begin{proof}
    This is a direct consequence of lemma \ref{moveCongDowPi} observing that $\twoheadrightarrow_{3} \subseteq \rightarrow_{2}$.
  \end{proof}
\end{theorem}


% We would like to prove that: $P\xrightarrow{\alpha}_{3}P^{'}$ imply that there exist a process $Q^{'}$ such that $P \xrightarrow{\alpha}_{2} Q^{'}\equiv P^{'}$. But this is false becuase: BECAUSE? E' VERO O FALSO?
% 
% 
% 
% 
% 
% \begin{theorem}
%   $P\xrightarrow{\alpha}_{3}P^{'}$ imply that there exist a process $Q^{'}$ such that $P \xrightarrow{\alpha}_{2} Q^{'}\equiv P^{'}$
%   \begin{proof}
% LA DIMOSTRAZIONE E' INCOMPLETA O FORSE ADDIRITTURA ASSURDA
%     The proof is by induction on the rules in table \ref{earlysemanticwithstructuralcongruence}.
%     \begin{description}
%       \item[$Out$]:
% 	\begin{center}
% 	  \begin{tabular}{lll}
% 	  $\inferrule* [left=\bf{Out}]{
% 	  }{
% 	    \overline{x}y.P \xrightarrow{\overline{x}y}_{3} P
% 	  }$
% 	  &
% 	  imply
% 	  &
% 	  $\inferrule* [left=\bf{Out}]{
% 	  }{
% 	    \overline{x}y.P \xrightarrow{\overline{x}y}_{2} P
% 	  }$
% 	  \end{tabular}
% 	\end{center}
%       \item[$EInp, Tau$] similar to the previous case.
%       \item[$Res$]:
% 	\begin{center}
% 	  \begin{tabular}{lll}
% 	    $\inferrule* [left=\bf{Res}]{
% 	      P \xrightarrow{\alpha}_{3} P^{'}
% 	    \\
% 	      z\notin n(\alpha)
% 	    }{
% 	    (\nu z) P \xrightarrow{\alpha}_{3} (\nu z) P^{'}
% 	    }$
% 	  &
% 	    imply
% 	  &
% 	    $\inferrule* [left=\bf{Res}]{
% 	      P \xrightarrow{\alpha}_{2} P^{'}
% 	    \\
% 	      z\notin n(\alpha)
% 	    }{
% 	    (\nu z) P \xrightarrow{\alpha}_{2} (\nu z) Q^{'}
% 	    }$
% 	  \end{tabular}
% 	\end{center}
% 	Note that we used the inductive hypothesis in: $P \xrightarrow{\alpha}_{3} P^{'}$ imply $P \xrightarrow{\alpha}_{2} Q^{'}\equiv P^{'}$. Since $Q^{'} \equiv P^{'}$ then $(\nu z)Q^{'} \equiv (\nu z)P^{'}$
%       \item[$Opn, ECom$] similar to the previous case.
%       \item[$Par$]:
% 		\[
% 		  \inferrule* [left=\bf{Par}]{
% 		      P\xrightarrow{\alpha}_{3}P^{'}
% 		    \\
% 		      bn(\alpha)\cap fn(Q)=\emptyset
% 		  }{
% 		      P|Q\xrightarrow{\alpha}_{3}P^{'}|Q
% 		  }
% 		\]	
% 		now we apply the inductive hypothesis to $P\xrightarrow{\alpha}_{3}P^{'}$ and get $P\xrightarrow{\alpha}_{2}Q^{'}$ for a $P^{''}$ such that $P^{'}\equiv Q^{'}$. The proof we want is
% 		\[
% 		      \inferrule* [left=\bf{ParL}]{
% 			  P\xrightarrow{\alpha}_{2}P^{''}
% 			\\
% 			  bn(\alpha)\cap fn(Q)=\emptyset
% 		      }{
% 			  P|Q\xrightarrow{\alpha}_{2}P|Q^{''}
% 		      }
% 		\]
% 		and $Q^{''}|P\equiv P|Q^{'}$
%       \item[$Sum$] similar since $Sum=SumL$
%       \item[$Cong$]:
% 	Having in mine lemma \ref{moveCongDowPi}, we proceed by cases on the last rule used to prove the structural congruence:
% 	\begin{description}
% 	  \item[$Symmetry$] 
% 	    the rules of structural congruence are symmetric so there is no rule for symmetry.
% 	  \item[$Commutativity\; of\; sum$]:
% 	    \[\inferrule* [left=\bf{Cong}]{
% 		\inferrule* [left=\bf{SumCom}]{
% 		}{
% 		  P + R \equiv R + P
% 		}
% 	      \\
% 		\inferrule* [left=\bf{Sum}]{
% 		    R \stackrel{\alpha}{\twoheadrightarrow}_{3} R^{'}
% 		  \\
% 		    bn(\alpha) \cap fn(P) = \emptyset
% 		}{
% 		  R + P \stackrel{\alpha}{\twoheadrightarrow}_{3} R^{'}
% 		}
% 	    }{
% 	      P + R \xrightarrow{\alpha}_{3} R^{'}
% 	    }\]
% 	    became
% 	    \[\inferrule* [left=\bf{SumR}]{
% 		    R \xrightarrow{\alpha}_{2} R^{'}
% 		  \\
% 		    bn(\alpha) \cap fn(P) = \emptyset
% 		}{
% 		  P + R \xrightarrow{\alpha}_{2} R^{'}
% 	    }\]
% 	  \item[$Commutativity\; of\; parallel$]
% 	    There are two cases to consider. The first is when the rule used in the premises of $Cong$ is $Par$. This case is similar to the previous. The second case is when the rule used in the premises is $ECom$:
% 	    \[\inferrule* [left=\bf{Cong}]{
% 		\inferrule* [left=\bf{ParCom}]{
% 		}{
% 		  P | R \equiv R | P
% 		}
% 	      \\
% 		\inferrule* [left=\bf{ECom}]{
% 		    R \xrightarrow{xy}_{3} R^{'}
% 		  \\
% 		    P \xrightarrow{\overline{x}y}_{3} P^{'}
% 		}{
% 		  R|P \stackrel{\tau}{\twoheadrightarrow}_{3} R^{'}|P^{'}
% 		}
% 	    }{
% 	      P|R \xrightarrow{\alpha}_{3} R^{'}|P^{'}
% 	    }\]
% 	    became
% 	    \[\inferrule* [left=\bf{EComR}]{
% 		    P \xrightarrow{\overline{x}y}_{3} P^{'}
% 		  \\
% 		    R \xrightarrow{xy}_{3} R^{'}
% 		}{
% 		  P|R \xrightarrow{\tau}_{2} P^{'}|R^{'}
% 	    }\]
% 	  \item[$Commutativity\; of\; restriction$]
% 	    Let $(\nu x)(\nu y)P \equiv (\nu y)(\nu x)P$ and $(\nu x)(\nu y)P \stackrel{\alpha}{\twoheadrightarrow}_{3} P^{'}$. The last two rule instance in a derivation of this transition can be: ($Opn$, $Res$), ($Res$, $Opn$) or ($Res$, $Res$). In each case we can swap the first instance with the second one and get $(\nu y)(\nu x)P \stackrel{\alpha}{\twoheadrightarrow}_{3} Q^{'}\equiv P^{'}$
% 	  \item[$Associativity\; of\; sum$]
% 	    \[\inferrule* [left=\bf{Cong}]{
% 		\inferrule* [left=\bf{Sum}]{
% 		    \inferrule* [left=\bf{Sum}]{
% 			P \stackrel{\alpha}{\twoheadrightarrow}_{3} R^{'}
% 		      \\
% 			bn(\alpha) \cap fn(R) = \emptyset
% 		    }{
% 		      P + R \stackrel{\alpha}{\twoheadrightarrow}_{3} R^{'}
% 		    }
% 		  \\
% 		    bn(\alpha) \cap fn(Q) = \emptyset
% 		}{
% 		  (P + R) + Q \stackrel{\alpha}{\twoheadrightarrow}_{3} R^{'}
% 		}
% 	    }{
% 	      P + (R + Q) \xrightarrow{\alpha}_{3} R^{'}
% 	    }\]
% 	    became
% 	    \[\inferrule* [left=\bf{SumL}]{
% 		    P \stackrel{\alpha}{\twoheadrightarrow}_{3} R^{'}
% 		  \\
% 		    bn(\alpha) \cap fn(R+Q) = \emptyset
% 		}{
% 		  P + (R+Q) \xrightarrow{\alpha}_{2} R^{'}
% 	    }\]
% 	    and
% 	    \[\inferrule* [left=\bf{Cong}]{
% 		\inferrule* [left=\bf{Sum}]{
% 		    P \stackrel{\alpha}{\twoheadrightarrow}_{3} R^{'}
% 		  \\
% 		    bn(\alpha) \cap fn(R+Q) = \emptyset
% 		}{
% 		  P + (R + Q) \stackrel{\alpha}{\twoheadrightarrow}_{3} R^{'}
% 		}
% 	    }{
% 	      (P + R) + Q \xrightarrow{\alpha}_{3} R^{'}
% 	    }\]
% 	    became
% 	    \[\inferrule* [left=\bf{SumL}]{
% 	      \inferrule* [left=\bf{SumL}]{
% 		    P \stackrel{\alpha}{\twoheadrightarrow}_{3} R^{'}
% 		  \\
% 		    bn(\alpha) \cap fn(R) = \emptyset
% 	      }{
% 		  P + R \xrightarrow{\alpha}_{2} R^{'}
% 	      }
% 	      \\
% 		bn(\alpha) \cap fn(Q) = \emptyset
% 	    }{
% 		(P + R) + Q \xrightarrow{\alpha}_{2} R^{'}
% 	    }\]
% 	  \item[$Associativity\; of\; parallel$]
% 	    There are two cases to consider. The first is when the rule used in the premises of $Cong$ is $Par$. This case is similar to the previous. The second case is when the rule used in the premises is $ECom$:
% 	    \[\inferrule* [left=\bf{Cong}]{
% 		\inferrule* [left=\bf{ECom}]{
% 		    P \xrightarrow{xy}_{3} P^{'}
% 		  \\
% 		    \inferrule* [left=\bf{Par}]{
% 		      R \xrightarrow{\overline{x}y}_{3} R^{'}
% 		    }{
% 		      R|Q \xrightarrow{\overline{x}y}_{3} R^{'}|Q
% 		    }
% 		}{
% 		  P|(R|Q) \stackrel{\tau}{\twoheadrightarrow}_{3} P^{'}|(R^{'}|Q)
% 		}
% 	    }{
% 	      (P|R)|Q\equiv P|(R|Q) \xrightarrow{tau}_{3} P^{'}|(R^{'}|Q)
% 	    }\]
% 	    became
% 	    \[\inferrule* [left=\bf{Par}]{\inferrule* [left=\bf{EComL}]{
% 		    P \xrightarrow{xy}_{2} P^{'}
% 		  \\
% 		    R \xrightarrow{\overline{x}y}_{2} R^{'}
% 		}{
% 		  P|R \xrightarrow{\tau}_{2} P^{'}|R^{'}
% 	    }}{
% 		(P|R)|Q \xrightarrow{\tau}_{2} (P^{'}|R^{'})|Q
% 	    }\]
% 	    The other case of associativity for parallel is similar.
% 	  \item[$Identifier$]
% 	    easy.
% 	  \item[$Congruence\; for\; output$]
% 	    easy.
% 	  \item[$Congruence\; for\; tau$]
% 	    easy.
% 	  \item[$Congruence\; for\; input$]
% 	  \item[$Congruence\; for\; restriction$]
% 	  \item[$Congruence\; for\; parallel$]
% 	  \item[$Congruence\; for\; sum$]
% 	\end{description}
% 		we proceed by cases on the premise $Q\xrightarrow{\alpha}_{3}P^{'}$. In the cases of prefix we can just use the appropriate prefix rule of $R_{2}$ and get rid of the $Cong$. In the other cases we can move upward the occurrence of $Cong$, after that we have one or two smaller derivation trees that are suitable to application of the inductive hypothesis and finally we apply some appropriate rules in $R_{2}$.
% 		\begin{description}
% 		  \item[Out]
% 		    Since we are using the rule $Out$, $Q=\overline{x}y.Q_{1}$ for some $Q_{1}$. $Q\equiv P$ means for the inversion lemma for structural congruence that $P=\overline{x}y.P_{1}$ for some $P_{1}\equiv Q_{1}$. The last part of the derivation tree is 
% 		    \[
% 		      \inferrule* [left=Str]{
% 			  \overline{x}y.P_{1}\equiv \overline{x}y.Q_{1}
% 			\\
% 			  \inferrule* [left=Out]{
% 			  }{
% 			    \overline{x}y.Q_{1}\xrightarrow{\overline{x}y}_{3}Q_{1}
% 			  }
% 		      }{
% 			\overline{x}y.P_{1}\xrightarrow{\overline{x}y}_{3}Q_{1}
% 		      }
% 		    \]
% 		    So we get 
% 		    \[
% 		      \inferrule* [left=Out]{
% 		      }{
% 			\overline{x}y.P_{1}\xrightarrow{\overline{x}y}_{2}P_{1}
% 		      }
% 		    \]
% 		    where $P_{1}\equiv Q_{1}$
% 		  \item[Tau] this is very similar to the previous case
% 		  \item[EInp]
% 		    Since we are using the rule $EInp$, $Q=x(y).Q_{1}$ for some $Q_{1}$. From $Q\equiv P$ using the inversion lemma for structural congruence we can have two cases:
% 		    \begin{itemize}
% 		      \item 
% 			$P=x(y).P_{1}$ for some $P_{1}\equiv Q_{1}$. The last part of the derivation tree is 
% 			\[
% 			  \inferrule* [left=Str]{
% 			      x(y).P_{1}\equiv x(y).Q_{1}
% 			    \\
% 			      \inferrule* [left=EInp]{
% 			      }{
% 				x(y).Q_{1}\xrightarrow{xw}_{3}Q_{1}\{w/y\}
% 			      }
% 			  }{
% 			    x(y).P_{1}\xrightarrow{xw}_{3}Q_{1}\{w/y\}
% 			  }
% 			\]
% 			So we get 
% 			\[
% 			  \inferrule* [left=EInp]{
% 			  }{
% 			    x(y).P_{1}\xrightarrow{xw}_{2}P_{1}\{w/y\}
% 			  }
% 			\]
% 			where $P_{1}\equiv Q_{1}$ implies $P_{1}\{w/y\}\equiv Q_{1}\{w/y\}$
% 		      \item
% 			$P=x(z).P_{1}$ for some $P_{1}\equiv Q_{1}\{z/y\}$. The last part of the derivation tree is 
% 			\[
% 			  \inferrule* [left=Str]{
% 			      x(z).P_{1}\equiv x(y).Q_{1}
% 			    \\
% 			      \inferrule* [left=EInp]{
% 			      }{
% 				x(y).Q_{1}\xrightarrow{xw}_{3}Q_{1}\{w/y\}
% 			      }
% 			  }{
% 			    x(z).P_{1}\xrightarrow{xw}_{3}Q_{1}\{w/y\}
% 			  }
% 			\]
% 			So we get 
% 			\[
% 			  \inferrule* [left=EInp]{
% 			  }{
% 			    x(z).P_{1}\xrightarrow{xw}_{2}P_{1}\{w/z\}
% 			  }
% 			\]
% 			where $P_{1}\equiv Q_{1}\{z/y\}$ implies $P_{1}\{w/z\}\equiv Q_{1}\{z/y\}\{w/z\}\equiv Q_{1}\{w/y\}$
% 		    \end{itemize}
% 		  \item[Par]
% 		    Since we are using the rule $Par$, $Q=Q_{1}|Q_{2}$ for some $Q_{1},Q_{2}$. $Q\equiv P$ means for the inversion lemma for structural congruence that $P=P_{1}|P_{2}$ for some $P_{1},P_{2}$ such that $P_{1}\equiv Q_{1}$ and $P_{2}\equiv Q_{2}$. The last part of the derivation tree is 
% 		    \[
% 		      \inferrule* [left=Str]{
% 			  P_{1}|P_{2}\equiv Q_{1}|Q_{2}
% 			\\
% 			  \inferrule* [left=Par]{
% 			      Q_{1}\xrightarrow{\alpha}_{3}Q_{1}^{'}
% 			    \\
% 			      bn(\alpha)\cap fn(Q_{2})=\emptyset
% 			  }{
% 			    Q_{1}|Q_{2}\xrightarrow{\alpha}_{3}Q_{1}^{'}|Q_{2}
% 			  }
% 		      }{
% 			P_{1}|P_{2}\xrightarrow{\alpha}_{3}Q_{1}^{'}|Q_{2}
% 		      }
% 		    \]
% 		    the first step is the creation of this proof tree:
% 		    \[
% 			  \inferrule* [left=Str]{
% 			      P_{1}\equiv Q_{1}
% 			    \\
% 			      Q_{1}\xrightarrow{\alpha}_{3}Q_{1}^{'}
% 			  }{
% 			    P_{1}\xrightarrow{\alpha}_{3}Q_{1}^{'}
% 			  }
% 		    \]
% 		    which is smaller then the inductive case, so we apply the inductive hypothesis and get $P_{1}\xrightarrow{\alpha}_{2}Q_{1}^{''}$ where $Q_{1}^{'}\equiv Q_{1}^{''}$. The last step is 
% 		    \[
% 		      \inferrule* [left=ParL]{
% 			  P_{1}\xrightarrow{\alpha}_{2}Q_{1}^{''}
% 			\\
% 			  bn(\alpha)\cap fn(P_{2})=\emptyset
% 		      }{
% 			P_{1}|P_{2}\xrightarrow{\alpha}_{2}Q_{1}^{''}|P_{2}
% 		      }
% 		    \]
% 		  \item[Sum] this case is very similar to the previous.
% 		  \item[ECom] this case is also similar to the $Par$ case.
% 		  \item[Res]
% 		    Since we are using the rule $Res$, $Q=(\nu z)Q_{1}$ for some $Q_{1}$ and some $z$. $(\nu z)Q_{1}\equiv P$ means thanks to the inversion lemma for structural congruence that one of the following cases holds:
%                     \begin{itemize}
% 		      \item
% 			$P=(\nu z)P_{1}$ for some $P_{1}$ such that $P_{1}\equiv Q_{1}$. The last part of the derivation tree is 
% 			\[
% 			  \inferrule* [left=Str]{
% 			      (\nu z)P_{1}\equiv (\nu z)Q_{1}
% 			    \\
% 			      \inferrule* [left=Res]{
% 				  Q_{1}\xrightarrow{\alpha}_{3}Q_{1}^{'}
% 				\\
% 				  z\notin n(\alpha)
% 			      }{
% 				(\nu z)Q_{1}\xrightarrow{\alpha}_{3}(\nu z)Q_{1}^{'}
% 			      }
% 			  }{
% 			    (\nu z)P_{1}\xrightarrow{\alpha}_{3}(\nu z)Q_{1}^{'}
% 			  }
% 			\]
% 			first we create the following proof:
% 			\[
% 			      \inferrule* [left=Str]{
% 				  P_{1}\equiv Q_{1}
% 				\\
% 				  Q_{1}
% 				    \xrightarrow{\alpha}_{3}
% 				      Q_{1}^{'}
% 			      }{
% 				P_{1}\xrightarrow{\alpha}_{3}Q_{1}^{'}
% 			      }
% 			\]
% 			now we can apply the inductive hypothesis and get $P_{1}\xrightarrow{\alpha}_{2}Q_{1}^{''}$ where $Q_{1}^{'}\equiv Q_{1}^{''}$. The last step is 
% 			\[
% 			  \inferrule* [left=Res]{
% 			      P_{1}\xrightarrow{\alpha}_{2}Q_{1}^{''}
% 			    \\
% 			      z\notin n(\alpha)
% 			  }{
% 			    (\nu z)P_{1}\xrightarrow{\alpha}_{2}(\nu z)Q_{1}^{''}
% 			  }
% 			\]
% 		      \item
% 			$P=(\nu y)P_{1}$ for some $P_{1}$ such that $P_{1}\{z/y\}\equiv Q_{1}$. The last part of the derivation tree is 
% 			\[
% 			  \inferrule* [left=Str]{
% 			      (\nu y)P_{1}\equiv (\nu z)Q_{1}
% 			    \\
% 			      \inferrule* [left=Res]{
% 				  Q_{1}\xrightarrow{\alpha}_{3}Q_{1}^{'}
% 				\\
% 				  z\notin n(\alpha)
% 			      }{
% 				(\nu z)Q_{1}\xrightarrow{\alpha}_{3}(\nu z)Q_{1}^{'}
% 			      }
% 			  }{
% 			    (\nu y)P_{1}\xrightarrow{\alpha}_{3}(\nu z)Q_{1}^{'}
% 			  }
% 			\]
% 			we create the following proof of $P_{1}\{z/y\}\xrightarrow{\alpha}_{3}Q_{1}^{'}$:
% 			\[
% 			  \inferrule* [left=Str]{
% 			      P_{1}\{z/y\}\equiv Q_{1}
% 			    \\
% 			      Q_{1}\xrightarrow{\alpha}_{3}Q_{1}^{'}
% 			  }{
% 			    P_{1}\{z/y\}\xrightarrow{\alpha}_{3}Q_{1}^{'}
% 			  }
% 			\]
% 			this proof tree is shorter then the one of $(\nu y)P_{1}\xrightarrow{\alpha}_{3}(\nu z)Q_{1}^{'}$ so we can apply the inductive hypothesis and get that there exists a process $Q_{1}^{''}$ such that 
% 			\[
% 			  P_{1}\{z/y\}\xrightarrow{\alpha}_{2}Q_{1}^{''}\; and\; Q_{1}^{''}\equiv Q_{1}^{'}
% 			\]
% 			now we can apply the rules $Res$ and $Alp$ to get the desired proof tree:
% 			\[
% 			  \inferrule* [left=Alp]{
% 				(\nu z)P_{1}\{z/y\}\equiv_{\alpha} (\nu y)P_{1}
% 			      \\
% 				\inferrule* [left=Res]{
% 				    P_{1}\{z/y\}
% 				      \xrightarrow{\alpha}_{2}
% 					Q_{1}^{''}
% 				  \\
% 				    z\notin(\alpha)
% 				  }{
% 				    (\nu z)P_{1}\{z/y\}
% 				      \xrightarrow{\alpha}_{2}
% 					(\nu z)Q_{1}^{''}
% 				  }
% 			  }{
% 			    (\nu y)P_{1}\xrightarrow{\alpha}_{2}(\nu z)Q_{1}^{''}
% 			  }
% 			\]
%                       \end{itemize}
% 		  \item[Opn]
% 		    Since we are using the rule $Opn$, $Q=(\nu z) Q_{1}$ for some $Q_{1}$. $(\nu z) Q_{1}\equiv P$ means for the inversion lemma for structural congruence that
% 		    \begin{itemize}
% 		      \item
% 			$P=(\nu z) P_{1}$ for some $P_{1}$ such that $P_{1}\equiv Q_{1}$. The last part of the derivation tree is 
% 			\[
% 			  \inferrule* [left=Str]{
% 			      (\nu z) P_{1}\equiv (\nu z) Q_{1}
% 			    \\
% 			      \inferrule* [left=Opn]{
% 				  Q_{1} \xrightarrow{\overline{x}z} Q_{1}^{'}
% 				\\
% 				  z\neq x
% 			      }{
% 				(\nu z) Q_{1} \xrightarrow{\overline{x}(z)} Q_{1}^{'}
% 			      }
% 			  }{
% 			    (\nu z) P_{1} \xrightarrow{\overline{x}(z)} Q_{1}^{'}
% 			  }
% 			\]
% 			first:
% 			\[
% 			      \inferrule* [left=Str]{
% 				  P_{1}\equiv Q_{1}
% 				\\
% 				  Q_{1}\xrightarrow{\overline{x}z}_{3}Q_{1}^{'}
% 			      }{
% 				P_{1}\xrightarrow{\overline{x}z}_{3}Q_{1}^{'}
% 			      }
% 			\]
% 			then we apply the inductive hypothesis and get $P_{1}\xrightarrow{\overline{x}z}_{2}Q_{1}^{''}$ where $Q_{1}^{'}\equiv Q_{1}^{''}$. The last step is 
% 			\[
% 			  \inferrule* [left=Res]{
% 			      P_{1}\xrightarrow{\overline{x}z}_{2}Q_{1}^{''}
% 			    \\
% 			      z\neq x
% 			  }{
% 			    (\nu z)P_{1}\xrightarrow{\overline{x}z}_{2}Q_{1}^{''}
% 			  }
% 			\]
% 		      \item
% 			$P=(\nu z) P_{1}$ for some $P_{1}$ such that $P_{1}\equiv Q_{1}$. The last part of the derivation tree is 
% 			\[
% 			  \inferrule* [left=Str]{
% 			      (\nu z) P_{1}\equiv (\nu z) Q_{1}
% 			    \\
% 			      \inferrule* [left=Opn]{
% 				  Q_{1} \xrightarrow{\overline{x}z} Q_{1}^{'}
% 				\\
% 				  z\neq x
% 			      }{
% 				(\nu z) Q_{1} \xrightarrow{\overline{x}(z)} Q_{1}^{'}
% 			      }
% 			  }{
% 			    (\nu z) P_{1} \xrightarrow{\overline{x}(z)} Q_{1}^{'}
% 			  }
% 			\]
% 			the first step is:
% 			\[
% 			      \inferrule* [left=Str]{
% 				  P_{1}\equiv Q_{1}
% 				\\
% 				  Q_{1}\xrightarrow{\overline{x}z}_{3}Q_{1}^{'}
% 			      }{
% 				P_{1}\xrightarrow{\overline{x}z}_{3}Q_{1}^{'}
% 			      }
% 			\]
% 			then we apply the inductive hypothesis and get $P_{1}\xrightarrow{\overline{x}z}_{2}Q_{1}^{''}$ where $Q_{1}^{'}\equiv Q_{1}^{''}$. The last step is 
% 			\[
% 			  \inferrule* [left=Res]{
% 			      P_{1}\xrightarrow{\overline{x}z}_{2}Q_{1}^{''}
% 			    \\
% 			      z\neq x
% 			  }{
% 			    (\nu z)P_{1}\xrightarrow{\overline{x}z}_{2}Q_{1}^{''}
% 			  }
% 			\]
% 		      \item
% 			$P=(\nu y) P_{1}$ for some $P_{1}$ such that $P_{1}\{z/y\}\equiv Q_{1}$. The last part of the derivation tree is 
% 			\[
% 			  \inferrule* [left=Str]{
% 			      (\nu y) P_{1}\equiv (\nu z) Q_{1}
% 			    \\
% 			      \inferrule* [left=Opn]{
% 				  Q_{1} \xrightarrow{\overline{x}z}_{3} Q_{1}^{'}
% 				\\
% 				  z\neq x
% 			      }{
% 				(\nu z) Q_{1} \xrightarrow{\overline{x}(z)}_{3} Q_{1}^{'}
% 			      }
% 			  }{
% 			    (\nu y) P_{1} \xrightarrow{\overline{x}(z)}_{3} Q_{1}^{'}
% 			  }
% 			\]
% 			we can create the following proof of $P_{1}\{z/y\} \xrightarrow{\overline{x}z}_{3} Q_{1}^{'}$:
% 			\[
% 			  \inferrule* [left=Str]{
% 			      P_{1}\{z/y\}\equiv Q_{1}
% 			    \\
% 			      Q_{1} 
% 				\xrightarrow{\overline{x}z}_{3} 
% 				  Q_{1}^{'}
% 			  }{
% 			    P_{1}\{z/y\} 
% 			      \xrightarrow{\overline{x}z}_{3} 
% 				Q_{1}^{'}
% 			  }
% 			\]
% 			this proof tree is shorter then the one of $(\nu y) P_{1} \xrightarrow{\overline{x}(z)}_{3} Q_{1}^{'}$ so we can apply the inductive hypothesis and get that there exists a process $Q_{1}^{''}$ such that
% 			\[
% 			  Q_{1}^{''}\equiv Q_{1}^{'}\; and\;
% 			    P_{1}\{z/y\} 
% 			      \xrightarrow{\overline{x}z}_{2} 
% 				Q_{1}^{''} 
% 			\]
% 			so now we only need to apply the rules $Opn$ and $Alp$:
% 			\[
% 			  \inferrule* [left=Alp]{
% 			      (\nu y) P_{1}\equiv_{\alpha} (\nu z) P_{1}\{z/y\}
% 			    \\
% 			      \inferrule* [left=Opn]{
% 				  P_{1}\{z/y\} \xrightarrow{\overline{x}z}_{2} Q_{1}^{''}
% 				\\
% 				  z\neq x
% 			      }{
% 				(\nu z) P_{1}\{z/y\} \xrightarrow{\overline{x}(z)}_{3} Q_{1}^{''}
% 			      }
% 			  }{
% 			    (\nu y) P_{1} \xrightarrow{\overline{x}(z)}_{2} Q_{1}^{''}
% 			  }
% 			\]
% 		  \end{itemize}
% 		\end{description}
%     \end{description}
%   \end{proof}
% \end{theorem}
% 
% 
% \begin{theorem}
%   If $P \xrightarrow{\alpha}_{3} P^{'}$ then $P \xrightarrow{\alpha}_{2} Q^{'}\equiv P^{'}$.
%   \begin{proof}
%     For lemma \ref{moveCongDowPi} and for reflexivity of structural congruence we can assume that $P\equiv R \stackrel{\alpha}{\twoheadrightarrow}_{3} P^{'}$. 
% %     However we can avoid to consider the case where the proof tree of $P\equiv R$ ends with an application of the reflexivity because $\twoheadrightarrow_{3} \subseteq \rightarrow_{2}$. We can also avoid the case where the proof tree of $P\equiv R$ ends with $Trans$ because if a proof tree contains:
% % 	    \begin{center}
% % 	      $\inferrule* [left=\bf{Cong}]{
% % 		  \inferrule* [left=\bf{Trans}]{
% % 		      P \equiv Q
% % 		    \\
% % 		      Q \equiv R
% % 		  }{
% % 		    P \equiv R
% % 		  }
% % 		\\
% % 		  R \xrightarrow{\alpha}_{3} R^{'}
% % 	      }{
% % 		P \xrightarrow{\alpha}_{3} R^{'}
% % 	      }$
% % 	    \end{center}
% %     then there is the following proof tree with the same conclusion
% % 	    \begin{center}
% % 	      $\inferrule* [left=\bf{Cong}]{
% % 		  P \equiv Q
% % 		\\
% % 		  \inferrule* [left=\bf{Cong}]{
% % 		      Q \equiv R
% % 		    \\
% % 		      R \xrightarrow{\alpha}_{3} R^{'}
% % 		  }{
% % 		    Q \equiv R
% % 		  }
% % 	      }{
% % 		P \xrightarrow{\alpha}_{3} R^{'}
% % 	      }$
% % 	    \end{center}
% %     and such that the size does not change, if in the size of a proof tree we count also the size of the proof trees of structural congruence.
%     We call $\delta_{1}$ the depth of the proof tree of $P\equiv R$ and $\delta_{2}$ the depth of the proof tree of $R\stackrel{\alpha}{\twoheadrightarrow}_{3} P^{'}$. The proof is an induction on $\delta_{1}+\delta_{2}$, we call $S_{1}$ the instance of the last rule of the proof tree of $P\equiv R$ and $S_{2}$ the instance of the last rule of the proof tree of $R \stackrel{\alpha}{\twoheadrightarrow}_{3} P^{'}$.
%     MANCANO I CASI PER LA REGOLA TRANS!
%     \begin{description}
%       \item[$\delta_{1}=0\; and\; \delta_{2}=0$]
% 	The only cases here are $P\equiv_{\alpha} R$ and $S_{2}\in \{Out, EInp, Tau\}$. For rule $Alp$ then $P \xrightarrow{\alpha}_{2} P^{'}$.
%       \item[$\delta_{1}=0\; and\; \delta_{2}>0$]
% 	There are various cases here for $(S_{1}, S_{2})$:
% 	\begin{description}
% 	  \item[$(Alp, \_)$]: in this cases we can use the rule $Alp$ of table \ref{transitionrelationearlywithalphaconversion}.
%     However we can avoid to consider the case where the proof tree of $P\equiv R$ ends with an application of the reflexivity because $\twoheadrightarrow_{3} \subseteq \rightarrow_{2}$. We can also avoid the case where the proof tree of $P\equiv R$ ends with $Trans$ because if a proof tree contains:
% 	    \begin{center}
% 	      $\inferrule* [left=\bf{Cong}]{
% 		  \inferrule* [left=\bf{Trans}]{
% 		      P \equiv Q
% 		    \\
% 		      Q \equiv R
% 		  }{
% 		    P \equiv R
% 		  }
% 		\\
% 		  R \xrightarrow{\alpha}_{3} R^{'}
% 	      }{
% 		P \xrightarrow{\alpha}_{3} R^{'}
% 	      }$
% 	    \end{center}
%     then there is the following proof tree with the same conclusion
% 	    \begin{center}
% 	      $\inferrule* [left=\bf{Cong}]{
% 		  P \equiv Q
% 		\\
% 		  \inferrule* [left=\bf{Cong}]{
% 		      Q \equiv R
% 		    \\
% 		      R \xrightarrow{\alpha}_{3} R^{'}
% 		  }{
% 		    Q \equiv R
% 		  }
% 	      }{
% 		P \xrightarrow{\alpha}_{3} R^{'}
% 	      }$
% 	    \end{center}
%     and such that the size does not change, if in the size of a proof tree we count also the size of the proof trees of structural congruence.
% 	  \item[$(SumAsc1, Sum)$]:
% 	  \item[$(SumAsc2, Sum)$]:
% 	  \item[$(SumCom, Sum)$]:
% 	  \item[$(ScpExtSum2, Sum)$]:
% 	  \item[$(ParAsc1, Par)$]:
% 	  \item[$(ParAsc2, Par)$]:
% 	  \item[$(ParCom, Par)$]:
% 	  \item[$(ScpExtPar2, Par)$]:
% 	  \item[$(ParAsc1, ECom)$]:
% 	  \item[$(ParAsc2, ECom)$]:
% 	  \item[$(ParCom, ECom)$]:
% 	  \item[$(ScpExtPar2, ECom)$]:
% 	  \item[$(Ide, Ide)$]:
% 	  \item[$(ResCom, Res)$]:
% 	  \item[$(ResCom, Opn)$]:
% 	  \item[$(ScpExtPar1, Par)$]:
% 	  \item[$(ScpExtPar1, ECom)$]:
% 	  \item[$(ScpExtSum1, Sum)$]:
% 	\end{description}
%       \item[$\delta_{1}>0\; and\; \delta_{2}=0$]
% 	There are various cases here for $(S_{1}, S_{2})$:
% 	\begin{description}	    
% 	  \item[$(CongOut,Out)$]:
% 	  \item[$(CongInp,EInp)$]:
% 	  \item[$(CongTau,Tau)$]:
% 	\end{description}	
%       \item[$\delta_{1}>0\; and\; \delta_{2}>0$]
% 	There are various cases here for $(S_{1}, S_{2})$:
% 	\begin{description}
% 	  \item[$(CongRes, Res)$]:
% 	  \item[$(CongRes, Opn)$]:
% 	  \item[$(CongPar, Par)$]:
% 	  \item[$(CongPar, ECom)$]:
% 	  \item[$(CongSum, Sum)$]:
% 	\end{description}	
%     \end{description}
%   \end{proof}
% \end{theorem}
% 
% 
\subsection{Equivalence of the late semantics}



\section{Bisimilarity, congruence and equivalence}

We present here some behavioural equivalences and some of their properties. In the following we will use the phrase $bn(\alpha)$ is fresh in a definition to mean that the name in $bn(\alpha)$, if any, is different from any free name occurring in any of the agents in the definition. We write $\rightarrow_{E}$ for the early semantic and $\rightarrow_{L}$ for the late semantic. It's not a concern which late semantic we are talking about because we have proved them equivalent.


\subsection{Late bisimilarity}

\begin{definition}
  A \emph{strong late bisimulation}(according to \cite{parrow}) is a binary simmetric relation $\mathbf{S}$ on processes such that for each process $P$ and $Q$, $P\mathbf{S}Q$ implies:
  \begin{itemize}
    \item
      if $P \xrightarrow{a(x)}_{L} P^{'}$ and $x\notin fn(P)\cup fn(Q)$ then there exists a process $Q^{'}$ such that $Q \xrightarrow{a(x)}_{L} Q^{'}$ and for all $u$ $P^{'}\{u/x\}\mathbf{S}Q^{'}\{u/x\}$
    \item 
      if $P \xrightarrow{\alpha}_{L} P^{'}$, $\alpha$ is not an input and $bn(\alpha) \cap (fn(P)\cup fn(Q)) = \emptyset$ then there exists a process $Q^{'}$ such that $Q \xrightarrow{\alpha}_{L} Q^{'}$ and $P^{'}\mathbf{S}Q^{'}$
  \end{itemize}
  $P$ and $Q$ are \emph{late bisimilar} written $P\dot{\sim}_{L}Q$ if there exists a strong late bisimulation $\mathbf{S}$ such that $P\mathbf{S}Q$.
\end{definition}

\begin{example}
  Strong late bisimulation is not closed under substitution in general:
  \[
    a(u).0|\overline{b}v.0\; \dot{\sim}_{L}\; a(u).\overline{b}v.0 + \overline{b}v.a(u).0
  \]
  and the bisimulation(without the simmetric part) is  the following:
  \[
    \{(a(u).0|\overline{b}v.0, a(u).\overline{b}v.0 + \overline{b}v.a(u).0),\;\; (a(u).0|0,a(u).0),\;\; (0|0,0),\;\; (0|\overline{b}v.0,\overline{b}v.0)\} 
  \]
  If we apply the substitution $\{a/b\}$ to each process then they are not strongly bisimilar anymore because $(a(u).0|\overline{b}v.0)\{a/b\}$ is $a(u).0|\overline{a}v.0$ and this process can perform an invisible action whether $(a(u).\overline{b}v.0 + \overline{b}v.a(u).0)\{a/b\}$ cannot.
\end{example}

We refer to strong late bisimulation as strong \emph{ground} late bisimulation, because it is not preserved by substitution.

\begin{proposition}
  If $P \dot{\sim} Q$ and $\sigma$ is injective then $P\sigma \dot{\sim} Q\sigma$
%   \begin{proof}
%     We have to establish that transitions are preserved by injective substitution, i.e., that for injective $\sigma$ 
%     \[
%       P\xrightarrow{\alpha}P^{'}\; \Rightarrow\; P\sigma\xrightarrow{\alpha\sigma}P^{'}\sigma
%     \]
%     by induction on the inference of $P\xrightarrow{\alpha}P^{'}$. For this purpose we consider the late semantic with structural congruence:
%     %The last rule of a derivation of $P\xrightarrow{\alpha}P^{'}$ can be:
%     \begin{description}
%       \item[$Pref$] 		
% 	An application of the rule pref let us prove that $\alpha\sigma.P\sigma\xrightarrow{\alpha\sigma}P\sigma$
%       \item[$Cong$]
% 	The premises of the rule tell us that: $P\xrightarrow{\alpha}Q$, $P^{'}\equiv P$ and $Q\equiv Q^{'}$. We apply the inductive hypothesis to the first premise and get $P\sigma\xrightarrow{\alpha\sigma}Q\sigma$. From the other two premises we get that $P^{'}\sigma\equiv P\sigma$ and $Q\sigma\equiv Q^{'}\sigma$. We should prove this separately. So applying the rule $Cong$ we get the result $P^{'}\sigma\xrightarrow{\alpha\sigma}Q^{'}\sigma$.
%       \item[$Sum$]
% 	
%       \item[$Par$]
%       \item[$Com$]
%       \item[$Res$]
%       \item[$Opn$]
%     \end{description}
%   \end{proof}
\end{proposition}

\begin{proposition}
  $\dot{\sim}_{L}$ is an equivalence
\end{proposition}

\begin{proposition}
  $\dot{\sim}_{L}$ is preserved by all operators except input prefix
\end{proposition}

\begin{definition}
  Two processes $P$ and $Q$ are \emph{strong late equivalent} written $P\sim_{L}Q$ is for each substitution $\sigma$ $P\sigma \dot{\sim}_{L}Q\sigma$
\end{definition}

\begin{example}
  If $z\notin fn(R)\cup \{x\}$ then $x(y).R \dot{\sim}_{L} (z)x(y).R$
\end{example}



\subsection{Early bisimilarity}

\begin{definition}
  A \emph{strong early bisimulation}(according to \cite{parrow})is a symmetric binary relation $\mathbf{S}$ on processes such that for each process $P$ and $Q$: $P\mathbf{S} Q$, $P \xrightarrow{\alpha}_{E} P^{'}$ and $bn(\alpha) \cap (fn(P) \cup fn(Q))=\emptyset$ implies that there exists $Q^{'}$ such that $Q \xrightarrow{\alpha}_{E} Q^{'}$ and $P^{'}\mathbf{S}Q^{'}$. $P$ and $Q$ are \emph{early bisimilar} written $P\dot{\sim}_{E}Q$ if there exists a strong early bisimulation $\mathbf{S}$ such that $P\mathbf{S}Q$
\end{definition}

\begin{definition}
  Two processes $P$ and $Q$ are \emph{strong early equivalent} written $P\sim_{E}Q$ if for each substitution $\sigma$ $P\sigma \dot{\sim}_{E}Q\sigma$
\end{definition}

\subsection{Congruence}

\begin{definition}
  We say that two agents $P$ and $Q$ are \emph{strongly congruent}, written $P\sim Q$ if
  \begin{center}
    $P\sigma \dot{\sim} Q\sigma$ for all substitution $\sigma$    
  \end{center}
\end{definition}

\begin{proposition}
  Strong congruence is the largest congruence in bisimilarity.
\end{proposition}


\subsection{Open bisimilarity}

\begin{definition}
  A \emph{distinction} is a finite symmetric and irreflexive binary relation on names. A substitution $\sigma$ \emph{respects} a distinction $D$ if for each name $a,b$ $aDb$ implies $\sigma(a)\neq \sigma(b)$. We write $D\sigma$ for the composition of the two relation.
\end{definition}


\begin{definition}
  An \emph{strong open simulation}(according to \cite{parrow}) is $\{S_{D}\}_{D\in \mathbb{D}}$ a family of binary relations on processes such that for each process $P, Q$, for each distinction $D\in \mathbb{D}$, for each name substitution $\sigma$ which respects $D$ if $P S_{D} Q$, $P\sigma \xrightarrow{\alpha} P^{'}$ and $bn(\alpha)\cap (fn(P\sigma)\cup fn(Q\sigma))=\emptyset$ then:
   \begin{itemize}
    \item 
      if $\alpha=\overline{a}(x)$ then there exists $Q^{'}$ such that $Q\sigma \xrightarrow{\overline{a}(x)} Q^{'}$ and $P^{'} S_{D^{'}} Q^{'}$ where $D^{'}=D\sigma \cup \{x\}\times (fn(P\sigma)\cup fn(Q\sigma)) \cup  (fn(P\sigma)\cup fn(Q\sigma))\times\{x\}$
    \item
      if $\alpha$ is not a bound output then there exists $Q^{'}$ such that $Q\sigma \xrightarrow{\alpha} Q^{'}$ and $P^{'} S_{D\sigma} Q^{'}$
  \end{itemize}
  $P$ and $Q$ are \emph{open D bisimilar}, written $P \dot{\sim}_{O}^{D} Q$ if there exists a member $S_{D}$ of an open bisimulation such that $P S_{D} Q$; they are \emph{open bisimilar} if they are open $\emptyset$ bisimilar, written $P \dot{\sim}_{O} D$.
 

\end{definition}
 
 















